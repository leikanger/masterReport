
\chapter{Synaptic Transmission and Plasticity in the Biological Neuron} %xxx Eller biologisk neurale sys. (er vel strengt tatt ikkje i nevronet, ELLER er det det?)
\label{appendixSynapticPlasticity}

%fra FDP. Direkte kopi. 15.02.2012

%fra NEVR3001 rapport om synPlast : (Kraftig omskrevet)

%\section{Introduction}

The background of memory and learning is a topic that have received much attention in the field of neuroscience. 
Learning on the neural level is represented by the alteration of synaptic weights in a neural network. 
This is called synaptic plasticity, and is one focus of this appendix.
%This appendix is ment to introduce aspects that are perceived to be important in synaptic transmission and plasticity.


% The background of memory and learning is a topic that receives much attention in the field of neuroscience.
% %It has recieved much attention lately, and the knowledge about synaptic plasticity and synaptic transmission has progressed significantly the last decades.
% In this appendix the different mechanisms behind synaptic transmission and plasticity will be in focus.
% We will see that the the ion $Ca^{2+}$ has a crusial role in both presynaptic transmission mechanisms and in postsynaptic mechanisms involved in synaptic plasticity.
 
% To get an understanding of synaptic plasticity, the reader first needs some background information on how chemical synapses propagates information.
% %This appendix introduces this subject thoroughly, and can be used to find additional information about the biological system.
% %
% Synaptic transmission and plasticity in chemical synapses depend on a multitude of mechanisms in the presynaptic axon terminal, the postsynaptic membrane and also in glial cells surrounding the synapse.
% This appendix therefore separate between these locations, and each element is presented in a separate section.
% A more compact summary is found in the last section of this appendix, \ref{appendixSynapticPlasticity:summary}.
% %A more compact version is found in the summary in section \ref{appendixSynapticPlasticity:summary}.
% %Synaptic plasticity can be separated into two groups.
% %Synaptic plasticity can be short--term, lasting from milliseconds to minites, or long--term alterations than last for a life time.

Synaptic transmission in chemical synapses is dependent on a multitude of different mechanisms situated in the presynaptic axon terminal, the postsynaptic membrane or in glial cells surrounding the synapse.
This appendix therefore separate between these locations, and each element is presented in a separate section.
A more compact summary is found in the last section of this appendix, \ref{appendixSynapticPlasticity:summary}.
% The mechanisms that are thought to be most important in synaptic transmission is introduced in this appendix.
Some elements involved in synaptic plasticity is also introduced, and one of the motivations for utilizing spiking neuron simulations is justified.
%We will also look at some elements involved in synaptic plasticity, or change in the size of the synaptic transmission. %TODO Skriv kvifor? FÅ RELEVANS!
% TODO Skriv noke slikt som: Because the ultimate goal of this appendix is to discuss the reason behind going on from
% XXX 	og end opp på at eg må beskrive syn.p. også.
% xxx  	Kan kanskje bare skrive at desse er tett sammenknytta. Nei, eg må forklare kvifor eg vil beskrive STDP.
%In this essay, some of the mechanisms involved in synaptic transmission will be described. 
%This is nessecary in order to say something of what changes the same mechanisms.


%Innleder synaptic transmission:
When the neuron is sufficiently depolarized, an action potential is initiated at the axon hillock.
The action potential propagate through the axon to the ``axon terminal'' where the presynaptic part of the synapse is found.
When there is a strong depolarization over the presynaptic membrane at the synapse, a cascade that ends up with a change in the postsynaptic neurons value is initiated\cite{PurvesNeuroscienceKAP05}.
This is cascate is often referred to as synaptic transmission. 
% This is what is referred to as synaptic transmission\cite{PurvesNeuroscienceKAP05}.

The size of postsynaptic change in depolarization after a transmission varies with the amount of neurotransmitter release and the number of neurotransmittor receptors in the postsynaptic membrane\cite{PurvesNeuroscienceKAP05}.
%The size of the synaptic transmission is dependent on the amount of neurotransmitters released from the presynaptic part of the synapse and the number of neurotransmitter resceptors in the postsynaptic membrane	\cite{PurvesNeuroscienceKAP05}. 
Synaptic plasticity, what is percieved as the basis of learning, can happen on a short--term of long--term timescale.
Short--term synaptic plasticity is transient, and can be said to involve changes of what can be perceived as ``the state'' of the synapse.
%Short--term synaptic plasticity normally only involves factors that can be seen as ``the state'' of the synapse. 
Factors as neuromodulators, amount of $Ca^{2+}$ on the in-- or out--side the neuron and amount of neurotransmitters available are examples of such factors.
Long--term synaptic plasticity involves ``lasting changes'' in the synapse, and can e.g. result in protein synthesis and the generation of new postsynaptic receptors\cite{PurvesNeuroscienceKAP8}. %todo har ikkje sett gjennom kap.8 når eg cite.
																										% Veit innholdet, men det er kanskje lurt å sjekke at alle desse elementa stå i purves kap 8.

\section{The Presynaptic Part of Synaptic Transmission}
\label{appendixSynapticPlasticity:presynapticMechanisms}%appendixSecPresynapticSynapticPartOfTransmission}
Propagation of the action potential along the axon works as a combination of active and passive electrochemical transmission.
%The propagation of the action potential along the axon, happens as a combination of passive and active transmission.
On the axon, there are voltage--gated $Na^+$ and $K^+$ channels, activated by the membrane potential being above some threshold.
These channels work so that the membrane potential is further depolarized, before the channels are closed after a short time interval.

The electrical potential is spreads passively along the axon, opening the voltage--gates channels at the site where active channels are located.
Such locations are called ``nodes of Ranviere'' for insulated neurons, and improves the transmission speed in comparison with un--insulated axons.
% These will open when the electrical membrane potential supasses a threshold, and will further increase the value of the potential. 
After a while the channels close, and a sequence that resets the potential to the base potential is initialized. %\cite{PrinciplesOfNeuralScience4edKAP09}. 
This ampifies the signal so that is may continue passively down the axon to the next site with voltage--gated channels\cite{PrinciplesOfNeuralScience4edKAP09}. 
% %and will not activate reamplification before it encounters special voltage--gated $Na^+$ and $K^+$ channels.
% %The axon is insulated by a special glia cell, to increase the speed of action potential propagation.
% %These are situated in gaps in the insulation, or glia
% This will reamplify the signal  so that is may continue passively down the axon to the next voltage gated channels.
% %In an action potential, electrical potential is first transmitted passively down the axon of a neuron. On the axon we have voltage--gated $Na^+$ and $K^+$ channels 
% %	that open when the electrical potential over the membrane surpasses a threshold\cite{PrinciplesOfNeuralScience4edKAP09}. 
% %This will enhance the signal so that it can continue passively to the next voltage gated channels.
 
% In the biological nervous system, insulative cells called myelin are located around neurons, and increase the distance that can be travelled passively for the signal.
% %Such cells comprise what is called myelin.
% %%In the nervous system, there are specialized insulative cells called myelin.
% In myelinated neurons, the voltage--gated channels are located in gaps in the myelin.
% These gaps are called ``nodes of Ranvier''\cite{PrinciplesOfNeuralScience4edKAP09}..
% %In myelinated neurons these voltage--gated channels are located in gaps in the myelin, called ``nodes of Ranvier'', in unmyelinated neurons the channels are located continously along the axon membrane. The channels will restore the signal, and constitutes the active part of current transduction along the axon\cite{PrinciplesOfNeuralScience4edKAP09}.


%*********************** OK så langt. ***********************

When the action potential reaches the axon terminal, the end of the axon down the signal path, 
	voltage gated $Ca^{2+}$ channels are opended at the active zones of the terminal.
This enables $Ca^{2+}$ to enter the cytosol of the presynaptic axon terminal\cite{PrinciplesOfNeuralScience4edKAP10}.
	%it will open voltage gated $Ca^{2+}$ channels at the active zones of the terminal. 
%This causes $Ca^{2+}$ to enter the cytosol of the axon terminal of the presynaptic neuron\cite{PrinciplesOfNeuralScience4edKAP10}.
%%
Calcium  cause synaptic vesicles, bag like organelles that contain e.g. neurotransmitters, to fuse with the membrane and releace its contents into the synaptic cleft.
%$Ca^{2+}$ causes synaptic vesicles to fuse with the membrane and release the contained neurotransmitters into the synaptic cleft.%\cite{PrinciplesOfNeuralScience4edKAP10}. 
The synaptic cleft is the space between the pre- and post-synaptic neuron and is fundamental for signal transmission in chemical synapses.
%There is a linear relationship between the amount of $Ca^{2+}$ entering the cytosol and the amount of synaptic vesicles fusing with the membrane (exocytosis).%\cite{PrinciplesOfNeuralScience4edKAP10}. 
Ecocytosis of a synaptic vesicle cause the neurotransmitters stored in it to be released into the synaptic cleft \cite{PrinciplesOfNeuralScience4edKAP10}. 
The amount of neurotransmitters released into the synaptic cleft therefore have a linear relationship with the amount of $Ca^{2+}$ entering the presynaptic axon terminal.
% TODO Ta med, eller ta vekk?      , given a constant amount of neurotransmitters stored in each synaptic vesicle.
%Exocytosis releases the content of the synaptic vesicle to the synaptic cleft\cite{PrinciplesOfNeuralScience4edKAP10}.

%XXX Viktig, men utafor scope av teksten:
%%%%%%The linear relation between the amount of $Ca^{2+}$ entering the axon terminal and the amount of synaptic vesicles undergoing excytosis was first proposed by Katz and Miledi, and later shown by Rodolfo Llinàs and colleges\cite{PrinciplesOfNeuralScience4edKAP14}. 
%KANSKJE:
% TODO Vettafaen om det skal være med:
% Dette gir at vi kan ha presynaptisk short--term syn.p., noke som er viktig argument for å innføre axo-axonic synapses (temporal synaptic modulatory system)
%Rodolfo Llinàs and colleges first showed the mechanisms of a linear relationship between the amount of $Ca^{2+}$ entering the cytosol of the presynaptic cytosol.
%They also found that the $Ca^{2+}$ channels are graded by the potential over the axon terminal membrane. 
%This further gives a graded responce of neurotransmitter release based on the preysnaptic membrane potential \cite{PrinciplesOfNeuralScience4edKAP14}.
%xxx kan ikkje ta vekk, lett. Bruker dette resultatet seinere.. ELLER?

% TODO Flytt alle \cite{} til slutten av avsnittet (dersom de påstandene på slutten også står her..)
% Har sjekka: alle påstandene er fra kap14 i Kandel.
There is a steady influx of $Ca^{2+}$ at axon terminals, through the L-type $Ca^{2+}$ channel. %\cite{PrinciplesOfNeuralScience4edKAP14}. 
This influx of calcium is graded by the potential over the presynaptic membrane.
When multiple synaptic transmissions happens within a short period of time, 
	the amount of $Ca^{2+}$ in the presynaptic axon terminal builds up and the following synaptic transmissions will give a successively larger effect on the postsynaptic potential.
This effect is called \emph{potentiation}, and can last from minites to more than an hour.
% Føler at neste linja ikkje heilt passer inn XXX:
% TODO Vær sikker på at axo--axonic synases er definert!
This mechanism also gives the effect of axo-axonic synapses in regulating the amount of neurotransmitter release for the next transmissions\cite{PrinciplesOfNeuralScience4edKAP14}.
%The axo-axonic synapses will not influence the firing of a neuron, only the membrane potential of the an axon terminal. %XXX KVA ER axo--axonic synapses? Inled dette for leser!
% % and thus the amount of neurotransmittors released by the following action potential\cite{PrinciplesOfNeuralScience4edKAP12}.
%This gives a mechanism for controling the postsynaptic exitatory postsynaptic potential between other neurons following an action potential.

%TODO TA vekk?
%When two action potentials reaches the axon terminal in fast succession it will cause the synapse to be stronger (give a larger postsynaptic response) for many minutes. 
%This is called \emph{potentiation}, and is thought to be partially because of the increase in presynaptic cytosol $Ca^{2+}$ levels\cite{PrinciplesOfNeuralScience4edKAP14}. In the mossy fiber pathway of the hippocampus, presynaptic $Ca^{2+}$ influx is an important mechanism for synaptic plasticity \cite{PrinciplesOfNeuralScience4edKAP63}. %XXX Sjekk! (mest viktige, eller bare viktig?)

%XXX TA VEKK? 
%A decrease in the number of synaptic vesicles undergoing exocytocis has been observed in sensory neurons of the \emph{Aplysia Californica} following LTD. %, by quantal analysis 
%The mechanisms for decrease in synaptic vesicle exocytosis is not known\cite{PrinciplesOfNeuralScience4edKAP63}.

% dette er kanskje interresant: Det kan også være en basis for STDP.. 
% Sjå om det skal takast vekk, dagen før innlevering.
An increase in the extracellular level of glutamate has been observed after LTP in CA3 neurons. 
The mechanisms behind this is debated, but evidense has been presented of \emph{retrograde messangers} from the postsynaptic neuron that will 
	give feedback to the presynaptic neuron after transmission \cite{PrinciplesOfNeuralScience4edKAP63}. 
This enables a  presynaptic component of \emph{long--term} synaptic plasticity.








\section{Postsynaptic Mechnisms of Synaptic Plasticity}
\label{appendixSynapticPlasticity:postsynapticMechanisms}
%There are tree groups of receptors in the postsynaptic membrane of a synapse. AMPA, .......NMDA, kainate, ....
Because neuroscience mainly have focused on excitatory glutamate synapses, the discussion about postsynaptic mechnisms behind synaptic plasticity will focus on glutamate transmission.

There are two groups of glutamate receptors: NMDA and non-NMDA receptors. 
The non-NMDA receptors consists of the AMPA and the kainate receptors. 
Most non-NMDA receptors are only permeable to $K^+$ and $Na^+$, while the NMDA receptor is permeaple to $Ca^{2+}$ in addition to $K^+$ and ${Na}^+$ \cite{PrinciplesOfNeuralScience4edKAP12}. 


The NMDA--receptor is an ion channel that is both voltage gated and ligand gated: 
	It requires both that the glutamate neurotransmitter is present in the extracellular fluid and a strong depolarization over the membrane to open \cite{PrinciplesOfNeuralScience4edKAP12}. 
When we get a transmission when the postsynaptic neuron is strongly depolarized, we therefore get an influx of calcium at the postsynaptic neuron.
$Ca^{2+}$ will activate calcium dependent enzymes and also protein kinases that leads to long--term synaptic plasticity\cite{PrinciplesOfNeuralScience4edKAP12}.
This is done as a result of the calcium dependent enzymes initiating synthesis of new AMPA receptors \cite{AMPARtrafficingArtikkel}. 
More receptors causes a larger probability of the glutamate neurotransmittor having an effect, and thus increases the effectivity of the synapse (the synaptic weight).

To conclude this section we will compare the statistical relationship between the postsynaptic neuron having a large depolarization at the time of transmission and the relative timing of the transmission, 
	in relation to the postsynaptic action potential. 
If the postsynaptic neuron is strongly depolarized at the time of transmission, this implies that the postsynaptic neuron will fire soon after.
This might be one of the basis of what has been known by the name Spike Time Dependent Plasticity (STDP).
% OMGJODT TIL HIT:  XXX XXX XXX XXX XXX XXX XXX XXX XXX XXX XXX XXX XXX XXX XXX XXX XXX XXX XXX XXX XXX XXX XXX XXX XXX XXX XXX XXX XXX XXX XXX XXX XXX 
% TODO KAnskje ta vekk resten (med unntak av Summary?)

%XXX XXX XXX XXX 
%If two transmissions happens in rapid succession, you will get a strong (lokal) depolarization around the postsynaptic receptors. This will cause the NMDA--channels to open at the second transmission, and admit $Ca^{2+}$ into the postsynaptic neuron. 
%TODO Skriv heller om at depolarisasjonen har mykje å seie. NEI, dette står allerede. Skriv korleis tid kan ha noke å seie (begrunn STDP med bakgrunn i teoien her). XXX Gjør eit poeng ut av kvifor eg har tatt med dette i appendixet.
% 			Dette kan helst gjøres etter neste setning.

%Also in the postsynaptic neuron, calcium has an important role in synaptic plasticity. 
%$Ca^{2+}$ will activate calcium dependent enzymes and also protein kinases that leads to long--term synaptic plasticity\cite{PrinciplesOfNeuralScience4edKAP12}.
%%Second messangers can also be activated by metabotropic receptors in addition to $Ca^{2+}$ and the same protein kinases are activated. 

% IKKJE RELEVANT:
%The calcium is thought to be important in both short-term potentiation by enhancing the response of AMPA receptors to glutamate\cite{PrinciplesOfNeuralScience4edKAP63}, 
% 	and also elicit ``permanent'' synaptic changes by receptor synthesis\cite{AMPARtrafficingArtikkel}. %Dette ER relevant, men skrevet over.
%This is thought to enhance the response of AMPA receptors to glutamate\cite{PrinciplesOfNeuralScience4edKAP63}, but also elicit longer lasting (``permanenet'') synaptic plasticity.

\section{Receptor Synthesis}
The rise in calsium levels in the postsynaptic cytosol activates postsynaptic plasticity\cite{AMPARtrafficingArtikkel}. 
One of the possible mechanisms behind postsynaptic LTP or LTD is the increase or decrease in postsynaptic receptors. There has been increased focus on receptor trafficing in the recent years, especially on the AMPA receptor. 

\begin{quote}
At early stages of development, synapses containing only NMDA type receptors are particularly common\cite{PrinciplesOfNeuralScience4edKAP12}.
\end{quote}
Synapses containing only the NMDA receptor is called ``silent synapses'' because they do not change the postsynaptic potential (PSP) unless the postsynaptic membrane is sufficiently depolarized. 
This makes them silent at normal resting membrane potential\cite{AMPARtrafficingArtikkel}. 
It has been observed that these ``silent synapses'' is converted into normal exitatory synapses by the insertion AMPA receptors into the postsynaptic membrane\cite{AMPARtrafficingArtikkel}. 

It has been shown that when synapses undergo LTD, the amount of AMPA receptors in the postsynaptic membrane decreases\cite{AMPARtrafficingArtikkel}. 
This is believed to be because of endocytosis of the receptors. If the dynamin-dependent endocytosis is blocked, LTD is also blocked in the samle\cite{AMPARtrafficingArtikkel}.

\section{Glial Modulation of Synaptic Transmission}
One way for asterocytes to modulate synaptic transmission is to release ATP, which is converted to adenosine extracellularly. 
%TODO Vær sikker på at det er signallingBetweenGlialAndNeuronsInSynapticPlasticity som meines. det stod signallingBetween Glial And Neurons In SynPlast
Adenosine inhibits the $Ca^{2+}$ channels in the presynaptic axon terminal membrane\cite{signallingBetweenGlialAndNeuronsInSynapticPlasticity}. 
This results in less exocytosis of synaptic vesicles in the presynaptic membrane, which gives less neurotransmitters in the synaptic cleft as a consequence\cite{signallingBetweenGlialAndNeuronsInSynapticPlasticity}.
%This also affects neighboring synapses\cite{signallingBetweenGlialAndNeuronsInSynapticPlasticity}. %men det er uklart om dette er pga slett andre plasser også, eller diffusjon.

For the NMDA receptor channels to open, three conditions has to be met:
\begin{enumerate}
	\item Glutamate needs to be present in the synaptic cleft.
	\item The postsynaptic membrane needs to be sufficiently depolarized.
	\item D-serine needs to be present in the synaptic cleft\cite{signallingBetweenGlialAndNeuronsInSynapticPlasticity}.
\end{enumerate}
The point about D-serine is interresting, since D-serine is absent in neurons. It is present in asterocytes.
One possible explanation it therefore that asterocytes release the D-serine required for the NMDA-R to open\cite{signallingBetweenGlialAndNeuronsInSynapticPlasticity}.  % D-Serine bindes til glycine-binding site.
This indicates that the asterocytes are important in modulating the synaptic plasticity induced by NMDA-R opening.

Glial cells are also important for synaptic transmission by being permeable to $K^+$ from the extracellular fluid of the synaptic cleft\cite{PrinciplesOfNeuralScience4edKAP07}, 
and by being in control of the reuptake of certain neurotransmittors (eg. glutamate)\cite{PrinciplesOfNeuralScience4edKAP15}. %kap 15 kandell, 
%This gives possible astrocyte mechanisms for modulating synaptic transmission.


% Ta vekk FRA HER TODO TODO 
\section{Synaptic Transmission an Plasticity Summary}
\label{appendixSynapticPlasticity:summary}
%TODO Skriv kvifor eg har skevet alt dette. Få relevans. (STDP, som er viktig argument for SANN)
The subject about synaptic plasticity is important for the understanding of neural systems.
We have presynaptic and postsynaptic elements of synaptic transmission, both subject to continous change. This gives two possible elements of synaptic plasticity. 

The presynaptic part of synaptic transmission can be regulated by changing the presynaptic voltage gated $Ca^{2+}$ channels. One way this is done is by axo-axonic synapses that depolarises the axon terminal before the action potential, and thus enhance/inhibit or prolong/shorten the influx of calcium. 
This results in a change in the amount of neurotransmittors released into the synaptic cleft.

The postsynaptic part of synaptic plasticity consists of short term changes, by changing the effect of AMPA-R with calcium, or long lasting changes involving protein synthesis and the insertion of new AMPA-R in the postsynaptic membrane. Both are dependent on calcium. Changing the postsynaptic influx of calcium is therefore an other plausible mechanism for synaptic plasticity.

The asterocytes maintains the environment for the synaptic transmission by maintaining the ion consentrations in the extracellular fluid in the synaptic cleft. 
Modulation of this will change the environment for synaptic transmission and be a way of changing the effect of synaptic transmission. 

The asterocytes are also responsible for removing some neurotransmittors from the synaptic cleft. 
This makes them in control of the time the neurotransmittor is in the synaptic cleft, and thereby the time it will be effective on the postsynaptic receptors.
This desides the postsynaptic effect of the transmission. Change of this is yet an other mechanism for synaptic plasticity.
%This opens for yet an other mechanism for the asterocytes to control the postsynaptic response of a transmission.


%\begin{figure}[!htbp]
%	\centering
%	\includegraphics[width=0.8\textwidth]{figurSTDP.jpeg}
%	\caption{Spike timing-dependent plasticity. a, Synapses are potentiated if the synaptic event precedes the postsynaptic spike. Synapses are depressed if the synaptic event follows the postsynaptic spike. b, The time window for synaptic modification. The relative amount of synaptic change is plotted versus the time difference between synaptic event and the postsynaptic spike. The amount of change falls off exponentially as the time difference increases. In addition, the amount of potentiation decreases for stronger synapses, whereas the relative amount of depression is independent of synaptic size.}
%\end{figure}

