\chapter{UML Class Diagrams}
	To make this report a better documentation of $auroSim$, the UML class diagrams of the most important classes have been included in this appendix.
	In addition to the different subelement classes, the UML diagram of $time\_class$ is presented in this appendix.
	All elements are derived from class $time\_interface$, making all elements inherit the pure virtual functions $doTask()$ and $doCalculation()$.
	Unless these are overloaded in the derived class, that class is also abstract and no instances of it can be made from it.
	
	
\newpage
\section{Time Class}
		\begin{figure}[hbt!p]
			\centerline{ \includegraphics[width=0.9\textwidth]{UML/classDiagramForTimeClass} }
			\caption[UML class diagram for \emph{time\_class}]{
				The main response of class \emph{time\_class} is all aspects of simulation time. 
				pWorkTaskQueue have all objects with tasks, including an object of time class, whose task's main responsibility is to iterate time.
				Most elements of \emph{time\_class} is declared \emph{static}, and have a class--wide scope.
					}
		\end{figure}

\newpage
\section{Node Subelement Classes}
\label{appendixUMLofAllNodeSubelementClasses}
	The artificial neuron has a design like the functional lay out of a biological neuron, as illustrated in fig. \ref{figFigurAvNeuronet}.
	This gives the design presented in fig. \ref{figUMLClassDiagramForASingleNeuron}.
	The UML class diagram of the different subelement classes is presented in this appendix.

		\begin{figure}[hbt!p]
			\centerline{ \includegraphics[width=0.9\textwidth]{UML/classDiagramForDendrite} }
			\caption[UML class diagram for dendrite subelement]{UML class diagram for the dendrite subelement.}
		\end{figure}

		\begin{figure}[hbt!p]
			\centerline{ \includegraphics[width=0.9\textwidth]{UML/classDiagramForAuronSubclass} }
			\caption[UML class diagram for auron subelement]{UML class diagram for the auron subelement. The sensory auron, used in the experiments, is a specialization of the auron.}
		\end{figure}

		\begin{figure}[hbt!p]
			\centerline{ \includegraphics[width=0.9\textwidth]{UML/classDiagramForAxon} }
			\caption[UML class diagram for axon subelement]{
						UML class diagram for axon subelement.
						It is found that $\kappa M$'s ability to schedule tasks can be used to simulate spatio--temporal effects like axonic transmission.
						The axon is therefore not nessecary in the $\kappa M$ implementation.
					}
		\end{figure}

		\begin{figure}[hbt!p]
			\centerline{ \includegraphics[width=0.9\textwidth]{UML/classDiagramForSynapse} }
			\caption[UML class diagram for synapse subelement]{
				UML class diagram for synapse subelement. 
				Both $K\_synapse$ and $s\_synapse$ is derived from the abstract class $i\_synapse$.
			}
		\end{figure}
\newpage
	
