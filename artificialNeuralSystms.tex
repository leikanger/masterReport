

\section{Artificial Neural Systems : A Review of ANN History}
	\label{ssecHistoryOfANN}
	The pragmatic use of neural network simulations started with the ``McCulloch--Pitts neuron'' in 1943.
	Warren McCulloch, an early neuroscientist, and the young mathematician Walter Pitts initiated a formalized discussion about the mechanics of the neuron and the use of neuron simulations in technology.
	This resulted in the first neuron emulator(artificial neuron). 
%
	Artificial Neural Networks based on the McCulloch--Pitts neuron model has later been referred to as the first generation ANN\cite{Maass97networksofSN}.
	Each node is modelled as a boolean device(with an on--off response), where the node sends output if the immediate level of input is large enough.
	The first generation ANN can therefore be said to be a network of simple threshold gates, 
		and does not take into consideration the depolarization state of each node. %, and is a tremendous simplification of the biological neuron.
	One famous example of a first generation ANN is Rosenblatt's \emph{Perceptron}\cite{HaykinANNbok}.

\begin{figure}[bth!p]
	\centering
	\includegraphics[width=0.75\textwidth]{sigmoidCurve}
	\caption[The sigmoid curve that is often used to compute the output of a node in second generation ANNs]{Sigmoid curve $\frac{1}{1+e^{-x}}$, often used as the activation funtion in second generation ANNs.}
	\label{figSigmoidCurve}
\end{figure}

%TODO TODO TODO TODO TODO TODO TODO TODO TODO TODO TODO TODO CITE TODO TODO TODO TODO TODO TODO TODO TODO TODO TODO TODO TODO Heile neste avsnitt!
	A better simulation of the neuron is done in the nodes of a second generation ANN\cite{Maass97networksofSN, NEVR3004OmModellane}.
	Each node computes the output level as a floating point number based on the immediate level of input to the node.
	From sec. \ref{secBiologicalNeuralSystems}, we have that the biological neuron sends discrete output pulses when its depolarization level goes above the firing threshold.
	A continuous propagation of a floating point number can therefore be said to represent the frequency of such transmissions as a function of the present input.
%%%
	The function used for computing the output is referred to as the \emph{activation function} of the node, and is found to give the best results if it is given by the continuously differentiable sigmoid function\cite{HaykinANNbok}.

\begin{equation}
	\sigma(x)=\frac{1}{1+e^{-x}}   %TODO TODO TODO TODO TODO TODO Lag heller en figur for å vise sigmoid function! (Bytt ut ligning med fig!) TODO TODO TODO TODO TODO TODO TODO
\end{equation}
	
	As the biological neuron has a state, defined by the depolarization value of the neuron, the stateless computation of the McCulloch--Pitts neuron is a gross simplification of the original system.
	It is more correct to consider the neuron as being stateless in the frequency domain, and the stateless computation of a second generation ANN can therefore be said to be more correct than in a first generation ANN\cite{Maass97networksofSN, FDP_report}.
	Since the concept of mean frequency only makes sense for time intervals of a certain size, precise simulations with small computational time steps does not necessarily give accurate simulation results for an ANN of the second generation. 
	%%%
	For more precise simulations of the neuron or simulations where temporal mechanisms are important, one has to simulate the neuron in the time domain by considering its depolarization.




\section{Spiking Artificial Neural Networks}
	\label{secSpikingANNBackgroundInfo}
	A node in a $SANN$ simulates the neuron with respect to its depolarization.
	When the depolarization value crosses the firing threshold, a spike is initiated and the signal is propagated in the neural network.
% 	When the depolarization value crosses the firing threshold from below, a spike is initiated and the signal is propagated through the neural network.
	Many formal spiking neuron models exist, where the most common is the $LIF$ neuron model\cite{florian03}. %TODO Sjekk om dette står der!
	
	
\begin{figure}[tb!hp]
	\centering
	\includegraphics[width=0.85\textwidth]{figRCcircuit}
	\caption[A schematic diagram of the $LIF$ neuron model]
			{A schematic diagram of the $LIF$ neuron model. 
			Each node can be modelled as the circuit inside the dashed circle on the right--hand side of the figure.
			Depolarizing input is represented by the input current $I(t)$, and when the potential over the capacitor is larger than the firing threshold at time $t_i^{(f)}$, a spike $\delta(t-t_i^{(f)})$ is generated. 
			The left--hand side of the figure shows a model of synaptic transmissions as a low--pass filtering of the presynaptic action potentials $\delta(t_j^{(t)})$, generating a input current $\alpha(t-t_j^{(f)})$ to neuron $i$
			\cite{gerstnerKistler2002}.
% TODO TODO Amund seier at denne referansen oppfattes som om det er figur pluss teks som kommer fra artikkelen. Gjøre om på dette?  Korleis?
% TODO TODO TODO TODO TODO TODO TODO TODO TODO TODO TODO TODO TODO TODO TODO TODO TODO TODO TODO TODO TODO TODO TODO TODO TODO TODO TODO TODO TODO TODO TODO TODO TODO 
			}
	\label{figRCcircuitAvNeuronet}
\end{figure}

	The $LIF$ model is a simple phenomenological model of the biological neuron, and is highly popular due to its simplicity.
	The leaky integration of the $LIF$ neuron model can be modelled by the electrical circuit shown in fig. \ref{figRCcircuitAvNeuronet}.
	When the membrane potential $v(t)$ crosses the neuron's firing threshold $\tau$, a spike is initiated and transmissions through all the neuron's output synapses is the result.
	The neuron's membrane potential is then reset to the resting membrane potential $v_r<\tau$ \cite{gerstnerKistler2002}. %\cite{gerstnerKistler2002KAP04}.

	%If the  TODO Innfør ligning 12 fra \ref{florian03}.
	% TODO Gjør det her!
\begin{equation}
	\lim\limits_{t\to t^{(f)}; t>t^{(f)}} v(t) = v_r
\end{equation}

	To the author's knowledge, the $LIF$ neuron has previously only been simulated by numerical integration in the computer.
	%To the author's knowledge, the only way of simulating the $LIF$ neuron in the digital computer is by numerical integration.
	The depolarization is integrated numerically by summing all synaptic input transmissions, and subtracting the leakage after each computational time step.
	%All depolarizing current, and the leakage is integrated numerically by summing all synaptic input transmissions, and subtracting the leakage after each computational time step.
	If the efficiency of the synapse from neuron $j$ to neuron $i$ is modelled by $\omega_{ij}$, the total input current during a time interval can be found by 
\begin{equation}
	I_i(t_n) = \sum_j \omega_{ij} \sum_f a(t_n - t_j^{(f)}) + \xi_i(t_n)
\end{equation}
	where $a(t- t_j^{(f)})$ is defined by the neuron's synaptic input currents and $\xi_i(t_n)$ comes from other sources of depolarizing input\cite{florian03}.
	%TODO TODO Slutten av neste setning er nok ikkje fra {florian03}. Sjekk dette, og isåfall endre neste setning.. TODO TODO Fiks det slit ag neste setning stemmer med florian03-referansen. Sjekk ligningene hans..
	Examples of this can be a current inserted through a probe, or the sensed signal of a sensory neuron.

	Leakage can be simulated by subtracting a fraction of the present membrane potential every time step.
	Because computational resources are limited, a finite temporal resolution(discrete time) is utilized.
	This involves discrete steps in simulation time, and the previously computed depolarization level has to be used to find the leakage $l(t_n)$.
	This introduces a delay, with the size of the computational time step, and establishes an error source for the simulation.
%	This introduces a delay, with the size of the computational time step, and is an error source for the simulation.
\begin{equation}
	%I_R(t_n) = \alpha v(t_{n-1})
	l(t_n) = \alpha v(t_{n-1})
\end{equation}
%%
	Thus, the discretization of time introduces an error defined by the computational time step ---
		if the size of the computational time step is increased, the simulation error becomes larger.
	More accurate simulations can therefore be accomplished by making the computational time steps smaller, but this simultaneously increase the computational load of the simulation.
% Veit ikkje om eg skal sitere dette. Det er forsåvidt mitt arbeid, men dei satte meg på ideen!
% 	\cite{PlesserStraubeMorrisonPlesser2007}.
	


	% 	The simplest form of the $LIF$ neuron utilize sample--and--hold numerical integration. %%% I think this statement can be found in  Maass97networksofSN



% TODO TODO TODO TODO TODO TODO TODO TODO TODO TODO TODO TODO TODO TODO TODO TODO TODO TODO TODO TODO TODO TODO TODO TODO TODO TODO TODO TODO TODO TODO TODO TODO TODO TODO TODO  Lag figuren som viser ANN på nytt! Litt slurvete, nå.
% TODO TODO TODO TODO TODO TODO TODO TODO TODO TODO TODO TODO TODO TODO TODO TODO TODO TODO TODO TODO TODO TODO TODO TODO TODO TODO TODO TODO TODO TODO TODO TODO TODO TODO TODO 
\begin{figure}[hbt!p]
    \centering
    \subfloat[Model of Spiking ANN Connections]{
        \label{figAuronE:subfigModel}
        \includegraphics[width=0.45\textwidth]{auronE_fig}
    }
    %%
    %%
    \subfloat[Depolarisation Time Course]{
        \includegraphics[width=0.55\textwidth]{auronE_depol}
        \label{figAuronE:subfigDepolarization}
    }
    \caption[An artificial neural circuit to illustrate numerical integration of the $LIF$ neuron. Schematic model and simulation results.]{
			(\ref{figAuronE:subfigModel}) A schematic model of the synaptic connections in an artificial neural circuit intended to illustrate neural integration.
			The synaptic connections in fig. \ref{figAuronE:subfigModel} are represented as a factor of the firing threshold.
			A single transmission through a synapse with $\omega_{ij}=1$ therefore cause the postsynaptic artificial neuron (``auron'') to fire.
			Thus, the neural circuit $[A1, A9]$ is self sustaining, and causes synaptic transmissions through the synapse from auron $[A*]$ to auron $E$.
			\mbox{(\ref{figAuronE:subfigDepolarization}) The} resulting depolarization curve for auron $E$.
			Every auron but $A7$ is connected to auron $E$, making the effect of leakage prominent.
			Note the small decrease in auron $E$'s depolarization, every ninth time step.
% 			Auron $A7$ is disconnected, causing a small decrease in auron $E$'s depolarization every ninth iteration.
%TODO TODO Dersom eg lager figuren på nytt: Skriv 
			(Figure \ref{figAuronE:subfigDepolarization} is generated by $auroSim_{N}$, the part of \emph{auroSim} with numerical integration) \cite{FDP_report}
		%	(Figure \ref{figAuronE:subfigDepolarization} is from the preliminary project\cite{FDP_report}, and is generated from $auroSim_{N}$, the part of \emph{auroSim} with numerical integration)
		%	(Figure \ref{figAuronE:subfigDepolarization} is from the preliminary project\cite{FDP_report}, and is generated from $auroSim_{N}$, the part of \emph{auroSim} that utilize numerical integration)
% 			(The figure is from \cite{FDP_report} and is generated by $auroSim_{N}$, the part of \emph{auroSim} that utilize numerical integration)
                % TODO TODO Make auron A7->A9. This is better for the text.. TODO TODO Also make figure figAuronE:subfigModel more pretty: redraw! TODO TODO TODO
				}
\end{figure}



		
% // vim:fdm=marker:fmr=//{,//}
