
% DISPOSISJON
% 	Intro: Begynn beldig vidt: "neuron" er en samlebetegnelse for en spesiell type celler med signalling properties (kan bli eksitert). [REF].
%  		IDE: -> Skriv litt om at den vide grupperingen gjør at simulering er vanskelig (men ikkje her: MEN i ANN-seciton!)
%  		- Dette ligger i komplekse neurale nettvert med "recurrent connections" og høg grad av 'connectivity'.(I CNS er det estimert til å være 1000+ input per neuron)
% 		- Koblingen mellom 
% 




% "Although the human brain contains an extraordinary number of these cells (in the order 10E11 neurons), which can be classified into at least a thousand different types, all nerve cells share the same architecture." Kandell kap 2.



\section{Biological Neural Systems}
	
	In the late 1800s, Camillo Golgi developed a way of staining nervous tissue so that complex networks became apperent in nervous tissue.
	%In the late 1800s, Camillo Golgi developed a way of staining nervous tissue so that complex networks of nerve cells (neurons) became apperent.
	Santiago Ramon Y Cajal used Golgi's technique in such a way that individual neurons could be separated, and it was observed that nervous tissue was not a continuous web but a network of discrete cells. 
	He proposed what has later been known as the neuron doctrine; That the basis of intelligence is individual ``brain cells'' that can process incoming transmissions and 
		can send output transmissions as a consequence of receiving enough depolarizing input transmissions.
	For their contribution, Ramon Y Cajal and Golgi shared the 1906 Nobel's price in Physiology and Medicine.
	\cite{NeuroscienceExploringTheBrain3edKAP2}
	%TODO Cite Bear kap. 2.

	Modern neuroscience follows the neuron doctrine. 
	Each node in a neural network is called a neuron and the connection between neurons are called synapses.
	When the presynaptic neuron ``fires'' an action potential, the following synaptic transmission cause the postsynaptic neuron to become excited or inhibited.
	%All synapses have direction(propagates information in one direction) and transmits if the presynaptic neuron ``fires'' an action potential.
	%Synapses can be exitatory or inhibitory.
	Transmissions in excitatory synapses increase the postsynaptic membrane potential, causing that neuron to approach firing. %sending an action potential.
	Inhibitory transmissions does the opposite, and inhibits the postsynaptic neuron with respect to firing.
	%When a neuron fires an action potential, a transmission is eventually initialized for all the neuron's output synapses.  % redundant? eller viktig å poengtere?
	When the neuron fires an action potential, a transmission in all the neuron's output synapses is the result.
	\cite{PrinciplesOfNeuralScience4edKAP02}
% TODO TODO TODO CITE!


% TODO TODO FINN BEDRE FIGUR! TODO TODO TODO 
\begin{figure}[hbt!p]
	\centering
	\includegraphics[width=0.65\textwidth]{ModellAvNeuronet}
	\caption{A illustrative model of the neuron. The signal propagation goes from the left to the right in this figure;
			Synaptic integration at the dendrites, action potential through the axon and finally transmission throught the output synapses. 
			The aspects of the cell body is not immediately relevant to signal processing, and is not taken into account in the model used. }
	\label{figFigurAvNeuronet}
\end{figure}
	
	In this section, the most important elements of neural signal processing are presented, enabling the reader to become more familiar with how neural networks process information.
	%It is recommended that the reader utilize this section as a reference work when different methods of neural simulation are presented.
	It is recommended that the reader utilize this section as a reference work when different methods of neural simulation are discussed, later in the text. % la til "later in this text", no.  Er det bra?
	% DERSOM NESTE SETNING skal være med: Skriv om "the neuron", slik at dette er sant!
	%Before discussing how the neuron processes information, the organization of the neuron have to be reviewed. %REVIEWED er dårlig TODO
	%We will start with the organization of the neuron, before moving on to a fundamental signal processing mechanism of the neuron; The electrochemical properties of the neuron membrane.




%	The most important elements of signal processing in neural systems is presented in this section,
%		enabling the reader to become familiar with how biological neural networks function.
%	Fundamental elements for the neural signal processing is presented, and it is recommended to use this section as a work of reference when 	we later discuss simulating the biological neuron.
																																				%going through the remainder of this text.
	



	

%Using Golgi's technique, Ramon Y Cajal, Santiago stained nervous tissue in such a way that individual neurons could be separated and 




		
% 	1) Membran som er stengt for ion
% 	2) Dendrite som som mottar depolarizing input 
% 	3) Axon
% 	4) Synapse
% Avslutt med å skrive litt om retninga til signalet: inn i dendritt (eller soma), integrert i intracellular fluid (ser på det nært axon hillock), overstiger terskel i axon hillock, AP i axon, syn.trans.
	
	\subsection{The Neuron}
		% TODO TODO TODO Dersom det passer: Skriv om lay-out til neuronet. (Dette er lovet fra setninga, over). Evt. fjærn denne setninga!)
		%
		In terms from graph theory, a biological neural network is a directed, cyclic graph.
		The nodes are called neurons and the edges between the nodes are called synapses.
		% Skriv om: neste setning! Vil heller formidle at i dette kapittelet skal eg gå gjennom geografien til neuronet.
		In addition to the synapse, the neuron contain three elements that are fundamental for signal processing.
		% skrive at det er eit PHOSPHOlipid bilayer?
		The most important element is the neuronal membrane. % that is a lipid bilayer with a low permeability to ions, enabling an electrical potential across the cell membrane.
		%The neuron membrane is a lipid bilayer that has a low permeability to ions, enabling an electrical potential across the cell membrane.
% 		%
%		The axon hillock initiates an action potential when the membrane potential becomes more positive than the firing threshold, causing an action potential to propagate through the axon.
%%
% 	Skal eg liste opp alle her, eller skal eg bare nevne det, og gå vidare til membran?		

		



	%xxx 1) Membran som er stengt for ion
		Each neuron is surrounded by a phospholipid bilayer cell membrane with a low permeability to ions, enabling a different concentration of ions over the membrane.
		%5													%%												%enabling the intracellular fluid to have a different consectration of different ions than the fluid outside the membrane.
		%under: ikkje "spesialiserte", men "dedidkerte"
		All neuron membranes has ionic pumps dedicated to create an ionic concentration gradient over the membrane.
		The different ionic pumps push the corresponding ions ``upstream'' in relation to the ionic concentration gradient, resulting in an electrochemical potential over the membrane.
		%The membrane potential at rest generally lies at about $-65mV$.
		The resting membrane potential of a neuron generally lies at about $-65mV$. 
		\cite{NeuroscienceExploringTheBrain3edKAP3} %XXX Dersom dei to avsnitta over blir kobla sammen: flytt denne under heile avsnittet. TODO Do this?
%TODO CITE: Bear kap 3 ?

		If specialized ionic gates permeable to certain ions are opened, these ions can flow freely through the gate.
		Depending on which ions are let thought, the neuron membrane is either hyperpolarized(more negative membrane potential) or depolarized(more positive membrane potential).
		%Depending on which ions are let thought, the neuron membrane to either be hyperpolarized(more negative membrane potential) or depolarized(more positive membrane potential).
		%This cause the neuron membrane to either be hyperpolarized(more negative membrane potential) or depolarized(more positive membrane potential), depending on which ions are let thought.
		If the membrane potential becomes more positive than the firing threshold of the neuron, an action potential is initiated at the axon hillock at the base of the neuron's axon. 
		\cite{PrinciplesOfNeuralScience4edKAP07}
		%\cite{NeuroscienceExploringTheBrain3edKAP3}\cite{PrinciplesOfNeuralScience4edKAP07} %XXX Dersom dei to avsnitta over blir kobla sammen. TODO Do this?
		% TODO Med unntak av siste linje kan alt finnes i Kandel:07. Siterer difor bare K:07. Er dette rett?
%TODO CITE: Bear kap 3 og 4. F.eks.     (Gjelder om det var kobla med det over.. No er det kanskje nok å cite en av kap. ?)
%TODO 		Eller: Kandell kap 7,9 (?)



%TODO TODO TODO Finn figur som handler om depol. og memb. pot. TODO TODO TODO




% Lag ny plan for det som no kommer. Skal jo skrive om denrite og axon i egene avsnitt. No trenger eg bare å skrive om korleis lay-out er på neuronet!
% 	- komme inn på at signalet går i en retning: fra dendrite til axon-terminal.
% 	- nevne såvidt kva som skjer ved axon terminal (synaptic vesicles og NT). NT spres gjennom noko kalla syn.cleft, og exiterer eller inhibits postsyn neuron.


	\subsection{The Axon and the Action Potential}

		Voltage--gated sodium and potassium channels are located along the membrane of the axon.
		If the membrane potential is more positive than the ``firing threshold'' of the neuron, these channels open, causing the membrane to have a transient positive increase in membrane potential.
		%In the membrane of the axon there are voltage--gated sodium and potassium channels that open if the level of depolarization is more than some value, referred to as the ``firing threshold'' of the neuron.
%This value is referred to as the ``firing threshold'' of the neuron.
		%When these channels are activated, the membrane potential will transiently have a large positive increase in membrane potential.
		Through passive transmission of the electrical charge due to diffusion of ions, the membrane potential at the next site of voltage gated channels becomes more positive than the firing threshold. %, and the process is repeated.
		The process is repeated, and the active propagation of the action potential continues until the end of the axon(the axon terminal).
		\cite{PrinciplesOfNeuralScience4edKAP09}
%%%Through passive transmission of the electrical potential, the next voltage gated channels will open as a result of going above the gate threshold. 
		%This establishes the active aspect of action potential propagation, and results in a self carrying propagating through the axon.
		%T0DO CITE: \cite{PrinciplesOfNeuralScience4edKAP09}.


		The two most important voltage gated channels for the active part of action potential propagation are the sodium and the potassium channels.
		%The $Na^{2+}$ channel is more responsive than the $K^+$ channel, and enables the $Na^{2+}$ ion that has the highest concentration outside the cell to flow into the cell.
		%This cause the membrane potential to become more positive.
		The $Na^{2+}$ channel is most responsive and enables the $Na^{2+}$ ion to flow freely. %, that has the highest concentration outside the cell, to flow into the cell.
		%The highest concentration of $Na^{2+}$ ions is on the outside of the neuron, resulting in a flow of positively charged ions into the neuron.
		The highest concentration of $Na^{2+}$ ions is on the outside of the neuron, resulting in an inflow of positively charged ions that depolarize the neuron.
		%%
		The potassium ion has the highest concentration inside the cell, and activation of the somewhat slower $K^+$ channel cause a flow of positively charged ions out of the cell.
		%Because both channels close after a short while, and the $K^+$ channel is slower than the $Na^{2+}$ channel, the action potential cause a membrane potential as shown in fig. \ref{figActionPotential}. 
		%Because both channels close after a short while, and the $K^+$ channel is slower than the $Na^{2+}$ channel, the action potential cause a transient membrane potential as shown in fig. \ref{figActionPotential}. 
		Because the $K^+$ channel is slightly less responsive than the $Na^{2+}$ channel, and that both channels only stay open for a short while, 
			the transient membrane potential of the action potential have a form as shown in fig. \ref{figActionPotential}. 
		\cite{PrinciplesOfNeuralScience4edKAP09}
		%TODO CITE: \cite{PrinciplesOfNeuralScience4edKAP09}.
	

\begin{figure}[hbt!p]
    \centering
    %\includegraphics[width=0.65\textwidth]{AP_IonFlow}
    \includegraphics[width=0.65\textwidth]{AP_IonFlow2}
 	  \caption{The action potential. Activation of the $Na^{2+}$ channel cause positively charged ions to flow into the neuron, depolarizing the neuron. The slower $K^+$ channel has the opposite effect.
				Both channels close after a short while, and the membrane potential returns to the resting value after a small overshoot.
				(Figure from \cite{PrinciplesOfNeuralScience4edKAP09}).
				%Figure taken from  \cite{PrinciplesOfNeuralScience4edKAP09}, page 158.
			%SKRIV MEIR! %TODO TODO TODO
			}
    \label{figActionPotential}
\end{figure}

% TODO TODO TODO TODO Før eller etter neste avsnitt: Skriv om at signalet er konstant gjennom heile aksonet, causing the syn.trans. til å være av samme størrelse. 
% TODO Trur eg skal skrive om dette heilt sist i dette avsnittet, for å lede inn på syn.trans.


		After a successful opening of the voltage--gated ion channels, internal mechanisms of the channels cause the ion channels to close again.
		The channels stay closed for a short while, enabling the active sodium--potassium pump to reestablish the ionic concentration gradient over the membrane.
		During this time, the voltage--gated channels remain closed, and it is impossible to elicit a new action potential.
		This mechanism cause what is referred to as the absolute refraction period for the neuron, and is important both to hinder the action potential to ``travel back'' along the axon and to limit the firing frequency of the neuron.
		\cite{NeuroscienceExploringTheBrain3edKAP4} %TODO TODO TODO Sjekk: les gjennom kapittelet, og se om alt stemmer! TODO TODO TODO
		%TODO CITE f.eks. Bear, kap 4

		%When the action potential reach the axon terminal at the distal end of the axon, 
		The axon is organized as a tree, with a trunk by the soma called the axon hillock.
		The branches of the ``axonic tree'' are called axon collaterals. % and split the axon in two. %%% TA MED SISTE?
		At the far end of each axon collateral, lies the axon terminal where the neuron's output synapses are located.
		\cite{NeuroscienceExploringTheBrain3edKAP2}
		% TODO CITE!

%		%TODO TODO TODO TODO TODO TODO TODO TODO TODO TODO TODO TODO TODO TODO TODO  Skriv ferdig siste avsnittet som skal lede leser over på synapsen! TODO TODO TODO TODO TODO TODO TODO TODO TODO TODO TODO TODO TODO TODO TODO 
%		An important result of the active signal propagation of the action potential is that the membrane potential of the action potential is the same OVER HEILE AKSONET:
% %TODO Heilt på slutten: For å lede leser over på synapsen:
% An important result of this is that the presynaptic membrane at the synapses receives an equal depolarization, independent of its location. 
% The release of neurotransmitters is a result of this depolarization, and the action potential enshures an equal depolarization of the presynaptic membrane.
% This gives that the transmission is dependent only on the synaptic connection, or ``the synaptic weight''.
% \cite{PrinciplesOfNeuralScience4edKAP09}














% //{ GAMMELT:
% ------------------
% GAMMELT:
% ------------------
% 	
% %	The electrochemical charge, represented by charged ions or molecules, spreads passively through the neuron.
% %	When the membrane potential at the axon hillock is more positive than the firing threshold, an action potential is initialized.
% %%%%%%%%
% 	The action potential is an active mechanism in the axon membrane that temporarily opens voltage--gated $K^+$ and $Na^{2+}$ channels as a result to the membrane potential to be more positive than the firing threshold. 
% % TODO Ikkje skriv for mykje om axonet: Spar litt til ssecTheAxon!
% 	This mechanism cause the membrane potential to become more positive, and an unstable mechanism that temporarily depolarize the axon membrane is the result.
% 	The ions diffuse in the intracellular fluid in the axon, causing $K^+$ and $Na^{2+}$ channels are opened further along the axon.
% 	This is repeated until the signal reach the axon terminal at the distal end of the axon.
% 
% 	Because the action potential propagation is active, the intracellular signal does not diminish along the axon.
% 	This cause the membrane potential to be of a relatively constant size, irrespective of distance traveled.
% 	This enables the axon to be LASKDJFLAKSJDFLAKSJDFLKJ, enabling the axon to 
% 
% %	SIDEN AP ER AKTIVT, minker ikkje signalet!
% 
% 
% % skrive meir? :	The $K^+$ channel ER RASKEST.
% 	
% 	The neuron's output synapses lies at different the axon terminals in the axon.
% 
% %-----------
% %	This mechanism cause the membrane potential to become more positive, and we have an unstable mechanism that cause the membrane potential to become temporally depolarized.
% %	%This cause the signal to actively be propagated along the axon, and makes the distance of a synapse along the axon irrelevant to the transmitted signal.
% %	When this signal reach the axon terminal, the synapses at the axon terminal is activated and synaptic transmission is the result.
% %	This signal does not decrease with the distance along the axon, causing distal synapses to receive the same signal as early synapses along the axon.
% 	
% 
% 	
% 	
% 
% 
% \newpage
% 
% 
% GAMMELT Fra into-delen:
% % TODO Drit i dendritt!
% 	%xxx 2) Dendrite som mottar depolarizing input
%  		%%% 						%% 		endre   receives   til eit anna ord!
% 		The dendrite is the part of the neuron that receives most of the excitatory input connections to the neuron, while most inhibitory input synapses arrive close to the cell body(soma).
% 		When the neuron receives an excitatory input transmission, the neuron is said to be ``\emph{excited}''.
% % TODO Skriv om ionekanaler? (Receptorer) ????
% 		This cause the membrane potential to become more positive, leading the neuron towards firing an action potential.
% 		Most inhibitory input synapses arrive close to the soma, and cause the membrane potential to become more negative, or hyperpolarized.
% 		This inhibits neuronal firing. %TODO Fortsett her, go CITE
% 		%Most excitatory input connections arrive at the dendrite. %CITE
% 
% 	%xxx 3) Axon
% 		When the neuron is depolarized sufficiently, an action potential is initiated at the axon hillock.
% 		The action potential is an unstable mechanism, causing a bolean (all--or--nothing) signal to spread through the axon.
% 		When this signal reaches the ``axon terminal'', where the presynaptic part of the synapse lies, different mechanisms cause neurotransmitters to be released into the synaptic cleft
%  			(see appendix \ref{appendixSecPresynapticSynapticPartOfTransmission} for a more complete discussion of the action potential and the presynaptic elements of synaptic transmission).
% 		The neurotransmitters diffuse passively through the synaptic cleft, %TODO Skriv kva syn.cleft er! TODO
% 		The ones that come in contact with postsynaptic ligand--gated receptors, activate these receptors and opens an ionic gate in the postsynaptic neuron.
% 		The resulting ion flow cause the postsynaptic neuron to become depolarized or hyperpolarized, changing the postsynaptic membrane potential.
% 		
% % 	4) Synapse
% 
% 
% 	
% 
% 
% 
% %		The neuron is a cell with a special property; 	The electrochemical properties of the neurons enables advanced signal processing
% 
%  //}



	\subsection{The Synapse}

		The size of the synaptic transmission is propotional to the membrane potential at the axon terminal.
		%TODO TODO TODO No er det for stort hopp! Gjør om, slik at det blir meir flytende overgang fra siste setninga, over, til neste setninga!
		One important result of the active signal propagation of the action potential is that the increase in neuron potential is independent of how long the signal have to travel.
		When the action potential reach the axon terminal, voltage--gated $Ca^{2+}$ channels in the active zone of the terminal opens and $Ca^{2+}$ enters the cytosol of the axon terminal. %\cite{PrinciplesOfNeuralScience4edKAP10}.
		The axon terminal contacts bag--like organelles called synaptic vesicles filled with different neurotransmitters.
		%The axon terminal contains synaptic vesicles, bag--like organelles that contain different neurotransmitters.
		When $Ca^{2+}$ enters the cytosol of the axon terminal, synaptic vesicles are pulled toward the membrane where they fuse and release its content into the synaptic cleft.
		%The neurotransmitters diffuse out in the fluid of the synaptic cleft, and some come in contact with postsynaptic ligand--gated channels that enable different ions to flow freely into/out of the postsynaptic neuron.
		The neurotransmitters diffuse out in the fluid of the synaptic cleft, and some come in contact with postsynaptic receptors. %ligand--gated channels that enable different ions to flow freely into/out of the postsynaptic neuron.
		When the right neurotransmitter bind to a specific group of receptors called ligand--gated channels, an ionic channel is opened in the postsynaptic membrane.
		Depending on the channel(and thus the ions that are let through), this can either depolarize(excite) or hyperpolarize(inhibit) the postsynaptic neuron.
						%% 											%%  , the postsynaptic neuron is either depolarized(excited) or hyperpolarized(inhibited).
		\cite{PrinciplesOfNeuralScience4edKAP10}
		%TODO Sjekk om det er rett å cite KANDELL:10
		
		%The synaptic vesicles are bag--like organelles that contain different neurotransmitters


\begin{figure}[hbt!p]
    \centering
    \includegraphics[width=0.95\textwidth]{synapticTransmissionFraKandell}	%Figure taken from  \cite{PrinciplesOfNeuralScience4edKAP10}, page 183.
 	  \caption{\textbf{Synaptic transmission in an excitatory synapse.} 
			An action potential arriving at the terminal of the presynaptic axon cause $Ca^{2+}$ to enter the presynaptic cytosol, causing synaptic vesicles to fuse with the membrane.
			The containing neurotransmitters are released into the synaptic cleft.
			The released neurotransmitters diffuse passively across the synaptic cleft and bind to specific receptors in the postsynaptic membrane, causing the connected ion channels to open.
			%This example depolarize the postsynaptic neuron. %% Redundant->sjå textbf(fig.-overskrifta)
			(Figure from \cite{PrinciplesOfNeuralScience4edKAP10}).
			}
    \label{figActionPotential}
\end{figure}

		%The size of one transmission is often referred to as the postsynaptic potential(PSP) following the transmission.
		In neural simulation, the size of a synaptic transmission at a particular synapse is often referred to as the synaptic weight of this synapse.
		The synaptic weight is plastic, and different mechanisms like long--term potentiation(LTP) and long--term depression(LTD) increase and decrease the synaptic weight.
		In neuroscience, synaptic plasticity is what is seen as the basis of learning and memory\cite{NeuroscienceExploringTheBrain3edKAP25}. %(DENNE CITE gjelder bare denne linja..) (Ta vekk?)
		%This is what is seen as the basis of learning\cite{NeuroscienceExploringTheBrain3edKAP25}. %(DENNE CITE gjelder bare denne linja..) (Ta vekk?)
		%The exact firing time have recently been found to have a large impact on synaptic plasticity.
		The subject of synaptic transmission and plasticity in biological systems have been covered extensively in appendix \ref{appendixSynPlast}.
	%TODO TODO TODO Skriv ferdig. Skal eg ha med STDP, her?
		%\cite{} MULIG: Neural Networks and Brain Function: KAP 1 (Introduction) s.7

	\subsection{Signal Propagation in the Biological Neuron}
		- oppsymmer AP, spatiotemporal effekt av axon, synaptic transmission, EPSP/IPSP, ...

% // vim:fdm=marker:fmr=//{,//}
