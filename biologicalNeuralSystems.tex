
% "Although the human brain contains an extraordinary number of these cells (in the order 10E11 neurons), which can be classified into at least a thousand different types, all nerve cells share the same architecture." Kandell kap 2.


% TODO TODO TODO TODO TODO TODO TODO TODO TODO TODO  Gå gjennom, og cite'er alle plassene eg innfører noko nytt. Type all ny info om neuronet. TODO TODO TODO TODO TODO TODO TODO TODO TODO TODO TODO TODO TODO

\section{Biological Neural Systems}
\label{secBiologicalNeuralSystems}	
	In the late 1800s, Camillo Golgi developed a way of staining nervous tissue so that complex networks became apparent in nervous tissue.
	Santiago Ramòn y Cajal used Golgi's technique in such a way that individual neurons could be separated, and it was observed that nervous tissue was not a continuous web but a network of discrete cells. 
	Ramòn y Cajal proposed what has later been known as the neuron doctrine; 
		that the computational capabilities of the brain comes from a network of individual ``brain cells'' that process incoming transmissions and sends output when its input history has a certain pattern.
	For their contribution, Ramòn y Cajal and Golgi shared the 1906 Nobel's price in Physiology and Medicine
	\cite{NeuroscienceExploringTheBrain3ed, PrinciplesOfNeuralScience4edKAP01}. %TODO Finn kapittel i Kandell, og ???, for å cite fleire kilder.

	Modern neuroscience follows the neuron doctrine. 
	Each node in a neural network is called a neuron, and the connection between neurons are called synapses.
	When the presynaptic neuron ``fires'' an action potential, the following synaptic transmission cause the postsynaptic neuron to become excited or inhibited.
	%All synapses have direction(propagates information in one direction) and transmits if the presynaptic neuron ``fires'' an action potential.
	%Synapses can be excitatory or inhibitory.
	Transmissions in excitatory synapses increase the postsynaptic membrane potential, causing that neuron to approach firing. %sending an action potential.
	Inhibitory transmissions does the opposite, and inhibits the postsynaptic neuron with respect to firing.
	% TODO Skriv at neuronet blir også resatt: Depol blir til null, og det blir klart til å lades opp på nytt!
	%When the neuron fires an action potential, a transmission in all the neuron's output synapses is the result. % redundant? eller viktig å poengtere?
	Firing of an action potential causes transmission in all the neuron's output synapses, and a resetting of the depolarization to the reset potential $v_r$ %TODO Sjekk. Skriv om. Avsluttinga er litt løs.
	\cite{PurvesNeuroscience, PrinciplesOfNeuralScience4edKAP01, PrinciplesOfNeuralScience4edKAP02}.


% TODO TODO FINN BEDRE FIGUR! TODO TODO TODO 
% TODO TODO TODO TODO TODO TODO TODO TODO TODO TODO TODO TODO TODO TODO TODO TODO TODO TODO TODO TODO TODO TODO TODO TODO TODO TODO TODO TODO TODO TODO TODO TODO TODO  Eller lag figuren sjølv!
% (Hugs at dersom eg siterer en plass, trur sensor at alt kommer derfra..
\begin{figure}[hbt!p]
	\centering
	\includegraphics[width=0.75\textwidth]{neuron_structure}%{ModellAvNeuronet}
	\caption[Illustrative model of the neuron]
			{Illustrative model of the neuron. The signal propagation goes from left to right in this figure;
			Synaptic integration at the dendrites, action potential through the axon and finally transmission through the neuron's output synapses. 
			(Figure from {\tiny http://biomedicalengineering.yolasite.com/neurons.php})
			%The aspects of the nucleus is not immediately relevant for signal processing, and is not taken into account in the simulation.
			}
% TODO TODO TODO TODO TODO TODO TODO TODO TODO TODO TODO TODO TODO TODO TODO TODO TODO TODO TODO TODO TODO TODO CITE!!! TODO TODO TODO TODO TODO TODO TODO TODO TODO TODO TODO TODO 
			% XXX Fra kor da? TODO Lag figur sjølv?
	\label{figFigurAvNeuronet}
\end{figure}
	
	In this section, the most important elements of neural signal processing are presented, enabling the reader to become more familiar with how neural networks process information.





		
% 	1) Membrane that is closed for ions
% 	2) Dendrite that receives depolarizing input 
% 	3) Axon
% 	4) Synapse
	

	\subsection{The Neuron}
		In terms from graph theory, a biological neural network is a directed, cyclic graph.
		The nodes are called neurons and the edges between nodes are called synapses.
		In addition to the synapse, the neuron contains some elements that are fundamental for signal processing.
		The most important element is the neuronal membrane. 



	%xxx 1) Membrane, (bilayer that is closed for ions)
		Each neuron is surrounded by a phospholipid bilayer cell membrane with a low permeability to ions, enabling a different concentration of ions over the membrane.
		All neuron membranes have ionic pumps dedicated to uphold an ionic concentration gradient over the membrane.
		Different ionic pumps push the corresponding ions ``upstream'' in relation to the ionic concentration gradient, resulting in an electrochemical potential over the membrane.
		The resting membrane potential of a neuron generally lies at about $-65mV$
		\cite{NeuroscienceExploringTheBrain3ed, PrinciplesOfNeuralScience4edKAP07}. %XXX Dersom dei to avsnitta over blir kobla sammen: flytt denne under heile avsnittet. TODO Do this?
%TODO CITE: Bear kap 3 ?

		When specialized ionic gates permeable to certain ions are opened, these ions can flow freely through the gate.
		Depending on which ions are let thought, the neuron membrane is either hyperpolarized(more negative membrane potential) or depolarized(more positive membrane potential).
		%Depending on which ions are let thought, the neuron membrane to either be hyperpolarized(more negative membrane potential) or depolarized(more positive membrane potential).
		%This cause the neuron membrane to either be hyperpolarized(more negative membrane potential) or depolarized(more positive membrane potential), depending on which ions are let thought.
		When the membrane potential becomes more positive than the firing threshold of the neuron, an action potential is initiated at the axon hillock, the base of the neuron's axon. %, the axon hillock.
%		If the membrane potential becomes more positive than the firing threshold of the neuron, an action potential is initiated at the axon hillock at the base of the neuron's axon
		\cite{PrinciplesOfNeuralScience4edKAP09, PurvesNeuroscience, PrinciplesOfNeuralScience4edKAP10}. 
		% ELLER:  				
% 		\cite{PrinciplesOfNeuralScience4edKAP07, PrinciplesOfNeuralScience4edKAP09, PrinciplesOfNeuralScience4edKAP10, PrinciplesOfNeuralScience4edKAP12, TrevesNeuralNetworks}. %, PrinciplesOfNeuralScience4edKAP07}.



%TODO TODO TODO Finn figur som handler om depol. og memb. pot. TODO TODO TODO





	\subsection{The Axon and the Action Potential}
	\label{ssecTheAxonAndActionPotential}
		Voltage--gated sodium and potassium channels are located along the membrane of the axon.
		If the membrane potential is more positive than the ``firing threshold'' of the neuron, these channels open, causing the membrane to have a transient positive increase in membrane potential.
		Through passive transmission of the electrical charge, due to diffusion of ions, the membrane potential at the next site of voltage gated channels becomes more positive than the firing threshold and the process is repeated.
		The size of the signal arriving at the synapse, at the distal end on the axon, is therefore independent of the total distance traveled
		\cite{PrinciplesOfNeuralScience4edKAP09}.


		The two most important voltage gated channels for the action potential is the sodium and the potassium channels.
		The $Na^{2+}$ channel is most responsive and opens and closes faster than the $K^+$ channel. 
		The highest concentration of $Na^{2+}$ ions is on the outside of the neuron, resulting in an inflow of positively charged ions that depolarize the neuron.
		The potassium ion has the highest concentration inside the cell, and activation of the $K^+$ channel cause a flow of positively charged ions out of the cell, repolarizing the neuron.
		Because the $K^+$ channel is slightly less responsive than the $Na^{2+}$ channel, and that both channels only stay open for a short while, 
			the transient membrane potential of the action potential has the form shown in fig.  \ref{figActionPotential}  % a form as shown in fig. \ref{figActionPotential}
		\cite{PrinciplesOfNeuralScience4edKAP09}.
	

\begin{figure}[hbt!p]
    \centering
    %\includegraphics[width=0.65\textwidth]{AP_IonFlow}
    \includegraphics[width=0.65\textwidth]{AP_IonFlow2}
 	  \caption[The transient axon membrane potential from an action potential]{
				The action potential. Activation of the $Na^{2+}$ channel cause positively charged ions to flow into the neuron, depolarizing the neuron. The slower $K^+$ channel has the opposite effect.
				Both channels close after a short while, and the membrane potential returns to the resting value after a small undershoot \cite{PrinciplesOfNeuralScience4edKAP09}.
				%\cite{PrinciplesOfNeuralScience4edKAP09}
			%SKRIV MEIR! %TODO TODO TODO
				}
    \label{figActionPotential}
\end{figure}



		After a successful opening of the voltage--gated channels in the axon membrane, internal mechanisms close the ion channels after a little while.
		The channels stay closed long enough to enable the active sodium--potassium pump to reestablish some of the ionic concentration gradient over the membrane.
		During this time, it is impossible to elicit a new action potential.
		This time interval is called the absolute refraction time for the neuron,
			and is important to prevent the action potential form ``travelling back'' along the axon\cite{NeuroscienceExploringTheBrain3ed}. %\cite{NeuroscienceExploringTheBrain3edKAP4}
			as well as limiting the maximal firing frequency of the neuron(see appendix \ref{appendixRefractionTimeAndSimulationTimeScale}). %TODO TODO TODO Sjekk: les gjennom kapittelet, og se om alt stemmer! 
%			 and is important both to prevent the action potential to ``travel back'' along the axon\cite{NeuroscienceExploringTheBrain3edKAP4} and to limit the maximal firing frequency of the neuron(see appendix \ref{appendixRefractionTimeAndSimulationTimeScale}). %TODO TODO TODO Sjekk: les gjennom kapittelet, og se om alt stemmer! TODO TODO TODO


	
		An important part of the active propagation of the action potential is that the signal is independent of the distance travelled.
		Because of this, the synapses located at various locations along the axon receives the same transmission--initiating signal. % than earlier synapses along the axon.
		The importance of this becomes clear when the mechanisms of synaptic transmissions are introduced. %discussed.
		%\cite{} % TODO CITE? 



% //{ GAMMELT:
% ------------------
% GAMMELT:
% ------------------
% 	
% %	The electrochemical charge, represented by charged ions or molecules, spreads passively through the neuron.
% %	When the membrane potential at the axon hillock is more positive than the firing threshold, an action potential is initialized.
% %%%%%%%%
% 	The action potential is an active mechanism in the axon membrane that temporarily opens voltage--gated $K^+$ and $Na^{2+}$ channels as a result to the membrane potential to be more positive than the firing threshold. 
% % TODO Ikkje skriv for mykje om axonet: Spar litt til ssecTheAxon!
% 	This mechanism cause the membrane potential to become more positive, and an unstable mechanism that temporarily depolarize the axon membrane is the result.
% 	The ions diffuse in the intracellular fluid in the axon, causing $K^+$ and $Na^{2+}$ channels are opened further along the axon.
% 	This is repeated until the signal reach the axon terminal at the distal end of the axon.
% 
% 	Because the action potential propagation is active, the intracellular signal does not diminish along the axon.
% 	This cause the membrane potential to be of a relatively constant size, irrespective of distance traveled.
% 	This enables the axon to be LASKDJFLAKSJDFLAKSJDFLKJ, enabling the axon to 
% 
% %	SIDEN AP ER AKTIVT, minker ikkje signalet!
% 
% 
% % skrive meir? :	The $K^+$ channel ER RASKEST.
% 	
% 	The neuron's output synapses lies at different the axon terminals in the axon.
% 
% %-----------
% %	This mechanism cause the membrane potential to become more positive, and we have an unstable mechanism that cause the membrane potential to become temporally depolarized.
% %	%This cause the signal to actively be propagated along the axon, and makes the distance of a synapse along the axon irrelevant to the transmitted signal.
% %	When this signal reach the axon terminal, the synapses at the axon terminal is activated and synaptic transmission is the result.
% %	This signal does not decrease with the distance along the axon, causing distal synapses to receive the same signal as early synapses along the axon.
% 	
% 
% 	
% 	
% 
% 
% \newpage
% 
% 
% GAMMELT Fra into-delen:
% % TODO Drit i dendritt!
% 	%xxx 2) Dendrite som mottar depolarizing input
%  		%%% 						%% 		endre   receives   til eit anna ord!
% 		The dendrite is the part of the neuron that receives most of the excitatory input connections to the neuron, while most inhibitory input synapses arrive close to the cell body(soma).
% 		When the neuron receives an excitatory input transmission, the neuron is said to be ``\emph{excited}''.
% % TODO Skriv om ionekanaler? (Receptorer) ????
% 		This cause the membrane potential to become more positive, leading the neuron towards firing an action potential.
% 		Most inhibitory input synapses arrive close to the soma, and cause the membrane potential to become more negative, or hyperpolarized.
% 		This inhibits neuronal firing. %TODO Fortsett her, go CITE
% 		%Most excitatory input connections arrive at the dendrite. %CITE
% 
% 	%xxx 3) Axon
% 		When the neuron is depolarized sufficiently, an action potential is initiated at the axon hillock.
% 		The action potential is an unstable mechanism, causing a bolean (all--or--nothing) signal to spread through the axon.
% 		When this signal reaches the ``axon terminal'', where the presynaptic part of the synapse lies, different mechanisms cause neurotransmitters to be released into the synaptic cleft
%  			(see appendix \ref{appendixSecPresynapticSynapticPartOfTransmission} for a more complete discussion of the action potential and the presynaptic elements of synaptic transmission).
% 		The neurotransmitters diffuse passively through the synaptic cleft, %TODO Skriv kva syn.cleft er! TODO
% 		The ones that come in contact with postsynaptic ligand--gated receptors, activate these receptors and opens an ionic gate in the postsynaptic neuron.
% 		The resulting ion flow cause the postsynaptic neuron to become depolarized or hyperpolarized, changing the postsynaptic membrane potential.
% 		
% % 	4) Synapse
% 
% 
% 	
% 
% 
% 
% %		The neuron is a cell with a special property; 	The electrochemical properties of the neurons enables advanced signal processing
% 
%  //}



	\subsection{The Synapse}
	\label{ssecTheBiologicalSynapse}

		When the action potential reaches an axon terminal, voltage--gated $Ca^{2+}$ channels in the active zone of the terminal opens and $Ca^{2+}$ enters the cytosol of the axon terminal.
		The axon terminal contains bag--like organelles called synaptic vesicles filled with different neurotransmitters.
		Free intracellular $Ca^{2+}$ cause these organelles to be pulled toward the neuron membrane.
%%
		The synaptic vesicles fuse into the neuron membrane when close enough, causing its content to be released into the synaptic cleft on the outside of the membrane.
		The neurotransmitters diffuse out in the fluid of the synaptic cleft, and some come in contact with postsynaptic receptors. 
		When the right neurotransmitter bind to a specific group of receptors, called ligand--gated channels, an ionic channel is opened in the postsynaptic membrane.
		Depending on the channel(and thus the ions that are let through), this can either depolarize(excite) or hyperpolarize(inhibit) the postsynaptic neuron%.
		\cite{PrinciplesOfNeuralScience4edKAP10}.

\begin{figure}[hbt!p]
    \centering
    \includegraphics[width=0.95\textwidth]{synapticTransmissionFraKandell}	%Figure taken from  \cite{PrinciplesOfNeuralScience4edKAP10}, page 183.
 	  \caption[Synaptic transmission in an excitatory synapse]
			{Synaptic transmission in an excitatory synapse; 
			An action potential arriving at the terminal of the presynaptic axon enables $Ca^{2+}$ to enter the presynaptic cytosol, causing synaptic vesicles to fuse with the membrane.
			The containing neurotransmitters are released into the synaptic cleft.
		 	These diffuse passively across the synaptic cleft and bind to transmittor--specific receptors in the postsynaptic membrane
% 			These diffuse passively across the synaptic cleft and bind to specific receptors in the postsynaptic membrane, causing the connected ion channels to open.
% 			The released neurotransmitters diffuse passively across the synaptic cleft and bind to specific receptors in the postsynaptic membrane, causing the connected ion channels to open.
			%This example depolarize the postsynaptic neuron. %% Redundant->sjå textbf(fig.-overskrifta)
			\cite{PrinciplesOfNeuralScience4edKAP10}.
			}
    \label{figSynapticTransmissionInExitatorySynapse}
\end{figure}


		The \emph{N-methyl-D-aspartic acid}($NMDA$) receptors are of a particular importance for learning in biological neural systems\cite{OmNMDAsekvenserForSynPlast}. 
		As opposed to \emph{non-NMDA receptors}, these channels enable $Ca^{2+}$ ions to flow into the neuron.
 		This is thought to take part in regulating the synthesis of new $AMPA$ receptors, and is considered important for synaptic plasticity as well as transmission\cite{AMPARtrafficingArtikkel}.
		Because $NMDA$ channels are blocked by a $Mg^{2+}$ ion covering the opening, the membrane potential has to be sufficiently depolarized to remove this block and let ions through\cite{PrinciplesOfNeuralScience4edKAP63}.
% %%%%%%%%%%%%%
  		Due to the number of $NMDA$ receptors and variations around the $Mg^{2+}$ blocks, this creates a graded magnitude of the $Ca^{2+}$ inflow and thus a graded synaptic plasticity. % TODO?
%		Due to the number of $NMDA$ receptors, and variations in local membrane potentials, this creates a  graded magnitude of the $Ca^{2+}$ inflow and thus a graded magnitude of synaptic plasticity. % TODO?
%  		Due to the number of $NMDA$ receptors and variations in $Mg^{2+}$ blocks, this creates a graded magnitude of the $Ca^{2+}$ inflow and thus a graded magnitude of synaptic plasticity. % TODO? \cite{CITE}.
		Synaptic plasticity can be modelled as a function of the postsynaptic membrane potential at the time of transmission\cite{PrinciplesOfNeuralScience4edKAP63, reviewSTDP, NEVR3003STDP}.
		Since this is closely linked with the relative time of firing for the pre-- and postsynaptic neurons, this mechanism is referred to as Spike Timing Dependent Plasticity(STDP).
 




% // vim:fdm=marker:fmr=//{,//}
