
\section{Design of Experiments to Assess Efficiency}

% 	%XXX Fra nedst på artificialNeuralSystms.tex
% 		If all nodes are updated each time step, the computational load scale linearly with the number of nodes and the inverse of the size of the computational time step.
% 		By halving the size of the computational time step, the computational load increase as if the number of nodes are doubled.
% 		By having precise simulation algorithms, fewer time iterations can be utilized to accomplish the same accuracy for the simulation.
% 		This explains that the accuracy of simulation algorithms can be used as a good measure of efficiency, and establish the motivation for having precise simulation algorithms.
% 		More sophisticated numerical integration techniques are therefore often used to accomplish a high accuracy in numerical simulations\cite{PlesserStraubeMorrisonPlesser2007}.





	To compare the accuracy of the two simulation models, low--resolution simulations of $\kappa M$ and $NIM$ can be compared to a simulation with much higher temporal resolution. %smaller time steps. % higher temporal resolution.
	In this work, the low--resolution simulations have less than $1000$ time steps per forcing function period, while the high--resolution $NIM$ simulation has $1.000.000$ time steps per period.
%	In this work, the low--resolution simulations have less than $1000$ time steps per forcing function period, while the high--resolution $NIM$ simulation has a resolution of $1.000.000$ time steps per period.
 	The high--resolution simulation results will be referred to as the simulated solution in the remainder of this text. %in this chapter. 

	The simulated solution has a number of time steps that is more than three orders of magnitude larger than for the low--resolution simulations.
	The simulated solution can therefore be considered to be the correct time course, for a temporal resolution up to that of the low--resolution simulations.
%	The simulated solution can therefore be considered to be the correct solution for the accuracy of the low--resolution simulations.
% 	The simulated solution can therefore be considered to be the correct answer up to a higher accuracy than what is given by the low--resolution simulations.
%%%
	To assess the accuracy of the two simulation models, one can therefore define the simulated solution to be the correct answer and find the errors for each of the two low--resolution simulations.
% 	To assess the accuracy of the two simulation models, one can therefore define the simulated solution to be the correct answer and find the errors of the two low--resolution simulations.


	The time course for the neuron's depolarization in the three simulations are compared in $Octave$, an open source numerical computing environment similar to $Matlab$.
% 	The depolarization time course of the two low--resolution simulations are compared to that of the simulated solution by Octave, an open source numerical computing environment similar to matlab.
	The considered variables are written to a log file during the execution of \emph{auroSim}, resulting in an executable $Octave$ script when a run of \emph{auroSim} is finished.
	All plots with the caption ``Generated by \emph{auroSim}'' are results of executing such log files in $Octave$. %the resulting log scripts after executions of the simulation software.
%	See appendix \ref{appendixLogForComparison} for more on \emph{auroSim}'s logging facility.
	The reader is referred to appendix \ref{appendixLogForComparison} for more on \emph{auroSim}'s logging facility.


\begin{figure}[htb!p]
	\centering
	\centerline{ %To make the figure lie at the center. Useful for figures that have different size than 1\textwidth
	\includegraphics[width=1.1\textwidth]{depolarizationInASensoryAuron}
	}
	\caption[The depolarization of a sensory neuron with a sinusoidal algebraic sensory function.]{
			Plot of a $NIM$ node's sensory function.
			The sensory function is set to be $f_s(t_n) = 2\tau\left(1-cos(\pi \cdot \frac{t_n}{1000}\right)$ for $t_n \in [0, 1500]$. After $t_n=1500$, the sensory function halves the amplitude and doubles the frequency.
			Firing is represented by a vertical line for the depolarization from $y=0$ to $y=1200$.
			(Generated by \emph{auroSim}) \cite{FDP_report}
			}
\end{figure}


	To make the experiments as comparable and reproducible as possible, the behaviour of a single node is simulated for the two simulation models.
	This node is implemented as a sensory node that receives depolarizing input defined by an externally applied signal $\xi_i(t_n)$. 
	For the sake of reproducibility, algebraic functions are utilized for all experiments in this work.
	For details on the design and implementation of the sensory node, see appendix \ref{appendixSensoryNode}.

	\subsection{Experiment 1: Idealized Situation}
	First consider an idealized situation, with a constant depolarizing inflow.
	This can be implemented as a sensory neuron with a forcing function $\xi(t_n)=1.1 \tau$. 
	This simple input flow simplifies analysis and shows whether the theory presented in chapter \ref{chDevelopmentOfANovelModel} can be used to simulate the neuron.

	This experiment can be used to assess whether the concept of \emph{time windows} and \emph{intra--iteration time resolution} works as designed.
	The concept of \emph{time windows} can be examined, since the activation level $\kappa$ is ``changed'' to the same value every time step.
	Each time $\kappa$ is changed, a new \emph{time window} is initialized and a new estimate for the next firing time is computed.
	This also enables an analysis of whether proactive firing time scheduling can be used to simulate the neuron's firing: %works as intended.
		If the spike is delayed as a result of having a time grid of possible spike times, the simulation error will have a step from before to after the spike.
	The concept of intra--iteration time accuracy therefore works as intended if the error after a spike is a linear continuation of the error curve before the spike.
	To make the effect observable in plots of the neuron's depolarization, a temporal resolution of only $100$ time steps is chosen for experiment 1.

	Because of the simple sensory function, the exact solution can be computed for the neuron's spike times.
	Experiment 1 can therefore be used to assess the accuracy of a simulation, up to a very high precision.
	This enables an analysis of the simulated solution's error, and a discussion of when it can be considered to be the correct solution for accuracy comparisons.
%	This enables an analysis of the simulated solution's error, and a discussion of when it can be considered to be the correct solution for later experiments.
% 	This enables an analysis of the simulated solution's error, and whether the presumption that it can be considered the solution holds.  %, and the error of the simulated solution can be analyzed.


	\subsection{Experiment 2: More Realistic Input Flow}
	\label{ssecExperiment2Design}
	Section \ref{ssecAnalysisOfErrorsForTheTwoModels} concludes  that the $\kappa M$ error is a result of the delay between an altered depolarizing flow and the initiation of a new \emph{time window}. %and the next computational time step.
	This implies that the error is constant for a constant forcing function.
	When designing an experiment for assessing the efficiency of the two simulation models, the form of the input should preferably affect both simulation models equally. 
	The best way to achieve this is to consider a forcing function where neither the value nor the derivative of any order is constant.
 
	Let the forcing function be defined by a trigonometric function that gives an activation level corresponding to $\kappa$ being above the firing threshold for the whole simulation.
	When $\kappa < \tau$, the simulated depolarization has the possibility to level out at a subthreshold value, suppressing the simulation error. 
% 	When $\kappa < \tau$, the artificial neuron's depolarization has the possibility to level out at a subthreshold value, suppressing the simulation error. 
	This is avoided to make the error from the two simulation models prominent. % in the experiment.
% 	This is avoided to make the error more prominent for the two simulations.
	The forcing function in experiment 2 is defined to be
\begin{equation}
	f(t) = (2.1 + sin\left(2\pi \cdot \frac{t_n}{l}\right)) \cdot \tau % TODO ER DENNE RETT? Sjekk!
	\label{eqSensorFunction}
\end{equation}
	where $l$ defines the temporal resolution of the simulation.
% 	This constant can be set by running $auroSim$ with the argument $-r[\text{temp.res.}]$.
%%
	%The neuron was simulated over one and a half period of \eqref{eqSensorFunction}, to enable a comparison of the error for the same phase of the forcing function.
% 	This is achieved by sending the argument $-n[\text{number of periods}]$ to \emph{auroSim}. %, and is done to find if there is any accumulation of error in simulations utilizing the two models. %for the two models.
	The neuron was simulated over one and a half period of \eqref{eqSensorFunction}, to enable a comparison of the error for two time instances where the forcing function is in the same phase.
%	The neuron was simulated over one and a half period of \eqref{eqSensorFunction} to enable a comparison of the error for two time instances where the forcing function was in the same phase.
	This was done to expose any cumulation of error for the two simulation models. %in the two simulations.
	
\begin{equation}
	\text{\emph{./auroSim.out -n1.5 -r\emph{[temporal resolution]}}}
	\nonumber
\end{equation}

% 	An experiment where the forcing function generates a $\kappa$ that goes below the firing threshold has also been conducted, and is presented in appendix \ref{appendixExperiment3Sec}.

 	It is important to emphasize that the experiment is conducted with the first chosen forcing function.
%	The author finds it important to emphasize that the experiment is conducted with the first chosen sensory function.
	No attempts have been made to optimize the results for any of the models.
	This can be done, and be the basis of a more thorough analysis of the two simulation models' error mechanisms.

%TODO Gjør om figuren, slik at den også har med simulated solution (plott av korleis depol. skal oppføre seg for de to eksperimenta TODO TODO
\begin{figure}[hbt!p]
	\centering
	\centerline{ %To make the figure lie at the center. Useful for figures that have different size than 1\textwidth
		\includegraphics[width=1.20\textwidth]{sensorFunctions}
	}
	\caption[Sensory functions for the two efficiency experiments.]{
				Sensory functions for the two experiments. 
				1) First experiment --- constant input, corresponding to inserting a constant current through a probe 
				\mbox{2) Second} experiment --- dynamic input, corresponding to one and a half period \mbox{of eq. \eqref{eqSensorFunction}}.
				(Generated by \emph{auroSim})	
			}
	\label{figSensorFunk}
\end{figure}



	\section{Results}
		\subsection{Static Input Flow}
		The primary motivation behind experiment 1 is to find whether $\kappa M$ can be utilized to simulate the neuron.
		The fundamental concept of \emph{time windows} is put to the test, since the activation level is changed(to the same value) every computational time step.
		Because initiation of a new \emph{time window} involves recalculation of the node's firing time estimate, this experiment can also be used to test whether proactive firing time scheduling works as designed. %%%%%%%%%%
		A plot of the results is presented in fig. \ref{figExperiment1}.
% 		The results are presented in fig. \ref{figExperiment1}.

\begin{figure}[hbt!p]
	\centering
	% Denne filen er oppdatert: Generert 03.02.2012.  Limit antall punkter til 10000 pkt.
	\centerline{
		\includegraphics[width=1.10\textwidth]{experiment1} 
	}
	\caption[Simulation results of experiment 1: static forcing function]{
			The transient time course of the artificial neuron's depolarization, simulated with $NIM$ and $\kappa M$.
%			The transient depolarization time course for the two simulation models. 
			The computational time step is set to $\Delta t = 1\%$, giving 100 time iterations for the two simulations. %$\kappa M$ and $NIM$ simulations. 
			The red curve shows the simulated solution of experiment 1.
%			The red curve shows the depolarization of the simulated solution.
			(Generated by \emph{auroSim})
			}
	\label{figExperiment1}
\end{figure}

		The algebraic solution for the neuron's spike times was found by adding \eqref{eqEstimateOfInterSpikePeriod} recursively to the previous firing time.
%		The algebraic solution for the neuron's spike times can be found by adding \eqref{eqEstimateOfInterSpikePeriod} recursively to the previous firing time.
		The results are presented in table \ref{tabSpikeTimesForKonstK}, alongside the simulation results from the $\kappa M_{100}$ simulation, with a temporal resolution $l = 100$, 
% 				and the results from a $NIM_{1.000.000}$ simulation.
				and the simulated solution.
% 		The simulated solution's absolute error has a monotonic increase of up to one time step for every spike,
 		The $NIM_{1.000.000}$ simulation's absolute error has a monotonic increase of up to one time step for every spike,
% GAMMEL:
% 		The $NIM_{1.000.000}$ simulation error increases with a value of up to one time step for every spike,
			while the $\kappa M_{100}$ simulation appears to give the correct spike times for all spikes in the simulation
%TODO TODO TODO Neste setning er conclusion! TODO Vurder om den skal være her, eller flytte til discussion for kapittelet.    Be certain that it is written in the chapter's discussion! TODO
% 		This can be seen as a confirmation that the concept of intra--iteration time accuracy works as intended.
%1 		It can be observed that the $NIM_{1.000.000}$ simulation diverge with a value of up to one time step every spike. 
%1		The $\kappa M_{100}$ simulation appears to give the correct spike time for all spikes in the simulation, implying that the concept of intra--iteration time accuracy works as intended.
	%	The $\kappa M_{100}$ simulation gives the correct spike time for all spikes in the simulation.
		
\begin{table}[hbt!p]
	\centering
	\begin{tabular}{|l|ccc|}
		\hline 
		Spike \#	& Analytic solution & 	$\kappa$N sim. 	& Simulated solution \\ %TODO Ikkje 'simulated solution, men SN100 ???  TODO    TODO TODO TODO TODO TODO TODO TODO SKRIV INN VERDIENE PÅ NYTT (med siste simuleringsresultat..)
		\hline
		1 			& 23.978953.. 		& 	23.978953.. 			& 23.9789 			\\	
		%1 			& 23.9789527279837 	& 	23.978952728 			& 23.9789 			\\	
		2 			& 47.957905.. 		& 	47.957905.. 			& 47.9578 			\\
		%2 			& 47.9579054559674 	& 	47.957905456 			& 47.9578 			\\
		3 			& 71.936858.. 	 	& 	71.936858.. 			& 71.9367 			\\
		%3 			& 71.9368581839511 	& 	71.936858184 			& 71.9367 			\\
		4 			& 95.915811.. 	 	& 	95.915811.. 			& 95.9156 			
		%4 			& 95.9158109119348 	& 	95.915810912 			& 95.9155 			\\
		%
		\\ \hline 
	\end{tabular}
	\caption{ 	Spike times for the artificial neuron. 
				The analytic solution is computed by adding \eqref{eqEstimateOfInterSpikePeriod} recursively to the previous spike time. 
				The $\kappa N$ simulation has a temporal resolution of $l=100$, %was simulated over $100$ time iterations.
					while the simulated solution is the result of a $NIM_{1.000.000}$ simulation with $l=1.000.000$.
%				Take note of the monotonic increase in error for the $NIM$ simulation.
			}
	\label{tabSpikeTimesForKonstK}
\end{table}




		\subsection{Dynamic Activation level}
			Experiment 2 considers a dynamic input current, defined as one and a half period of \eqref{eqSensorFunction}.
			The simulation results are presented as points in fig. \ref{figExperiment2} whenever a new value is available.
			Note that the $NIM$ simulation is conducted with the temporal resolution $l = 1.000$, while the $\kappa M$ simulation only has $100$ time steps per forcing function period.
% 			A $NIM_{100}$ simulation produced large errors, and have been excluded from the figure to increase readability.

	%TODO TODO TODO TODO TODO TODO TODO TODO TODO TODO TODO TODO TODO TODO TODO TODO TODO TODO TODO TODO TODO TODO Lag eit plot av NIM_100 også, og legg det i appendix. Referer til appendix, her(etter siste linja, over..)



\begin{figure}[hbt!p]
 	\centering
	\centerline{ %To make the figure lie at the center. Useful for figures that have different size than 1\textwidth
 		\includegraphics[width=1.2\textwidth]{experiment2HalvannenPeriode}
	}
 	\caption[Simulation results of experiment 2: dynamic forcing func.]{
			 	The neuron's depolarization curve in a $NIM_{1.000}$ simulation and a $\kappa M_{100}$ simulation.
				The two simulations have a number of time steps that differ with one order of magnitude.
				The red curve shows the simulated solution of experiment 2.
%%%
				(Generated by \emph{auroSim})
			}
 	\label{figExperiment2}
\end{figure}

			
			Since the depolarization value is written to log every time it is updated, the number of points from each simulation indicate the temporal resolution of that simulation.
			Spikes are represented by a vertical line from $x=1050$ to $x=1200$ when the neuron fires.
%% 			%%
			The spikes in the figure indicates that the simulation error is larger in the second period of the forcing function than in the first period.
 			To enable further analysis of this effect, the spike time errors have been isolated and is presented in fig. \ref{figSpikeTimeErrorExperiment2}.
%  			To make this effect prominent and enable further examination, the spike time error has been isolated and presented in fig. \ref{figSpikeTimeErrorExperiment2}.

			The error in spike times for the $NIM_{1.000}$ simulation shows the hypothesized cumulative property of the $NIM$ error.
% 			The error in spike times for the $NIM_{1.000}$ simulation implies a cumulative property for the $NIM$ error.
			To examine the extent of the two models' error properties, experiment 2 was simulated over a time interval that is ten times as long.
% 			To examine the extent of the hypothesized cumulative property of the $NIM$ error, experiment 2 was simulated over a time interval that is ten times as long.
			A plot of the resulting spike time errors is presented in fig. \ref{figExperiment2ErrorInTenSineOscillations}.
			
\begin{figure}[hbt!p]
	\centering
  		\includegraphics[width=0.90\textwidth]{errorInFiringTimesOneAndHalfPeriod}
  	\caption[Spike time error for all $26$ spikes of experiment 2]{
			 	The spike time error for all $26$ spikes in the $\kappa M_{100}$ and the $NIM_{1.000}$ simulations.
				From fig. \ref{figExperiment2}. it can be seen that the second period of the forcing function starts at spike number $15$.
				An indication of the cumulation of error can therefore be found by comparing the spike time error for spike number $5$ and spike number $20$ for the two models. % for the two plots.
% 				The cumulation of error for the two models can therefore be indicated by comparing the spike time error for spike number $5$ and spike number $20$ for the two plots.
				(Generated from log files generated by \emph{auroSim})
			}
  	\label{figSpikeTimeErrorExperiment2}
\end{figure}
			
			


\begin{figure}[hbt!p]
	\centering
	\centerline{ %To make the figure lie at the center. Useful for figures that have different size than 1\textwidth
		\includegraphics[width=1.2\textwidth]{errorInFiringTimesTenOscillations}
	}
	%TODO TODO TODO Bli heilt sikker på kva y-aksen representerer! TODO TODO TODO
	\caption[Spike time error for all spikes from an extended run of experiment 2. The simulation time interval is ten times as long as the forcing function in experiment 2 to make the accumulation of error prominent.]{
			 	The error in spike times for the $\kappa M_{100}$, $NIM_{1.000}$ and $NIM_{10.000}$  simulations, simulated over a time interval that is ten times as long as in experiment 2. 
				Due to the number of spikes, the simulated solution was found by a $NIM_{1E8}$ simulation to make sure the solution's error is acceptable. 
				%The y-axis of the figure represents percent of one forcing function period. %of experiment 2's simulation length.
				The $NIM_{10.000}$ and the $\kappa M_{100}$ simulations gave the correct $228$ spikes, 
					while the $NIM_{1.000}$ simulation produced one spike less. %only produced $227$ spikes.
				A $NIM_{100}$ simulation resulted in only $224$ spikes, where the largest error was $-33.6$. 
				(Generated from log files generated by \emph{auroSim})
		% NIM10e6 	: 228 spikes
		% KN100 	: 228 spikes
		% NIM10.000	: 228 spikes
		% NIM3000 	: 228 spikes
		% NIM1000 	: 227 spikes
		%%%NIM100 	: 224 spikes
			}
	\label{figExperiment2ErrorInTenSineOscillations}
\end{figure}

%XXX XXX
\newpage
%XXX XXX

	\section{Discussion of Experimental Results}

		The primary motivation for the first experiment is to assess whether the theory discussed in chapter \ref{chDevelopmentOfANovelModel} makes it possible to implement a spiking neuron simulator based on synaptic flow.
		%%
		The concept of time windows, as defined in sec. \ref{ssecTheAlgebraicSolution}, enables the use of the algebraic solution for simulation of the neuron's depolarization. 
		In the implementation used in this work, a new time window is initiated every computational time step, making it irrelevant whether the activation level is constant or dynamic. 
		This makes the results from experiment 1 pertinent for error analysis. 

		The simple form of the neuron's forcing function in the first experiment enables a precise error analysis for the its spike times.
		It is possible to compute the neuron's firing times algebraically, enabling an analysis of the simulated solution's error.
		The simulated solution has a cumulative error that increases with up to one computational time step for every spike, given this level of input.
%%%%%%%%
		In a simulation with only $26$ spikes, this gives a maximum spike time error for the simulated solution, $f_{e, max} = \frac{26}{1000000} = 2.6\cdot10^{-5}$. 
%%%%%%%%%%%%%% 			%%						%%				%%									%			%$f_{e, max} = \frac{26}{1500000} = 1.73\cdot10^{-5}$. 
		Thus, in experiment 2, the theoretical maximum spike time error for the simulated solution is much smaller than the computational time step in both low--resolution simulations. %a simulation with a temporal resolution $l = 1.000$.
% ENTEN: Dei 2 over, eller den under:
%		Thus, in experiment 2, the theoretical maximum spike time error for the simulated solution is much smaller than the computational time step in a simulation with a temporal resolution $l = 1.000$.
		$$ \Delta t_{NIM,1.000} = \frac{1}{1000} = 10^{-3}$$
		This shows that the simulated solution can be considered to be the correct solution, up to an accuracy defined by the low--resolution simulation with finest granularity, $NIM_{1.000}$.
%		This shows that the simulated solution can be considered to be the correct solution, up to an accuracy defined by the low--resolution simulation with the finest temporal resolution, $NIM_{1.000}$.
%%%%%%%%%
% 		Since the temporal resolution for the simulated solution is more than $1.000$ times more than that of the low--resolution simulations, this is considered to be an acceptable error for a simulation with only $26$ spikes. %error is acceptable for a simulation with only $26$ spikes.
		In the second part of experiment 2, where the experiment is simulated over a time interval that gives $228$ spikes for the neuron, a $NIM_{100.000.000}$ simulation is used to define the simulated solution. %%%%% 
	%In the second part of experiment 2, where the experiment is simulated over a time interval that gives $228$ spikes for the neuron, a $NIM_{100.000.000}$ simulation is used to find the simulated solution.


 
		Experiment 2 considers a sinusoidal input flow corresponding to an activation level that varies between $1.1\tau$ and $3.1\tau$.
		Since the forcing function has the property that no aspect of the signal is constant in the time domain, 
			the results from experiment 2 is more valid for an efficiency analysis than experiment 1.
		The experiment shows that the $\kappa M_{100}$ simulation generally is more accurate than the $NIM_{1.000}$ simulation.
%		The experiment shows that the  $\kappa M_{100}$ simulation generally produce an error of smaller magnitude than that of the $NIM_{1.000}$ simulation. %, in the course of the simulation.
		The comparative efficiency improvement is larger when the same experiment is simulated over a time interval that is ten times as long.
%  		This effect is larger when the same experiment is simulated over a time interval that is ten times as long.
		The absolute error becomes larger in the $NIM_{10.000}$ simulation than in the $\kappa M_{100}$ simulation before the simulation is over.
		This implies a considerable efficiency improvement, since the $NIM$ simulation utilizes a number of time steps that is two orders of magnitude larger than the $\kappa M$ simulation.
%		This implies a considerable efficiency improvement, as the $NIM$ simulation needs a number of time steps that is two orders of magnitude larger than the $\kappa M$ simulation.
% TA VEKK? TODO Vurder å ha med? : (neiss, denne er bedre i konklusjon..)
%		It is believed that this trend continues for longer simulations. 

		%TODO Drøfte kvifor amplituden i svingningene i spike time error blir større? (sjå fig. \ref{figExperiment2ErrorInTenSineOscillations}).
