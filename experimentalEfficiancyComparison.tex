
\section{General Design of Experiments to Assess Efficiency}
	%The efficiency of a simulator can be measured by the time resolution required to accomplish some accuracy goal. %XXX FLYTTA NED 9 linjer..
	%TODO Write a new intro! TODO som handler om error!
	%A higher temporal resolution, represented by smaller time steps, produce smaller truncation errors for the simulation. %Dette må eg referere! Helst til intro i dette kapittelet.

	The number of computations per time step is given by the number of simulated neurons in the ANN.
	The number of time steps is defined by the temporal resolution of the simulation.
	These two factors therefore affect the number of computations in a similar manner; 
		If the temporal resolution of the simulation is doubled, the effect on the computational load is the same as if an ANN of double size is simulted with the same temporal resolution.
		%If the temporal resolution of the simulation is doubled, the effect on the computational load is the same as if the size of the ANN is doubled with the same temporal resolution in the simulation.

%Nyaktigheit er difor langt viktigare enn computational efficiency! 	%TODO Skriv om dette i intro! Referer til artikler på numerikk! \\

	To compare the accuracy of the depolarization time course produced by the two simulation models, low--resolution solutions of $\kappa M$ and $NIM$ are compared to the a simulation with a much larger temporal resolution.
%TODO Skriv at i section \label{secSpikingANNBackgroundInfo} skriver eg at feilen er en funksjon av temporal resolution. Dette er grunnen til at innholdet i neste stening er bra.
	In this work, the low--resolution simulations have a temporal resolution of less than $1000$ time steps per forcing function period while the high--resolution simulation has a resolution of $1.000.000$ time steps per period.
	The high--res simulation results will be referred to as the simulated solution in the remainder of this text.

	The simulated solution has a temporal resolution that is more than three orders of magnitude larger than the low--res simulations, and can be considered the correct answer %for short simulations 
		with a resolution up to that of the low--res simulation. %defined by the low--res simulations.
	%The experiments designed in this chapter can therefore assess the accuracy of the two models by comparing the simulation results with the simulated solution.
	To assess the accuracy of the two simulation models, one can define the high--res simulation results as the correct answer and find the error between the simulated solution and the two low--res simulations.
	This enables a comparative efficiency analysis of the two simulation models, as the temporal resolution required to accomplish some accuracy goal is closely related with the computational resources needed.
%%%%
	%To assess the accuracy of the two simulation models, one can thus define the high--res simulation as the correct answer and find the error by comparing the results of the low--res simulations to this.
	%To assess how correct the simulation results of the two simulation models are, one can thus define the high--res simulation as the correct answer and experimentally find the error between the low--res simulations and the simulated solution.

	The depolarization time course of the two low--res simulations are compared to that of the simulated solution by octave, an open source numerical computing environment similar to matlab.
	The considered variables are written to a log file during the execution of \emph{auroSim}, resulting in an executable octave script when \emph{auroSim} is finished.
	All plots with the caption "Generated by \emph{auroSim}" are results of running such log files. %the resulting log scripts after executions of the simulation software.
	%This is done by executing the octave log scrips created during each run of the simulation software.
	%%Every auron object have a private file stream that can be used to write to the auron's log file. 
	It is referred to appendix \ref{appendixLogForComparison} for more on \emph{auroSim}'s logging facility.
%	The later approach is utilized in this work, as this gives a more intuitive understanding of the connection between simulation accuracy and efficiency of the simulation model.


%TODO TODO Lag også andre plott for noen av de andre funksjonene i sensoryFunctions.h, og legg dei i appendix!     -kappa og depol.      TODO TODO TODO TODO TODO TODO TODO 
% SENSORFUNKSJONEN ER FEIL? Trur ikkje det lenger. Sjekk sensory funtions .h ! 		Det ser ut som om f_s(T) går til 3\tau i starten, ikkje 4\tau slik som den lista funksjonen vil gjøre. Finn ut av dette, en siste gang for å være sikker..
% Kanskje skriv sensory function for siste bit av plottet?
\begin{figure}[htb!p]
	\centering
	\centerline{ %To make the figure lie at the center. Useful for figures that have different size than 1\textwidth
% TODO Gjør width=1.2\textwidth om dette blir finare.. Bra med litt større, men no føkker det opp litt.
	\includegraphics[width=1.1\textwidth]{depolarizationInASensoryAuron}
	}
	\caption[The depolarization of a sensory neuron with a sinusoidal algebraic sensory function.]{
			Plot of a $NIM$ node's sensory function.
			The sensory function is set to be $f_s(t_n) = 2\tau\left(1-cos(\pi \cdot \frac{t_n}{1000}\right)$ for $t_n \in [0, 1500]$. After $t_n=1500$, the sensory function halves the amplitude and doubles the frequency.
			Firing is represented by a vertical line for the depolarization from $y=0$ to $y=1200$.
			%(Results produced in preliminary project to this MSc thesis).
			(Figure is from the preliminary project. Generated by \emph{auroSim}).
			}
\end{figure}


	To make the experiments as comparable and reproducible as possible, the behaviour of a single node is simulated for the two simulation models.
	This node is implemented as a sensory node that receives depolarizing input defined by an externally applied signal $\xi_i(t_n)$, %as discussed in sec. \ref{appendixSensoryNode}.
		as this enable more reproducible experiments and a simpler analysis.
	If $\xi_i(t_n)$ is defined by an algebraic function, the algebraic solution to the node's depolarization can be found. %, the results are more suited for analysis.
	For the sake of reproducibility, algebraic functions are utilized for the experiments in this work.
	%In this way, the results are simpler to analyze, as $\xi_i(t_n)$ can be defined by an algebraic function.
	%The experiments are also more reproducible, as networks of neurons with chaotic behaviour is avoided.
	For more on the design and implementation of the sensory node, it is referred to appendix \ref{appendixSensoryNode}.

	\subsection{Experiment 1: Idealized Situation}
	First consider an idealized situation, with a constant depolarizing inflow.
	This can be implemented with a sensory neuron as discussed in appendix \ref{appendixSensoryNode} with a forcing function $\xi(t_n)=1.1 \tau$. 
	%In the first experiment, the forcing function $\xi(t_n)$ gives a flow corresponding to an activation level of $\kappa = 1.1 \tau$.
	The simple form of input simplifies analysis and shows whether the theory presented in chapter \ref{chapDevelopmentOfANovelModel} can be used to simulate the neuron.
%the concept of 'time windows' and intra--iteration time accuracy works as intended.
	%The simple form of input simplifies analysis and makes it simpler to examine whether the concept of 'time windows' and intra--iteration time accuracy works as intended.

% 	%%This enables an analysis of different aspects of the $\kappa M$ implementation; 
% 	%	From the simulation results, it can be examined whether the concept of 'time windows' and intra--iteration time accuracy works as intended.
% 	The aspect of 'time windows' and the hypothesized intra--iteration time accuracy are important aspects to be assessed in this experiment. 
% 	The concept of time windows is put to the test as the node's activation level $\kappa$ is ``changed'' to the same value, each time step.
	The concept of 'time windows' is the most important aspect to be assessed in this experiment, and can be examined as the activation level $\kappa$ is ``changed'' to the same level each computational time step.
	Every time $\kappa$ is changed, a new 'time window' is initialized and a new estimate for the next firing time is computed.
	%The estimate of the next spike time is also updated, letting us examine whether the proactive firing time scheduling works as designed;
	This also enables us to examine whether proactive firing time scheduling works as intended;
		If the spike is delayed as a result of having a time grid of possible spike times, the simulation error will have a step from before to after the spike.
	%If the error is a continuous function, there are no steps in the depolarizing error caused by the spike. %, indicating that the neuron fires at the computed time(intra--iteration spike time accuracy).
	The concept of intra--iteration time accuracy therefore works as intended if the error after a spike is a linear continuation of the error curve before the spike.
	If the error level have a step each time the neuron fires, it can on the other hand be concluded that the neuron fires at the wrong time and that the implementation or the discussion about intra--iteration time accuracy may be faulty.
	To make this effect observable in plots of the simulation results, a temporal resolution of only $100$ time steps is chosen for experiment 1.
	%If the neuron fires at the wrong time, the absolute depolarization error would step up to a new level.

	Because of the simple sensory function, the exact solution can be computed and the algebraic solution can be utilized as the correct answer.
	%The algebraic solution can therefore be used to assess the accuracy of the two models, with a very high accuracy.
	Experiment 1 can therefore be used to assess the accuracy of the simulated solution with a very high precision.
	%A simulation of the $\kappa M$ is also compared, to have an other approach for assessing the simulation error of the novel model. XXX Dropper å skrive dette nå: Heller når eg lager tabellen i results! XXX

%	If the efficiency of the simulator is measured by the time resolution required to accomplish some accuracy goal, ...

%TODO Gjør om figuren, slik at den også har med simulated solution (plott av korleis depol. skal oppføre seg for de to eksperimenta TODO TODO
\begin{figure}[hbt!p]
	\centering
	\includegraphics[width=0.75\textwidth]{sensorFunctions}
	\caption[Sensory functions for the two efficiency experiments.]{
				Sensory functions for the two experiments. 1) constant input, corresponding to inserting a constant current through a probe 
				2) dynamic input, corresponding to one and a half period of eq. \eqref{eqSensorFunction}.
				(Generated by \emph{auroSim})	
			}
	\label{figSensorFunk}
\end{figure}

	\subsection{Experiment 2: More Realistic Input Flow}
	\label{ssecExperiment2Design}
	It is likely that the $\kappa M$ error is a result of the delay between an altered depolarizing flow and the next computational time step.
	This implies that the error or its differential is constant for a constant forcing function.
	%To make an experiment to assess the efficiency of the two simulation models, an experiment intended to be more just to both models is devised.
	To make an experiment to compare the efficiency of the two simulation models, an experiment intended to be equal for both models is devised.

% 	To make the comparison between the two models just, it might be an idea to have a forcing function where neither the value nor the differentiated of any order is constant.
% 	For the sake of reproducibility, stochastic functions are avoided.
% 	%To truly assess the accuracy of the two models, an experiment is designed where neither the value nor the differentiated of any order is constant.
% 	A group of such algebraically defined functions is the trigonometric functions.
 
	Let the forcing function be defined by a flow that gives an activation level corresponding to $\kappa$ being above the firing threshold for the whole simulation.
	%When $\kappa < \tau$, the depolarization of the node have the possibility to level out at a subthreshold depolarization, something that can function as a buffer for the error.
	When $\kappa < \tau$, the depolarization of the node have the possibility to level out at a subthreshold depolarization, something that can suppress the simulation error.
	This is avoided to make the error more prominent for the two simulations.
	The algebraic forcing function of experiment 2 is defined to be 
\begin{equation}
	f(t) = (2.1 + sin\left(2\pi \cdot \frac{t_n}{l}\right)) \cdot \tau % TODO ER DENNE RETT? Sjekk!
	\label{eqSensorFunction}
\end{equation}
	where $l$ is the temporal accuracy of the simulation and can be set by sending the argument $-r[\text{temp.res.}]$ to \emph{auroSim}.

	The sensory node is simulated over one and a half period of \eqref{eqSensorFunction} to be able to compare the error for the same phase of the applied signal.
	This can be achieved by sending the argument $-r[\text{number of periods}]$ to \emph{auroSim}, and is done to find if there is any accumulation of error in simulations utilizing the two models. %for the two models.
	%Any accumulation of error will thus be prominent in the simulation results.

% 	A simulation over one and a half period of \eqref{eqSensorFunction} is 
% 	The simulation is set to last for one and a half period of the forcing function, set by sending the argument $-n[\text{number of periods}]$ to \emph{auroSim}.
% 	This is done to be able to compare the error for the same phase of the applied signal, and compare a possible accumulation of error for the two models.
% 	%This makes it possible to observe differences in the accumulation of error for the two simulation models.

	The author find it important to emphasize that the experiment is conducted with the first chosen sensory function.
	No attempts have been made to optimize the results for any of the models.
	This can be done, and be the basis of a more thorough analysis of the two simulation models' error mechanisms.
	%This can be done, and be the basis of a more thorough analysis of the error mechanisms of the two simulation models.


	\section{Experiment Results}
		\subsection{Static Input Flow}
		The primary motivation behind experiment 1 is to find whether $\kappa M$ can be utilized to simulate the neuron.
		The fundamental concept of 'time windows' is put to the test, as the activation level is changed(to the same value) every computational time step.
		As initiation of a new time window demands recalculation of the node's firing time estimate, this can also be used to find whether the proactive firing time scheduling scheme works as designed.

\begin{figure}[hbt!p]
	\centering
	% Denne filen er oppdatert: Generert 03.02.2012.  Limit antall punkter til 10000 pkt.
	\includegraphics[width=0.95\textwidth]{experiment1} %{staticInput_alpha0_1} %TODO TODO TODO TODO TODO TODO TODO TODO TODO TODO TODO TODO OPPDATER PLOTT! Nå er står det SN og KN. Skal stå "NIM simulation" og "KM simulation"
	\caption[Simulation results of experiment 1: static forcing function]{
			The transient depolarization time course for the two simulation models. The time step is set to $\Delta t = 1\%$, giving 100 time iterations for the $\kappa M$ and $NIM$ simulations. 
			%The red curve is the time course of the depolarization of a numerical simulation with 1.000.000 iterations.
			%The red curve is the depolarization time course of a $NIM$ simulation with 1.000.000 iterations.
			The red curve shows the depolarization of the simulated solution.
			(Generated by \emph{auroSim})
			}
	\label{figExperiment1}
\end{figure}

		The exact spike times can be found by adding \eqref{eqEstimateOfInterSpikePeriod} recursively to the previous spike time.
		The results are presented in table \ref{tabSpikeTimesForKonstK}, alongside the simulation results from a $\kappa M_{100}$ simulation with a temporal accuracy of $100$ and the simulated solution.
		It can be observed that the $NIM_{1.000.000}$ simulation diverge with a value of up to one time step every spike. 
		The $\kappa M_{100}$ simulation appears to give the correct spike time for all spikes in the simulation, implying that the concept of intra--iteration time accuracy works as intended.
	%	The $\kappa M_{100}$ simulation gives the correct spike time for all spikes in the simulation.
		
\begin{table}
	\centering
	\begin{tabular}{|l|ccc|}
		\hline 
		Spike \#	& Analytic solution & 	$\kappa$N sim. 	& Simulated solution \\ %TODO Ikkje 'simulated solution, men SN100 ???  TODO    TODO TODO TODO TODO TODO TODO TODO SKRIV INN VERDIENE PÅ NYTT (med siste simuleringsresultat..)
		\hline
		1 			& 23.978953.. 		& 	23.978953.. 			& 23.9789 			\\	
		%1 			& 23.9789527279837 	& 	23.978952728 			& 23.9789 			\\	
		2 			& 47.957905.. 		& 	47.957905.. 			& 47.9578 			\\
		%2 			& 47.9579054559674 	& 	47.957905456 			& 47.9578 			\\
		3 			& 71.936858.. 	 	& 	71.936858.. 			& 71.9367 			\\
		%3 			& 71.9368581839511 	& 	71.936858184 			& 71.9367 			\\
		4 			& 95.915811.. 	 	& 	95.915811.. 			& 95.9156 			
		%4 			& 95.9158109119348 	& 	95.915810912 			& 95.9155 			\\
		%
		\\ \hline 
	\end{tabular}
	\caption{Spike times for sensor neuron. 
				Analytic solution is computed by adding eq. \eqref{eqEstimateOfInterSpikePeriod} to the previous spike time. 
				$\kappa N$ is simulated over $100$ time iterations.  The simulated solution is a $NIM$ simulation with a temporal accuracy of $10^6$ iterations.}
	\label{tabSpikeTimesForKonstK}
\end{table}

		\subsection{Dynamic Activation level}
%TODO TODO TODO TODO Har eg skrevet at eg ser på KM_100 og NIM_1000? 	TODO TODO TODO TODO TODO TODO 
			Experiment 2 considers a dynamic input current, defined to be one and a half period of \eqref{eqSensorFunction}.
			The simulation results are presented in fig. \ref{figExperiment2}, where the simulation results are presented as points whenever the value is updated.
			Note that the $NIM$ simulation is conducted with a temporal resolution of $1000$, while the $\kappa M$ simulation has the same resolution as in experiment 1(resolution of $100$).
			The result from a $NIM_{100}$ simulation produce large errors, and have been excluded from the figure as it gives a messy plot. %TODO legg det inn i appendix? TODO
%and have been excluded as this makes the figure confusing. %XXX Endre siste ord: ROTETE..

\begin{figure}[hbt!p]
 	\centering
 		%\label{figExperiment2:depolarizationPlot}
	\centerline{ %To make the figure lie at the center. Useful for figures that have different size than 1\textwidth
 		\includegraphics[width=1.2\textwidth]{experiment2HalvannenPeriode}
	}
 	%	\label{figExperiment2:spikeTimeErrorPlot}
 	\caption[Simulation results of experiment 2: dynamic forcing func.]{
			 	The neuron's depolarization curve in a $NIM_{1000}$ simulation and a $\kappa M_{100}$ simulation.
 				%The two simulation results are produced in simulations that have a number of time steps that differ with one order of magnitude.
				The two simulations have a number of time steps that differ with one order of magnitude.
 				%The red curve is the result of a high-resolution simulation and is seen as the true depolarization curve.
 				The red curve shows the simulation results from a $NIM_{1.000.000}$ simulation, and is seen as the true depolarization time course for the neuron.
				All simulations produce the $26$ spikes that is though to be the correct number of action potentials. % of the simulation.
				(Generated by \emph{auroSim})
			}
 	\label{figExperiment2}
\end{figure}

			The depolarization value is written to log every time it is updated, so the number of points in each figure thus indicate the temporal resolution of the simulations.
			%The temporal resolution of the two simulations can be observed as the number of points in each figure, as the depolarization is written to log every time the value is updated.
%% 			%%
			It seems like the simulated value of the second period have a larger error than in the first period of the forcing function.
			%It seems like the spikes of the second period have a different spike time error than in the first period of the forcing function.
			To make this effect more prominent and examine it further, the spike time error have been isolated and presented in fig. \ref{figSpikeTimeErrorExperiment2}.
			%It can be observed that the spikes of the second period of the forcing function has a larger error than in the first period for the $NIM_{1000}$ simulation.
			%To make this effect more obvious, the spike time error is presented in fig. \ref{figSpikeTimeErrorExperiment2}

			The error in spike times for the $NIM_{1000}$ simulation implies a cumulative property for the $NIM$ error.
			%Fig. \ref{figSpikeTimeErrorExperiment2} implies a cumulative property for the $NIM$ error.
			%To test the extent of a possible cumulative error for the two simulation models, \eqref{eqSensorFunction} is simulated over $10$ times the length of experiment 2.
			To test the extent of a hypothesized cumulative error(see section \ref{ssecAnalysisOfErrorsForTheTwoModels}), this experiment is simulated for ten times the simulation length of experiment 2.
			%To test the extent of a hypothesized cumulative error(see section \ref{ssecAnalysisOfErrorsForTheTwoModels}), experiment 2 is simulated for ten times the simulation length of experiment 2.
			%To test the extent of a hypothesized cumulative error in section \ref{ssecAnalysisOfErrorsForTheTwoModels}, eq. \eqref{eqSensorFunction} is simulated over $10$ times the length of experiment 2.
			%Due to the clear accumulation of error for the $NIM$ simulation, \eqref{eqSensorFunction} is simulated over $10$ times the length of experiment 2.
			A plot of the simulations' errors in spike times is presented in fig. \ref{figExperiment2ErrorInTenSineOscillations}.
			
\begin{figure}[hbt!p]
	\centering
  		\includegraphics[width=0.90\textwidth]{errorInFiringTimesOneAndHalfPeriod}
  	\caption[Spike time error for all $26$ spikes of experiment 2]{
			 	The spike time error for all $26$ spikes in the $\kappa M_{100}$ and the $NIM_{1000}$ simulation.
				From fig. \ref{figExperiment2}. it can be seen that the second period of the forcing function starts at spike number $15$.
				Compare the spike time error for spike number $5$ and number $20$ to observe cumulation of errors for the two models.
				%Compare the spike time error of the $NIM_{1000}$ simulation at spike number $5$ and the $20$'th spike to observe cumulation of errors.
				%Note the difference in firing time error of the $NIM$ simulation, between spike number $5$ and spike $20$.
				(Generated from log files generated by \emph{auroSim})
			}
  	\label{figSpikeTimeErrorExperiment2}
\end{figure}
			
			


\begin{figure}[hbt!p]
	\centering
	\centerline{ %To make the figure lie at the center. Useful for figures that have different size than 1\textwidth
		\includegraphics[width=1.2\textwidth]{errorInFiringTimesTenOscillations}
	}
	%TODO TODO TODO Bli heilt sikker på kva y-aksen representerer! TODO TODO TODO
	\caption[Spike time error for all spikes from an extended run of experiment 2. The simulation time interval is ten times as long as the forcing function in experiment 2 to make the accumulation of error prominent.]{
			 	The error in spike times for the $\kappa M_{100}$, $NIM_{1000}$ and $NIM_{10.000}$  simulations over ten times the length of experiment 2. 
				Due to the number of spikes, the correct spike times are found by a $NIM_{1E8}$ simulation to make sure the solution's error is acceptable. %TODO TODO TODO TODO TODO Vær heilt sikker på at det er NIM_{1E9}. Sto: 1.5E9 før..
				%The y-axis of the figure represents percent of one forcing funtion period. %of experiment 2's simulation length.
				%The error is measured in percent of experiment 2's simulation length.
				The $NIM_{10.000}$ and the $\kappa M_{100}$ simulations gave the correct $228$ spikes, 
					while the $NIM_{1000}$ simulation produced one spike less. %only produced $227$ spikes.
				%The $NIM_{100}$ simulation gave an error that spoiled the plot and have not been included. 
				It might be important to mention that the $NIM_{100}$ simulation results only have $224$ spikes, 
					where the largest error is $-33.6$. %%% 			%%%
				(Generated from log files generated by \emph{auroSim})
		% NIM10e6 	: 228 spikes
		% KN100 	: 228 spikes
		% NIM10.000	: 228 spikes
		% NIM3000 	: 228 spikes
		% NIM1000 	: 227 spikes
		%%%NIM100 	: 224 spikes
			}
	\label{figExperiment2ErrorInTenSineOscillations}
\end{figure}

	\section{Discussion of Experimental Results}
	% eller Analysis of Experimental Results   eller 	Analysis of Simulation Results.
	%\section{Analysis of Simulation Results} %.. Det som står i artikkel::Discussion..


		%TODO TODO TODO Ta vekk dette første avsnittet! Blir for mykje GPP! TODO TODO TODO Skriv det til ei setning, og gå rett på neste avsnitt..
		The primary motivation behind the first experiment is to assess whether the theory discussed in chapter \ref{chapDevelopmentOfANovelModel} enable the use of flow simulation of a $LIF$ neuron's depolarization.
		%As can be seen in fig. \ref{figExperiment1}, the $\kappa M$ simulation is at least as precise as the $NIM$ simulation of the node's depolarization.
		%This is expected, and can be seen as a confirmation that the novel simulation model works as designed.
		%%
		The concept of time windows, as defined in sec. \ref{ssecTheAlgebraicSolution}, enables the use of the algebraic solution for simulation of the neuron's depolarization. 
		In the implementation used in this work, a new time window is initiated every computational time step, making it irrelevant whether the activation level is constant or dynamic. 
		This makes the result from experiment 1 pertinent for error analysis. 
		%%
		The simple form of the input flow makes it possible to extract information about fundamental mechanisms of neuron emulation,
			and e.g. the aspect of spike time error can be examined separately. %, without disturbing elements.
		Figure \ref{figExperiment1} indicates that all spikes happen at the right time, showing that the concept of intra--iteration time accuracy works as intended.
		Table \ref{tabSpikeTimesForKonstK} confirm this, as the simulated spike times of the $\kappa M$ neuron are the same as the algebraically computed spike times for all spikes.
	%%
		%For constant activation level $\kappa$, table \ref{tabSpikeTimesForKonstK} shows that the $\kappa M_{100}$ simulation actually produce more precise simulation results than the $NIM_{1.000.000}$ simulation.
		For constant activation level $\kappa$, the $\kappa M_{100}$ simulation therefore produce more precise simulation results than the $NIM_{1.000.000}$ simulation. %, confirming the $\kappa M$ error analysis in \ref{ssecAnalysisOfErrorsForTheTwoModels}.

		The simulated solution has a cumulative error that increase with up to one computational time step every spike.
		Because the simulated solution has more than $1000$ times the number of time steps than the low--res simulations, it is still possible to utilize the results from the $NIM_{1.000.000}$ simulation as the solution as long as a simulation produce a number of spikes that is much smaller than $1000$.
		For the second part of experiment 2 where the neuron is simulated over $228$ spikes, 
			the simulation result considered to be the correct spike times are found by a $NIM_{100.000.000}$ simulation.
		%For the second part of experiment 2 where the neuron is simulated over $228$ spikes, the reference values used as the correct spike times are found by a $NIM_{100.000.000}$ simulation.
 
		Experiment 2 considers a sinusoidal input flow corresponding to an activation level that varies between $1.1\tau$ and $3.1\tau$.
		As the forcing function has the property that no aspects of the signal is constant(in the time domain), %something that enable a fairer comparative efficiency analysis between the two models.
			it is more likely that experiment 2 gives balanced results of the efficiency of the two models.
			%it is more likely that experiment 2 gives a balanced results of the comparative efficiency for the two models.
			%it is likely that the experiment can be used to make a proper analysis of the efficiency of the two models.
		The results shows that the $\kappa M_{100}$ simulation produce an error of smaller size than that of the $NIM_{1.000}$ simulation. %, in the course of the simulation.
		%The results shows that the $\kappa M_{100}$ simulation produce a smaller error than the $NIM_{1.000}$ simulation in the course of the simulation.
		%The results from experiment 2, presented in fig. \ref{figExperiment2}, shows that the $\kappa M_{100}$ simulation gives a more accurate simulation than a $NIM_{1.000}$ simulation.
		%The results presented in figure \ref{figExperiment2} shows that a $\kappa M_{100}$ simulation produce a smaller error than a $NIM_{1.000}$ simulation. 
		This effect is more prominent when the same experiment is simulated over a time interval that is ten times as large.
		%The absolute error becomes larger in the $NIM_{10.000}$ simulation than in the $\kappa M_{100}$ simulation.
		The absolute error becomes larger in the $NIM_{10.000}$ simulation than in the $\kappa M_{100}$ simulation before the simulation is over.
		This implies a considerable efficiency improvement, as the $NIM$ simulation needs a number of time steps that is two orders of magnitude greater than the $\kappa M$ simulation.
		%This implies a considerable efficiency improvement by the $\kappa M$, as a $NIM$ simulation needs a number of time steps that is two orders of magnitude greater in the course of this simulation.
		It is believed that this trend continues for longer simulations. 

		%TODO Drøfte kvifor amplituden i svingningene i spike time error blir større? (sjå fig. \ref{figExperiment2ErrorInTenSineOscillations}).
