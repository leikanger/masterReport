

\section{Introduction}

% Motivasjon: Kvifor er dette gjort?
% Problemområde: Kva er gjort?
% Kva har andre gjort? 		(skive om SANN)
% Mi avgrensning.  --Kva har eg utelatt. Kvifor. osv..
% Disposisjon: Utvida innholdsfortegnelse.
	% Skriv: Eg har plassert eit kapittel i appendix om synaptisk overføring i appendix. Dette er for å få en meir helhetlig oversikt over det biologiske systemet, men siden det ikkje er direkte knyttet til oppgaven(dårlig formulert) er det ikkje med som en del av hovedteksten. Dette kan brukes for utfyllende informasjon og motivasjon for SANN.

Torebror: jo det er kjekt :)
	har du noke praktisk ide om korleis desse kunstige neurona kan nyttast?
me: Tja.. Tenkte på det på toget i natt. Tre hovedelement: ANN, neurosimulering for nevrovitenskap(teste teorier og slikt)
	og en til.
	ANN(Artificial Neural Networks) er brukt mykje for å gi datamaskina assosiative egenskaper("hmm, det føler eg at eg har sett, minner meg på ...") og for maskinlæring. Spesiellt mykje brukt i bildegjenkjenning
Torebror: (...ANN også mykje brukt i geomagnetisk "vær"melding :P )
me: Nice. Mønstegjenkjenning og assosiative egenskaper!
	(kanskje..)
	Neiss, eg har ikkje tenkt så mykje på bruken. Eg bare liker å sysle med det..
	Kanskje noken har bruk for det..
	:)
Torebror: ja, med basis på målingar fra satellittar 1 time herfra mot sola (dvs 1 time for solpartiklar)
	kan tenke meg det kan brukast mykje i medisinsk kyb
me: Tja, ikkje enda..
	Det var den tredje bruksområdet..
	Dei har klart å lage elektroder som kan brukes som interface direkte til nerve-bunter i kroppen.
	Dette signalet må behandles på en måte, og da tenker eg det kan være nyttig å gjøre dette ved å la det "snakke" med kunstige neuron i datamaskina. Men dette er veldig visjonært, og ingen har gjort det.
	Fryktlig spennande! SciFi!



I FDP ble det teoretisert at 
	for certain situations, $\kappa M$ is more effective than $NIM$.
Der skriver eg også litt om kor  vanskelig det er å sjå for seg kor effektivt noke er, spesiellt for ANN.
Eg har dermed videreført dette prosjektet i dette arbeidet.




%Kort motivasjon for å lese vidare. Tenk at dette er dritkjedelig for sensor, og at eg må motivere han til å lese vidare (eit avsnitt)
%Tidligare arbeid. 
%  Ting som må være med :
% 		A statement med målet til teksten. Kvifor vart prosjektet gjort?
% 		Nok bakgrunnsinfo for å forstå kvifor det er viktig å lese vidare.
% 		Proper accnowledgement of previous work on which I am building. Nok referanser til at leser kan gå til biblioteket og finne støttelitteratur før han leser vidare.
% 		The introduction should be focused on the thesis question(s).  All cited work should be directly relevent to the goals of the thesis.  
% 		Scope of project: Kva er med, og kva er ikkje!
% 		Verbal table of context. Vær sikker på at det er veldig klart kva som er bakgrunnsinfo og kva som er mitt arbeid.



Motivasjon! Kvifor simulere neuronet!

Problemet!  - (Såppass kaotisk med ANN at feil kan føre til enorme utslag)
			- feil som gir feil oppførsel [VEID OPPIMOT]  for treig


Skriv kven denne teksten er skrevet for. Skriv at eg ikkje går ut fra noe neuroscience bakgrunn, men en grunnleggende matematisk bakgrunn og en relativt god oversikt over C++ er gått ut fra at leser har..

%Skriv kva gruppe eg skriv til. Kva bakgrunn ser eg for meg at dei har. Eg har tatt med litt bakgrunnsinformasjon om nevro dersom leser ikkje har utpregende kjennskap til dette området. Eg har også forsøkt å ta med litt meir i avsnittene som omhandler C++, ettersom leser også kan være fra neuroscience minjøer uten utpreget kjennskap til programmering. Tilfeller vil difor oppstå der leser har god kunnskap til området, og i dette tilfellet bes leser å skim these sections.

% // vim:fdm=marker:fmr=//{,//}
