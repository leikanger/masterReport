

% Trenger "Table of abbrevations"
% 	- LTP, LTD, STDP
% 	- "spike" ? (at det betyr Action potential)?
% 	- SANN  - spiking ANN
% 	- SN 	- Spiking Node, a node in SANN.
% 	- KANN, KN.
% 	- LIF neuron: 	Leaky--Integrate and Fire  neuron




% Eller kanskje eg treng stikkordsregister: Trenger:
% 	- Det over og:
% 	- axo-axonic synapses
%
% 	- KN, KANN node : 	node based on a the consept of the $\kappa$ model of the neuron. Nodes of KANN
% 	- SN, SANN node : 	node based on a direct simulation of a spikin neuoron. Nodes of SANN
% 	- suptrathreshold levels : 	value higher than the firing threshold of a node.
% 	- STDP
% 	- LIF neuron
% 	- synaptic weight.
% 	- AuroSim 		:  	The implementation written in this project.
%
%


\documentclass[b5paper,11 pt]{report}
% TODO Er dette rett? Sjekk dette..


\usepackage[utf8]{inputenc}
\usepackage{graphicx} 

% sudo apt-get install texlive
%\usepackage[font=small,format=plain,labelfont=bf,up,textfont=it,up]{caption}
\usepackage[font=footnotesize,format=plain,labelfont=bf,up,textfont=it,up]{caption}


\usepackage{amsmath}
\usepackage{mathrsfs} %brukt for krølle-L : laplace

\usepackage{subfig}   % for subfigures.
\usepackage{listings} %for c++ kode
\lstset{language=c++}


\usepackage{amsthm}
\newtheorem{mydef}{Definition}


%\usepackage{pstricks} % For teikning. 		FOR mykje styr å lære..
%\usepackage{pstricks-add}
 %\let\psgrid\relax %XXX Fjærner GRID på alle pstricks teikninger. (global removal)

%\usepackage{epstopdf} % for å få med eps i pdf-latex (?)    funkerIkkje.. Gjør det manuellt for filene..



%Kanskje dette er naudsynt? For metapost:
%\DeclareGraphicsRule{*}{mps}{*}{}









%%%%%%%% Fra Kristoffer sin master:
\lstset{ %
basicstyle=\footnotesize,       % the size of the fonts that are used for the code
%numbers=left,                   % where to put the line-numbers
%numberstyle=\footnotesize,      % the size of the fonts that are used for the line-numbers
%stepnumber=1,                   % the step between two line-numbers. If it's 1 each line will be numbered
%numbersep=5pt,                  % how far the line-numbers are from the code
showspaces=false,               % show spaces adding particular underscores
showstringspaces=false,         % underline spaces within strings
showtabs=false,                 % show tabs within strings adding particular underscores
%captionpos=b,                   % sets the caption-position to bottom
%breakatwhitespace=false,        % sets if automatic breaks should only happen at whitespace
%escapeinside={\%*}{*)}          % if you want to add a comment within your code
%
%frame=single,                % adds a frame around the code
tabsize=4,                % sets default tabsize to 2 spaces
%breaklines=true,                % sets automatic line breaking
}


\author{Per R. Leikanger}
%\title{Development and Assessment of a Novel Model for Artificial Neural Networks}
\title{Development and Assessment of a Novel Model for Neural Simulation}
\date{\today}     



\begin{document}   



% TODO Neste linje skal kanskje være over abstract og maketitle? TODO
\pagestyle{empty} %get rid of header/footer for toc page

\maketitle

\begin{abstract}
abstract: Her.
\end{abstract}






\pagestyle{empty} 

\tableofcontents %put toc in
\cleardoublepage %start new page %TODO Brukes dersom [report] brukes for dokumentet.
\clearpage

\setcounter{page}{1} %reset the page counter
\pagestyle{plain} % put headers/footers back on















\section{Introduction}

	\subsection{Dette skal heuilt sist, f.eks. i diskurs!}
	Skriv en-eller-annen--plass at når man gjekk fra 1. og 2. generasjon ANN til 3. generasjon ANN, det er mulig å sei at man begynte å "consider a Moore automata of the neuron".
	(Dette er en litt løs tolking, siden ouput er gitt som discrete pulser. Eg trur likevel at den er gyldig..)
	(Her er det bare state som gir ouput).
	
	I min nye modell, tar eg dette videre, og innfører at output også er gitt av present input til neuronet. Dette gir oss en Mealy automata av neuronet.
	Skriv litt om Moore vs. Mealy automata!


% // vim:fdm=marker:fmr=//{,//}


\chapter{Background Theory}
	Introduksjon til kapittel: motiver leser til å få bakgrunnsinfo!
	-Neurale systemet er en shitbra signal-prosesseringsenhet som har en del egenskaper som digitale signal-processore ikkje har- List opp..
	
% DISPOSISJON
% 	Intro: Begynn beldig vidt: "neuron" er en samlebetegnelse for en spesiell type celler med signalling properties (kan bli eksitert). [REF].
%  		IDE: -> Skriv litt om at den vide grupperingen gjør at simulering er vanskelig (men ikkje her: MEN i ANN-seciton!)
%  		- Dette ligger i komplekse neurale nettvert med "recurrent connections" og høg grad av 'connectivity'.(I CNS er det estimert til å være 1000+ input per neuron)
% 		- Koblingen mellom 
% 




% "Although the human brain contains an extraordinary number of these cells (in the order 10E11 neurons), which can be classified into at least a thousand different types, all nerve cells share the same architecture." Kandell kap 2.



\section{Biological Neural Systems}
	
	In the late 1800s, Camillo Golgi developed a way of staining nervous tissue so that complex networks of nerve cells (neurons) became apperent.
	Ramon Y Cajal, Santiago used Golgi's technique in such a way that individual neurons could be separated, and it was observed that nervous tissue was not a continuous web but a network of discrete cells. 
	He proposed what has later been known as the neuron doctrine; That the basis of intelligence is individual ``brain cells'' that can process incoming transmissions and send ouput transmissions in certain situations.
	For their contribution, Ramon Y Cajal and Golgi shared the 1906 Nobel's price in Physiology and Medicine. %CITE: Kandell kap 2. 

	Modern neuroscience follows the neuron doctrine. 
	Each node in a neural network is called a neuron and the connection between neurons are called synapses.
	All synapses have direction(propagates information in one direction) and transmitts if the presynaptic neuron ``fires'' an action potential.
	Synapses can be exitatory or inhibitory.
	Transmissions in excitatory synapses increase the postsynaptic membrane potential, causing that neuron to approach firing. %sending an action potential.
	Inhibitory transmissions does the opposite, and inhibits the postsynaptic neuron with respect to firing.
	When a neuron fires an action potential, a transmission is initialized for all the neuron's output synapses.  % redundant? eller viktig å poengtere?
% TODO TODO TODO CITE!
	
	In this section, the most important elements of neural signal processing are presented, enabling the reader to become more familiar with how neural networks process information.
	%It is recommended that the reader utilize this section as a reference work when neural simulation is discussed.
	It is recommended that the reader utilize this section as a reference work when different methods of neural simulation is presented.
	Before discussing how the neuron processes information, the organization of the neuron have to be reviewed. %REVIEWED er dårlig TODO
	%We will start with the organization of the neuron, before moving on to a fundamental signal processing mechanism of the neuron; The electrochemical properties of the neuron membrane.




%	The most important elements of signal processing in neural systems is presented in this section,
%		enabling the reader to become familiar with how biological neural networks function.
%	Fundamental elements for the neural signal processing is presented, and it is recommended to use this section as a work of reference when 	we later discuss simulating the biological neuron.
																																				%going through the remainder of this text.
	



	

%Using Golgi's technique, Ramon Y Cajal, Santiago stained nervous tissue in such a way that individual neurons could be separated and 

	
	\subsection{The Neuron}
		%TODO TODO TODO TODO Begynn med å beskrive korleis neuronet er (tenk på den figuren eg hadde i FDP). Legg ved figuren.
		% TODO     Etterpå, fortsett som under..
	
		Each neuron is surrouded by a cell membrane with a low permeability to ions, enabling the intracellular fluid to have a different consectration of different ions than the fluid outside the membrane.
		%under: ikkje "spesialiserte", men "dedidkerte"
		All neuron membranes has ionic pumps dedicated to uphold an ionic concentration gradient over the membrane.
		The different ionic pumps push the corresponding ions ``upstream'' in relation to the ionic concentration gradient, resulting in an electrochemical potential over the membrane.
		If specialized ionic gates permeable to certain ions are opened, these ions can flow freely through the gate. % and the neuron membrane is either depolarized of further polarized, depending on which ions are let thought.
		Depending on which ions are let thought, this cause the neuron membrane to either be depolarized of further polarized with respect to the ionic concentration.
		%This cause the neuron membrane to either be depolarized or further polarized with respect to the ionic concentration.
		If the membrane potential becomes more positive than what is referred to as the firing threshold of the neuron, the neuron fires an action potential.
%TODO CITE: Bear kap 3 og 4. F.eks.
%TODO 		Eller: Kandell kap 7,9 (?)
	%FEILAKTIG: Det varierer! (resting pot: mellom neuron, og firing th.: var. i tid)	The membrane potential al rest typically lies around $-65mV$, and the firing threshold is at a more positive membrane potential.

		When a neuron
		%When the membrane potential is above a threshold, an ``action potential'' is fired, and the neuron sends output through all its output synapses.
	



		The neuron is a cell with a special property; It can
			The electrochemical properties of the neurons enables advanced signal processing
			It has the capability of processing signals due to its
			It have electrochemical mechanisms for processing 

		% Membranen - eksiterbar 
		% 
		% 
		% 


	To understand the signalling property of the neuron, 

	\subsection{The Axon and the Action Potential}

	\subsection{The Synapse}

	\subsection{Signal Propagation}
		- oppsymmer AP, spatiotemporal effekt av axon, synaptic transmission, EPSP/IPSP, ...


	
%\section{Artificial Neural Networks}

% SKRIV OM! FOkuser på at det er noken ting som ikkje er så bra å gjøre i PC.
% 	Så skriv litt om pragmatic ANN (som løysing på dette)

%	\subsection{A Review of ANN History}



\section{Artificial Neural Systems : A Review of ANN History}
	\label{ssecHistoryOfANN}
	The pragmatic use neural network simulations started with the ``McCulloch--Pitts neuron'' in 1943.
	%Warren McCulloch, an early neuroscientist and the young mathematician Walter Pitts formalized the models of the neuron and proposed the first artificial neuron simulator. %artificial neural network.
	Warren McCulloch, an early neuroscientist and the young mathematician Walter Pitts initiated a formalized discussion about the mechanics of the neuron and the use of this in technology. %artificial neural network.
	This resulted in the first neuron emulator(artificial neuron). %, later referred to as the McCulloch--Pitts neuron. 
	%When a network of nodes consisting of the artificial neural was set up, McCulloch and Pitts created the very first artificial neural network(ANN).\cite{MccullochPittsHistorie} %TODO Sjekk referansen! TODO
%
	%TODO Glatt ut: Gjør slik at det eg god flyt i teksten i det som står under her! TODO
	Artificial Neural Networks based on the McCulloch--Pitts neuron model has later been referred to as the first generation ANN\cite{Maass97networksof}.
	%What has later been referred to as the first generation ANN is based on the McCulloch--Pitts neuron\cite{Maass97networksof}.
	%One example of a first generation ANN is the Rosenblatt's Preceptron\cite{HaykinANNbok}.
	Each node is modelled as a boolean device(with an on--off response), where the node sends output if the immediate input level is large enough.
	The first generation ANN therefore can be said to be a network of simple filters called threshold gates.
	This does not take into consideration the depolarization (state) of each node, and is a tremendous simplification of the biological neuron.
	%The first generation ANN thus does not take into consideration the depolarization of the neuron, and is a great simplification of the biological neuron.
	%%%The node sends output if the immediate input is large enough, and does not take into consideration the depolarization of the neuron.
	One famous example of an ANN classified as a first generation ANN is Rosenblatt's Perceptron\cite{HaykinANNbok}.


	The second generation ANN gives a better simulation of the neuron. % in the frequency domain.
	%A better simulation of the neuron considers the neuron in the frequency domain.
	Each node computes the output level as a floating point number based on the immediate input to the node.
	From sec. \ref{secBiologicalNeuralSystems}, the biological neuron is introduced as a node that sends output when the depolarization goes to suprathreshold levels.
	A continuous propagation of a floating point number can therefore only be said to represent the frequency of such transmissions as a function of present input.
%%%
	The function used for computing the output is referred to as the \emph{activation function} of the node.
	%A common activation function is the continuously differentiable \emph{sigmoid function}, that also limits the maximal output\cite{HaykinANNbok}.
	The activation function is found to give the best results if the function is a continuously differentiable sigmoid function\cite{HaykinANNbok}.
	\begin{equation}
		\sigma(x)=\frac{1}{1+e^{-x}}   %TODO TODO TODO TODO TODO TODO Lag heller en figur for å vise sigmoid function! (Bytt ut ligning med fig!) TODO TODO TODO TODO TODO TODO TODO
	\end{equation}
	Because it is more right to consider the neuron as stateless in the frequency domain, the stateless computation in a second generation ANN is more correct than the stateless computation in the McCulloch--Pitts neuron model.
	As the concept of frequency only makes sense for time intervals of a certain size, precise simulations with small computational time steps does not necessarily give accurate simulation results.
	%As the concept of frequency only makes sense for time intervals of a certain size, a second generation ANN can not be used for accurate simulations with small computational time steps.
	%%%
		%%the second generation ANN only give accurate simulations for computational time intervals where is makes sense to talk about mean frequency.
%%%%%%%%%%
	%It is intuitive that the frequency representation only give good simulation results for time intervals where it makes sense to talk about mean frequency.
	%This model thus only gives accurate simulations for a very coarse temporal resolution(large computational time steps), and does not take into account temporal effects caused by the time of firing. %%
	%%This simulation can therefore only be said to only give an accurate simulation for a very coarse temporal resolution(large computational time steps).
	For more precise simulations of the neuron or simulation where temporal mechanisms in the neuron are important, the frequency representation in second generation ANNs can therefore not be used.  %TODO Finn noe å CITE!
	%For more precise simulations of the neuron or of temporal mechanisms, the frequency representation in second generation ANNs therefore can not give good simulation results. %TODO Finn noe å CITE!
	%%This representation therefore is unsuitable for more precise simulations of neural networks and of mechanisms that depend on temporal elements(like STDP learning rules). 
%TODO TODO TODO TODO Cite :  Finn steder dette står/ting å cite! TODO TODO TODO TODO


\begin{figure}[hbt!p]
	\centering
	\includegraphics[width=0.75\textwidth]{sigmoidCurve}
	\caption{Sigmoid curve $\frac{1}{1+e^{-x}}$ for the domain $x\in [-5,5]$}
	\label{figFigurAvNeuronet}
\end{figure}
	
	A direct simulation of the signal propagation mechanisms of the neuron is often referred to as a ``spiking'' artificial neuron model. %\cite{Maass97networksof}.
	The depolarization of the neuron is simulated by numerical integration of all events that change the neuron's depolarization.
	The most commonly used model for spiking neuron simulations is the Leaky Integrate--and--Fire(LIF) neuron, where the depolarization of a neuron is simulated as a leaky integration of depolarizing input\cite{florian03}.
	Because all aspects that are considered important in signal processing are simulated, this model can be used to test theories about neural signal propagation.
	%TODO LINJA over: Endre litt, og cite. F.eks. Maass97networksof?? (XXX Last leddsetning does not tell what I intended..)

	The LIF model describes the depolarization of a neuron as a leaky integration of the neuron's excitatory and inhibitory input, where the depolarization value diminish(towards the resting membrane potential) over time.
	%The leaky aspect of the neuron can be implemented by subtracting a certain ration of the last computed depolarization, every time iteration.
%%	Artificial Neural Networks that utilize this simulation model for its nodes is sometimes referred to as Spiking ANN(SANN) and belong to the \emph{third generation ANN}. %TODO Cite en art. av Wulfram Gerstner
	%%Artificial Neural Networks that utilize this simulation model for its nodes is sometimes referred to as Spiking ANN(SANN) and belong to what is referred to as the \emph{third generation ANN}. %TODO Cite en art. av Wulfram Gerstner
%%	
	When the simulated depolarization of a node is excited above the firing threshold, a spike is initiates, causing transmission through all the node's output edges. %synapses.
	The signal is propagated as discrete spikes, very similar to the signal processing of a biological neuron\cite{Kunkle02pulsedneural}.
	Artificial Neural Networks with this simulation model for its nodes are sometimes referred to as Spiking ANN(SANN) and belong to the \emph{third generation ANN}\cite{Maass97networksof}. %TODO Cite en art. av Wulfram Gerstner
	%%Artificial Neural Networks that utilize this simulation model for its nodes is sometimes referred to as Spiking ANN(SANN) and belong to what is referred to as the \emph{third generation ANN}. %TODO Cite en art. av Wulfram Gerstner
	

	To summarize this section about ANN history, there are three generations of artificial neural networks, each getting closer to the biological neuron in behaviour.
	%What propagates thought the network, how this is computed and what it represents differs  
	The first generation of artificial neurons where so--called threshold gates, with a boolean output that was [true] if the summed input were above some threshold. %TODO CITE!
	Nodes of the second generation gave, in some respects, a better simulation of the biological neuron.
	The output is not given as discrete states given by the input but as a continuous function that can be interpreted as the firing frequency of the node. % of the level of input to the node.
	With this interpretation it can be said that ANNs of this generation gives a simulation that is closer to the biological neuron in behaviour.
	%If the transmission through the output synapse of a neuron is seen as the firing frequency of that neuron, it can be said that this generation of ANN gives a simulation that is closer to the biological neuron.
	%The third generation ANN is as supposed to be an accurate simulation of the neuron, and is as close to the biological neuron as possible.
	The third generation ANN is as supposed to give an accurate simulation of the neuron, and achieves this by simulation the neuron's depolarization directly. 
	The neuron has an internal state representing the depolarization and fires if this value goes to supra--threshold levels.
	The signal is propagated in the same manner as in the biological neuron, where excitatory synapses increase the postsynaptic depolarization and inhibitory synapses decrease the postsynaptic depolarization.
	%Errors in the simulation comes as a consequence of numerical errors or from the neuron model used.
	Only numerical errors in the digital simulation and errors in the neuron model used separates the simulated result from the behaviour of a real neuron.
	%Only elements like truncation errors in the digital simulation and errors in the neuron model used separates the simulated result from the behaviour of the real neuron.

%%%	XXX TA MED?    Den observante leser vil dermed se at med kvar ny generasjon ANN, så kommer vi nermere bio-neuronet. 




% 2.gen er bedre enn første, fordi begge er "state less". For tidsdomenet blir dette heilt feil. For frekvensdomenet blir det mindre feil. (Heilt rett dersom du ser bort fra syn.p. og modulatory neurotransmitters.

%It is actually so close that the word ``simulation'' will occasionally be used in this report.  	% ".. actually so closa that .." DÅRLIG. Fiks?

%In the third generation ANN it is the action potentials or the "spikes", that is responsible for information processing.  %TODO Skriv om slutten / Feil ord.. 		".. or the "spikes" that is responsible for the information flow.
%This ANN model is therefore often referred to as ``Spiking Artificial Neural Network''(SANN).

%If the transmissions between nodes is viewed as the firing frequency of the neuron, we can say that the continuous output value represents the output frequency as a function of the input frequency over the time step in the simulation.

%The nodes of the third generation ANN became even more similar to the biological neuron, as the output of a node depend solely on the state of the node.






%	\subsection{Synaptic Plasticity and motivation for SANN}
%		- skiv om Hebbian learning: ustabilt. \\
%		- skiv om STDP og at dette er en viktig grunn til å bruke SANN. \\
%		Det er truleg at begge 'learning rules' har sannhet. Det ville difor vært bra å kunne benytte begge, ivertfall i forskningssamanheng.

\subsection{Depolarization Simulation by Numerical Integration}
	The depolarization of a node is a time integral of all depolarizing input and the total ``leakage'' of depolarization value, and can be implemented by numerical integration.
	The effect of leakage can be implemented by subtracting a scaled version of the neuron's depolarization value every time iteration.
	%The last computed value, that is used for computing the size of the leakage is delayed by the size of the computational time step before the leakage can be computed.
	The size of the leakage is computed by the last computed value for the neuron's depolarization, delayed by the size of the computational time step.
%	This cause an erroneous value for the leakage, and defines one aspect of the local truncation error.
	%This value is delayed up The last computed value, that is used as the present value of the neuron is delayed by the size of the computational time step before the leakage can be computed.
%%
	%The error induced by doing it this way in discrete--time simulations is discussed in section \ref{TODO TODO TODO XXX}. %TODO TODO TODO Referer til denne plassen!
%%%%
	The error from each iteration, referred to as the local truncation error, thus increases with the size of the computational time step.
	%It is possible to get accurate simulation results by decreasing the size of the computational time step.
	Accurate simulations can therefore be designed by decreasing the size of the computational time step.

	If all nodes are updated every iteration, the computational load scales linearly with the number of nodes and the inverse of the size of the computational time step.
	By halving the size of the computational time step, the computational load therefore increase as if the number of nodes are doubled.
	This explains that the accuracy of the simulator can be used as a good measure of efficiency, and establish the motivation for having precise simulation algorithms.
	More sophisticated numerical integration techniques are often used to accomplish this.





% 	A leaky integrator can be implemented by integrating all depolarizing input, and subtracting each iteration's leakage.
% 	Excitatory input is cause an increase in the postsynaptic neuron's depolarization and inhibitory input cause a decrease in the postsynaptic node's value.
% 	The leakage is computed by the current depolarization level of the neuron, scaled by a leakage constant.
% 
% 
 

	The corresponding electrical circuit to the LIF neuron model consists of a capacitor $C$ in parallel with a resistor $R$. %, driven by a current $I(t)$.
	Depolarizing input to the neuron, either in form of externally applied current or in the form of excitatory synaptic input is modelled as the $I(t)$. %    cause the membrane potential to increase(the capacitor is charged).
	A leakage current $I_l(t) = -\frac{v(t)}{R}$ cause the depolarization value of the neuron to decrease, and can be modelled as a current through the circuit's resistor. %as a function of the node's present value.
	The equivalent current in the RC circuit is then given by the equation
	\begin{equation}
		I_{tot}(t) = I(t) + I_l(t) = I(t) - \frac{v(t)}{R}
	\end{equation}
	
	When implemented in a discrete--time simulator by numerical integration, discrete time cause a delay of one time step for $v(t)$.
	%The error caused by this every time step is referred to as the local truncation error
	As the value is integrated, the local truncation errors caused by this accumulate, giving an increasingly erroneous depolarization value for the neuron. 
	This cause the simulated neuron to fire at the wrong time.
	An erroneous inter--spike internal cause an error for the neuron's firing frequency with a size defined by the temporal resolution (number of time steps) of the simulation. %, and ??? XXX
	%If the local truncation error is systematic, the erroneous inter--spike interval also produce an error for the neuron's firing frequency.
	%%	This can be implemented in a discrete--time simulator as a discrete integration of all input ($I(t)$) minus the present leakage current $I_l(t) = \frac{v(t)}{R}$.	








		
% // vim:fdm=marker:fmr=//{,//}

% 	\input{NIM-simulations}

%
\section{Simulating the Neuron's Depolarization}

	\subsection{Simulation by Numerical Integration of Input} %\section{Numerical Integration Method}
	\subsection{Simulation utilizing the algebraic equation}



\appendix
	
\chapter{Synaptic Transmission and Plasticity in the Biological Neuron} %xxx Eller biologisk neurale sys. (er vel strengt tatt ikkje i nevronet, ELLER er det det?)
\label{appendixSynPlast}

%fra FDP. Direkte kopi. 15.02.2012

%fra NEVR3001 rapport om synPlast : (Kraftig omskrevet)

%\section{Introduction}

The background of memory and learning is an important topic in the field of neuroscience. 
It has recieved much attention lately, and the knowledge about synaptic plasticity and synaptic transmission has progressed significantly the last decades.
In this appendix the different mechanisms behind synaptic transmission and plasticity will be in focus.
We will see that the the ion $Ca^{2+}$ has a crusial role in both presynaptic transmission mechanisms and in postsynaptic mechanisms involved in synaptic plasticity.

% todo Ta vekk (Skreiv det bare inn for å forklare en kybb-leser. Trur syn.p. er kjendt etter rapporten min).
%First some expressions will have to be defined. Synaptic plasticity referst to change in the size of each of the synapse's transmissions.
%Synaptic plasticity may be short--term, lasting from milliseconds to minites %TODO finn ut kva noken seier, og referer dette (Bear, purves, kandel)
% 	, or long--term, giving us ``memories'' that might last from hours to a lifetime.

Synaptic transmission in chemical synapses is dependent on a multitude of different mechanisms situated in the presynaptic axon terminal, the postsynaptic membrane or in glial cells surrounding the synapse.
%Synaptic transmission in chemical synapses is based on many mechanisms both in the presynaptic axon terminal, the postsynaptic neuron, and in glial cell surrounding the synapse.
Some of these mechanisms involved in synaptic transmission will be discussed in this appendix.
We will also look at some elements involved in synaptic plasticity, or change in the size of the synaptic transmission. %TODO Skriv kvifor? FÅ RELEVANS!
% TODO Skriv noke slikt som: Because the ultimate goal of this appendix is to discuss the reason behind going on from
% XXX 	og end opp på at eg må beskrive syn.p. også.
% xxx  	Kan kanskje bare skrive at desse er tett sammenknytta. Nei, eg må forklare kvifor eg vil beskrive STDP.
%In this essay, some of the mechanisms involved in synaptic transmission will be described. 
%This is nessecary in order to say something of what changes the same mechanisms.


%Innleder synaptic transmission:
When the neuron is sufficiently depolarized, an action potential is initiated at the axon hillock.
The action potential will propagate through the axon to the ``axon terminal'' where the presynaptic part of the synapse is found.
When we get a strong depolarization over the presynaptic membrane at the synapse, a cascade that ends up with a change in the postsynaptic neurons value is initiated.
This is what is referred to as synaptic transmission\ref{PurvesNeuroscienceKAP05}.

The size of the synaptic transmission is dependent on the amound of neurotransmitters released from the presynaptic part of the synapse and the number of neurotransmitter resceptors in the postsynaptic membrane
		\ref{PurvesNeuroscienceKAP05}. 
Synaptic plasticity, what is percieved as the basis of learning, can happen on a short--term of long--term timescale.
Short--term synaptic plasticity normally only involves factors that can be seen as ``the state'' of the synapse. 
Factors as neuromodulators, amount of $Ca^{2+}$ in-- or outside the neuron and amount of neurotransmitters available are examples of such factors.
Long--term synaptic plasticity involves ``lasting changes'' in the synapse. One example is such protein synthesis is generation of new postsynaptic receptors\ref{PurvesNeuroscienceKAP8}. %todo har ikkje sett gjennom kap.8 når eg ref.
																										% Veit innholdet, men det er kanskje lurt å sjekke at alle desse elementa stå i purves kap 8.

\section{The Presynaptic Part of Synaptic Transmission}
\label{appendixSecPresynapticSynapticPartOfTransmission}
The propagation of the action potential along the axon, happens as a combination of passive and active transmission.
The electrical potential is transmitted passively along the axon. On the axon we have voltage--gated $Na^+$ and $K^+$ channels.
These will open when the electrical membrane potential supasses a threshold, and will further increase the value of the potential. 
After a while the channels will close in a sequence that resets the potential to the base potential (with a small overshoot) \cite{PrinciplesOfNeuralScience4edKAP09}. 
%and will not activate reamplification before it encounters special voltage--gated $Na^+$ and $K^+$ channels.
%The axon is insulated by a special glia cell, to increase the speed of action potential propagation.
%These are situated in gaps in the insulation, or glia
This will reamplify the signal  so that is may continue passively down the axon to the next voltage gated channels.
%In an action potential, electrical potential is first transmitted passively down the axon of a neuron. On the axon we have voltage--gated $Na^+$ and $K^+$ channels 
%	that open when the electrical potential over the membrane surpasses a threshold\cite{PrinciplesOfNeuralScience4edKAP09}. 
%This will enhance the signal so that it can continue passively to the next voltage gated channels.

In the nervous system we have specialized insulative cells, called myelin.
In myelinated neurons, the voltage--gated channels are located in gaps in the myelin, These gaps are called ``nodes of Ranvier''.
%skriv også om situasjonen for umyeliniserte axon? Nei.
The voltage--gated channels will restore the action potential, and constitutes the active element of signal tranmission along the axon \cite{PrinciplesOfNeuralScience4edKAP09}.
%In myelinated neurons these voltage--gated channels are located in gaps in the myelin, called ``nodes of Ranvier'', in unmyelinated neurons the channels are located continously along the axon membrane. The channels will restore the signal, and constitutes the active part of current transduction along the axon\cite{PrinciplesOfNeuralScience4edKAP09}.


%*********************** OK så langt. ***********************

When the action potential reaches the axon terminal, the end of the axon down the signal path, it will open voltage gated $Ca^{2+}$ channels at the active zones of the terminal. 
This causes $Ca^{2+}$ to enter the cytosol of the axon terminal of the presynaptic neuron\cite{PrinciplesOfNeuralScience4edKAP10}.

$Ca^{2+}$ causes synaptic vesicles to fuse with the membrane and release the contained neurotransmitters into the synaptic cleft.%\cite{PrinciplesOfNeuralScience4edKAP10}. 
There is a linear relationship between the amount of $Ca^{2+}$ entering the cytosol and the amount of synaptic vesicles fusing with the membrane (exocytosis).%\cite{PrinciplesOfNeuralScience4edKAP10}. 
Ecocytosis of a synaptic vesicle will release the neurotransmitters stored in it into the synaptic cleft \cite{PrinciplesOfNeuralScience4edKAP10}. 
The amount of neurotransmitters released into the synaptic cleft therefore have a linear relationship with the amount of $Ca^{2+}$ entering the presynaptic axon terminal.
% TODO Ta med, eller ta vekk?      , given a constant amount of neurotransmitters stored in each synaptic vesicle.
%Exocytosis releases the content of the synaptic vesicle to the synaptic cleft\cite{PrinciplesOfNeuralScience4edKAP10}.

%XXX Viktig, men utafor scope av teksten:
%%%%%%The linear relation between the amount of $Ca^{2+}$ entering the axon terminal and the amount of synaptic vesicles undergoing excytosis was first proposed by Katz and Miledi, and later shown by Rodolfo Llinàs and colleges\cite{PrinciplesOfNeuralScience4edKAP14}. 
%KANSKJE:
% TODO Vettafaen om det skal være med:
% Dette gir at vi kan ha presynaptisk short--term syn.p., noke som er viktig argument for å innføre axo-axonic synapses (temporal synaptic modulatory system)
%Rodolfo Llinàs and colleges first showed the mechanisms of a linear relationship between the amount of $Ca^{2+}$ entering the cytosol of the presynaptic cytosol.
%They also found that the $Ca^{2+}$ channels are graded by the potential over the axon terminal membrane. 
%This further gives a graded responce of neurotransmitter release based on the preysnaptic membrane potential \cite{PrinciplesOfNeuralScience4edKAP14}.
%xxx kan ikkje ta vekk, lett. Bruker dette resultatet seinere.. ELLER?

% TODO Flytt alle \cite{} til slutten av avsnittet (dersom de påstandene på slutten også står her..)
% Har sjekka: alle påstandene er fra kap14 i Kandel.
There is a steady influx of $Ca^{2+}$ at axon terminals, through the L-type $Ca^{2+}$ channel. %\cite{PrinciplesOfNeuralScience4edKAP14}. 
This influx of calcium is graded by the potential over the presynaptic membrane.
When multiple synaptic transmissions happens within a short period of time, 
	the amount of $Ca^{2+}$ in the presynaptic axon terminal builds up and the following synaptic transmissions will give a successively larger effect on the postsynaptic potential.
This effect is called \emph{potentiation}, and can last from minites to more than an hour.
% Føler at neste linja ikkje heilt passer inn XXX:
% TODO Vær sikker på at axo--axonic synases er definert!
This mechanism also gives the effect of axo-axonic synapses in regulating the amount of neurotransmitter release for the next transmissions\cite{PrinciplesOfNeuralScience4edKAP14}.
%The axo-axonic synapses will not influence the firing of a neuron, only the membrane potential of the an axon terminal. %XXX KVA ER axo--axonic synapses? Inled dette for leser!
% % and thus the amount of neurotransmittors released by the following action potential\cite{PrinciplesOfNeuralScience4edKAP12}.
%This gives a mechanism for controling the postsynaptic exitatory postsynaptic potential between other neurons following an action potential.

%TODO TA vekk?
%When two action potentials reaches the axon terminal in fast succession it will cause the synapse to be stronger (give a larger postsynaptic response) for many minutes. 
%This is called \emph{potentiation}, and is thought to be partially because of the increase in presynaptic cytosol $Ca^{2+}$ levels\cite{PrinciplesOfNeuralScience4edKAP14}. In the mossy fiber pathway of the hippocampus, presynaptic $Ca^{2+}$ influx is an important mechanism for synaptic plasticity \cite{PrinciplesOfNeuralScience4edKAP63}. %XXX Sjekk! (mest viktige, eller bare viktig?)

%XXX TA VEKK? 
%A decrease in the number of synaptic vesicles undergoing exocytocis has been observed in sensory neurons of the \emph{Aplysia Californica} following LTD. %, by quantal analysis 
%The mechanisms for decrease in synaptic vesicle exocytosis is not known\cite{PrinciplesOfNeuralScience4edKAP63}.

% dette er kanskje interresant: Det kan også være en basis for STDP.. 
% Sjå om det skal takast vekk, dagen før innlevering.
An increase in the extracellular level of glutamate has been observed after LTP in CA3 neurons. 
The mechanisms behind this is debated, but evidense has been presented of \emph{retrograde messangers} from the postsynaptic neuron that will 
	give feedback to the presynaptic neuron after transmission \cite{PrinciplesOfNeuralScience4edKAP63}. 
This enables a  presynaptic component of \emph{long--term} synaptic plasticity.








\section{Postsynaptic Mechnisms of Synaptic Plasticity}
\label{appendixSynPlast:postsynapticMechanisms}
%There are tree groups of receptors in the postsynaptic membrane of a synapse. AMPA, .......NMDA, kainate, ....
Because neuroscience mainly have focused on excitatory glutamate synapses, the discussion about postsynaptic mechnisms behind synaptic plasticity will focus on glutamate transmission.

There are two groups of glutamate receptors: NMDA and non-NMDA receptors. 
The non-NMDA receptors consists of the AMPA and the kainate receptors. 
Most non-NMDA receptors are only permeable to $K^+$ and $Na^+$, while the NMDA receptor is permeaple to $Ca^{2+}$ in addition to $K^+$ and ${Na}^+$ \cite{PrinciplesOfNeuralScience4edKAP12}. 


The NMDA--receptor is an ion channel that is both voltage gated and ligand gated: 
	It requires both that the glutamate neurotransmitter is present in the extracellular fluid and a strong depolarization over the membrane to open \cite{PrinciplesOfNeuralScience4edKAP12}. 
When we get a transmission when the postsynaptic neuron is strongly depolarized, we therefore get an influx of calcium at the postsynaptic neuron.
$Ca^{2+}$ will activate calcium dependent enzymes and also protein kinases that leads to long--term synaptic plasticity\cite{PrinciplesOfNeuralScience4edKAP12}.
This is done as a result of the calcium dependent enzymes initiating synthesis of new AMPA receptors \cite{AMPARtrafficingArtikkel}. 
More receptors causes a larger probability of the glutamate neurotransmittor having an effect, and thus increases the effectivity of the synapse (the synaptic weight).

To conclude this section we will compare the statistical relationship between the postsynaptic neuron having a large depolarization at the time of transmission and the relative timing of the transmission, 
	in relation to the postsynaptic action potential. 
If the postsynaptic neuron is strongly depolarized at the time of transmission, this implies that the postsynaptic neuron will fire soon after.
This might be one of the basis of what has been known by the name Spike Time Dependent Plasticity (STDP).
% OMGJODT TIL HIT:  XXX XXX XXX XXX XXX XXX XXX XXX XXX XXX XXX XXX XXX XXX XXX XXX XXX XXX XXX XXX XXX XXX XXX XXX XXX XXX XXX XXX XXX XXX XXX XXX XXX 
% TODO KAnskje ta vekk resten (med unntak av Summary?)

%XXX XXX XXX XXX 
%If two transmissions happens in rapid succession, you will get a strong (lokal) depolarization around the postsynaptic receptors. This will cause the NMDA--channels to open at the second transmission, and admit $Ca^{2+}$ into the postsynaptic neuron. 
%TODO Skriv heller om at depolarisasjonen har mykje å seie. NEI, dette står allerede. Skriv korleis tid kan ha noke å seie (begrunn STDP med bakgrunn i teoien her). XXX Gjør eit poeng ut av kvifor eg har tatt med dette i appendixet.
% 			Dette kan helst gjøres etter neste setning.

%Also in the postsynaptic neuron, calcium has an important role in synaptic plasticity. 
%$Ca^{2+}$ will activate calcium dependent enzymes and also protein kinases that leads to long--term synaptic plasticity\cite{PrinciplesOfNeuralScience4edKAP12}.
%%Second messangers can also be activated by metabotropic receptors in addition to $Ca^{2+}$ and the same protein kinases are activated. 

% IKKJE RELEVANT:
%The calcium is thought to be important in both short-term potentiation by enhancing the response of AMPA receptors to glutamate\cite{PrinciplesOfNeuralScience4edKAP63}, 
% 	and also elicit ``permanent'' synaptic changes by receptor synthesis\cite{AMPARtrafficingArtikkel}. %Dette ER relevant, men skrevet over.
%This is thought to enhance the response of AMPA receptors to glutamate\cite{PrinciplesOfNeuralScience4edKAP63}, but also elicit longer lasting (``permanenet'') synaptic plasticity.

\section{Receptor Synthesis}
The rise in calsium levels in the postsynaptic cytosol activates postsynaptic plasticity\cite{AMPARtrafficingArtikkel}. 
One of the possible mechanisms behind postsynaptic LTP or LTD is the increase or decrease in postsynaptic receptors. There has been increased focus on receptor trafficing in the recent years, especially on the AMPA receptor. 

\begin{quote}
At early stages of development, synapses containing only NMDA type receptors are particularly common\cite{PrinciplesOfNeuralScience4edKAP12}.
\end{quote}
Synapses containing only the NMDA receptor is called ``silent synapses'' because they do not change the postsynaptic potential (PSP) unless the postsynaptic membrane is sufficiently depolarized. 
This makes them silent at normal resting membrane potential\cite{AMPARtrafficingArtikkel}. 
It has been observed that these ``silent synapses'' is converted into normal exitatory synapses by the insertion AMPA receptors into the postsynaptic membrane\cite{AMPARtrafficingArtikkel}. 

It has been shown that when synapses undergo LTD, the amount of AMPA receptors in the postsynaptic membrane decreases\cite{AMPARtrafficingArtikkel}. 
This is believed to be because of endocytosis of the receptors. If the dynamin-dependent endocytosis is blocked, LTD is also blocked in the samle\cite{AMPARtrafficingArtikkel}.

\section{Glial Modulation of Synaptic Transmission}
One way for asterocytes to modulate synaptic transmission is to release ATP, which is converted to adenosine extracellularly. 
Adenosine inhibits the $Ca^{2+}$ channels in the presynaptic axon terminal membrane\cite{signallingBetweenGlialAndNeuronsInSynPlast}. 
This results in less exocytosis of synaptic vesicles in the presynaptic membrane, which gives less neurotransmitters in the synaptic cleft as a consequence\cite{signallingBetweenGlialAndNeuronsInSynPlast}.
%This also affects neighboring synapses\cite{signallingBetweenGlialAndNeuronsInSynPlast}. %men det er uklart om dette er pga slett andre plasser også, eller diffusjon.

For the NMDA receptor channels to open, three conditions has to be met:
\begin{enumerate}
	\item Glutamate needs to be present in the synaptic cleft.
	\item The postsynaptic membrane needs to be sufficiently depolarized.
	\item D-serine needs to be present in the synaptic cleft\cite{signallingBetweenGlialAndNeuronsInSynPlast}.
\end{enumerate}
The point about D-serine is interresting, since D-serine is absent in neurons. It is present in asterocytes.
One possible explanation it therefore that asterocytes release the D-serine required for the NMDA-R to open\cite{signallingBetweenGlialAndNeuronsInSynPlast}.  % D-Serine bindes til glycine-binding site.
This indicates that the asterocytes are important in modulating the synaptic plasticity induced by NMDA-R opening.

Glial cells are also important for synaptic transmission by being permeable to $K^+$ from the extracellular fluid of the synaptic cleft\cite{PrinciplesOfNeuralScience4edKAP07}, 
and by being in control of the reuptake of certain neurotransmittors (eg. glutamate)\cite{PrinciplesOfNeuralScience4edKAP15}. %kap 15 kandell, 
%This gives possible astrocyte mechanisms for modulating synaptic transmission.


% Ta vekk FRA HER TODO TODO 
\section{Synaptic Transmission an Plasticity Summary}
%TODO Skriv kvifor eg har skevet alt dette. Få relevans. (STDP, som er viktig argument for SANN)
The subject about synaptic plasticity is important for the understanding of neural systems.
We have presynaptic and postsynaptic elements of synaptic transmission, both subject to continous change. This gives two possible elements of synaptic plasticity. 

The presynaptic part of synaptic transmission can be regulated by changing the presynaptic voltage gated $Ca^{2+}$ channels. One way this is done is by axo-axonic synapses that depolarises the axon terminal before the action potential, and thus enhance/inhibit or prolong/shorten the influx of calcium. 
This results in a change in the amount of neurotransmittors released into the synaptic cleft.

The postsynaptic part of synaptic plasticity consists of short term changes, by changing the effect of AMPA-R with calcium, or long lasting changes involving protein synthesis and the insertion of new AMPA-R in the postsynaptic membrane. Both are dependent on calcium. Changing the postsynaptic influx of calcium is therefore an other plausible mechanism for synaptic plasticity.

The asterocytes maintains the environment for the synaptic transmission by maintaining the ion consentrations in the extracellular fluid in the synaptic cleft. 
Modulation of this will change the environment for synaptic transmission and be a way of changing the effect of synaptic transmission. 

The asterocytes are also responsible for removing some neurotransmittors from the synaptic cleft. 
This makes them in control of the time the neurotransmittor is in the synaptic cleft, and thereby the time it will be effective on the postsynaptic receptors.
This desides the postsynaptic effect of the transmission. Change of this is yet an other mechanism for synaptic plasticity.
%This opens for yet an other mechanism for the asterocytes to control the postsynaptic response of a transmission.


%\begin{figure}[!htbp]
%	\centering
%	\includegraphics[width=0.8\textwidth]{figurSTDP.jpeg}
%	\caption{Spike timing-dependent plasticity. a, Synapses are potentiated if the synaptic event precedes the postsynaptic spike. Synapses are depressed if the synaptic event follows the postsynaptic spike. b, The time window for synaptic modification. The relative amount of synaptic change is plotted versus the time difference between synaptic event and the postsynaptic spike. The amount of change falls off exponentially as the time difference increases. In addition, the amount of potentiation decreases for stronger synapses, whereas the relative amount of depression is independent of synaptic size.}
%\end{figure}




\bibliography{bibliografi}
%\bibliographystyle{abbrvnat}
\bibliographystyle{plain}
\end{document}

% // vim:fdm=marker:fmr=//{,//}
