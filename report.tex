

% Trenger "Table of abbrevations"
% 	- LTP, LTD, STDP
% 	- "spike" ? (at det betyr Action potential)?
% 	- SANN  - spiking ANN
% 	- SN 	- Spiking Node, a node in SANN.
% 	- KANN, KN.
% 	- LIF neuron: 	Leaky--Integrate and Fire  neuron




% Eller kanskje eg treng stikkordsregister: Trenger:
% 	- Det over og:
% 	- axo-axonic synapses
%
% 	- KN, KANN node : 	node based on a the consept of the $\kappa$ model of the neuron. Nodes of KANN
% 	- SN, SANN node : 	node based on a direct simulation of a spikin neuoron. Nodes of SANN
% 	- suptrathreshold levels : 	value higher than the firing threshold of a node.
% 	- STDP
% 	- LIF neuron
% 	- synaptic weight.
% 	- AuroSim 		:  	The implementation written in this project.
%
%


\documentclass[b5paper,11 pt]{report}
% TODO Er dette rett? Sjekk dette..


\usepackage[utf8]{inputenc}
\usepackage{graphicx} 

% sudo apt-get install texlive
%\usepackage[font=small,format=plain,labelfont=bf,up,textfont=it,up]{caption}
\usepackage[font=footnotesize,format=plain,labelfont=bf,up,textfont=it,up]{caption}


\usepackage{amsmath}
\usepackage{mathrsfs} %brukt for krølle-L : laplace

\usepackage{subfig}   % for subfigures.
\usepackage{listings} %for c++ kode
\lstset{language=c++}


\usepackage{amsthm}
\newtheorem{mydef}{Definition}


%\usepackage{pstricks} % For teikning. 		FOR mykje styr å lære..
%\usepackage{pstricks-add}
 %\let\psgrid\relax %XXX Fjærner GRID på alle pstricks teikninger. (global removal)

%\usepackage{epstopdf} % for å få med eps i pdf-latex (?)    funkerIkkje.. Gjør det manuellt for filene..



%Kanskje dette er naudsynt? For metapost:
%\DeclareGraphicsRule{*}{mps}{*}{}









%%%%%%%% Fra Kristoffer sin master:
\lstset{ %
basicstyle=\footnotesize,       % the size of the fonts that are used for the code
%numbers=left,                   % where to put the line-numbers
numberstyle=\footnotesize,      % the size of the fonts that are used for the line-numbers
%stepnumber=1,                   % the step between two line-numbers. If it's 1 each line will be numbered
%numbersep=5pt,                  % how far the line-numbers are from the code
showspaces=false,               % show spaces adding particular underscores
showstringspaces=false,         % underline spaces within strings
showtabs=false,                 % show tabs within strings adding particular underscores
%captionpos=b,                   % sets the caption-position to bottom
%breakatwhitespace=false,        % sets if automatic breaks should only happen at whitespace
%escapeinside={\%*}{*)}          % if you want to add a comment within your code
%
frame=single,                % adds a frame around the code
tabsize=4,                % sets default tabsize to 2 spaces
breaklines=true,                % sets automatic line breaking
}


\author{Per R. Leikanger}
%\title{Development and Assessment of a Novel Model for Artificial Neural Networks}
\title{Development and Assessment of a Novel Model for Neural Simulation}
\date{\today}     



\begin{document}   



% TODO Neste linje skal kanskje være over abstract og maketitle? TODO
\pagestyle{empty} %get rid of header/footer for toc page

\maketitle

%\begin{abstract}
%abstract: Her.
%\end{abstract}






\pagestyle{empty} 

\tableofcontents %put toc in
\cleardoublepage %start new page %TODO Brukes dersom [report] brukes for dokumentet.
%\clearpage

\setcounter{page}{1} %reset the page counter
\pagestyle{plain} % put headers/footers back on















\section{Introduction}

	\subsection{Dette skal heuilt sist, f.eks. i diskurs!}
	Skriv en-eller-annen--plass at når man gjekk fra 1. og 2. generasjon ANN til 3. generasjon ANN, det er mulig å sei at man begynte å "consider a Moore automata of the neuron".
	(Dette er en litt løs tolking, siden ouput er gitt som discrete pulser. Eg trur likevel at den er gyldig..)
	(Her er det bare state som gir ouput).
	
	I min nye modell, tar eg dette videre, og innfører at output også er gitt av present input til neuronet. Dette gir oss en Mealy automata av neuronet.
	Skriv litt om Moore vs. Mealy automata!


% // vim:fdm=marker:fmr=//{,//}


\chapter{Background Theory}
	Before we looking av Artificial Neural Networks(ANN), different aspects of ANN history and biology have to be introduced.
	This chapter is intended to introduce the reader to the necessary background information to make the theory in this work comprehendable.

	First, theory about the simulated system is ... OSV (sjå FDP)...

	Introduksjon til kapittel: motiver leser til å få bakgrunnsinfo!
	-Neurale systemet er en shitbra signal-prosesseringsenhet som har en del egenskaper som digitale signal-processore ikkje har- List opp..
	
% DISPOSISJON
% 	Intro: Begynn beldig vidt: "neuron" er en samlebetegnelse for en spesiell type celler med signalling properties (kan bli eksitert). [REF].
%  		IDE: -> Skriv litt om at den vide grupperingen gjør at simulering er vanskelig (men ikkje her: MEN i ANN-seciton!)
%  		- Dette ligger i komplekse neurale nettvert med "recurrent connections" og høg grad av 'connectivity'.(I CNS er det estimert til å være 1000+ input per neuron)
% 		- Koblingen mellom 
% 




% "Although the human brain contains an extraordinary number of these cells (in the order 10E11 neurons), which can be classified into at least a thousand different types, all nerve cells share the same architecture." Kandell kap 2.



\section{Biological Neural Systems}
	
	In the late 1800s, Camillo Golgi developed a way of staining nervous tissue so that complex networks of nerve cells (neurons) became apperent.
	Ramon Y Cajal, Santiago used Golgi's technique in such a way that individual neurons could be separated, and it was observed that nervous tissue was not a continuous web but a network of discrete cells. 
	He proposed what has later been known as the neuron doctrine; That the basis of intelligence is individual ``brain cells'' that can process incoming transmissions and send ouput transmissions in certain situations.
	For their contribution, Ramon Y Cajal and Golgi shared the 1906 Nobel's price in Physiology and Medicine. %CITE: Kandell kap 2. 

	Modern neuroscience follows the neuron doctrine. 
	Each node in a neural network is called a neuron and the connection between neurons are called synapses.
	All synapses have direction(propagates information in one direction) and transmitts if the presynaptic neuron ``fires'' an action potential.
	Synapses can be exitatory or inhibitory.
	Transmissions in excitatory synapses increase the postsynaptic membrane potential, causing that neuron to approach firing. %sending an action potential.
	Inhibitory transmissions does the opposite, and inhibits the postsynaptic neuron with respect to firing.
	When a neuron fires an action potential, a transmission is initialized for all the neuron's output synapses.  % redundant? eller viktig å poengtere?
% TODO TODO TODO CITE!
	
	In this section, the most important elements of neural signal processing are presented, enabling the reader to become more familiar with how neural networks process information.
	%It is recommended that the reader utilize this section as a reference work when neural simulation is discussed.
	It is recommended that the reader utilize this section as a reference work when different methods of neural simulation is presented.
	Before discussing how the neuron processes information, the organization of the neuron have to be reviewed. %REVIEWED er dårlig TODO
	%We will start with the organization of the neuron, before moving on to a fundamental signal processing mechanism of the neuron; The electrochemical properties of the neuron membrane.




%	The most important elements of signal processing in neural systems is presented in this section,
%		enabling the reader to become familiar with how biological neural networks function.
%	Fundamental elements for the neural signal processing is presented, and it is recommended to use this section as a work of reference when 	we later discuss simulating the biological neuron.
																																				%going through the remainder of this text.
	



	

%Using Golgi's technique, Ramon Y Cajal, Santiago stained nervous tissue in such a way that individual neurons could be separated and 

	
	\subsection{The Neuron}
		%TODO TODO TODO TODO Begynn med å beskrive korleis neuronet er (tenk på den figuren eg hadde i FDP). Legg ved figuren.
		% TODO     Etterpå, fortsett som under..
	
		Each neuron is surrouded by a cell membrane with a low permeability to ions, enabling the intracellular fluid to have a different consectration of different ions than the fluid outside the membrane.
		%under: ikkje "spesialiserte", men "dedidkerte"
		All neuron membranes has ionic pumps dedicated to uphold an ionic concentration gradient over the membrane.
		The different ionic pumps push the corresponding ions ``upstream'' in relation to the ionic concentration gradient, resulting in an electrochemical potential over the membrane.
		If specialized ionic gates permeable to certain ions are opened, these ions can flow freely through the gate. % and the neuron membrane is either depolarized of further polarized, depending on which ions are let thought.
		Depending on which ions are let thought, this cause the neuron membrane to either be depolarized of further polarized with respect to the ionic concentration.
		%This cause the neuron membrane to either be depolarized or further polarized with respect to the ionic concentration.
		If the membrane potential becomes more positive than what is referred to as the firing threshold of the neuron, the neuron fires an action potential.
%TODO CITE: Bear kap 3 og 4. F.eks.
%TODO 		Eller: Kandell kap 7,9 (?)
	%FEILAKTIG: Det varierer! (resting pot: mellom neuron, og firing th.: var. i tid)	The membrane potential al rest typically lies around $-65mV$, and the firing threshold is at a more positive membrane potential.

		When a neuron
		%When the membrane potential is above a threshold, an ``action potential'' is fired, and the neuron sends output through all its output synapses.
	



		The neuron is a cell with a special property; It can
			The electrochemical properties of the neurons enables advanced signal processing
			It has the capability of processing signals due to its
			It have electrochemical mechanisms for processing 

		% Membranen - eksiterbar 
		% 
		% 
		% 


	To understand the signalling property of the neuron, 

	\subsection{The Axon and the Action Potential}

	\subsection{The Synapse}

	\subsection{Signal Propagation}
		- oppsymmer AP, spatiotemporal effekt av axon, synaptic transmission, EPSP/IPSP, ...


	
%\section{Artificial Neural Networks}

% SKRIV OM! FOkuser på at det er noken ting som ikkje er så bra å gjøre i PC.
% 	Så skriv litt om pragmatic ANN (som løysing på dette)

%	\subsection{A Review of ANN History}



\section{Artificial Neural Systems : A Review of ANN History}
	\label{ssecHistoryOfANN}
	The pragmatic use neural network simulations started with the ``McCulloch--Pitts neuron'' in 1943.
	%Warren McCulloch, an early neuroscientist and the young mathematician Walter Pitts formalized the models of the neuron and proposed the first artificial neuron simulator. %artificial neural network.
	Warren McCulloch, an early neuroscientist and the young mathematician Walter Pitts initiated a formalized discussion about the mechanics of the neuron and the use of this in technology. %artificial neural network.
	This resulted in the first neuron emulator(artificial neuron). %, later referred to as the McCulloch--Pitts neuron. 
	%When a network of nodes consisting of the artificial neural was set up, McCulloch and Pitts created the very first artificial neural network(ANN).\cite{MccullochPittsHistorie} %TODO Sjekk referansen! TODO
%
	%TODO Glatt ut: Gjør slik at det eg god flyt i teksten i det som står under her! TODO
	Artificial Neural Networks based on the McCulloch--Pitts neuron model has later been referred to as the first generation ANN\cite{Maass97networksof}.
	%What has later been referred to as the first generation ANN is based on the McCulloch--Pitts neuron\cite{Maass97networksof}.
	%One example of a first generation ANN is the Rosenblatt's Preceptron\cite{HaykinANNbok}.
	Each node is modelled as a boolean device(with an on--off response), where the node sends output if the immediate input level is large enough.
	The first generation ANN therefore can be said to be a network of simple filters called threshold gates.
	This does not take into consideration the depolarization (state) of each node, and is a tremendous simplification of the biological neuron.
	%The first generation ANN thus does not take into consideration the depolarization of the neuron, and is a great simplification of the biological neuron.
	%%%The node sends output if the immediate input is large enough, and does not take into consideration the depolarization of the neuron.
	One famous example of an ANN classified as a first generation ANN is Rosenblatt's Perceptron\cite{HaykinANNbok}.


	The second generation ANN gives a better simulation of the neuron. % in the frequency domain.
	%A better simulation of the neuron considers the neuron in the frequency domain.
	Each node computes the output level as a floating point number based on the immediate input to the node.
	From sec. \ref{secBiologicalNeuralSystems}, the biological neuron is introduced as a node that sends output when the depolarization goes to suprathreshold levels.
	A continuous propagation of a floating point number can therefore only be said to represent the frequency of such transmissions as a function of present input.
%%%
	The function used for computing the output is referred to as the \emph{activation function} of the node.
	%A common activation function is the continuously differentiable \emph{sigmoid function}, that also limits the maximal output\cite{HaykinANNbok}.
	The activation function is found to give the best results if the function is a continuously differentiable sigmoid function\cite{HaykinANNbok}.
	\begin{equation}
		\sigma(x)=\frac{1}{1+e^{-x}}   %TODO TODO TODO TODO TODO TODO Lag heller en figur for å vise sigmoid function! (Bytt ut ligning med fig!) TODO TODO TODO TODO TODO TODO TODO
	\end{equation}
	Because it is more right to consider the neuron as stateless in the frequency domain, the stateless computation in a second generation ANN is more correct than the stateless computation in the McCulloch--Pitts neuron model.
	As the concept of frequency only makes sense for time intervals of a certain size, precise simulations with small computational time steps does not necessarily give accurate simulation results.
	%As the concept of frequency only makes sense for time intervals of a certain size, a second generation ANN can not be used for accurate simulations with small computational time steps.
	%%%
		%%the second generation ANN only give accurate simulations for computational time intervals where is makes sense to talk about mean frequency.
%%%%%%%%%%
	%It is intuitive that the frequency representation only give good simulation results for time intervals where it makes sense to talk about mean frequency.
	%This model thus only gives accurate simulations for a very coarse temporal resolution(large computational time steps), and does not take into account temporal effects caused by the time of firing. %%
	%%This simulation can therefore only be said to only give an accurate simulation for a very coarse temporal resolution(large computational time steps).
	For more precise simulations of the neuron or simulation where temporal mechanisms in the neuron are important, the frequency representation in second generation ANNs can therefore not be used.  %TODO Finn noe å CITE!
	%For more precise simulations of the neuron or of temporal mechanisms, the frequency representation in second generation ANNs therefore can not give good simulation results. %TODO Finn noe å CITE!
	%%This representation therefore is unsuitable for more precise simulations of neural networks and of mechanisms that depend on temporal elements(like STDP learning rules). 
%TODO TODO TODO TODO Cite :  Finn steder dette står/ting å cite! TODO TODO TODO TODO


\begin{figure}[hbt!p]
	\centering
	\includegraphics[width=0.75\textwidth]{sigmoidCurve}
	\caption{Sigmoid curve $\frac{1}{1+e^{-x}}$ for the domain $x\in [-5,5]$}
	\label{figFigurAvNeuronet}
\end{figure}
	
	A direct simulation of the signal propagation mechanisms of the neuron is often referred to as a ``spiking'' artificial neuron model. %\cite{Maass97networksof}.
	The depolarization of the neuron is simulated by numerical integration of all events that change the neuron's depolarization.
	The most commonly used model for spiking neuron simulations is the Leaky Integrate--and--Fire(LIF) neuron, where the depolarization of a neuron is simulated as a leaky integration of depolarizing input\cite{florian03}.
	Because all aspects that are considered important in signal processing are simulated, this model can be used to test theories about neural signal propagation.
	%TODO LINJA over: Endre litt, og cite. F.eks. Maass97networksof?? (XXX Last leddsetning does not tell what I intended..)

	The LIF model describes the depolarization of a neuron as a leaky integration of the neuron's excitatory and inhibitory input, where the depolarization value diminish(towards the resting membrane potential) over time.
	%The leaky aspect of the neuron can be implemented by subtracting a certain ration of the last computed depolarization, every time iteration.
%%	Artificial Neural Networks that utilize this simulation model for its nodes is sometimes referred to as Spiking ANN(SANN) and belong to the \emph{third generation ANN}. %TODO Cite en art. av Wulfram Gerstner
	%%Artificial Neural Networks that utilize this simulation model for its nodes is sometimes referred to as Spiking ANN(SANN) and belong to what is referred to as the \emph{third generation ANN}. %TODO Cite en art. av Wulfram Gerstner
%%	
	When the simulated depolarization of a node is excited above the firing threshold, a spike is initiates, causing transmission through all the node's output edges. %synapses.
	The signal is propagated as discrete spikes, very similar to the signal processing of a biological neuron\cite{Kunkle02pulsedneural}.
	Artificial Neural Networks with this simulation model for its nodes are sometimes referred to as Spiking ANN(SANN) and belong to the \emph{third generation ANN}\cite{Maass97networksof}. %TODO Cite en art. av Wulfram Gerstner
	%%Artificial Neural Networks that utilize this simulation model for its nodes is sometimes referred to as Spiking ANN(SANN) and belong to what is referred to as the \emph{third generation ANN}. %TODO Cite en art. av Wulfram Gerstner
	

	To summarize this section about ANN history, there are three generations of artificial neural networks, each getting closer to the biological neuron in behaviour.
	%What propagates thought the network, how this is computed and what it represents differs  
	The first generation of artificial neurons where so--called threshold gates, with a boolean output that was [true] if the summed input were above some threshold. %TODO CITE!
	Nodes of the second generation gave, in some respects, a better simulation of the biological neuron.
	The output is not given as discrete states given by the input but as a continuous function that can be interpreted as the firing frequency of the node. % of the level of input to the node.
	With this interpretation it can be said that ANNs of this generation gives a simulation that is closer to the biological neuron in behaviour.
	%If the transmission through the output synapse of a neuron is seen as the firing frequency of that neuron, it can be said that this generation of ANN gives a simulation that is closer to the biological neuron.
	%The third generation ANN is as supposed to be an accurate simulation of the neuron, and is as close to the biological neuron as possible.
	The third generation ANN is as supposed to give an accurate simulation of the neuron, and achieves this by simulation the neuron's depolarization directly. 
	The neuron has an internal state representing the depolarization and fires if this value goes to supra--threshold levels.
	The signal is propagated in the same manner as in the biological neuron, where excitatory synapses increase the postsynaptic depolarization and inhibitory synapses decrease the postsynaptic depolarization.
	%Errors in the simulation comes as a consequence of numerical errors or from the neuron model used.
	Only numerical errors in the digital simulation and errors in the neuron model used separates the simulated result from the behaviour of a real neuron.
	%Only elements like truncation errors in the digital simulation and errors in the neuron model used separates the simulated result from the behaviour of the real neuron.

%%%	XXX TA MED?    Den observante leser vil dermed se at med kvar ny generasjon ANN, så kommer vi nermere bio-neuronet. 




% 2.gen er bedre enn første, fordi begge er "state less". For tidsdomenet blir dette heilt feil. For frekvensdomenet blir det mindre feil. (Heilt rett dersom du ser bort fra syn.p. og modulatory neurotransmitters.

%It is actually so close that the word ``simulation'' will occasionally be used in this report.  	% ".. actually so closa that .." DÅRLIG. Fiks?

%In the third generation ANN it is the action potentials or the "spikes", that is responsible for information processing.  %TODO Skriv om slutten / Feil ord.. 		".. or the "spikes" that is responsible for the information flow.
%This ANN model is therefore often referred to as ``Spiking Artificial Neural Network''(SANN).

%If the transmissions between nodes is viewed as the firing frequency of the neuron, we can say that the continuous output value represents the output frequency as a function of the input frequency over the time step in the simulation.

%The nodes of the third generation ANN became even more similar to the biological neuron, as the output of a node depend solely on the state of the node.






%	\subsection{Synaptic Plasticity and motivation for SANN}
%		- skiv om Hebbian learning: ustabilt. \\
%		- skiv om STDP og at dette er en viktig grunn til å bruke SANN. \\
%		Det er truleg at begge 'learning rules' har sannhet. Det ville difor vært bra å kunne benytte begge, ivertfall i forskningssamanheng.

\subsection{Depolarization Simulation by Numerical Integration}
	The depolarization of a node is a time integral of all depolarizing input and the total ``leakage'' of depolarization value, and can be implemented by numerical integration.
	The effect of leakage can be implemented by subtracting a scaled version of the neuron's depolarization value every time iteration.
	%The last computed value, that is used for computing the size of the leakage is delayed by the size of the computational time step before the leakage can be computed.
	The size of the leakage is computed by the last computed value for the neuron's depolarization, delayed by the size of the computational time step.
%	This cause an erroneous value for the leakage, and defines one aspect of the local truncation error.
	%This value is delayed up The last computed value, that is used as the present value of the neuron is delayed by the size of the computational time step before the leakage can be computed.
%%
	%The error induced by doing it this way in discrete--time simulations is discussed in section \ref{TODO TODO TODO XXX}. %TODO TODO TODO Referer til denne plassen!
%%%%
	The error from each iteration, referred to as the local truncation error, thus increases with the size of the computational time step.
	%It is possible to get accurate simulation results by decreasing the size of the computational time step.
	Accurate simulations can therefore be designed by decreasing the size of the computational time step.

	If all nodes are updated every iteration, the computational load scales linearly with the number of nodes and the inverse of the size of the computational time step.
	By halving the size of the computational time step, the computational load therefore increase as if the number of nodes are doubled.
	This explains that the accuracy of the simulator can be used as a good measure of efficiency, and establish the motivation for having precise simulation algorithms.
	More sophisticated numerical integration techniques are often used to accomplish this.





% 	A leaky integrator can be implemented by integrating all depolarizing input, and subtracting each iteration's leakage.
% 	Excitatory input is cause an increase in the postsynaptic neuron's depolarization and inhibitory input cause a decrease in the postsynaptic node's value.
% 	The leakage is computed by the current depolarization level of the neuron, scaled by a leakage constant.
% 
% 
 

	The corresponding electrical circuit to the LIF neuron model consists of a capacitor $C$ in parallel with a resistor $R$. %, driven by a current $I(t)$.
	Depolarizing input to the neuron, either in form of externally applied current or in the form of excitatory synaptic input is modelled as the $I(t)$. %    cause the membrane potential to increase(the capacitor is charged).
	A leakage current $I_l(t) = -\frac{v(t)}{R}$ cause the depolarization value of the neuron to decrease, and can be modelled as a current through the circuit's resistor. %as a function of the node's present value.
	The equivalent current in the RC circuit is then given by the equation
	\begin{equation}
		I_{tot}(t) = I(t) + I_l(t) = I(t) - \frac{v(t)}{R}
	\end{equation}
	
	When implemented in a discrete--time simulator by numerical integration, discrete time cause a delay of one time step for $v(t)$.
	%The error caused by this every time step is referred to as the local truncation error
	As the value is integrated, the local truncation errors caused by this accumulate, giving an increasingly erroneous depolarization value for the neuron. 
	This cause the simulated neuron to fire at the wrong time.
	An erroneous inter--spike internal cause an error for the neuron's firing frequency with a size defined by the temporal resolution (number of time steps) of the simulation. %, and ??? XXX
	%If the local truncation error is systematic, the erroneous inter--spike interval also produce an error for the neuron's firing frequency.
	%%	This can be implemented in a discrete--time simulator as a discrete integration of all input ($I(t)$) minus the present leakage current $I_l(t) = \frac{v(t)}{R}$.	








		
% // vim:fdm=marker:fmr=//{,//}

% 	\input{NIM-simulations}




%\chapter{Articles}
%	

\section{Introduction}

	\subsection{Dette skal heuilt sist, f.eks. i diskurs!}
	Skriv en-eller-annen--plass at når man gjekk fra 1. og 2. generasjon ANN til 3. generasjon ANN, det er mulig å sei at man begynte å "consider a Moore automata of the neuron".
	(Dette er en litt løs tolking, siden ouput er gitt som discrete pulser. Eg trur likevel at den er gyldig..)
	(Her er det bare state som gir ouput).
	
	I min nye modell, tar eg dette videre, og innfører at output også er gitt av present input til neuronet. Dette gir oss en Mealy automata av neuronet.
	Skriv litt om Moore vs. Mealy automata!


% // vim:fdm=marker:fmr=//{,//}

%	\input{artikkel_06.02.2012/theory}
%	\input{artikkel_06.02.2012/metode}
%	\input{artikkel_06.02.2012/results}
%	
%  		- Skriv om at med dEstimatedTaskTime-opplegget er det ikkje nødvendig å simulere spatiotemporal delay. I aksonet for eksempel. Dette er kjempebra for neuronsimulering.
% 			- Mulighet for "multi-compartment model with single compartment implementation".
% 		
% 		- Skriv om at det var teit å bruke deriverte ved overføringer. Bedre ville det vore å bare definert det slik at kvart neuron oppdaterte(rekalkulerte) kvar iterasjon. 
%
% 		- Directions for further work


%TODO Bruk: 	"This study shows ..." 	Veldig rett formulering! XXX


%XXX skrive om Lovelace and Cios(2008) som "proposed a very simple spiking neuron(VSSN) model, og "Simplicity an Efficiency of Integrate--and-fire neuron models" mener er drit. Denne bruker en forenkling av SANN, men diesmann og Plesser mener den er dårlig. Skriv om at eg går motsatt retning, og bruker meir avanserte metoder for å finne meir correct resultat(som betyr meir effektiv simulering).



% TODO Synaptisk plasticity! Kan bruke både aktivitetsbasert og spikebasert!   TODO 



\section{Summary}

The mechanisms of biological neuron networks, the computational system of biological beings, is not fully understood.
% How biological neural networks, the computational system of biological beings, function is not fully understood.
On a low level, neuroscientists have found that networks of neurons propagate information by discrete action potentials.
An action potential causes a transmission through all the neuron's output synapses, leading to the increase or decrease in the postsynaptic neuron's value.
This value, referred  to as \emph{the depolarization} of the neuron, is the result of a leaky integration of synaptic input transmissions.
% This value, referred to as \emph{depolarization}, can be considered to be the result of a leaky integration of synaptic input transmissions.
% This value, referred to as \emph{depolarization}, can be seen as a leaky integration of synaptic input transmissions.

Digital simulations have discrete time, and a neuron's depolarization is often simulated by numerical integration.
%Digital simulations have discrete time, and continuous mechanisms like leakage can be simulated by numerical integration.
This is done by adding synaptic input and subtracting an estimate of the neuron's leakage.
In this work, the previous time step's value is utilized when computing leakage for the $NIM$ model (\emph{sample--and--hold integration}). % in the implementation that utilizes numerical integration

This study shows that the error from each computational time step varies like a stochastic variable, and that the total error is defined as the integral of all local errors. 
This results in a diverging simulation error, unless the local truncation error has an expectancy value of zero.
In an attempt to avoid this, a novel simulation scheme has been developed that does not involve numerical integration.
% In order to avoid this, ideas from systems theory have been utilized to develop a new simulation scheme that does not depend on numerical integration.
Using the concept of \emph{time windows}, time intervals where the neuron's depolarizing inflow is held constant, a neural simulator was developed that utilize the algebraic value equation in these intervals.
% By the concept of \emph{time windows}, time intervals where the neuron's depolarizing inflow is held constant, a neural simulator was developed that utilize the algebraic value equation in these intervals.
% %%%%
Software intended to make differences in design of the two simulation schemes have been designed and implemented, $auroSim$.
The artificial neuron has the functional lay--out of the biological neuron, with four distinct subelement types, [$i\_dendrite$, $i\_auron$, $i\_axon$, $i\_synapse$].
The abstract \emph{i\_\{element\}} types are inherited to \emph{s\_\{element\}} and \emph{K\_\{element\}}, model specific classes.
All common aspects between the two simulation models can thus be placed in the ancestor \emph{i\_\{element\}} class, making principal differences in design of the two simulation schemes prominent.

It is shown experimentally that although the $\kappa M$ simulation scheme is computationally more complex, the simulation is more effective. %it gives a more effective simulation.
Because the $\kappa M$ simulation scheme produces less errors, longer computational time steps can be used to achieve the same accuracy.
This makes it possible to utilize fewer computational time steps to achieve the same degree of simulator accuracy, enabling a more effective simulation.
It is also shown that the absolute error of the algebraic simulation scheme is bounded, something that could be of importance in complex ANN simulations.
% It is also shown that the algebraic simulation scheme has a bounded error, something that might be important in complex ANN simulations.


%XXX XXX XXX XXX XXX XXX XXX XXX XXX XXX XXX 
% \section{Contribution of Thesis}
% 	-modell
% 	-simulation scheme
% 	-implications of results
% 	-future directions


\section{Discussion}  
	One question that presents itself is the importance of a gradually increasing cumulative error.
	The most immediate errors are the ones that alter the length of an inter--spike interval.
	These are represented as the derivative of the spike--time error curves in fig. \ref{figSpikeTimeErrorExperiment2}; % or \ref{figExperiment2ErrorInTenSineOscillations};
		when an inter--spike interval has an erroneous length, the spike--time error is changed by this amount.
	Fig. \ref{figSpikeTimeErrorExperiment2} shows that in the first period of the forcing function, the $\kappa M_{100}$ spike--time error change with about the same rate as the $NIM_{1.000}$ error.
	After spike nr. $20$, the derivative of the spike--time error is larger in the $NIM_{1.000}$ simulation than in the $\kappa M_{100}$ simulation.
	This illustrates a significant efficiency improvement, as the $NIM$ simulation has a temporal resolution that involves ten times as many time steps as the $\kappa M$ simulation.
% 	The $NIM$ simulation has a temporal resolution that involves ten times as many time steps, and illustrates a significant efficiency improvement by utilizing the $\kappa M$ simulation model.
% 	Since the $NIM$ simulation have a temporal resolution that involves ten times as many time steps as the $\kappa M$ simulation, this illustrates a significant efficiency improvement for the novel model.
% 	Since the $NIM$ simulation has ten times as many time steps, this shows a significant efficiency improvement when utilizing the $\kappa M$ simulation model.

	Fig. \ref{figExperiment2ErrorInTenSineOscillations} shows the spike time errors for the same experiment, simulated over a longer time interval. 
%	Fig. \ref{figExperiment2ErrorInTenSineOscillations} shows the spike time error in an artificial neuron simulated over a longer time interval.
	One can observe the cumulative property of the $NIM$ error as a gradual increase in the absolute spike--time error. %, over the duration of the experiment. 
%	One can observe the cumulative property of the $NIM$ error, since the absolute spike--time error increases over the duration of the experiment. 
% XXX Remove the next two sentences? XXX
% 	Also note that the rate of change, represented by the magnitude of the oscillations in the spike--time error, increases in the course of the experiment.
% 	The reason for this is unknown. %xxx Ta med? This is important for discussion! TODO
%%%
	To compare the $\kappa M_{100}$ simulation's spike--time error with the $NIM_{10.000}$ simulation's error, the difference in absolute error is presented in fig. \ref{appendixDifferenceInErrorFig}.
	This figure shows that in the second half of the experiment, the $\kappa M_{100}$ error is generally less than the $NIM_{10.000}$ simulation's error. %for the $NIM_{10.000}$ simulation.
	This implies an even greater efficiency improvement, as the $NIM$ simulation has a number of time steps that is two orders of magnitude larger than the $\kappa M$ simulation's.
	In all conducted experiments, this effect becomes larger for longer simulations.
% 	It appears that this effect becomes larger for longer simulations.

	Reproducibility has been an important element in the conducted experiments in this work.
% 	In the experiments conducted in this work, reproducibility of results have been in focus.
%%%	The design of the comparison software, and the implementation of fundamental simulation elements have therefore been documented in this text.
	The most important elements of the simulation software are well documented, and the forcing functions in the experiments are represented by algebraic functions.
% 	The forcing functions in the conducted experiments are represented as algebraic functions.
% 	The forcing functions are therefore represented as algebraic functions.
	It is possible that the use of algebraic forcing functions limits the validity of the results, since the input to a node in a neural network is far from being a smooth algebraic function.
	Experiment 2 considers a sinusoidal forcing function, where neither the value nor the derivative is constant for any time interval. % at any time.
	This can be used as a basis in a Fourier series to produce any periodic signal.
	This forcing function can therefore be seen as a component in any signal, and is considered to be an appropriate algebraic function for efficiency measurements. %be a good signal for efficiency mesurements.
	A stochastic Wiener process could also be used, but this would make the experiments harder to validate for others.
% 	
% 	The use of algebraic input functions makes reproduction of the results simpler, but might be a limitation for the validity of the results.
%%
	To simplify further analysis and for a thorough study of the implementation, $auroSim$ has been published under \emph{GPL}.
%	To simplify further analysis and for a thorough study of the conducted experiments, $auroSim$ has been published under \emph{GPL}.
	The source code can be found under branch \emph{master} in the git repository located at \emph{https://github.com/leikanger/masterProject} \cite{gitRepoCommit}.


	One element that could be worth examining, is the ability of the $\kappa M$ simulation model to simulate the neuron by other formal neuron models.
% 	Something that could be worth examining is the ability of $\kappa M$ to simulate the neuron by other formal neuron models.
	The $LIF$ neuron model is often used because it is simple, and does not involve complex operations.
	Other neuron models are reported to produce more accurate simulation results \cite{gerstnerKistler2002}. %\cite{gerstnerKistler2002KAP04}.
% 	Some other neuron models are reported to give more accurate simulation results\cite{gerstnerKistler2002}.
% 	Other formal neuron models is reported to give more accurate simulation results\cite{gerstnerKistler2002}.
	The $\kappa M$ simulation scheme is thought to be applicable for any neuron model where the depolarization is described by an ordinary differential equation.
	As long as the value equation is defined as a function of a single variable, \emph{time windows} can be defined, and the $\kappa M$ simulation model can be utilized.
	The use of $\kappa M$ for systems defined by partial differential equations or sets of ordinary differential equations, is also an area that could be worth examining.
% 	These models can probably be simulated with $\kappa M$ by substituting the $LIF$ neuron's equation with the alternative model's value equation. %, in the implementation.
	%These models can be simulated by substituting the $LIF$ neuron's value equation with the alternative model's equation. %, in the implementation.
% 	It is probable that a $\kappa M$ simulation of these models is as precise as for the $LIF$ model, enabling equally effective simulations of these neuron models.
% 	This can be done by substituting the $LIF$ neuron's depolarization equation with an other neuron model's in the implementation.

	When edge transmission is implemented as the derivative of synaptic flow, transmissions are only needed when there is an altered activation level for the presynaptic neuron.
% 	When edge transmissions as the derivative is implemented, transmissions are only needed when there is an altered activation level for the presynaptic neuron.
% 	Edge transmissions as the derivative enables the synapse to transmit, only when there is an altered activation level for the presynaptic neuron.
%	Edge transmissions as the derivative enables the synapse to transmit, only when the presynaptic neuron has an altered activation level.
% 	Edge transmissions as the derivative enables the synapse to conduct transmissions only when the presynaptic node has an altered activation level.
% 	The use of edge transmissions as the derivative, causes transmissions to be needed, only when the presynaptic node has an altered action level.
% 	The use of edge transmissions as the derivative seemed like a good idea, since transmissions are only needed when the presynaptic node has an altered activation level.
	When a double precision floating point data type is used, with the smallest increase defined to be $10^{-308}$, it is highly unlikely that the activation level of a node remains constant over any time interval.
% 	Because a double precision floating point data type is used, with a precision down to $10^{-308}$, it is highly unlikely that the activation level of a node is constant over a time step.
	The concept of edge transmission as the derivative does not decrease the efficiency of a simulation, but it does not improve it either.
% 	The concept of edge transmission as the derivative therefore does not improve efficiency noticeably.
	It does increase the complexity of the design/implementation, and is recommended to be removed for further uses of $auroSim$.
%%%% Ta med? Ta vekk? ???
% 	Because it does not affect the simulation results, or other comparisons in this report, edge transmission for the nodes has not been altered
%	Because it does not affect the simulation results or other comparisons in this report, edge transmission as the derivative has not been altered in this work.
	
	%todo todo todo todo todo todo todo todo todo todo Skriv om bruk av andre integrasjonsmodeller: Burde sjekka f.eks. runkegutta' todo todo todo todo todo todo todo todo todo todo 


% 	Why the oscillations in spike--time error grows in longer $NIM$ simulations is unknown.
% 	- Muligheten for å bruke KM for andre neuron-models. Skriv at så lenge man har en algebraisk funksjon kan KM brukes. Får en 'bounded error'.
% 	- Mulighet for å interface'e med 2.gen. ANN og andre filter. (skrevet dette før?) Sjekk, og evt. skriv her..
	


\section{Conclusion}
	%TODO Konsklusjon: NIM error auker mens KM error er stabil. Som forutsett..

	This work introduces an entirely new way of considering a neuron's activation level.
	The novel formalism considers what the neuron's depolarization would approach, $\kappa$, if no firing interrupts it.
	The $\kappa$--formalism enables the use of algebra to find the neuron's depolarization, as well as the immediate firing frequency of the neuron.
	Combined with the concept of \emph{time windows}, time intervals where the depolarizing inflow is held constant, spiking neuron simulations can be conducted without the use of numerical integration.
% 	Combined with the concept of \emph{time windows}, it is possible to conduct a spiking neuron simulation that does not utilize numerical integration.



	The traditional Numerical Integration Model($NIM$) and $\kappa M$ is compared theoretically and experimentally in this report. % analitically in this report.
%	Section \ref{ssecAnalysisOfErrorsForTheTwoModels} shows that because the neuron has a positive 
	The analysis of the $NIM$ model shows that the local truncation error has stochastic elements, and that the global truncation error diverges unless the local errors have a expectancy value $\hat{e} = 0$. %that is zero.
	The $\kappa M$ error is a result of a delayed update from a variable that varies within a bounded domain, producing a bounded error. % causing the error to be limited.

% 	The neuron has a potential unlimited increase in depolarization value, since this value has a positive increase till the firing threshold before being reset, every inter--spike interval.
% 	This happens every inter--spike interval, causing a potentially unlimited increase in value. %(positive derivative).
% 	Any systematic local truncation error based on this value therefore produce a diverging error for the simulated depolarization. % neuron's depolarization.

	The two simulation models were implemented in a common framework, and accuracy comparisons was conducted. % on the resulting software.
	These comparisons are relevant since the differences between the models were isolated and potential faults in the common framework affects both models equally.
%FORRIGE(amund endra foreslår å endre. Dette prøver eg å gjør, over) GAMML:	Accuracy comparisons thus give more valid results, as the differences between the models were isolated and potential faults in the implementation of a simulator affects both simulations.
%	Accuracy comparisons thus give more valid results, as the differences between the models were isolated and other implementation details are shared.
%% Ta med? Ta vekk?
% 	The time course of the simulated neuron's depolarization has been compared to a high--resolution simulation to find the two models' error.
	It is shown that in the course of $15$ periods  of a sinusoidal forcing function, the $\kappa M_{100}$ simulation, a $\kappa M$ simulation with $100$ time steps per forcing function period,
		generally produces more accurate results than a $NIM_{10.000}$ simulation.
% 		produces more accurate results than a $NIM_{10.000}$ simulation.
	This is a significant efficiency improvement, as the $NIM_{10.000}$ simulation has a number of time steps that is two orders of magnitude larger than for the $\kappa M_{100}$ simulation.
	All results imply that this effect becomes larger for longer simulations, making the $\kappa M$ simulation model a significant improvement of today's spiking neuron simulation model.

%	
%1	Experiments conducted on the resulting software, $auroSim$, implies that the conducted error analysis is accurate.
%1	The $NIM$ simulations have an accumulation of error, while the $\kappa M$ simulations have a bounded error.
%1	In only $15$ periods of the sinusoidal forcing function, it is shown that the $\kappa M_{100}$, a $\kappa M$ simulation with $100$ time steps per sinus period, 
%1		produce a more accurate result than a $NIM_{10.000}$ simulation.
%1	This implies a significant efficiency improvement, as the $NIM_{10.000}$ simulation has a number of time steps that is two orders of magnitude larger than the $\kappa M_{100}$ simulation.
%1	All results implies that this effect grow larger for longer simulations.


% //{ KOMMENTERT UT!
% \chapter{UFERDIG.. kladd:}
% 
% 	If for example all inter--spike intervals are increased by the same factor, one can say that this is a similarity transform of 
% 		the real inter--spike intervals. % and the results can be transformed back.
% 	For ``offline simulations'' with e.g. scientific intent, the resulting depolarization and spike times can thus be transformed back.
% 	If the emulator is to be used for real--time applications, it is hard to avoid that the transformed results are utilized instead of the correct values.
% 	The gradually increasing spike time error affects the activity level of the neuron, altering the mean firing frequency of the neuron;
% 		When the firing time error increase by some factor $C_{e}$ in the course of the interval $\Delta t$, 
% 		the resulting firing frequency error for that interval have a magnitude $e_f = \frac{C_e}{\Delta t}$.
% 
% 	\begin{itemize}
% 		\item[-] Årsak til at feilen er mindre for KANN: intra--iteration time accuracy, use of algebraic function, 
% 		\item[-] Bruk av edge transmission: Kvifor er dette bortkasta arbeid? Skriv også at det BARE auker kompleksiteten på implementasjonen.
% 		\item[-] Limitations: bruk av algebraisk test-funk. Kanskje dette gjør eksperimentet ubra? Snakk om Fourier-series som lineærkombinasjon av sinusoidal functions.
% 		\item[-] Sammenligning: burde kanskje også sammenligna med en etablert software, som NEURON eller [det fra ås]. Dette ville gjort resultata av effektivitetsanalysen meir overbevisandes.
% 				\emph{vart ikkje gjort fordi dette ikkje var hovudelemented i prosjektet}.
% 		\item[-] Burde sjekka meir avanserte integrasjonsmetoder for NIM.
% 		\item[-] Task-scheduler: muliggjør meir effektiv simulering av spatio-temporal effects.
% 		\item[-] Diskuter kvifor oscillations in the error increase for longer simulations.
% %Diskuter litt om kvifor oscillations i feilen auker, for SANN. Kaffaen er dette? Det ser ut som om feilen svinger meir(deriverte er større) etterkvart, for SANN. Dermed er det ikkje bare den absolutte feilen som auker, men også svingningene av feilen..
% 		\item[-] 
% 	\end{itemize}
% %By considering the neuron's activation level by the value the depolarization would approach in the absence of a firing threshold, 
% 	
% 	\
% 	\subsection{Limitations}
% 	\subsection{Implications of Results}
% 
% 	
% \section{gammel discussion}
% 
% As the neuron fires when the value crosses the firing threshold and an algebraic equation is utilized to find the depolarization value,
% 	the exact firing time $t^{(f)}$ can be estimated by the equation $v(t^{(f)}) = \tau$. 
% 	%the exact firing time can be estimated by an equation that equals the algebraic formula to the firing threshold. %TODO Skriv litt om på slutten.
% The $\kappa$ formalism can be used to simulate the neuron by utilizing the concept of 'time windows', intervals where $\kappa$ is constant.
% A changed $\kappa$ initializes a new time window, and the initial value for the value equation is found as the last value in the previous time window.
% %A changed $\kappa$ initializes a new time window, and the firing time estimate needs to be updated.
% Time windows are thus fundamental for the use of the $\kappa$ formalism in a neural simulator.
% %When a new time window is initialized, the depolarization value and firing time estimate is updated for the neuron.
% 
% % TODO TA vekk: Dette er ikkje en diskurs!
% % When a new time window is initialized, the depolarization value and a new firing time estimate is computed.
% % %When $\kappa$ is updated, the depolarization value of this time is computed and saved along the time of initiation of the new time window.
% % The computation of a new firing time estimate and all other aspects involved in initiating a new time window are only computed once, 
% % 	at the end of a computational time step.
% % This saves much computational resources, and cause these computations to be executed as often as the computations of leakage in $NIM$.
% % The computational complexity still makes the $\kappa M$ simulation more demanding on the system, and the per--iteration efficiency is lower than for the $NIM$ model.
% % %It also generates a smaller error OR SOMETHING..
%  
% 
% 
% Two experiment have been set up to assess the discussed theory.
% The first experiment considers a constant depolarizing inflow, and illustrates that the $\kappa$ formalism can be used to simulate the depolarization of the neuron. 
% The $\kappa M_{100}$ simulation produce the algebraically correct spike times for all spikes, 
% 	indicating that the $\kappa M$ error analysis varies with the change in activation level.
% As the $NIM_{100}$ simulation produce a cumulative error of notable size, the first experiment justifies the conclutions from the theoretical error analysis in sec. \ref{ssecAnalysisOfErrorsForTheTwoModels}.
% 
% 
% The second experiment considers a single sensory neuron with a forcing function given by one and a half period of a sine function. %TODO sine?
% This study shows that errors of approximately the same magnitude is achieved for a $\kappa M_{100}$ simulation with a temporal resolution 
% 	of $100$ time steps per sensory function period as a $NIM_{1.000}$ simulation with ten times the number of time steps.
% The results also indicate that the $NIM$ simulation produce a cumulative error, while the $\kappa M$ simulation have 
% 	an error that only varies with the phase of the forcing function.
% To test the extend of the hypothesized cumulation of error for $NIM$, % and the stability property of the $\kappa M$ error,
% 	the experiment was repeated with a simulation time interval that is ten times as long.
% The cumulative property of $NIM$ is prominent, and the results also verifies the hypothesized stability property of the $\kappa M$ error;
% 	Instead of having an accumulation of truncation errors from numerical integration, $\kappa M$ error varies within constant bounds. %a constant interval. 
% An error from a delayed update can also possibly be dampened by methods from systems theory or numerical estimation.
%  	
% 
% %%% XXX HAR FLYTTET OPP...
% 	One question that presents itself is the importance of a gradually increasing cumulative error.
% 	%One question that presents itself is the importance of a gradually increasing accumulation of error.
% 	If for example all inter--spike intervals are increased by the same factor, one can say that this is a similarity transform of 
% 		the real inter--spike intervals. % and the results can be transformed back.
% 	For ``offline simulations'' with e.g. scientific intent, the resulting depolarization and spike times can thus be transformed back.
% 	%For simulations with a scientific intent, the resulting depolarization and spike times can thus possibly be transformed back.
% 	If the emulator is to be used for real--time applications, it is hard to avoid that the transformed results are utilized instead of the correct values.
% 	%This is hard if the emulator is to be used for real--time applications.
% %
% %	Problems arise when this is to be used for real time applications;
% %		Unless the simulation results are transformed back, the transformed results are utilized instead of the correct values.
% %	Transforming the results back demands algebraic equations for the error, and is hard to achieve for $NIM$ simulations.
% %%%%%%%%
% %	Another problem with having a gradually increasing error is that the mean firing frequency is affected by this gradual increase in absolute error.
% %	If the firing time error increase by some factor $C$ in the course of the interval $\Delta t$, 
% %		the resulting firing frequency error for that interval have a size $e_f = \frac{C}{\Delta t}$.
% %	%The implications of errors are therefore hard to predict in a complex neural network.
% %	%If there is a motivation for doing accurate simulations with a maximal error, the $\kappa M$ simulation scheme is a contribution to the field of computational neuroscience	and/or neural--inspired  cybernetics.
% %%%%%
% %%  %%
% 	%If there is a gradually increasing error, the mean firing frequency of the neuron is affected.
% 	The gradually increasing spike time error affects the activity level of the neuron, altering the mean firing frequency of the neuron;
% 		When the firing time error increase by some factor $C_{e}$ in the course of the interval $\Delta t$, 
% 		the resulting firing frequency error for that interval have a magnitude $e_f = \frac{C_e}{\Delta t}$.
% 	%If it is important that the neuron simulations are accurate without a gradual increase in error,
% 	%	the $\kappa M$ simulation scheme accomplish a small and limited absolute error at a relatively low computational cost.
% %%%
% 	
% 
% The concept of edge transmission, where the signal is propagated as the derivative of synaptic transmission seems like a good idea.
% Only the subset of input synapses with a changed transmission level is considered by the postsynaptic node.
% % Because floating point precision is utilized for representing $\kappa$, 
% % 	it is highly unlikely that the activation level remains constant to this resolution.
% % The concept of edge transmission increase the complexity of the design, and it is recommended that information is propagated as 
% % 	synaptic flow instead of its derivative.
% When $\kappa$ is represented with a floating point precision, it is unlikely that the activation level remains constant 
% 	to this accuracy from one computational time step to the next.
% The concept of edge transmission as the derivative thus only increase the complexity of the design, without impoving simulator efficiency.
% %The concept of edge transmission as the derivative therefore does not involve any efficiency improvement, and only increases the complexity of the design.
% It is therefore recommended that information propagation is implemented as synaptic flow instead of its derivative in future work.
% % It is highly unlikely that the activation level remains constant to the resolution of the floating point data format,
% % 	the consept of edge transmission as the derivative therefore does not involve any efficiency improvement.
% % Transmission as the derivative therefore only increase the complexity of the design, and it is recommended that information is propagated
% % 	as synaptic flow instead of its derivative.
% 
% 
% 	%TODO TODO Diskuter det med umiddelbar vs. transient-kurve for overføring! Kanskje KM er bedre for dette? (bruker ikkje dirac-delta)
% 
% \section{Limitations}
% 	%Her kan eg DISKUTERE begrensninger med dette arbeidet(diskutere frem og tilbake. Kanskje ende opp positivt?
% 	%VELDIG bra for å få full score på kor reflektert arbeidet er!
% 
% 	%\subsection{The Model} xxx
% 		%The $\kappa M$ simulation scheme is possible after a couple of simple concepts woven together to make is possible to utilize the algebraic equation for the neuron's depolarization.
% 		%Because the simulation model utilize the algebraic solution to the $LIF$ neuron's differential equations, the author can only think of two aspects that can be limiting for the $\kappa M$ simulation model.
% 		The concept of 'time windows' makes it possible to utilized algebraic equations to simulate the depolarization of the neuron.
% 		The model used in this work is the $LIF$ neuron model.
% 		This is a simple neuron model with many abstractions, and gives less correct simulation results than more advanced models\cite{CITE}.
% 		It is still used in this work, as this model is the most commonly used neuron model in computational neuroscience\cite{CITE}.
% 
% %%TODO Skriv 
% %%TODO [leikanger] hevder at sånnOgSånn, difor har eg brukts Sånn.. 		Hevder at, raporterer at, sier at, har gjort, ...
% 
% 		[CITEAUTHOR] reports that e.g. the nonlinear IF model is a more precise and balanced neuron model\cite{CITE}.
% 		It is likely that all models that can be represented by a funtion of a single variable can be simulated by the novel simulation scheme. %$\kappa M$.
% %		In this case, the alternating model's equation for depolarization is used instead of that of the $LIF$ model.
% 		%The alternating model's equation for depolarization have to be used instead of that of the $LIF$ model.
% 		%The algebraic solution to the $LIF$ model only has to be substituted with the equation for the alternative model.
% %		As these models are harder to simulate numerically, it is believed that the efficiency improvement could be even larger for these models. 	%%%
% 		This is something that aught to be tested.		
% 
% %		%The nonlinear IF model have not been examined, but could be especially [EGNET]suited for simulation by the $\kappa M$.
% %		%[SKRIVE MEIR OM DETTE]
% %		%The nonlinear integrate--and--fire model have not been examined, but the native use of the exponential function in $\kappa M$ makes
% %		%	it probable that $\kappa M$ can be relatively computationally more effective than a $NIM$ design.
% %		This might be the case for other neuron models as well, and is worth further examination.
% 
% 		An other aspect that might be a limitation for the $\kappa M$ implementation, is how synaptic transmission is modelled.
% 		Because $\kappa M$ propagates information as a second generation ANN, with transmission of the activation level, 
% 			it is hard to imagine whether all aspects of the propagation of spikes are preserved.
% 		With the $NIM$ design, synaptic transmission can e.g. be implemented as transient transmission curves instead of instantaneous transmissions.
% 		It is possible that this is possible for $\kappa M$ as well, by letting $\kappa_{ij}$ have a transient curve.
% 		%It is possible that $\kappa M$ also makes this possible, by letting $\kappa_{ij}$ have a transient curve.
% 		It is therefore recommended that this aspect is examined further.
% 
% %		The third aspect comes from the implementation and computational complexity of the $\kappa M$ simulation scheme.
% 
% %	\begin{itemize}
% %		\item Bruk av $LIF$ modellen:
% %			\begin{itemize}
% %				\item [-] ikkje så bra modell
% %				\item [+] mest brukte modellen
% %			\end{itemize}
% %		\item Diskuter korleis KM kan utvides for f.eks. nonlinear model.
% %			\small{KM kan gjøre det lettare å bruke transiente overføringskurver(enn dirac delta)}
% %		\item Synaptisk plasticity
% %	\end{itemize}
% %		\subsubsection{Further Work (-ikkje med, men viktig å hugse-)}
% 
% 	%\subsection{The Implementation} xxx
% 		An important focus for the experiments conducted in this work have been to make it simpler to verify the results.
% 		%An important focus in the experiments conducted in this work have been on reproducibility.
% 		The source code is therefore freely available\cite{gitRepoCommit} and all experiments have been conducted with an algebraic forcing function.
% 		It is thus simple to examine the source code and verify the results, as well as examine other elements/situations.
% 		%It is thus possible to examine the source code and reproduce the results as well as further examine other elements.
% 		Other sensory functions can be examined by declaring them in the file \emph{sensoryFuntions.h} 
% 			and constructing sensory aurons with the pointer to that function as argument.
% 		% All experiments have been conducted with an algebraic forcing function and with a focus on making it possible for others to reproduce the results. 
% 
% % //{ KOMMENTERT UT
% %	This work [OMFATTER] two different projects.
% %	The first, and in the beginning the most important project, have been to compare the two $LIF$ neuron simulation schemes.
% %	Both simulators have been implemented from scratch, in order to emphasize the differences between the two models.
% %	%Both simulators was implemented from scratch, to be able to see important differences between the two models.
% %	
% %	The second part of the report that considers the efficiency comparison between $\kappa M$ and $NIM$, % the two simulation models, 
% %		and utilize the resulting simulator software from the first part.
% %	%The second part of the project, involving efficiency comparison between the two models used this implementation when comparing $\kappa M$ and $NIM$.
% %	Both implementations have nodes that are designed as the biological neuron, with four subelements representing the functionality
% %		of [dendrite, soma, axon, synapse].
% %	%Both implementations are designed as a biological neural system, with four sub--compartment in each node.
% %	It can thus be said that both models are designed so that spatiotemporal delay is simulated directly in the artificial neuron.
% %			% as a direct simulation of the biological neuron.
% %%% 	%%	  %%%
% %	It is found that this is not necessary, and only introduce more computational load for the simulation.
% %	Because the task scheduler can be utilized for both neural simulation models, this has not been persuaded any further in the efficiency comparison. %XXX persuaded? Sjekk om dette er feil skrivemåte.. (forfulgt..)
% %	It is possible that implementing the task scheduler for the $NIM$ simulation model introduce more computational load, but it is unlikely that this is more demanding than simulating intracellular signal propagation.
% %	%Delay can instead be implemented by scheduling tasks to happen after the defined delay, and no more computational resources have to be used.
% %	As the efficiency improvement from utilizing the task scheduler only affects the run time and not the results,
% %		utilizing the task scheduler would not affect the simulation results.
% %	The efficiency comparison done in this work would therefore be unaffected by doing this.
% %%the result of a comparison that only considers the accuracy of the simulation results.
% %	It is still recommended that further development of $auroSim$ utilize the task scheduler developed in this work%
% %	(the source code of $auroSim$ can be found at \cite{gitRepoCommit}).
% %	%(for the source code of $auroSim$, if is referred to \cite{gitRepoCommit}).
% %	%It is still recommended that future uses of $auroSim$ utilize the task scheduler developed in this work.
% %	%This would only affect the total run time of the simulation, not the accuracy of the simulation results.
% %
% %	%It is also believed that a multiple--compartment model can be simulated in a single--compartment implementation.
% %	%This is an important direction for further work, as it can greatly improve the efficiency of scientific simulations utilizing multiple--compartment models of the neuron.
% %	
% %	%An other aspect that makes the $\kappa M$ implementation more complex is the use of edge transmission as the derivative.
% %	%TODO Skrive om dette?
% %
% %%TODO Ha med, eller ikkje? XXX:
% %%	The results from the efficiency comparison might have been more credible if an established simulation software was used for the $NIM$ simulation.
% %%	%The best approach, when comparing the efficiency of the two models could be to use established software for the $NIM$ simulation.
% %%	As both implementations are designed and written by the author, the $NIM$ implementation can be criticized for being 
% %%		less accurate than other $NIM$ implementations.
% %%	The approach where both simulation models are implemented from a common framework have still been utilized, 
% %%		to minimize the effect of errors originating from the common framework on the comparison. %on the results.
% %%		%as errors originating from the common framework thus affects both simulation models equally.
% %%	%As both simulation models are implemented in a common framework, and it has been important that all common aspects between the
% %%	%	two models are similar, this approach have still been utilized.
% %%	This causes suboptimal elements from the framework to affect both models, and the two models have a more equal [UTG.PKT].
% %%	%It is believed that all elements that are sub--optimal affects both models equally.
% %
% %
% %%	The efficiency comparison done in this work considers the error from $\kappa M$ and $NIM$ simulations.
% %%	The results from a $NIM$ simulation with a temporal resolution that is two orders of magnitude larger than for the compared simulations is used as the correct time course for the neuron's depolarization.
% %%	%The results that are used as the correct time course for the neuron's depolarization is found by a $NIM$ simulation with a number of time steps that is two orders of magnitude larger than the compared simulations.
% %%	%This approach to finding the correct value is validated, as it is shown that the truncation error is proportional to the size of the computational time step.
% %%	The node is designed to emulate a sensory neuron, where the change in depolarization varies with the sensed signal, enabling algebraic 
% %%		functions to define each experiment.
% %	//}
% 
% %	Algebraic input functions are chosen to make the experiments replicable.
% 	By an infinite sum of different sine functions with different phase and frequency, any signal can be modelled exactly\cite{CITE}.
% 	The forcing function utilized when assessing efficiency is therefore chosen to be a sine function.
% 	%The forcing function utilized when assessing efficiency have been used as this can be considered a single [LEDD] in a Fourier series.
% 	To assess all possible signals, however, an infinite amount of these experiments have to be considered.
% %%%%%%
% 	Because the experiment does not define the time scale, the aspect of different frequencies is irrelevant in this context.
% 	The size of $\kappa$ in relation to $\tau$ is defined and gives the inter--spike interval, but the amplitude of the sine function is not linked with the time scale of the simulation.
% 	%The amplitude of the forcing function oscillations is not directly linked to the time scale, and implies that the experiment done in this work is but an example of simulator efficiency.
% 	This implies that the efficiency experiment is but an example of simulator efficiency.
% 	%The domain of the forcing function is chosen so that errors are prominent while still being plausible.
% 	The forcing function is chosen so that it is a relevant example that is plausible for neural networks.
% 	%The forcing function is chosen so that the example is relevant and plausible for neural network, so that it is a relevant example.
% 	Because the forcing function of the efficiency comparison in chapter \ref{chExperimentalEfficiencyMeasurement} has the constraint that $\kappa$ is above threshold for the whole simulation, the experiment has been repeated with a forcing function without this constraint.
% 	%Because the forcing function of the efficiency comparison in chapter \ref{chExperimentalEfficiencyMeasurement} is above threshold for the whole simulation to make errors prominent, the experiment has been repeated with a forcing function without this constraint.
% 	%Because it is above the firing threshold for the whole simulation to make differences prominent, a simulation has also been conducted with a forcing funtion that goes below this value.
% 	For the sake of completeness, method and results of this experiment are presented in appendix \ref{appendixExperiment3}.
% % 		, and shows a similar effect.
% %	The results are presented in appendix \ref{appendixExperiment3} for the sake of completeness, and shows a similar effect.
% 	%The results from this experiment have not been included in the main text, but a plot of the results is presented in appendix \ref{appendixExperiment3}.
% 
% 	%Algebraic input functions are chosen to facilitate
% 	%Algebraic input functions are chosen for the sake of replication
% 	%Algebraic input functions enable a easier repetition of experiments, and also makes analysis of e.g. the simulated solution possible.
% 	A neuron's spiking input from other nodes in a neural network have a complex character, and resembles white noise in the time domain.
% 	%The chaotic input from a network of neurons can almost be seen as white noise in the time domain.
% 	%The chaotic input from a network of neurons can almost be seen as white noise.
% 	The resulting depolarization of the receiving $LIF$ neuron is the integral of this input, and can thus resemble a Wiener process.
% 	To test more realistic neural input flows, one should focus less on replication, and utilize a stochastic process to define neural input.
% 	%To test a more realistic neural input flow, one could step away from optimizing experiments for replication, and e.g. utilize a Wiener process to define neural input.
% %%%%%	
% %	An algebraic forcing function can be seen as only an example of all possible input functions found in white noise.
% %	All functions can be represented as a Fourier series of trigonometric functions[CITE].
% %	The author therefore thinks that the use of a sine forcing function makes the experiment more general, 
% %		and that this particular input can give general results. %answers.
% %	%As all functions can be represented as a Fourier series of trigonometric functions[CITE], 
% %	%	the author believes that the use of the sine function makes the forcing function more general.
% %	It is left for further work to assess forcing functions that gives a sub--threshold $\kappa$. %TODO Eller ta det inn i denne rapporten?
% %	%The use of an algebraic function defined as a sine function can thus be seen as only one 
% 	It is left for further research to assess forcing functions that are defined by e.g. a Wiener process or by synaptic input from an ANN.
% %	It is left for further work to assess forcing functions that also gives sub--threshold activation levels, 
% %		and to assess the two models when a node receives input from an ANN.
% 	To simplify further analysis, the source code of $auroSim$ have been published under GPL and the version presented in this report can be found under branch \emph{master} in the git repository found in \emph{https://github.com/leikanger/masterProject}\cite{gitRepoCommit}.
% 	%To simplify further analysis, the software developed in this work have been published under GPL,
% 	%	and the version presented in this report can be found with the commit id $5e1e609ef_{\ldots}$ at 
% 		%\cite{gitRepoCommit}.
% 	%	\emph{github.com/leikanger/masterProject}\cite{gitRepositorySida}\cite{gitRepoCommit}.
% 	%	\emph{github.com/leikanger/masterProject}\cite{gitRepositorySida}. %TODO TODO TODO Cite viser ikkje vev-sida. Fiks webcitation!
% 	
% 	
% 
% 
% 
% 
% %	\begin{itemize}
% %		\item Har ikkje implementert synaptic plasticity
% %		\item Kanskje skrive om effektivitetsanalysen: at KM er implementert for likt NIM, og vil difor 'suffer' av dette.
% %			KM vil difor bruke meir comp. resources enn nødvendig, noke som kan ha sett KM i dåligare lys.\\
% %			MEN det er alikevel valget å gjøre det slik, siden hovedfokuset i denne teksten er en teoretisk sammenligning(og utvikling av KM)
% %		\item Eg burde brukt meir bibliotek, istadenfor å implementere alt selv (i C++)
% %		\item Eg kunne sammenligna KM med etablert software for effektivitetssammenligning.
% %			\begin{itemize}
% %				\item[+] Dette ville gjort resultata mindre avhengig av min implementasjon av $NIM$.
% %				\item[-] Implementasjonene ville da blitt meir ulike med tanke på f.eks. effektiviteten av tids-simuleringa. Dette ville difor kunne bli feilaktig.
% %				\item[-] Den teoretiske sammenligningen ville ikkje blitt like insiktsfull: mange små detaljer ville forsvunnet.
% %			\end{itemize}
% 
% %		\item bare sett på enkelt-neuron(ikkje ANN)
% %			\begin{itemize}
% %				\item[+] Bedre for å reprodusere resultat!
% %				\item[-] Mulig det ikkje gir eit velbalansert svar. I kvit støy finner man alle mulige situasjoner mens
% %					algebraisk funk. vil være rimelig smalt i fohold.
% %				\item[-/+] Diskutere korleis en synaptic flow vil opptrå. Skrive at denne vil ha en kontinuerlig funk. Så kvit støy er umulig.
% %					Heller den integrerte av kvit støy -- wiener process.
% %%				\item[+] Mulig å finne algebraisk løysing
% %			\end{itemize}
% % TODO TODO TODO Skal eg også ta med neste? (har eg ikkje allerede skrevet om dette?) TODO TODO TODO
% %		\item Bruk av edge transmission as the derivative. 
% %			\begin{itemize}
% %				\item[-] Øker kompleksiteten til design uten å forbedre effektivitet
% %				\item[-] Fører til små feil som må handteres for Kappa,  uten å forbedre effektivitet
% %				\item[+] Tvang meg til å tenke gjennom desse aspektene, og lære at dette er bortkastet.
% %			\end{itemize}
% %	\end{itemize}
% 
% 
% \section{Concluding Remarks}
% 	%(Skriv into om K-formalismen først) TODO
% 
% 	In this work, a new formalism is developed to denote the activation level of a neuron.
% % NESTE SETN ER DÅRLIG: litt feilaktig. TODO TODO Skriv meir presist/ bedre/ rettere. (Det er ikkje at man benytter K-formalismen, men time windows.. TODO
% 	By utilizing the concept of time windows and synaptic flow, it is shown that algebraic equations can be used directly in spiking neuron simulators.
% 	%By utilizing the concept of time windows and synaptic flow, it is shown that algebraic equations can be used to make activation based spiking neuron simulators.
% 	%By utilizing the concept of time windows and synaptic flow, it is shown that a mechanistic model can be used for activation based spiking neuron simulators.
% %	By utilizing this formalism, it is shown that a mechanistic model can be used to design activation based neural simulations.
% %	%By utilizing the new $\kappa$ formalism to denote the activation level of the neuron, 
% %TODO Skriv neste stetninga slika at alle aspekt ved den er interesant og har innhold! TODO
% 	By computing the neuron's depolarization every time the activation level is altered, the neuron's spike time estimate can also be updated.
% 	%As the neuron's depolarization is updated every time the activation level is altered, this can be used as the initial value in the equations to estimate the neuron's spike time.
% 	When the neuron is estimated to fire in the course of the present time step, this estimate can be used as the simulated firing time as $\kappa$ is defined to be constant during a computational time step.
% %	When the the neuron is estimated to fire in the present time step, this estimate will not be altered, and the node can fire.
% 	%By updating the neuron's depolarization every time the activation level is altered, the neuron's spike time can be can not only be found on a reactive basis(firing after threshold crossing), but the spike can be scheduled at the estimated spike time.
% %XXX No gir den heller ikkje heilt meining.. "Ka vil han med dette?"
% 	%By updating the neuron's depolarization every time the activation level is altered, the neuron's spike time can be fond on a proactive basis, not only by firing after the depolarization variable's threshold crossing(reactive basis).
% 	This makes it possible to have spike times with an arbitrary precision, and what is called intra--iteration time accuracy is the result.
% %	This makes it possible for the neuron to spike at a time instance with an arbitrary resolution. %a floating point variable's resolution.
% 	Theoretical analysis conclude that this gives more accurate simulation results.
% 	%It is theoretzised that this gives more accurate simulation results.
% %	For all conducted experiments, it is shown that this gives more accurate simulation results.
% 
% 	Theoretical and experimental analysis indicate that the error varies with the derivative of the neuron's activation level in $\kappa M$. %,
% 	%	and the absolute simulation error is bounded since synaptic inflow is bounded.
% 	This results in a bounded error for $\kappa M$ simulations, as opposed to the error produced when simulating the neuron by numerical integration.
% 	%TODO Fortesett på dette: Veldig bra intro til å skrive at man trenger 10 ganger så mange 
% %%%%%%%%%%%%%%%%%%%%%%%%%%%
% 	If efficiency can be measured by the temporal resolution needed to accomplish some simulation accuracy requirement, this study shows that utilizing the $\kappa M$ results in a large increase in efficiency.
% %	If the size of the computational time step is used to measure efficiency, this study shows that a large increase in efficiency can be the result of utilizing the $\kappa M$ simulation scheme.
% 	In the course of one and a half period of a sinusoidal input, the $\kappa M$ simulation gives more accurate results than a $NIM$ 
% 		simulation with ten times the number of computational time steps.
% % 	In the course of one and a half period of a sinusoidal input, a $\kappa M$ simulation gives more accurate results than a simulation
% % 		that utilize numerical integration of input, with ten times the number of time steps.
% % 		%the currently used simulation model based on numerical integration of input.
% 	The comparative improvement is further increased if the experiment is simulated over $15$ periods of this input;
% 		The absolute error is generally less in the $\kappa M_{100}$ simulation than in the $NIM_{10.000}$ simulation. %, with a computational time step that is $1\%$ of $\kappa M_{100}$'s time steps.
% %		In the second half of this experiment, the absolute error is less in the $\kappa M_{100}$ simulation than in a $NIM_{10.000}$ simulation with a computational time step that is $1\%$ of $\kappa M_{100}$'s time steps.
% 	The $NIM_{10.000}$ simulation has a computational time step that is $1\%$ of the $\kappa M_{100}$'s, and the $\kappa M$ can be said to be ``as efficient as'' a $NIM$ simulation that has a number of time steps that is two orders of magnitude larger.
% 	This involves a simulator efficiency improvement of an increasing magnitude for longer simulations.
% 	%The $\kappa M$ simulation thus generates a smaller error in this experiment than a $NIM$ simulation that have a number of time steps that is two orders of magnitude larger.
% %%%%%%% TODO TODO DODO TODOD TODO TODO SKRIV SISTE TO SETN OM! TODO TODO TDOO TODO
% %	It is concluded that the $\kappa M$ error varies within a bounded domain, while the $NIM$ error is cumulative without an upper limit.
% %	%%					%%							%%									%and does not have an upper limit.
% %	%It is concluded that the comparative efficiency improvement of the $\kappa M$ is partially a result of the stability property of the $\kappa M$ error and the cumulative property of the $NIM$ error.
% %	% It is concluded that the comparative efficiency of the two models is further improved for longer simulations.
% %	The effect of the error properties of the two models are prominent already after 15 periods of a sinusoidal input,
% %		and all results implicate that these difference will diverge further for longer simulations.
% %	%The effect of the error properties of the two models are prominent even thought 15 periods of sinusoidal input can be considered a short simulation.
% %	%Fifteen periods of a sinusoidal input can be considered a short simulation, but the effect of the error properties of the two models are prominent. 
% %	%The effect will be larger for longer simulations, implying that the $\kappa M$ could be a break through when it comes to neural simulation.
% %	%The observed effect will be larger for longer simulations.
% %	%The comparative efficiency of the two models will thus be further improved in longer simulations.
% %	
% %
% %	%The $\kappa$ formalism ... %skrive om at det er mulig å bruke til anna også(?)
% %	%A new formalism for modelling a neuron's activation level is the result.
% %	%The $\kappa$ mathematics enables a more precise discussion about a neuron's activation level.
% %	%skriv f.eks. om kor forvirra dette er no, og kor teit det er å bruke gj.sn. fyringsfreq. som umiddelbart aktivitetsmål!
% %
% 		
% 	
% 
% 	The novel spiking ANN model propagate information like a second generation ANN, while still being capable of computing spikes.
% 	This is something that makes theory from frequency based ANN relevant.
% 	Second generation ANNs or other digital filters, can therefore be interfaced directly by letting its floating point output give $\kappa$ for the interface node.
% 	%This can also be done by a ``sensory node'', as in the experiments conducted in this work, and is thus valid for $NIM$ as well.
% 	By utilizing a ``sensory node'', similar effect can be achieved for input to a $NIM$ node, but only $\kappa M$ is able to produce a signal that can be utilized by a second generation ANN without signal processing.
% %	
% %	This is valid for both $NIM$ and $\kappa M$, but only $\kappa M$ is able of producing a floating point output signal without signal processing and/or estimation.
% 	Elements from second generation ANNs can also be utilized direcly in a $\kappa M$ spiking ANN;
% 		$\kappa$ can be used to compute the neuron's ``immediate frequency'' as well as the present and future depolarization values. %, and $\kappa M$ can thus be used for simulating second generation ANNs as well as spiking ANNs.
% 	%E.g. learning rules can be applied directly, as $\kappa$ can be used to compute the neuron's ``immediate frequency'' as well as the present and future depolarization values.
% 	With a few constraints, $\kappa M$ can therefore be used for either a second generation ANN or a spiking ANN.
% 	This makes $\kappa ANN$ very useful for examining theoretical neuroscience, as a full $\kappa M$ simulation can find info om begge..
% 
% %	%Begrunn meir: skriv at 2.gen er godt utvikler, mens SANN er heilt nytt.
% 
% %TODO TODO TODO TODO TODO TODO TODO TODO TODO TODO TODO TODO TODO TODO TODO  MEIR FUTT I SLUTT! TODO TODO TODO TODO TODO TODO TODO TODO TODO TODO TODO TODO 
% 
% 
% 
% 
% 
% 
% 
% 
% % Negative ting med arbeidet mitt:
% % 	- Begge simulatorene er designet som NIM-simulatoren, der axonet simulerer spatiotemporal effekter. Dette er dårlig, og eg gjekk vekk fra dette designet. Det henger likevel igjen fra tidlig i prosjektet, 
% % 		og bør designes på nytt i videre arbeid.
% % 	- Synaptisk transmission i KM er implementert som den deriverte. Dette var for å effektivisere oppdateringen av postsynaptisk neuron, da bare de som var oppdatert trengte å resummeres. 
% % 		(Selv om summering er en lett operasjon kan neuronet ha veldig mange input-synapser).
% % 		Mengden med inputsynapser fører selfølgelig til at de aller fleste neuron (så mange at man kan kalle det "alle") får endret K kvar iter.
% % 		Det ville difor vært bedre å bruke en enklere direkte implementasjon, da dette er mindre "error prone" og er lettere å vedlikeholde. Dette er eit viktig aspekt i videreføringen av dette arbeidet!
% % 	- KM har meir konstant workload, men er dette egentlig bra? Diskuter om resultatet alltid er maks(meir enn det burde være).
% % 		Bra diskurs-materiale! (fordi eg trur at det ender med en følelse om at det er bra, spess for real-time applications)
% % 	- Skriv at når man 'for the case of reproducibility' går vekk fra å bruke eit kaotisk ANN, over til å bruke en enkelt node kan dette tviste effektivitetsanalysen. Dette er lite truleg, pga formen på signalet.
% % 		Hovedfokus i dette prosjektet var å sammenligne accuracy. Dette blir ikkje endra om det kommer fra $\xi(t)$ eller fra synaptisk input. Den analytiske input funksjonen er også designa til å være slik at verken verdien eller den deriverte av en vilkårlig grad er konstant. Bra.
% % 		Diskuter frem og tilbake.. slutt med å diskuter for denne approach.
% 
% % Det meste som var felles (i i\_auron, i\_...) var skriving til logfil osv. (STEMMER DETTE?) Kan isåfall diskutere at det aller meste er heilt ulikt for KM!
% 
% % Skrive om at eg ser i etterkant at eg ikkje burde implementert KM noden som NIM noden. KM noden trenger ikkje simulere intracellular delay, men kan bruke scheduler. (Dette kan for så vidt NIM også)
% 
% % Siden dette også var en sammenligning av design av de to SANN simulator modellene, har to implementasjoner blitt skrevet fra grunnen og sammenlignet. Det hadde kanskje vore bedre å sammenlignet KM med SANN implementert av andre når effektivitetssammenligne, men dette ville tatt bort en del fra første aspektet ved denne oppgaven(teoretisk sammenligning av modellene). BLA BLA. Men kunne være interresant. Men sammenligningen ville vært mellom KM-sample--and--hold og NIM med meir avanserte integrasjonsmetoder.
% 
% 
% % VIDARE ARBEID:
% % 	- Undersøke om det er mulig å gjøre det samme for andre neuron modeller. Kanskje KM er meir egnet til f.eks. exponential neuron model?
% % 	- Transduction mellom generasjoner av ANN(og også andre filter)
% % 	- Har bare sett på det enkle eksempelet med sensor-neuron. Dette er bra for reproducibility, men kan kanskje gjøre noko anna i dette forferdelig komplekse systemet. VIDERE FORSKNING kan difor være å undersøke om dette også stemmer for synaptisk input(ANN).
% % 	- Eg går bare ut ifra at begge modellene har nytte av å bruke meir avanserte numeriske metoder. Kor stor denne nytten er bør undersøkes. Dette har eg satt som utenfor the scope of this project. Dette er viktig element i for further recearch!
% 
%//} 
 
 
% // vim:fdm=marker:fmr=//{,//}

%	\input{artikkel_06.02.2012/concludingRemarks}
%
% 	\appendix
% 	
% TODO TODO TODO TODO TODO TODO TODO TODO TODO TODO TODO TODO TODO TODO TODO  Skriv om denne fila: Skriv tekst og greier. No er det replikat av artikkel-appendix
\chapter{Mathematical Derivations}

\section{Algebraic Solution to the LIF Neuron's Depolarisation}
\label{appendixAlgebraicSolution}
	The subthreshold behaviour of the LIF neuron model can be modelled as a general leaky integrator.
	\begin{equation}
		\begin{split}
			\dot{v}(t)&= \dot{v}_{in}(t) - \dot{v}_{out}(t) \\
				&= I - \alpha v(t)
		\end{split}
		%\nonumber
		\label{appendix:eqDifferentialEquation}
	\end{equation}
		where I represents the input flow and $\alpha$ represents the leakage constant of the value.

		Laplace transform gives
	% TODO TODO TODDO TODO TODO Legg utledning av uttrykk i appendix?
	\begin{equation}
		\begin{split}
			sV(s)-v_0 		&= \frac{I}{s} - \alpha V(s) 			\qquad, \; \qquad v_0 = v(t_0) 				\\
			(s+\alpha)V(s) 	&= \frac{I}{s} + v_0 														\\
			V(s) 			&= \frac{1}{s+\alpha}\left( \frac{I}{s} + v_0 \right)
		\end{split}
		\nonumber
	\end{equation}

	And 
	\begin{equation}
		\begin{split}
			v(t)  	&= 		\mathscr{L}^{-1}\bigg\{ V(s) \bigg\}  									\\
			 		&=		\frac{I}{\alpha} - \frac{I}{\alpha} e^{-\alpha t_w} + v_0 e^{-\alpha t_w} \qquad, \; t_w = t - t_0
% TODO TODO TODO TODO Sjekk om det stemmer at init-value blir trukket fra, som over (t_w = t - t_0)
		\end{split}
		\label{appendix:eqValueEquationUTLEDING}
 	\end{equation}

	The value equation for the leaky integrator with initial value $v_0$ is only valid for time intervals where $I$ and $\alpha$ remain constant.
	This includes any time window, as defined in sec. \ref{ssecTheAlgebraicSolution}.
	%The value equation for the leaky integrator with initial value $v_0$ is only valid for time intervals where $\kappa$ and $\alpha$ remain constant.
	%Such an interval is referred to as a time window, as defined in sec. \ref{ssecTheAlgebraicSolution}.
%	%%We arrive at the value equation for the leaky integrator with initial value $v_0$.
%	%%It is important to emphasize that the value equation \eqref{appendix:eqValueEquationUTLEDING} is only valid for time intervals where $\alpha$ and $I$ remain constant.
%%
	The variable that represents time is measured from the start of the current time window, $t_w = t - t_0$.



% FLYTTA TIL TEKSTEN: \section{Firing Time for the LIF Neuron}
% 	Difor er \label{appendixFiringTime} bore..



\section{Refration time and simulator time scale}
%\label{ssecValueOfAlpha}

The inter--spike interval for a neuron consists of two phases. 
The absolute refraction period and the depolarizing phase (se sec. \ref{ssecTheActionPotential}).
% % % 
Equation \eqref{eqPeriodeligningForKonstIntraPeriodKAPPA} models the interval of the depolarizing phase of the neuron. % , $p_d(\kappa)$.
The equation for the whole inter--spike interval is given by
\begin{equation}
	p_{isi}(\kappa) = p_d(\kappa) + t_r
	\label{eqHeilePerioden}
\end{equation}

Where $t_r$ is the refraction period of the neuron. % , and $p_d(\kappa)$ is given in \eqref{eqPeriodeligningForKonstIntraPeriodKAPPA}.
If we consider the firing frequency of the neuron, $f(\kappa) = p_{isi}^{-1}(\kappa)$ we can see that the asymptote is given by
\begin{equation}
	\begin{split}
		\lim_{\kappa->\infty}{ f(\kappa)} &= \lim_{\kappa->\inf}\left( \frac{-\alpha}{\ln \left( \frac{\kappa - \tau}{\kappa} \right) - \alpha t_r} \right)   \qquad = \frac{1}{t_r} \\ 
		%\lim_{\kappa->\infty}{ f(\kappa)} &= \frac{1}{t_r}
	\end{split}
	\label{eqFrekvensLlim} 
\end{equation}

From this analysis it can be concluded that the refraction period of the neuron will limit the output frequency of the neuron.
This can be seen in fig. \ref{figFrekvensMedOgUtenRefractionPeriod}.

\begin{figure}[bhtp]
	\begin{center}
		\includegraphics[width=0.7\textwidth]{frekvensPlotRefractionPeriod}
	\end{center}
	\caption{Frequency for a neuron, with and without refraction period for the neuron.}
	\label{figFrekvensMedOgUtenRefractionPeriod}
\end{figure}

%We can see from this analysis that the refraction period of the neuron is fundamental for restricting the neurons output frequency (se fig. \ref{figFrekvensMedOgUtenRefractionPeriod}).
For biological neurons, the maximum firing frequency is about 1000 Hz \cite{NeuroscienceExploringTheBrain3edKAP4}. %s 79
\begin{equation}
	\lim_{\kappa->\infty}{ f(\kappa}) \approx 1000 \, \text{Hz}
\end{equation}
%If we define the maximum firing frequency to be 1000, equation \ref{eqFrekvensLlim} gives us the absolute refraction period as
If we define the maximum firing frequency for the artificial neuron to be 1000 Hz, from equation \ref{eqFrekvensLlim} we get the corresponding refraction period $t_r$:
\begin{equation}
	t_r = \frac{1}{1000 \text{Hz}} = 1 \, \text{m}s %= 0.001 s = 
\end{equation}

%TODO Skriv at dette er en kjendt størrelse i neuroscience (finn, referer), og er en indikasjon på rettheten til lingningene (?)
% 		Kanskje også skrive litt om at dette er "absolute refraction period". Det er også en mild refraction period etter dette (finn,referer). Dette kan implementeres ved 2ms refraction period for auronet. 
% 		TODO TODO Sjekk andre linja her, og gjør en bestemmelse i forhold til mine ANN. (1 eller 2 ms refraction period?).
If we define the time step of the simulation to be 1 m$s$, the refraction period will be one time step in the simulation.
%With a time step of 1 m$s$, the absolute refraction period (the time interval where it is impossible to exite the neuron) can be set to one time step. 
With a time step of 1 m$s$, the simulation of the refraction period can be done by blocking the input for the durion of one time iteration.
%For SANN nodes, this means that the node will not change its value for the duration of the next time step. 
%For $\kappa$ANN this can be implemented more effective by incrementing the estimated firing time by one time iteration. 



\section{Activation level recalculation} 		%todo todo todo todo todo todo todo todo todo todo todo todo todo todo todo 
\label{appendixRecalculateKappaClass}
	The concept of edge transmissions as the derived potentially gives an increase in the efficiency of the simulation, as only the necessary additions have to be executed.
	The value is found as the sum of all such edge transmissions, and the effect of an altered activation level is computed after the time step.
	As the activation level is found as the sum of all preceding edge transmissions, small numerical errors is also integrated and could give a large deviation from the correct activation level.
	Because of this, an adaptive mechanism for recalculation of the activation variable $\kappa$ is devised.

	The size of the error is hard to estimate, as it can vary with the hardware architecture, the system load and the number of input transmissions to the node in question.
	%The size of the error from one time step is hard to estimate, as this varies with the number of inputs in the course of the time step.
	Because of this, the number of time steps between each recalculation in a node is designed to be adaptive.
	When the activation variable have a small deviation from the actual activation level, the interval to the next recalculation can be set higher than if the deviation is large.

	It is important to limit both the minimal and maximal period between recalculation of $\kappa$.
	This is achieved by the altered sigmoid function \eqref{eqIntervalToNextRecalculationOfKappa}. %, shown in fig. \ref{figIntervalToNextRecalculationOfKappa}.
	
\begin{equation}
	p_e(E) = (c_1 + c_2) - \frac{c_2}{1+e^{-(c_4\cdot E - c_3)}}
	\label{eqIntervalToNextRecalculationOfKappa}
\end{equation}

	From equation \eqref{eqIntervalToNextRecalculationOfKappa}, it can be observed that the altered sigmoid function has a maximal value of $c_1+c_2$.
	In fig. \ref{figIntervalToNextRecalculationOfKappa}, $c_1=100$ and $c_2=250$ gives the maximal interval of $350$ time steps between recalculation.
	Because of a small value for the $\kappa$ errors while experimenting with this aspect, the minimal period between recalculations was set to $c_1 = 100$ iterations.
	This can easily be adjusted if $\kappa$ errors become an issue.
	 
	%As indicated in fig. \ref{figIntervalToNextRecalculationOfKappa}, 
	%	this function gives a maximal interval defined by $c_1+c_2$ when the error is zero and a minimal period of size $c_1$ when the error $E\to\infty$.
	%The altered sigmoid function can therefore easily be adjusted to give a different recalculation interval.

\begin{figure}[bhtp]
	\centering
	\includegraphics[width=0.9\textwidth]{intervalToNextRecalculationOfKappa}
	\caption{Plot of the altered sigmoid function \eqref{eqIntervalToNextRecalculationOfKappa} with $c_1=100$, $c_2=250$, $c_3=10$ and $c_4=0.5$.
			The minimal interval is given by $c_1$ and the maximum period by $c_1+c_2$.  }
	\label{figIntervalToNextRecalculationOfKappa}
\end{figure}

	\subsection{Implementation of \emph{recalcKappaClass}}
	% todo todo todo todo todo todo todo todo todo todo todo todo todo todo todo todo todo todo todo todo todo todo todo todo todo todo todo todo todo todo todo todo todo 

% // vim:fdm=marker:fmr=//{,//}



\chapter{Development of A Novel Neuron Simulation Scheme}
	% TODO Write an intro to this chapter
	The subthreshold build--up of membrane potential of a LIF node have the form of a general leaky integrator.
	In the preliminary work to the final thesis, the author proposed a novel simulation scheme based directly on the solution to the LIF neuron's differential equations.
	In this work, the simulation model has been further developed and tested experimentally.
	This chapter will focus on the mathematical modelling and different necessary concepts for the resulting equations to be utilized in a simulator. %, before an experiment is put up in chapter ??
	%	In this section, the mathematical modelling of the LIF neuron and different necessary concepts are introduced.
%%	In the preliminary work to the final thesis, the author proposed a novel simulation technique that utilize the algebraic solution to the LIF neuron's differential equations.
%%%%%%Because the system can be modelled as a capacitor circuit, the author have experimented on utilizing the leaky integrator of the familiar capacitor equation to simulate the neuron's depolarization instead of numerical integration.


%\section{Mathematical Modelling}
%%	In the preliminary work to the final thesis, the author proposed a novel simulation technique that utilize the algebraic solution to the LIF neuron's differential equations.

If the solution to the capacitor differential equations are to be used for computing the neuron's depolarization value, it is found that depolarizing input have to be represented as a continuous flow.
%Instead of considering the discrete synaptic transmissions, 
After the equations for the leaky integrator have been adjusted for simulation of the LIF neuron in sec. \ref{ssecTheAlgebraicSolution} and \ref{ssecTheActionPotential},
	the concept of synaptic transmission as a continuous flow is introduced in sec. \ref{ssecSynapticFlow}.
The second part of this chapter is dedicated to a discussion about the error mechanisms for the two models; Simulation by numerical integration and simulation by the novel model.
%Different errors mechanisms of the two models are discussed in sec. \ref{ssecAnalysisOfErrorsForTheTwoModels}.

	

	

%	TODO TODO TODO TODO Kjør masse sitering: \cite{FDP_report} TODO TODO TODO TODO TODO 

\section{Synaptic Flow: New Formalism for Neural Acivity} %Activation Level}
%\section{Spiking Neuron Simulation Based on Synaptic Flow}
	\label{secDevelopmentOfTheNovelANNmodel}
	A system that behaves like a leaky itegrator is a bucket with a set of small holes at the bottom.
	%An intuitive leaky integrator is a bucket with a set of small holes at the bottom.
	If the LIF neuron is visualized as a leaky bucket with input from a gutter, excitatory synaptic input can be represented by an agent pouring cups of water into that gutter.
	%If the LIF neuron is visualized as a leaky bucket with input from a gutter, synaptic transmissions is represented by pouring cups of water into this gutter.
	When the number of agents pouring water into the gutter becomes very large and the size of each transmission is small, this can again be visualized as rain.
	%The resulting water level in the leaky bucket can be simulated by either counting the number of raindrops or by estimating the corresponding flow in the input gutter and utilizing the algebraic solution to find the water level.

	The resulting water level in the leaky bucket can either be simulated by counting the number of raindrops(and computing the size of the leakage in every computional time step)
	%The resulting water level in the leaky bucket can be simulated by either counting the number of raindrops(and computing the leakage after each computational time step)
		or by estimating the corresponding flow through the gutter and utilizing the algebraic solution to find the water level.
%%%%%
	%If the simulation has a bounded temporal accuracy(discrete time), the author believes that a more accurate simulation result can be achieved when the algebraic solution is utilized to simulate the systems value. 
	For simulations with a bounded temporal accuracy(discrete time), more accurate simulations may be achieved by utilizing the algebraic solution to simulate the systems value.
	This is tested in chapter \ref{chExperimentalEfficiencyMeasurement}.
	%Experiments test this is set up in chapter \ref{chExperimentalEfficiencyMeasurement}.
	%For simulations with a bounded temporal accuracy(discrete time), it is found that a more accurate simulation result can be achieved when the algebraic solution is utilized to simulate the systems value.
	%TODO Dropp setninga over? NEI: første biten er jævla bra! 			Men det etter her er litt dårlig: avslører for mykje om kva eg finner ut?
%%                 %%                                                          %%                                                  %%                                              % input is represented as a flow.
	%This implies that fewer iterations are needed to accomplish some accuracy goal, and a more efficient simulator model is the result.
	In this section, the mathematics and necessary concepts for a flow simulation is developed and presented.




% 	The subthreshold integration of a LIF neuron can be visualized as a leaky bucket with input from a gutter.
% 	Excitatory synaptic input can further be represented by an agent pouring cups of water into that gutter.
% 	%The subthreshold integration of a LIF neuron can be thought of as a leaky bucket with small holes at the bottom.
% 	%If the LIF neuron is modelled as a leaky bucket with input from a gutter, excitatory synaptic input can be represented by pouring cups of water into that gutter. %this gutter.
% 	When the number of incoming synaptic connections are very large and the size of each transmission is small, this can again be visualized as rain.
% 	The resulting water level in the bucket can either be simulated by counting the number of rain drops and estimating the size of each, or by estimating the corresponding flow out of the gutter and utilizing the 
% 		algebraic solution to find the water level. %algebraic solution to the differential equations to find the water level.
% 	If the simulation has a bounded temporal resolution(discrete time), it is found that a more accurate simulation can be achieved by considering depolarizing flow instead of discrete synaptic transmissions.
% 	In this section, the mathematics and necessary concepts for flow simulation are developed and presented.

	\subsection{Algebraic Solution for the LIF Neuron's Depolarization}
	\label{ssecTheAlgebraicSolution}
		Subthreshold integration in the LIF neuron is defined by general leaky integrator's differential equations\cite{gerstnerKistler2002KAP04}.
		\begin{equation}
			\begin{split}
				\dot{v}(t)&= \dot{v}_{in}(t) - \dot{v}_{out}(t) \\
					&= I(t) - \alpha v(t)
			\end{split}
			%\nonumber
			\label{eqDifferentialEquation}
		\end{equation}
		The inflow is represented by $\dot{v}_{in}(t) = I(t)$, and $\dot{v}_{out}(t)$ represents the ``leakage'' of the neuron's depolarization value.
		The leakage is thus given as the neuron's present depolarization level scaled by the system's leakage constant $\alpha$.
		The algebraic solution to \ref{eqDifferentialEquation} is derived in appendix \ref{appendixAlgebraicSolution}.
		For time intervals where $\kappa$ and $\alpha$ are constant, it is found that the system's subthreshold depolarization is given by %can be found by TODO Skriv om! "is given by" er dårlig!
		\begin{equation}
			v(t_v) = \kappa - \left( \kappa - v_0 \right) e^{-at_v} 	\quad,\; \kappa = \frac{I}{\alpha} % \quad,\;t_v = t-t_0
			\label{eqValueEquation}
		\end{equation}

		The variable $v_0$ represents the initial value for the neuron's depolarization and $t_v$ is the time from the start of the considered time interval\mbox{($t_v = t - t_0$)}.
% Var sammenkobla med Recall that equation \ref{eqValueEquation} ...
%TODO Lag figur på nytt! Endre litt på teksten som står (t_p -- time from start of period    er dårlig. Bl.a.)
\begin{figure}[htb!p]
    \centering
    \includegraphics[width=0.65\textwidth]{demonstrasjonAvUlikeKappaforVerdifunksjonen}
 	  \caption[Illustration of how time windows can be utilized to simulated the neuron by the algebraic equation]{
	%		A leaky integrator can be simulated by utilizing the concept of time windows.
			The figure shows how the concept of time windows enables the use of \eqref{eqValueEquation} for simulating the neuron's depolarization.
			In the time interval $t_p = [0, 100]$, $\kappa_0 = 0.7$ is valid.
			At time $t_p = 100$, $\kappa$ is changed to $\kappa_1 = 0.5$, before it finally is set to $\kappa_2 = 1$ at time $t_p = 150$.
			}
\end{figure}
		Recall that equation \ref{eqValueEquation} only is valid for time intervals where $\kappa$ and $\alpha$ remain constant.
		To formalize such an interval for later discussions, the concept of time windows is introduced. % defined.
		\begin{mydef}
			A time window is a time interval where $\kappa$ and $\alpha$ are constants, within one inter--spike period.
			\label{defTimeWindow}
		\end{mydef}

		When the neuron's input flow is changed or the neuron fires an action potential, a new time window is initialized.
		The initial value $v_0$ can be found by computing the last value of the previous time window, and $t_0$ is acquired by saving the time of initiation for the new time window\cite{FDP_report}.



	\subsection{The Action Potential Discontinuity}
	\label{ssecTheActionPotential}
%TODO TODO TODO TODO TODO TODO TODO TODO   Legg inn plott av K som viser kappa og fyring: FDP::fig.3.4    TODO TODO TODO TODO TODO TODO TODO
	As introduced in sec. \ref{secBiologicalNeuralSystems}, the neuron fires an action potential when the depolarization value crosses the firing threshold.
	%In continuous time, 
	The firing time for a neuron in continuous time can therefore be found by the equation $v(t_w^{(f)}) = \tau$, where $\tau$ is the firing threshold for the neuron.

	\begin{equation}
		\begin{split}
				v\left(t_w^{(f)}\right)			 							&= \tau \qquad 										\\	%,\qquad\qquad\tau = \text{firing threshold}
				\kappa - \left( \kappa - v_0 \right) e^{-at_w^{(f)}}  		&= \tau 											\\
		%		(v_0-\kappa)e^{-\alpha t^^{(f)}}							&= \tau-\kappa 										\\
				e^{-\alpha t_w^{(f)}} 			 						&= \frac{\kappa - \tau}{\kappa - v_0} 					\\
				t_w^{(f)}													&= -\alpha^{-1} \, \ln \left( \frac{\kappa - \tau}{\kappa - v_0} \right) 					
		\end{split}
		\label{eqDevelopmentOfFiringTimeEstimateEq}
	\end{equation}

	If an absolute refraction time $t_r$ is defined for the neuron where the depolarization remain constant after firing, this value is added to eq. \eqref{eqDevelopmentOfFiringTimeEstimateEq}.
	%If an absolute refraction time $t_r$ is defined for the neuron where the depolarization remain constant after firing, $t_r$ has to be added to eq. \eqref{eqDevelopmentOfFiringTimeEstimateEq}.
	An other way of viewing the resulting equation is as the remainder of current inter--spike interval, $p_r(\kappa, v_0)$.

	%It is shown in appendix \ref{appendixFiringTime} that the firing time, represented as the remainder of the current inter--spike period can be estimated by % is given by
\begin{equation}
	p_r(\kappa, v_0)  	= -\alpha^{-1} \, \ln \left( \frac{\kappa - \tau}{\kappa - v_0} \right) + t_r
	\label{eqEstimatedTimeToFiring}
\end{equation}

	As eq. \eqref{eqEstimatedTimeToFiring} is derived from \eqref{eqValueEquation}, the same constraints are valid;
		The estimate for the remainder of the current inter--spike interval is only valid until a new time window is initialized.
	%This means that when a new time window is initiated, the old firing time estimate becomes invalid.
%%
	If depolarizing inflow is defined to be constant during a computational time step, a firing time estimate during the current time step can not change before the estimated time. % the neuron fires.
	%If depolarizing inflow is defined to be constant during a computational time step, a firing time estimate in the current iteration can not change before the estimated time. % the neuron fires.
	%If depolarizing inflow is defined to be constant during a computational time step, a firing time estimate in the present time step can not change before that time. %the neuron fires.
%%%%%%%%%
%	The estimated firing time can therefore be utilized as the simulation's firing time, and an action potential can be initiated with an intra--iteration time accuracy defined by the data format used in the computations. %, e.g. the \emph{double} data formate. %given by e.g. the \emph{float} data format.
	The estimated firing time can therefore be utilized as the emulated neuron's firing time. %, giving a set of possible spike times that has a near--continuous temporal resolution.
	If the double precision floating point formate is utilized, this gives a near--continuous temporal resolution for the neuron's firing times.
%%
%	The set of possible spike times therefore has a near--continuous temporal resolution, only limited by the accuracy of the format used. %e.g. the double precision floating point format.

% asdf@jeje12

% XXX Er det for langt hopp? Vil gjerne gå over til neste section: synaptic flow of activation level.
	An inter--spike interval is finalized by the neuron firing an action potential, after which the neuron's depolarization is reset to the membrane resting potential before the process starts anew.
	The immediate estimate of the total inter--spike interval can be computed by eq. \eqref{eqEstimatedTimeToFiring}, from the neuron's reset potential $v_r$.
	%The current estimate of the total inter--spike interval can be computed by eq. \eqref{eqEstimatedTimeToFiring} from the neuron's reset potential $v_r$.
	%An immediate estimate of the total inter--spike interval can be found by computing eq. \eqref{eqEstimatedTimeToFiring} from the neuron's reset potential $v_r$.
	%The total inter--spike interval can therefore be estimated as the remainder of the inter--spike period from the neuron's reset potential $v_r$.
\begin{equation}
	p_{isi}(\kappa) = p_r(\kappa, v_r)% IKKJE: + t_r
	\label{eqEstimateOfInterSpikePeriod}
\end{equation}
	This equation will show important when we next consider synaptic flow of activation level.
	
	%This process can be modelled by 


    \subsection{Synaptic Flow}
	\label{ssecSynapticFlow}
%	Neural input that changes the neuron's depolarization can be separated into two sets, a subclass of synaptic input that changes the postsynaptic neuron's depolarization and other depolarizing input.
%	Synaptic depolarizing input can be mediated through ligand--gated channels, as introduced in section \ref{ssecTheBiologicalSynapse}.
%	%The synaptic part of depolarizing input can be mediated through ligand--gated channels, as introduced in section \ref{ssecTheBiologicalSynapse}.
%	This is what will be referred to as synaptic input in the remainder of this text.

%todo todo todo todo todo todo todo todo todo todo todo todo todo todo todo todo todo todo todo todo todo todo todo todo todo todo todo todo todo todo todo todo todo 
%todo todo todo todo           Lag en figur som viser skematisk kva input eit neuron har(K_ij og xi_i)                   todo todo todo todo todo todo todo todo todo 
%todo todo todo todo todo todo todo todo todo todo todo todo todo todo todo todo todo todo todo todo todo todo todo todo todo todo todo todo todo todo todo todo todo 

	Let all synaptic input be modelled as the flow $\kappa_{ij}$, where $j$ represents the presynaptic neuron and $i$ the receiving neuron.
	Other input that changes neuron $i$'s depolarization is represented by $\xi_i(t)$.
	The final value for the neuron's depolarization, $\kappa_i = \frac{I_i}{\alpha}$, is defined as the sum of all the neuron's input flows.
	%The final value for the neuron's depolarization $\kappa_i$ is defined by the sum of all input flows for neuron $i$.
	If $\mathscr{D}$ is the set of integers representing neuron $i$'s presynaptic neurons, the total inflow during the $n$'th iteration can be written as
	%The total inflow in the $n$'th iteration can therefore be written as

		\begin{equation}
% TODO HUGS: K = I/a : dermed må I være sum(k_ij + xi)*alpha
			% I_{i, t_n} = \sum_{j} \kappa_{ij, t_n} + \xi_{i, t_n}
			\begin{split}
			I_{i, t_n} 	&= \kappa_{i,t_n} \cdot \alpha \\
						&= \left( \sum_{j} \kappa_{ij, t_n} + \xi_i(t_n) \right) \alpha \quad,\; j\in\mathscr{D}
			\end{split}
			\label{eqSynapticIntegrationForKANN}
		\end{equation}

	Synaptic input $\kappa_{ij}$ is the most important element for neural signal processing\cite{PrinciplesOfNeuralScience4edKAP10}, and is the main focus of this section.
	%Synaptic input $\kappa_{ij}$ is the most important neural input for signal processing\cite{PrinciplesOfNeuralScience4edKAP10}, and will be the main focus of this section.
	%The most important depolarizing input for neural signal processing is synaptic input\cite{PrinciplesOfNeuralScience4edKAP10}. Synaptic input will therefore be the main focus in this section.
	%The main focus of this section is therefore synaptic transmissions.
	%The main focus of this section will therefore be synaptic transmissions.
	%TODO Skriv om neste setn., litt. (rundt "have")
	The funtion $\xi_i(t)$, representing other input, can have different forms for different depolarizing sources.
	This element therefore has to be modelled separately for each such source.
	%The funtion $\xi_i(t)$, representing other input, can have different forms for different depolarizing sources and have to be modelled separately for each such source.
	%Other input $\xi_i(t)$ have different forms for different sources and have to be modelled separately for different such mechanisms.
	%The form of other input $\xi_i(t)$ varies for different sources of the signal and have to be modelled separately for each such mechanism. 
	%XXX BRA XXX: One example of another source for changing a neuron's depolarization is the instrumentation done by sensory neurons. %TODO TODO TODO Skriv om dette en plass, og referer dit!  

\begin{figure}[hbt!p]
	\centering
	\includegraphics[width=0.70\textwidth]{epsp_ipsp}
	\caption[Illustration of neural integration of synaptic input]{
			A simulation of neural integration of synaptic input. 
			Excitatory Postsynaptic Potentials(EPSP) increase the membrane potential of the postsynaptic neuron and thus excite the neuron toward firing.
			Inhibitory Postsynaptic Potentials(IPSP) hyperpolarizes the postsynaptic neuron, and inhibits the postsynaptic neuron with respect to firing.
			When the membrane potential at the axon hillock crosses the firing threshold, set to $-10mV$, an action potential is fired.
			%Figuren kommer fra http://techlab.bu.edu/resources/software_view/epsp_ipsp/
			%The simulation result presented in the figure is produced with the educational ``\emph{EPSP IPSP}'' software intended to illustrate EPSP and IPSP after synaptic transmissions.
			(The figure is found on the website of the educational ``\emph{EPSP IPSP}'' software intended for illustration of EPSP and IPSP after synaptic transmissions).
			% TODO Gjør forrige setninga mindre, og FÅ MED AT DET IKKJE ER EG SOM HAR LAGA DEN!
				}
	\label{figIllustrationOfEPSPandIPSP}
\end{figure}


	Let the synaptic weight $\omega_{ij}$ be defined as the postsynaptic change in depolarization after one synaptic transmission. 	
	%Let the synaptic weight $\omega_{ij}$ be defined as the postsynaptic change in depolarization after one transmission in the synapse.
	Synapse $j$'s contribution to the total change in depolarization after a time interval $\Delta t$ can therefore be written as the number of transmissions in that interval scaled by the synaptic weight $\omega_{ij}$.
	%In discrete time simulations, this can be written as
	\begin{equation}
% TODO Skriv det som N
%		\Delta v_i(\Delta t) = f_j(t_{n-1})\Delta t \cdot\omega_{ij} = \frac{\omega_{ij}}{p_{isi}(t_{n-1}}
		%\Delta v_{i, t_n}(\Delta t) = N_{j,t_n}\cdot\omega_{ij, t_n} %								%= f_j(t_{n-1})\Delta t \cdot\omega_{ij} % = \frac{\omega_{ij}}{p_{isi}(t_{n-1}}
		\Delta v_{i}(\Delta t_n) = N_{j,\Delta t}\cdot\omega_{ij, t_{n-1}} \qquad,\; j\in\mathscr{D}%								%= f_j(t_{n-1})\Delta t \cdot\omega_{ij} % = \frac{\omega_{ij}}{p_{isi}(t_{n-1}}
	\end{equation}
	where $N_{j,t_n}$ represents the number of transmissions in the synapse from neuron $j$ to neuron $i$ in the time interval $\Delta t_n$, 
	and $\omega_{ij,t_{n-1}}$ the synaptic weight updated at time $t_{n-1}$.
	%where the variable $N_{j,t_n}$ represents the number of transmissions from neuron $j$ in time interval $\Delta t_n$, and $\omega_{ij, t_{n-1}}$ represents the synaptic weight updated at time $t_{n-1}$.
	%where the number of transmissions is found by the last computed firing frequency of the presynaptic neuron $f_j(t_{n-1})$ multiplied by the length of the time interval $\Delta t$.

	In $\kappa M$, a continuous variable representing the present estimate of the inter--spike interval can be sent instead of the integer number of transmissions. 
	%In the flow simulation model($\kappa M$), a continuous variable representing the present estimate of the inter--spike interval can be sent instead of the integer number of transmissions. 
	This enables a higher resolution for the propagated signal and thus less discretization errors. %XXX Kanskje bedre enn den under?
%	This enables a higher resolution for the propagated signal and thus a more accurate simulation. %XXX Kanskje litt drøyt: more accurate - dette kan også ha med andre ting, som f-eks- modellen som er brukt..
	%For a time interval where the presynaptic activation level $\kappa_j$ is constant(a time window for the presynaptic neuron), synaptic flow of activation level can be written as
	For a time interval where the presynaptic activation level $\kappa_j$ is constant, synaptic flow of activation level can be written as
	\begin{equation}
	%	\kappa_{ij} = \frac{ \omega_{ij} }{ p_{isi}(\kappa_{j})}\Delta t
		\kappa_{ij, t_n} = \frac{ \omega_{ij, t_n} }{ p_{isi}(\kappa_{j, t_n}) } \Delta t \qquad,\; j\in\mathscr{D}% TODO SKRIV kva \Delta t   er for noke! TODO TODO SKVIVE DET SOM FREKVENS, først? = f(t) \omega \cdot \Delta t
	\end{equation}

	For a simulation with constant computational time steps $\Delta t = C_t$, this constant can be further be incorporated into the variable that represents synaptic weight $\omega_{ij}$. % equation for synaptic flow $\kappa_{ij}$.
%	For a simulation with constant computational time steps $\Delta t = C_t$, this constant can be incorporated into the equation for synaptic flow $\kappa_{ij}$.
	%
	%If we let the simulation be carried out with constant time steps $\Delta t = C_t$, this constant can be incorporated into the equation for synaptic flow $\kappa_{ij}$.
	% ELLER:
	%Let the simulation be carried out with constant time steps $\Delta t = C_t$.
	%This constant can then be incorporated into the equation for synaptic flow $\kappa_{ij}$.
	We arrive at the equation for synaptic flow of activation level for constant time steps:
	\begin{equation}
		\kappa_{ij} = \frac{ \omega_{ij} }{ p_{isi}(\kappa_{j})} \qquad,\;j\in\mathscr{D}
		\label{eqSynapticTransmissionForKANN}
	\end{equation}
	
	When synaptic plasticity is introduced, it is important to remember that synaptic weight is scaled by the constant $C_t$.
	For consistency, it is important to scale synaptic plasticity by the same factor.

%[Her stod tidligare en analyse av feilen for de to ANN modellene]. Dette er flyttet inn i analysisOfTheTwoModels.tex

	%TODO La denne være discussion for dette kapittelet! (Ta "Summary" inn i denne section!)TODO
	\section{Implications of $\kappa$--Mathematics}
		Algebraic analysis of a node's activation level is possible when neural input is represented as a continuous flow.
		The propagation of information as activation level can be modelled as the distribution of a changed $\kappa$, 
			and algebraic transfer functions can be set up for a neural network.
		This makes it possible to utilize a less confusing jargon when talking about neural activation level.
		%In ANNs, this might actively be used to compute propagation of activation level.

		Combined with the concept of synaptic flow and time windows, the $\kappa$ formalism enables a new neural simulation scheme, the $\kappa$ simulation model($\kappa M$).
%		The $\kappa$ mathematics also enables an entirely new neural simulation scheme, the $\kappa$ simulation Model -- $\kappa M$. 
		By letting the activation level $\kappa$ be propagated as a mechanistic function for the presynaptic neuron's firing frequency,
		%By letting the activation level $\kappa$ be propagated as a function of the presynaptic neuron, 
			neural network dynamics can be simulated and eq. \ref{eqValueEquation} can be used to find the neuron's depolarization.
%		This makes it possible to utilize a propagation of activation level like in a 2. generation ANN, to simulate a spiking neuron.
		The $\kappa$ simulation model thus has elements from second as well as third generation ANNs.
%		$\kappa M$ thus has elements from second as well as third generation ANNs.

		The concept of time windows from definition \ref{defTimeWindow} makes it possible to utilize equation \ref{eqValueEquation}
			to simulate the neuron's depolarization;
		Every time the neuron's activation level is altered, a new time window is initialized by updating the initial depolarization 
			value $v_0$ and saving the time of initiation, $t_0$.
		The depolarization value can therefore be found for any time $t$ in a time window, by the equation
$$v(t_v) = \kappa - \left( \kappa - v_0 \right) e^{-at_v} 	\quad,\; t_v = t - t_0 . $$
%%%%%%%%
		%The size of synaptic flow can be found by utilizing equation \ref{eqSynapticTransmissionForKANN}, making it possible to create Artificial Neural Networks with nodes simulated by the flow simulation model, $\kappa M$.

		The next firing time can be found by eq. \ref{eqEstimatedTimeToFiring} every time a new time window is initialized.
		This variable can be used to have spike times with an intra--iteration time resolution, and a near--continuous resolution for possible spike times is the result.




% 	\section{Summary, $\kappa M$ neuron simulation model}
% 	% - Kvar node er ansvarlig for å oppdatere sin depol. som funksjon av depolarizing flow.
% 	% - 
% 		In this chapter, a novel neuron simulation model based on the algebraic solution to the $LIF$ neuron's depolarization is developed.
% 		The algebraic equation is found by solving the system's differential equations. %, and necessary concepts like synaptic flow and time windows enable a simulation model based on this equation.
% 		When considering depolarizing and hyperpolarizing input as flows, this equation can be utilized for simulating the neuron's depolarization.
% 		The concept of time windows enable these depolarization--altering flows to be dynamic, and an activation based ANN model similar to a second generation ANN can be used to simulate nodes with third generation ANN facilities.
% 		% TODO i siste discussion: Skriv om muligheten for 'transduction' av signal mellom 2.gen og 3.gen ANN ved KM.
% 		
% 		From equation \ref{eqEstimatedTimeToFiring} and \ref{eqEstimateOfInterSpikePeriod}, the next firing in addition to the immediate inter--spike interval can be estimated with a floating point accuracy.
% 		This enables a higher resolution of firing time and synaptic signal propagation, and a more accurate depolarization simulation is thought to be the result.
% 		% TODO TODO TODO Neste setning er litt upassende, her! TODO TODO TODO Skriv om! (Dette skal bare være en oppsummering, men eg føler at eg trenger noko meir enn setn. over..
% 		To further examine this element, simulation software is designed with the intention to compare the two models.
% 		The design and results of this comparison is presented in chapter \ref{chDesignAndTheroeticalComparison}.
% 		%To further examine this, simulation software is designed with the intention of comparing the different ways of simulating a spiking neuron.
% 		%To test whether this is the case, the two models are implemented and efficiency experiments are designed in chapter \ref{chExperimentalEfficiencyMeasurement}.
% 		
% 	%	Even if the $\kappa M$ neuron simulation model has the capability to compute the spike time of the neuron, spikes are not used as a means to propagate the signal through the neural network.
% 	%	The simulated firing is not directly involved in signal propagation(as in the $NIM$ model), but can be considered as an extra proficiency of the $\kappa M$ simulation model.
% 		
	

	
% 	\section{New Aspects to be Considered for the Novel Model}
% 		The use of the theory presented in this chapter introduce new aspects that have to be considered as well as opportunities for the simulator. % implementation.
% 		%The use of the theory presented in this chapter introduce a some new considerations and opportunities for the simulator. % implementation.
% 		%The use of the theory presented in this chapter introduce a some new considerations and opportunities for the implementation.
% 		Because the activation level $\kappa$ is updated many times before the neuron fires, time windows have to be utilized to be able to simulate utilizing the $\kappa M$. %a spiking neuron by $\kappa M$.
% 		From equation \ref{eqEstimatedTimeToFiring} and \ref{eqEstimateOfInterSpikePeriod}, the next firing and the inter--spike period can be estimated with a floating point accuracy.
% 		This enables a synaptic signal propagation of a number with a higher resolution, and the execution of an action potential at the computed firing time instant.
% 		
% 		When equation \ref{eqEstimatedTimeToFiring} have given an estimate that is in the present computational time step, an action potential is simulated.
% 		%When the estimated firing time is in the present time iteration, an action potential is simulated.
% 		The simulated firing is not involved in signal propagation as in the $NIM$ model, but is an additional capability for the $\kappa M$ simulation model.
% 		%The main reason for simulating the action potential in the $\kappa M$ is to compute mechanisms like STDP, as presented in appendix \ref{appendixSynapticPlasticity}.
% 		The neuron fires when the estimated task time is in the present computational time step.
% 		To be able to efficiently make use of this proactive firing time simulation scheme, a task scheduler have to be devised specifically for this purpose.
% 		%This proactive firing scheme in $\kappa M$ requires a task scheduler to be able to efficiently simulate the neuron.





%TODO  Skriv noke nytt, her. Skal flytte "Task Scheduling" til "General Design of Simulator"::"Time" Det er fortidlig å ha det her.
		
% 		\subsection{Task Scheduling}
% 			
% 			Two alternatives for scheduling tasks have been tested for the simulator.
% 			The first is based on a continuously updated linked list of linked lists with tasks. %that can be considered a variable array.
% 			When a task is scheduled for execution e.g. in the iteration after the next, the object's pointer is inserted into the second inner list of the outer linked list.
% 			Before every time step, the first element of the outer list is popped and all the tasks of the inner list is inserted into \emph{pWorkTaskQueue}.
% 			This gives a list of lists that gives the relative time of scheduled tasks, where each list contains jobs scheduled for future time iterations.
% 			
% 			An alternative approach is to implement time scheduling by letting the \emph{time\_interface} abstract class have a variable \emph{double dEstimatedTaskTime}.
% 			This element is updated every time the neuron's firing time estimate is updated and checked by \emph{time\_class::doTask()} when time is iterated:
% 				If an element is scheduled for execution during the next computational time step, the pointer to that element is inserted into \emph{pWorkTaskQueue}.
% 			As introduced in section \ref{ssecTime}, this causes the task to be executed during the correct computational time step, 
%  			%This causes the task to be executed during the correct computational time step, 
% 				and the double precision floating point variable \emph{dEstimatedTaskTime} enables an intra--iteration time accuracy for tasks if \emph{pWorkTaskQueue} is
% 				 sorted by this variable.
% 
% 			The two methods was tested by comparing the total run time for a similar experiment set up.
% 			Because the second alternative is simpler to implement and thus simpler to maintain,
% 				and it was found to have about the same grade of efficiency(almost $5\%$ faster for the conducted experiment),
% 				%and have about the same grade of efficiency(about $5\%$ faster for the conducted experiment), 
% 				this approach is used for time scheduling in the implementation.
% 				%the alternative with the \emph{time\_interface::dEstimatedTaskTime} is used for time scheduling in this implementation.
% 			%The second alternative was somewhat more efficient($<5\%$ faster run time) in addition to being simpler to implement and maintain.
% 			%This alternative was therefore chosen.
% 			
% 			\subsubsection{Task Scheduling for Other Tasks}
% 				%The task scheduler utilize a variable from \emph{
% 				As the task scheduler use a member variable from \emph{class time\_interface}, task scheduling can be used for all classes that is part of the simulation.
% 				An important example of this is the \emph{synapse}: % The synaptic transmission for all output synapses of a node can therefore 
% 					When the neuron fires, the auron object of the node can write to all the node's output synapses' \emph{dEstimatedTaskTime} variable.
% 				The time can be written to the present time plus the predefined axonic delay before that synapse's transmission.
% 				In this way, a more efficient axon delay can be simulated with floating point accuracy.





	
% // vim:fdm=marker:fmr=//{,//}

		% Skal inneholde:  		\section{The novel model: Modelling}  : 	\label{ssecModellingRefractionTimeToLimitFiringFrequency}
	%\include{timeAndError} ... eller noke



\appendix
	
\chapter{Synaptic Transmission and Plasticity in the Biological Neuron} %xxx Eller biologisk neurale sys. (er vel strengt tatt ikkje i nevronet, ELLER er det det?)
\label{appendixSynPlast}

%fra FDP. Direkte kopi. 15.02.2012

%fra NEVR3001 rapport om synPlast : (Kraftig omskrevet)

%\section{Introduction}

The background of memory and learning is an important topic in the field of neuroscience. 
It has recieved much attention lately, and the knowledge about synaptic plasticity and synaptic transmission has progressed significantly the last decades.
In this appendix the different mechanisms behind synaptic transmission and plasticity will be in focus.
We will see that the the ion $Ca^{2+}$ has a crusial role in both presynaptic transmission mechanisms and in postsynaptic mechanisms involved in synaptic plasticity.

% todo Ta vekk (Skreiv det bare inn for å forklare en kybb-leser. Trur syn.p. er kjendt etter rapporten min).
%First some expressions will have to be defined. Synaptic plasticity referst to change in the size of each of the synapse's transmissions.
%Synaptic plasticity may be short--term, lasting from milliseconds to minites %TODO finn ut kva noken seier, og referer dette (Bear, purves, kandel)
% 	, or long--term, giving us ``memories'' that might last from hours to a lifetime.

Synaptic transmission in chemical synapses is dependent on a multitude of different mechanisms situated in the presynaptic axon terminal, the postsynaptic membrane or in glial cells surrounding the synapse.
%Synaptic transmission in chemical synapses is based on many mechanisms both in the presynaptic axon terminal, the postsynaptic neuron, and in glial cell surrounding the synapse.
Some of these mechanisms involved in synaptic transmission will be discussed in this appendix.
We will also look at some elements involved in synaptic plasticity, or change in the size of the synaptic transmission. %TODO Skriv kvifor? FÅ RELEVANS!
% TODO Skriv noke slikt som: Because the ultimate goal of this appendix is to discuss the reason behind going on from
% XXX 	og end opp på at eg må beskrive syn.p. også.
% xxx  	Kan kanskje bare skrive at desse er tett sammenknytta. Nei, eg må forklare kvifor eg vil beskrive STDP.
%In this essay, some of the mechanisms involved in synaptic transmission will be described. 
%This is nessecary in order to say something of what changes the same mechanisms.


%Innleder synaptic transmission:
When the neuron is sufficiently depolarized, an action potential is initiated at the axon hillock.
The action potential will propagate through the axon to the ``axon terminal'' where the presynaptic part of the synapse is found.
When we get a strong depolarization over the presynaptic membrane at the synapse, a cascade that ends up with a change in the postsynaptic neurons value is initiated.
This is what is referred to as synaptic transmission\ref{PurvesNeuroscienceKAP05}.

The size of the synaptic transmission is dependent on the amound of neurotransmitters released from the presynaptic part of the synapse and the number of neurotransmitter resceptors in the postsynaptic membrane
		\ref{PurvesNeuroscienceKAP05}. 
Synaptic plasticity, what is percieved as the basis of learning, can happen on a short--term of long--term timescale.
Short--term synaptic plasticity normally only involves factors that can be seen as ``the state'' of the synapse. 
Factors as neuromodulators, amount of $Ca^{2+}$ in-- or outside the neuron and amount of neurotransmitters available are examples of such factors.
Long--term synaptic plasticity involves ``lasting changes'' in the synapse. One example is such protein synthesis is generation of new postsynaptic receptors\ref{PurvesNeuroscienceKAP8}. %todo har ikkje sett gjennom kap.8 når eg ref.
																										% Veit innholdet, men det er kanskje lurt å sjekke at alle desse elementa stå i purves kap 8.

\section{The Presynaptic Part of Synaptic Transmission}
\label{appendixSecPresynapticSynapticPartOfTransmission}
The propagation of the action potential along the axon, happens as a combination of passive and active transmission.
The electrical potential is transmitted passively along the axon. On the axon we have voltage--gated $Na^+$ and $K^+$ channels.
These will open when the electrical membrane potential supasses a threshold, and will further increase the value of the potential. 
After a while the channels will close in a sequence that resets the potential to the base potential (with a small overshoot) \cite{PrinciplesOfNeuralScience4edKAP09}. 
%and will not activate reamplification before it encounters special voltage--gated $Na^+$ and $K^+$ channels.
%The axon is insulated by a special glia cell, to increase the speed of action potential propagation.
%These are situated in gaps in the insulation, or glia
This will reamplify the signal  so that is may continue passively down the axon to the next voltage gated channels.
%In an action potential, electrical potential is first transmitted passively down the axon of a neuron. On the axon we have voltage--gated $Na^+$ and $K^+$ channels 
%	that open when the electrical potential over the membrane surpasses a threshold\cite{PrinciplesOfNeuralScience4edKAP09}. 
%This will enhance the signal so that it can continue passively to the next voltage gated channels.

In the nervous system we have specialized insulative cells, called myelin.
In myelinated neurons, the voltage--gated channels are located in gaps in the myelin, These gaps are called ``nodes of Ranvier''.
%skriv også om situasjonen for umyeliniserte axon? Nei.
The voltage--gated channels will restore the action potential, and constitutes the active element of signal tranmission along the axon \cite{PrinciplesOfNeuralScience4edKAP09}.
%In myelinated neurons these voltage--gated channels are located in gaps in the myelin, called ``nodes of Ranvier'', in unmyelinated neurons the channels are located continously along the axon membrane. The channels will restore the signal, and constitutes the active part of current transduction along the axon\cite{PrinciplesOfNeuralScience4edKAP09}.


%*********************** OK så langt. ***********************

When the action potential reaches the axon terminal, the end of the axon down the signal path, it will open voltage gated $Ca^{2+}$ channels at the active zones of the terminal. 
This causes $Ca^{2+}$ to enter the cytosol of the axon terminal of the presynaptic neuron\cite{PrinciplesOfNeuralScience4edKAP10}.

$Ca^{2+}$ causes synaptic vesicles to fuse with the membrane and release the contained neurotransmitters into the synaptic cleft.%\cite{PrinciplesOfNeuralScience4edKAP10}. 
There is a linear relationship between the amount of $Ca^{2+}$ entering the cytosol and the amount of synaptic vesicles fusing with the membrane (exocytosis).%\cite{PrinciplesOfNeuralScience4edKAP10}. 
Ecocytosis of a synaptic vesicle will release the neurotransmitters stored in it into the synaptic cleft \cite{PrinciplesOfNeuralScience4edKAP10}. 
The amount of neurotransmitters released into the synaptic cleft therefore have a linear relationship with the amount of $Ca^{2+}$ entering the presynaptic axon terminal.
% TODO Ta med, eller ta vekk?      , given a constant amount of neurotransmitters stored in each synaptic vesicle.
%Exocytosis releases the content of the synaptic vesicle to the synaptic cleft\cite{PrinciplesOfNeuralScience4edKAP10}.

%XXX Viktig, men utafor scope av teksten:
%%%%%%The linear relation between the amount of $Ca^{2+}$ entering the axon terminal and the amount of synaptic vesicles undergoing excytosis was first proposed by Katz and Miledi, and later shown by Rodolfo Llinàs and colleges\cite{PrinciplesOfNeuralScience4edKAP14}. 
%KANSKJE:
% TODO Vettafaen om det skal være med:
% Dette gir at vi kan ha presynaptisk short--term syn.p., noke som er viktig argument for å innføre axo-axonic synapses (temporal synaptic modulatory system)
%Rodolfo Llinàs and colleges first showed the mechanisms of a linear relationship between the amount of $Ca^{2+}$ entering the cytosol of the presynaptic cytosol.
%They also found that the $Ca^{2+}$ channels are graded by the potential over the axon terminal membrane. 
%This further gives a graded responce of neurotransmitter release based on the preysnaptic membrane potential \cite{PrinciplesOfNeuralScience4edKAP14}.
%xxx kan ikkje ta vekk, lett. Bruker dette resultatet seinere.. ELLER?

% TODO Flytt alle \cite{} til slutten av avsnittet (dersom de påstandene på slutten også står her..)
% Har sjekka: alle påstandene er fra kap14 i Kandel.
There is a steady influx of $Ca^{2+}$ at axon terminals, through the L-type $Ca^{2+}$ channel. %\cite{PrinciplesOfNeuralScience4edKAP14}. 
This influx of calcium is graded by the potential over the presynaptic membrane.
When multiple synaptic transmissions happens within a short period of time, 
	the amount of $Ca^{2+}$ in the presynaptic axon terminal builds up and the following synaptic transmissions will give a successively larger effect on the postsynaptic potential.
This effect is called \emph{potentiation}, and can last from minites to more than an hour.
% Føler at neste linja ikkje heilt passer inn XXX:
% TODO Vær sikker på at axo--axonic synases er definert!
This mechanism also gives the effect of axo-axonic synapses in regulating the amount of neurotransmitter release for the next transmissions\cite{PrinciplesOfNeuralScience4edKAP14}.
%The axo-axonic synapses will not influence the firing of a neuron, only the membrane potential of the an axon terminal. %XXX KVA ER axo--axonic synapses? Inled dette for leser!
% % and thus the amount of neurotransmittors released by the following action potential\cite{PrinciplesOfNeuralScience4edKAP12}.
%This gives a mechanism for controling the postsynaptic exitatory postsynaptic potential between other neurons following an action potential.

%TODO TA vekk?
%When two action potentials reaches the axon terminal in fast succession it will cause the synapse to be stronger (give a larger postsynaptic response) for many minutes. 
%This is called \emph{potentiation}, and is thought to be partially because of the increase in presynaptic cytosol $Ca^{2+}$ levels\cite{PrinciplesOfNeuralScience4edKAP14}. In the mossy fiber pathway of the hippocampus, presynaptic $Ca^{2+}$ influx is an important mechanism for synaptic plasticity \cite{PrinciplesOfNeuralScience4edKAP63}. %XXX Sjekk! (mest viktige, eller bare viktig?)

%XXX TA VEKK? 
%A decrease in the number of synaptic vesicles undergoing exocytocis has been observed in sensory neurons of the \emph{Aplysia Californica} following LTD. %, by quantal analysis 
%The mechanisms for decrease in synaptic vesicle exocytosis is not known\cite{PrinciplesOfNeuralScience4edKAP63}.

% dette er kanskje interresant: Det kan også være en basis for STDP.. 
% Sjå om det skal takast vekk, dagen før innlevering.
An increase in the extracellular level of glutamate has been observed after LTP in CA3 neurons. 
The mechanisms behind this is debated, but evidense has been presented of \emph{retrograde messangers} from the postsynaptic neuron that will 
	give feedback to the presynaptic neuron after transmission \cite{PrinciplesOfNeuralScience4edKAP63}. 
This enables a  presynaptic component of \emph{long--term} synaptic plasticity.








\section{Postsynaptic Mechnisms of Synaptic Plasticity}
\label{appendixSynPlast:postsynapticMechanisms}
%There are tree groups of receptors in the postsynaptic membrane of a synapse. AMPA, .......NMDA, kainate, ....
Because neuroscience mainly have focused on excitatory glutamate synapses, the discussion about postsynaptic mechnisms behind synaptic plasticity will focus on glutamate transmission.

There are two groups of glutamate receptors: NMDA and non-NMDA receptors. 
The non-NMDA receptors consists of the AMPA and the kainate receptors. 
Most non-NMDA receptors are only permeable to $K^+$ and $Na^+$, while the NMDA receptor is permeaple to $Ca^{2+}$ in addition to $K^+$ and ${Na}^+$ \cite{PrinciplesOfNeuralScience4edKAP12}. 


The NMDA--receptor is an ion channel that is both voltage gated and ligand gated: 
	It requires both that the glutamate neurotransmitter is present in the extracellular fluid and a strong depolarization over the membrane to open \cite{PrinciplesOfNeuralScience4edKAP12}. 
When we get a transmission when the postsynaptic neuron is strongly depolarized, we therefore get an influx of calcium at the postsynaptic neuron.
$Ca^{2+}$ will activate calcium dependent enzymes and also protein kinases that leads to long--term synaptic plasticity\cite{PrinciplesOfNeuralScience4edKAP12}.
This is done as a result of the calcium dependent enzymes initiating synthesis of new AMPA receptors \cite{AMPARtrafficingArtikkel}. 
More receptors causes a larger probability of the glutamate neurotransmittor having an effect, and thus increases the effectivity of the synapse (the synaptic weight).

To conclude this section we will compare the statistical relationship between the postsynaptic neuron having a large depolarization at the time of transmission and the relative timing of the transmission, 
	in relation to the postsynaptic action potential. 
If the postsynaptic neuron is strongly depolarized at the time of transmission, this implies that the postsynaptic neuron will fire soon after.
This might be one of the basis of what has been known by the name Spike Time Dependent Plasticity (STDP).
% OMGJODT TIL HIT:  XXX XXX XXX XXX XXX XXX XXX XXX XXX XXX XXX XXX XXX XXX XXX XXX XXX XXX XXX XXX XXX XXX XXX XXX XXX XXX XXX XXX XXX XXX XXX XXX XXX 
% TODO KAnskje ta vekk resten (med unntak av Summary?)

%XXX XXX XXX XXX 
%If two transmissions happens in rapid succession, you will get a strong (lokal) depolarization around the postsynaptic receptors. This will cause the NMDA--channels to open at the second transmission, and admit $Ca^{2+}$ into the postsynaptic neuron. 
%TODO Skriv heller om at depolarisasjonen har mykje å seie. NEI, dette står allerede. Skriv korleis tid kan ha noke å seie (begrunn STDP med bakgrunn i teoien her). XXX Gjør eit poeng ut av kvifor eg har tatt med dette i appendixet.
% 			Dette kan helst gjøres etter neste setning.

%Also in the postsynaptic neuron, calcium has an important role in synaptic plasticity. 
%$Ca^{2+}$ will activate calcium dependent enzymes and also protein kinases that leads to long--term synaptic plasticity\cite{PrinciplesOfNeuralScience4edKAP12}.
%%Second messangers can also be activated by metabotropic receptors in addition to $Ca^{2+}$ and the same protein kinases are activated. 

% IKKJE RELEVANT:
%The calcium is thought to be important in both short-term potentiation by enhancing the response of AMPA receptors to glutamate\cite{PrinciplesOfNeuralScience4edKAP63}, 
% 	and also elicit ``permanent'' synaptic changes by receptor synthesis\cite{AMPARtrafficingArtikkel}. %Dette ER relevant, men skrevet over.
%This is thought to enhance the response of AMPA receptors to glutamate\cite{PrinciplesOfNeuralScience4edKAP63}, but also elicit longer lasting (``permanenet'') synaptic plasticity.

\section{Receptor Synthesis}
The rise in calsium levels in the postsynaptic cytosol activates postsynaptic plasticity\cite{AMPARtrafficingArtikkel}. 
One of the possible mechanisms behind postsynaptic LTP or LTD is the increase or decrease in postsynaptic receptors. There has been increased focus on receptor trafficing in the recent years, especially on the AMPA receptor. 

\begin{quote}
At early stages of development, synapses containing only NMDA type receptors are particularly common\cite{PrinciplesOfNeuralScience4edKAP12}.
\end{quote}
Synapses containing only the NMDA receptor is called ``silent synapses'' because they do not change the postsynaptic potential (PSP) unless the postsynaptic membrane is sufficiently depolarized. 
This makes them silent at normal resting membrane potential\cite{AMPARtrafficingArtikkel}. 
It has been observed that these ``silent synapses'' is converted into normal exitatory synapses by the insertion AMPA receptors into the postsynaptic membrane\cite{AMPARtrafficingArtikkel}. 

It has been shown that when synapses undergo LTD, the amount of AMPA receptors in the postsynaptic membrane decreases\cite{AMPARtrafficingArtikkel}. 
This is believed to be because of endocytosis of the receptors. If the dynamin-dependent endocytosis is blocked, LTD is also blocked in the samle\cite{AMPARtrafficingArtikkel}.

\section{Glial Modulation of Synaptic Transmission}
One way for asterocytes to modulate synaptic transmission is to release ATP, which is converted to adenosine extracellularly. 
Adenosine inhibits the $Ca^{2+}$ channels in the presynaptic axon terminal membrane\cite{signallingBetweenGlialAndNeuronsInSynPlast}. 
This results in less exocytosis of synaptic vesicles in the presynaptic membrane, which gives less neurotransmitters in the synaptic cleft as a consequence\cite{signallingBetweenGlialAndNeuronsInSynPlast}.
%This also affects neighboring synapses\cite{signallingBetweenGlialAndNeuronsInSynPlast}. %men det er uklart om dette er pga slett andre plasser også, eller diffusjon.

For the NMDA receptor channels to open, three conditions has to be met:
\begin{enumerate}
	\item Glutamate needs to be present in the synaptic cleft.
	\item The postsynaptic membrane needs to be sufficiently depolarized.
	\item D-serine needs to be present in the synaptic cleft\cite{signallingBetweenGlialAndNeuronsInSynPlast}.
\end{enumerate}
The point about D-serine is interresting, since D-serine is absent in neurons. It is present in asterocytes.
One possible explanation it therefore that asterocytes release the D-serine required for the NMDA-R to open\cite{signallingBetweenGlialAndNeuronsInSynPlast}.  % D-Serine bindes til glycine-binding site.
This indicates that the asterocytes are important in modulating the synaptic plasticity induced by NMDA-R opening.

Glial cells are also important for synaptic transmission by being permeable to $K^+$ from the extracellular fluid of the synaptic cleft\cite{PrinciplesOfNeuralScience4edKAP07}, 
and by being in control of the reuptake of certain neurotransmittors (eg. glutamate)\cite{PrinciplesOfNeuralScience4edKAP15}. %kap 15 kandell, 
%This gives possible astrocyte mechanisms for modulating synaptic transmission.


% Ta vekk FRA HER TODO TODO 
\section{Synaptic Transmission an Plasticity Summary}
%TODO Skriv kvifor eg har skevet alt dette. Få relevans. (STDP, som er viktig argument for SANN)
The subject about synaptic plasticity is important for the understanding of neural systems.
We have presynaptic and postsynaptic elements of synaptic transmission, both subject to continous change. This gives two possible elements of synaptic plasticity. 

The presynaptic part of synaptic transmission can be regulated by changing the presynaptic voltage gated $Ca^{2+}$ channels. One way this is done is by axo-axonic synapses that depolarises the axon terminal before the action potential, and thus enhance/inhibit or prolong/shorten the influx of calcium. 
This results in a change in the amount of neurotransmittors released into the synaptic cleft.

The postsynaptic part of synaptic plasticity consists of short term changes, by changing the effect of AMPA-R with calcium, or long lasting changes involving protein synthesis and the insertion of new AMPA-R in the postsynaptic membrane. Both are dependent on calcium. Changing the postsynaptic influx of calcium is therefore an other plausible mechanism for synaptic plasticity.

The asterocytes maintains the environment for the synaptic transmission by maintaining the ion consentrations in the extracellular fluid in the synaptic cleft. 
Modulation of this will change the environment for synaptic transmission and be a way of changing the effect of synaptic transmission. 

The asterocytes are also responsible for removing some neurotransmittors from the synaptic cleft. 
This makes them in control of the time the neurotransmittor is in the synaptic cleft, and thereby the time it will be effective on the postsynaptic receptors.
This desides the postsynaptic effect of the transmission. Change of this is yet an other mechanism for synaptic plasticity.
%This opens for yet an other mechanism for the asterocytes to control the postsynaptic response of a transmission.


%\begin{figure}[!htbp]
%	\centering
%	\includegraphics[width=0.8\textwidth]{figurSTDP.jpeg}
%	\caption{Spike timing-dependent plasticity. a, Synapses are potentiated if the synaptic event precedes the postsynaptic spike. Synapses are depressed if the synaptic event follows the postsynaptic spike. b, The time window for synaptic modification. The relative amount of synaptic change is plotted versus the time difference between synaptic event and the postsynaptic spike. The amount of change falls off exponentially as the time difference increases. In addition, the amount of potentiation decreases for stronger synapses, whereas the relative amount of depression is independent of synaptic size.}
%\end{figure}


	
% TODO TODO TODO TODO TODO TODO TODO TODO TODO TODO TODO TODO TODO TODO TODO  Skriv om denne fila: Skriv tekst og greier. No er det replikat av artikkel-appendix
\chapter{Mathematical Derivations}

\section{Algebraic Solution to the LIF Neuron's Depolarisation}
\label{appendixAlgebraicSolution}
	The subthreshold behaviour of the LIF neuron model can be modelled as a general leaky integrator.
	\begin{equation}
		\begin{split}
			\dot{v}(t)&= \dot{v}_{in}(t) - \dot{v}_{out}(t) \\
				&= I - \alpha v(t)
		\end{split}
		%\nonumber
		\label{appendix:eqDifferentialEquation}
	\end{equation}
		where I represents the input flow and $\alpha$ represents the leakage constant of the value.

		Laplace transform gives
	% TODO TODO TODDO TODO TODO Legg utledning av uttrykk i appendix?
	\begin{equation}
		\begin{split}
			sV(s)-v_0 		&= \frac{I}{s} - \alpha V(s) 			\qquad, \; \qquad v_0 = v(t_0) 				\\
			(s+\alpha)V(s) 	&= \frac{I}{s} + v_0 														\\
			V(s) 			&= \frac{1}{s+\alpha}\left( \frac{I}{s} + v_0 \right)
		\end{split}
		\nonumber
	\end{equation}

	And 
	\begin{equation}
		\begin{split}
			v(t)  	&= 		\mathscr{L}^{-1}\bigg\{ V(s) \bigg\}  									\\
			 		&=		\frac{I}{\alpha} - \frac{I}{\alpha} e^{-\alpha t_w} + v_0 e^{-\alpha t_w} \qquad, \; t_w = t - t_0
% TODO TODO TODO TODO Sjekk om det stemmer at init-value blir trukket fra, som over (t_w = t - t_0)
		\end{split}
		\label{appendix:eqValueEquationUTLEDING}
 	\end{equation}

	The value equation for the leaky integrator with initial value $v_0$ is only valid for time intervals where $I$ and $\alpha$ remain constant.
	This includes any time window, as defined in sec. \ref{ssecTheAlgebraicSolution}.
	%The value equation for the leaky integrator with initial value $v_0$ is only valid for time intervals where $\kappa$ and $\alpha$ remain constant.
	%Such an interval is referred to as a time window, as defined in sec. \ref{ssecTheAlgebraicSolution}.
%	%%We arrive at the value equation for the leaky integrator with initial value $v_0$.
%	%%It is important to emphasize that the value equation \eqref{appendix:eqValueEquationUTLEDING} is only valid for time intervals where $\alpha$ and $I$ remain constant.
%%
	The variable that represents time is measured from the start of the current time window, $t_w = t - t_0$.



% FLYTTA TIL TEKSTEN: \section{Firing Time for the LIF Neuron}
% 	Difor er \label{appendixFiringTime} bore..



\section{Refration time and simulator time scale}
%\label{ssecValueOfAlpha}

The inter--spike interval for a neuron consists of two phases. 
The absolute refraction period and the depolarizing phase (se sec. \ref{ssecTheActionPotential}).
% % % 
Equation \eqref{eqPeriodeligningForKonstIntraPeriodKAPPA} models the interval of the depolarizing phase of the neuron. % , $p_d(\kappa)$.
The equation for the whole inter--spike interval is given by
\begin{equation}
	p_{isi}(\kappa) = p_d(\kappa) + t_r
	\label{eqHeilePerioden}
\end{equation}

Where $t_r$ is the refraction period of the neuron. % , and $p_d(\kappa)$ is given in \eqref{eqPeriodeligningForKonstIntraPeriodKAPPA}.
If we consider the firing frequency of the neuron, $f(\kappa) = p_{isi}^{-1}(\kappa)$ we can see that the asymptote is given by
\begin{equation}
	\begin{split}
		\lim_{\kappa->\infty}{ f(\kappa)} &= \lim_{\kappa->\inf}\left( \frac{-\alpha}{\ln \left( \frac{\kappa - \tau}{\kappa} \right) - \alpha t_r} \right)   \qquad = \frac{1}{t_r} \\ 
		%\lim_{\kappa->\infty}{ f(\kappa)} &= \frac{1}{t_r}
	\end{split}
	\label{eqFrekvensLlim} 
\end{equation}

From this analysis it can be concluded that the refraction period of the neuron will limit the output frequency of the neuron.
This can be seen in fig. \ref{figFrekvensMedOgUtenRefractionPeriod}.

\begin{figure}[bhtp]
	\begin{center}
		\includegraphics[width=0.7\textwidth]{frekvensPlotRefractionPeriod}
	\end{center}
	\caption{Frequency for a neuron, with and without refraction period for the neuron.}
	\label{figFrekvensMedOgUtenRefractionPeriod}
\end{figure}

%We can see from this analysis that the refraction period of the neuron is fundamental for restricting the neurons output frequency (se fig. \ref{figFrekvensMedOgUtenRefractionPeriod}).
For biological neurons, the maximum firing frequency is about 1000 Hz \cite{NeuroscienceExploringTheBrain3edKAP4}. %s 79
\begin{equation}
	\lim_{\kappa->\infty}{ f(\kappa}) \approx 1000 \, \text{Hz}
\end{equation}
%If we define the maximum firing frequency to be 1000, equation \ref{eqFrekvensLlim} gives us the absolute refraction period as
If we define the maximum firing frequency for the artificial neuron to be 1000 Hz, from equation \ref{eqFrekvensLlim} we get the corresponding refraction period $t_r$:
\begin{equation}
	t_r = \frac{1}{1000 \text{Hz}} = 1 \, \text{m}s %= 0.001 s = 
\end{equation}

%TODO Skriv at dette er en kjendt størrelse i neuroscience (finn, referer), og er en indikasjon på rettheten til lingningene (?)
% 		Kanskje også skrive litt om at dette er "absolute refraction period". Det er også en mild refraction period etter dette (finn,referer). Dette kan implementeres ved 2ms refraction period for auronet. 
% 		TODO TODO Sjekk andre linja her, og gjør en bestemmelse i forhold til mine ANN. (1 eller 2 ms refraction period?).
If we define the time step of the simulation to be 1 m$s$, the refraction period will be one time step in the simulation.
%With a time step of 1 m$s$, the absolute refraction period (the time interval where it is impossible to exite the neuron) can be set to one time step. 
With a time step of 1 m$s$, the simulation of the refraction period can be done by blocking the input for the durion of one time iteration.
%For SANN nodes, this means that the node will not change its value for the duration of the next time step. 
%For $\kappa$ANN this can be implemented more effective by incrementing the estimated firing time by one time iteration. 



\section{Activation level recalculation} 		%todo todo todo todo todo todo todo todo todo todo todo todo todo todo todo 
\label{appendixRecalculateKappaClass}
	The concept of edge transmissions as the derived potentially gives an increase in the efficiency of the simulation, as only the necessary additions have to be executed.
	The value is found as the sum of all such edge transmissions, and the effect of an altered activation level is computed after the time step.
	As the activation level is found as the sum of all preceding edge transmissions, small numerical errors is also integrated and could give a large deviation from the correct activation level.
	Because of this, an adaptive mechanism for recalculation of the activation variable $\kappa$ is devised.

	The size of the error is hard to estimate, as it can vary with the hardware architecture, the system load and the number of input transmissions to the node in question.
	%The size of the error from one time step is hard to estimate, as this varies with the number of inputs in the course of the time step.
	Because of this, the number of time steps between each recalculation in a node is designed to be adaptive.
	When the activation variable have a small deviation from the actual activation level, the interval to the next recalculation can be set higher than if the deviation is large.

	It is important to limit both the minimal and maximal period between recalculation of $\kappa$.
	This is achieved by the altered sigmoid function \eqref{eqIntervalToNextRecalculationOfKappa}. %, shown in fig. \ref{figIntervalToNextRecalculationOfKappa}.
	
\begin{equation}
	p_e(E) = (c_1 + c_2) - \frac{c_2}{1+e^{-(c_4\cdot E - c_3)}}
	\label{eqIntervalToNextRecalculationOfKappa}
\end{equation}

	From equation \eqref{eqIntervalToNextRecalculationOfKappa}, it can be observed that the altered sigmoid function has a maximal value of $c_1+c_2$.
	In fig. \ref{figIntervalToNextRecalculationOfKappa}, $c_1=100$ and $c_2=250$ gives the maximal interval of $350$ time steps between recalculation.
	Because of a small value for the $\kappa$ errors while experimenting with this aspect, the minimal period between recalculations was set to $c_1 = 100$ iterations.
	This can easily be adjusted if $\kappa$ errors become an issue.
	 
	%As indicated in fig. \ref{figIntervalToNextRecalculationOfKappa}, 
	%	this function gives a maximal interval defined by $c_1+c_2$ when the error is zero and a minimal period of size $c_1$ when the error $E\to\infty$.
	%The altered sigmoid function can therefore easily be adjusted to give a different recalculation interval.

\begin{figure}[bhtp]
	\centering
	\includegraphics[width=0.9\textwidth]{intervalToNextRecalculationOfKappa}
	\caption{Plot of the altered sigmoid function \eqref{eqIntervalToNextRecalculationOfKappa} with $c_1=100$, $c_2=250$, $c_3=10$ and $c_4=0.5$.
			The minimal interval is given by $c_1$ and the maximum period by $c_1+c_2$.  }
	\label{figIntervalToNextRecalculationOfKappa}
\end{figure}

	\subsection{Implementation of \emph{recalcKappaClass}}
	% todo todo todo todo todo todo todo todo todo todo todo todo todo todo todo todo todo todo todo todo todo todo todo todo todo todo todo todo todo todo todo todo todo 

% // vim:fdm=marker:fmr=//{,//}



\bibliography{bibliografi}
%\bibliographystyle{abbrvnat}
\bibliographystyle{plain}
\end{document}

% // vim:fdm=marker:fmr=//{,//}
