

% Trenger "Table of abbrevations"
% 	- LTP, LTD, STDP
% 	- "spike" ? (at det betyr Action potential)?
% 	- SANN  - spiking ANN
% 	- SN 	- Spiking Node, a node in SANN.
% 	- KANN, KN.
% 	- LIF neuron: 	Leaky--Integrate and Fire  neuron




% Eller kanskje eg treng stikkordsregister: Trenger:
% 	- Det over og:
% 	- axo-axonic synapses
%
% 	- KN, KANN node : 	node based on a the consept of the $\kappa$ model of the neuron. Nodes of KANN
% 	- SN, SANN node : 	node based on a direct simulation of a spikin neuoron. Nodes of SANN
% 	- suptrathreshold levels : 	value higher than the firing threshold of a node.
% 	- STDP
% 	- LIF neuron
% 	- synaptic weight.
% 	- AuroSim 		:  	The implementation written in this project.
%
%


\documentclass[b5paper,11 pt]{report}
% TODO Er dette rett? Sjekk dette..


\usepackage[utf8]{inputenc}
\usepackage{graphicx} 

% sudo apt-get install texlive
%\usepackage[font=small,format=plain,labelfont=bf,up,textfont=it,up]{caption}
\usepackage[font=footnotesize,format=plain,labelfont=bf,up,textfont=it,up]{caption}


\usepackage{amsmath}
\usepackage{mathrsfs} %brukt for krølle-L : laplace

\usepackage{subfig}   % for subfigures.
\usepackage{listings} %for c++ kode
\lstset{language=c++}


\usepackage{amsthm}
\newtheorem{mydef}{Definition}


%\usepackage{pstricks} % For teikning. 		FOR mykje styr å lære..
%\usepackage{pstricks-add}
 %\let\psgrid\relax %XXX Fjærner GRID på alle pstricks teikninger. (global removal)

%\usepackage{epstopdf} % for å få med eps i pdf-latex (?)    funkerIkkje.. Gjør det manuellt for filene..



%Kanskje dette er naudsynt? For metapost:
%\DeclareGraphicsRule{*}{mps}{*}{}









%%%%%%%% Fra Kristoffer sin master:
\lstset{ %
basicstyle=\footnotesize,       % the size of the fonts that are used for the code
%numbers=left,                   % where to put the line-numbers
%numberstyle=\footnotesize,      % the size of the fonts that are used for the line-numbers
%stepnumber=1,                   % the step between two line-numbers. If it's 1 each line will be numbered
%numbersep=5pt,                  % how far the line-numbers are from the code
showspaces=false,               % show spaces adding particular underscores
showstringspaces=false,         % underline spaces within strings
showtabs=false,                 % show tabs within strings adding particular underscores
%captionpos=b,                   % sets the caption-position to bottom
%breakatwhitespace=false,        % sets if automatic breaks should only happen at whitespace
%escapeinside={\%*}{*)}          % if you want to add a comment within your code
%
%frame=single,                % adds a frame around the code
tabsize=4,                % sets default tabsize to 2 spaces
%breaklines=true,                % sets automatic line breaking
}


\author{Per R. Leikanger}
%\title{Development and Assessment of a Novel Model for Artificial Neural Networks}
\title{Development and Assessment of a Novel Model for Neural Simulation}
\date{\today}     



\begin{document}   



% TODO Neste linje skal kanskje være over abstract og maketitle? TODO
\pagestyle{empty} %get rid of header/footer for toc page

\maketitle

\begin{abstract}
abstract: Her.
\end{abstract}






\pagestyle{empty} 

\tableofcontents %put toc in
\cleardoublepage %start new page %TODO Brukes dersom [report] brukes for dokumentet.
%\clearpage

\setcounter{page}{1} %reset the page counter
\pagestyle{plain} % put headers/footers back on















\section{Introduction}


%	TODO Skriv om årsaken til like sitering: at all uncited material har eg funnet fram til sjølv. Viktig å poengtere dette i intro!\\


	The digital computer can compute tasks that can be stated as simple algorithms.
	It can compute algebra much faster than any human being, and can be an efficient tool. %is an efficient tool for ???.
	Digital systems are not directly capable of performing associative tasks or tasks involving pattern recognition or learning\cite{CITE}.
	An example of this is the anti--bot service ``citeUlike'' that makes it hard for so--called bots to e.g. register account on web sites%%
	\cite{CITE?_artikkel_som_snakker_om_citeUlike, Heimesida_til_citeUlike}.
%	To execute such tasks, the digital computer needs to run algorithms that emulates such skills.
%	%To be able to execute such tasks, the digital computer needs algorithms that enable this.
	Pattern recognition is useful in tasks that involve recognition of objects\cite{RobotisertIndustri, Bildegjenkjenning} and patterns \cite{artikkelSomViserMoenstergj.kj., enTilOmDet, endaEn, endaEn}
% TODO   Lån bok "mønstegjenkjenning" på bib, og skriv om dette! TODO 


	Bionics is a common description for technology inspired by nature\cite{CITE}. %XXX
	Artificial Neural Networks(ANNs) is an example of such technology;
		ANNs emulates the signal processing mechanism of a network biological neurons by letting each node be a simulation of the neuron\cite{CITE}. %XXX
	%Such networks of biological neurons is what will be referred to as Biological Neural Networks(BNNs) in this text.
	%Such networks of neural simulators emulates the signalling processing capabilities of networks of biological neurons, referred to as Biological Neural Networks(BNN) in this text.
	While the digital computer process information by algorithms, networks of neurons can be said to process information by 
		pattern recognition on its input\cite{CITE}. % the network's input\cite{CITE}.
	The two presented computational systems thus utilize different computational schemes, 
		but both can emulate the computational scheme of the other to accomplish certain tasks. % to be able to perform certain tasks.
	%The two presented computational systems thus utilize different computational schemes, but both can emulate the computational scheme utilized by the other. % to be able to perform certain tasks.
	By classification of input and producing output based on previously learned patterns and memory, biological neural systems are capable of performing algorithmic tasks. % with serial execution.
	Digital systems are likewise able to emulate neural abilities by simulating networks of artificial neurons\cite{CITE}. %TODO finn CITE
	%Digital systems are likewise able to emulate neural networks by utilizing ANN algorithms\cite{CITE}. %TO DO finn noke som passer!
	%Digital systems can likewise emulate neural networks by the use of ANNs\cite{CITE?}.
	%Like a biological agent is able of performing algorithms by emulating a serial computation/execution, the digital computer is able of emulating the biological computation by emulating an ANN by algorithms.
%%%%%

	%TODO Nevne litt om kor mykje det er brukt. Gjennomgang: ANN i teknologi! TODO (og da kommer det fram at det er 2.gen ANN TODO

	Neurons propagate information by discrete output transmissions\cite{CITE}.
	%As presented in sec. \ref{secBiologicalNeuralSystems}, neurons propagate information by discrete output transmissions.
	Transmissions are initialized when the depolarization of the neuron, given as a leaky integral of input, goes beyond some threshold\cite{CITE}.
%	When the integral of input goes beyond some threshold, referred to as the firing threshold, the output synapses have a transmission and the node's state is reset\cite{CITE}.
	%A transmission happens when the integral of input goes above some threshold\cite{CITE}.
	This cause an action potential in the neuron that gives transmission through all the neuron's output synapses.
	%Synaptic transmissions occur after what is referred to as action  potentials of the presynaptic neuron.
	The size of these transmissions does not vary with the magnitude of the neuron's input, 
		but is thought to be defined by the strength of the synaptic connection alone\cite{CITE}. %TODO Sitere? Eller bare hene på forrige?
%		but is thought to be a function of the strength of the synaptic connection only\cite{CITE}. %TOD O Sitere? Eller bare hene på forrige?
	It is possible to propagate information as e.g. the frequency or the exact timing of these action potentials(``spikes'')\cite{ frekvensCITE, ExactTimingCITE}.
%	It is possible to propagate information as e.g. the frequency or the exact timing of synaptic transmissions\cite{ frekvensCITE, ExactTimingCITE}.
	%It is possible to propagate information as e.g. the frequency of synaptic transmissions or the exact timing of synaptic transmissions\cite{ frekvensCITE, ExactTimingCITE}.
	%It is possible to transmit graded information in this system, e.g. as the frequency of such transmissions or the exact timing of action\cite{ frekvensCITE, ExactTimingCITE}.
	%Information can, however, be said to propagate as a consequence of the frequency of such transmissions\cite{CITE}. %TODO
%	The main branch of ANN used for technology models information propagation by a floating point number defined by the immediate input, and can thus be said to represent the neuron in the frequency domain\cite{FDP_report}. %emulated neurons. %TODO
	The main branch of ANN technology models information propagation by a floating point number, given as a function of the neuron's immediate input\cite{CITE}. %TODO CITE
	These ANNs can thus only be said to represent the neuron in the frequency domain\cite{FDP_report}.
		%and can thus be said to simulate the neuron in the frequency domain\cite{FDP_report}. %emulated neurons. %TODO
%	The main branch of ANN technology models information propagation by a floating point number. This variable is given as a funtion of the neuron's immediate input, and can thus only be said to represent the neuron in the frequency domain\cite{NEVR3004OmModellane, FDP_report}

	%Simulating the neuron in the frequency domain is a great simplication of the system.
%\begin{quote}
%	When modelling neurons in the frequency domain, information of spike timing is lost\cite{FDP_report}.
%\end{quote}

	Simulating the neuron in the frequency domain is a major simplification of the system, 
		as all information about timing is lost. %the spike timing is lost.
		%and all information of spike timing is lost in an ANN that only propagates the activation level of neurons.
	Such models can therefore not be used for exact simulations of the neuron or where the relative spike time of neurons is important\cite{NEVR3004OmModellane}. %TODO TODO Finn anna referanse. Dette er lett å finne CITE på! TODO
% TODO Skriv om Hebb, gå over mot STDP etterkvart.. 
%	Simulating the neuron in the frequency domain is a great simplification of the system, and all information of spike timing is lost.
%	An ANN that propagates a floating point number between its nodes is therefore a great simplification of the system, 
%		and can not be used for simulations to be used with scientific intent of where the relative spike time of neurons is important.
% TODO TODO TODO Skriv neste setn. på nytt. Få heller fokus på at Hebb er ustabilt! TODO TODO TODO TODO
	%As one of the main reasons behind utilizing ANNs is learning and adaptation, mechanisms for synaptic plasticity is important for ANNs.
	An element of particular importance is synaptic plasticity, what is seen as learning in neuroscience\cite{CITE, NEVR3001synPlast, FDP_report}. %KANDEL? Originalartikkel?   og kanskje FDP
%	An element of particular importance is synaptic plasticity, what is seen as the background of learning in neuroscience\cite{CITE, CITE2}.
	%XXX Local learning rule? Var ikkje det noke?
	In frequency based ANNs, local learning rules can e.g. be a funtion of the presynaptic and postsynaptic spike frequency, $r_j'$ and $r_i$.
	%\cite{CITE}.
	%In frequency based ANNs, a local learning rule has to be a funtion of the presynaptic and postsynaptic firing frequency.
	%In ANNs, this is often modelled as a function of the presynaptic and postsynaptic firing frequency\cite{CITE}.
\begin{equation}
	\Delta \omega_{ij} = \sum_j C r_i r_j' \quad,\qquad
	\begin{tabular}{l}
 		$r_j'$ 			\tiny{ is the presynaptic neuron' s firing frequency} \\
		$r_i$  			\tiny{ is the receiving neuron's firing frequency}  \\
		$\omega_{ij}$ 	\tiny{ it the synaptic weight between neuron $j$ and $i$} \\ %\tiny{ is the synaptic weight form neuron $j$ to neuron $i$} \\
		$C > 0$ \\
	\end{tabular}
	\label{eqHebbsPostulate}
\end{equation}
	where $\omega_{i,j}$ represents the magnitude of the synaptic connection between neuron $j$ and neuron $i$.
	This is a mathematical interpretation of what is referred to as ``Hebbian learning'' after Donald A. Hebb who first proposed this mechanism.
\begin{quote}
	When an axon of cell A is near enough to excite a cell B and repeatedly or persistently takes part in firing it, some growth process or metabolic change takes place in one or both cells such that A’s efficiency, as one of the cells firing B, is increased. \cite{Hebb1949Kap4}
\end{quote}

	Hebb's postulate or the mathematical interpretation presented in equation \eqref{eqHebbsPostulate} only
		describes positive change in synaptic weight and is obviously unstable;
	Any correlation between the two neurons' firing frequency would cause a stronger connection between them and thus increase the correlation.
%	Any correlation between the two neurons would cause a stronger connection between them and thus increase the correlation in firing frequency.
	Numerous attempts have been made create stable learning rules in frequency based ANNs, with still increasing complexity\cite{CITE, MASSE, MEIR, EndaFleir - EnTil}. %TODO Finn ref. FINN UT korleis eg lager cite: [2, 4-6] (streken).
%%
	%One mechanism that requires special attention is Spike Timing Dependent Plasticity(STDP)\cite{NEVR3004OmModellane}.

%TODO Finn figur. Legg inn figur om STDP, her! (eller skal det være fig. i intro?		TODO

%TODO TODO TODO Sjekk alle referanser i neste avsnitt! Har ikkje sjekka. Bare stjåle fra NEVR3003. TODO TODO TODO
	In 1987, Gustafsson et al. proposed that the synaptic weight gain from a transmission varies with the postsynaptic neuron's depolarization at the time of transmission\cite{Gustafsson03011987}. %TODO Sjekk artikkelen igjen. Er veldig usikker på om dette stemmer!
%	This opened for a graded increase in synaptic weight.
	%TODO Spesielt viktig å sjekke neste! TODO
	At about the same time, Levy and Steward found that synaptic transmission could cause Long--Term Depression, a decrease in synaptic weight\cite{Levy1983791LTDetterSTDP}. % as a consequence of synaptic transmission\cite{Levy1983791LTDetterSTDP}.
	%At about the same time, Levy and Steward found that synaptic connections could undergo Long--Term Depression, a decrease in synaptic weight as a result of synaptic transmissions\cite{Levy1983791LTDetterSTDP}.
%TODO Neste setning er litt dårlig TODO skriv om!
	These two findings makes stable local learning rules possible, as a single transmission can give graded synaptic plasticity from negative to positive weight change\cite{CITE}.
	%These two findings makes it possible with stable local learning rules, as the synapse can undergo a graded synaptic plasticity from negative to positive weight change after a single synaptic transmission\cite{CITE}.
%	These two findings give that a synapse can undergo a graded synaptic plasticity ranging from positive to negative weight change, and could be the basis of a stable local learning rule\cite{CITE}.
	%These two findings explains that a synapse can undergo a graded synaptic plasticity ranging from positive to negative weight change after transmission, and could be the basis of a stable local learning rule\cite{CITE}.
	%These two findings explains how it is possible with a graded synaptic plasticity ranging from positive to negative change in synaptic weight. %CITE? XXX
	The N-methyl-D-aspartic acid($NMDA$) receptor is found to be of a particular importance in this context.
	%The $NMDA$ receptor is found to be important in this context.
%	 %\cite{RossumStableHebbVedSTDP}, %TODO Sjekk teksten til artikkl, ikkje bare tittel..
%	\cite{NEVR3003STDP, RossumStableHebbVedSTDP},

	$NMDA$ receptors consists of ionic gates that open when the receptor is exposed to the right neurotransmitters\cite{CITE}.
	%In addition to other ions involved in neural depolarizing, this channel lets $Ca^{2+}$ ions flow into the neuron. Calcium is an ion that takes part in regulating synthesis of new receptors\cite{CITE,C}. 
	As opposed to the $AMPA$ receptor, this channel enable $Ca^{2+}$ ions to flow into the neuron.
	This ion is thought to take part in regulating the synthesis of new $AMPA$ receptors and is considered important for synaptic plasticity as well as transmission\cite{CITE}.
	%The $Ca^{2+}$ ion is thought to take part in regulating the synthesis of new $AMPA$ receptors, and is considered important for synaptic transmission and plasticity\cite{CITE}.
%%
	%The negative resting membrane potential of about  $-65mV$ pulls positively charged ions to it from outside\cite{CITE}. 
	The $NMDA$ channel is blocked by a $Mg^{2+}$ ion that covers the opening, stopping all ion flow through the channel\cite{CITE}.
	%The $NMDA$ channel is blocked by a $Mg^{2+}$ ion covering the opening, stopping all ion flow through the channel\cite{CITE}.
%%%%%%%%%%%%%
	%Outside of the $NMDA$ channel lies a $Mg^{2+}$ ion that covers the opening and blocks any flow of ions thought the channel\cite{CITE}.
	%Outside the opening, there is a $Mg^{2+}$ ion that inhibits a flow of ions thought the channel\cite{CITE}.
	%There is a $Mg^{2+}$ ion that lies in the opening and inhibits the flow of ions thought the opening\cite{CITE}.
	When the (negative) membrane potential is sufficiently depolarized, the block is no longer pulled towards the gate and ions is able to pass through the channel. %TODO TODO TODO VÆR HEILT SIKKER på at det er neg. potensial inni i forhold til utafor! TODO
	%When the (negative) membrane potential is sufficiently depolarized, the block is pushed away from the gate and ions can flow through the channel\cite{CITE}.
	%The probability of pushing away the block can be seen as a stochastic function of the membrane potential. %FINN CITE? EVT BEGRUNN MEIR!
	Due to the number of $NMDA$ receptors and variations in $Mg^{2+}$ blocks, this creates a graded magnitude of the $Ca^{2+}$ inflow and thus variations in synaptic plasticity\cite{CITE}.
	%Due to the number of $NMDA$ receptors and variations in the block in $NMDA$ receptors, this creates a graded response for the magnitude of the $Ca^{2+}$ inflow as a function of the membrane potential at the time of transmission\cite{CITE}. %TODO
	Synaptic plasticity can thus be modelled as a function of the postsynaptic membrane potential at the time of transmission\cite{CITE}.
%	Synaptic plasticity thus have a graded response of the postsynaptic membrane potential at the time of transmission\cite{CITE}.
%	It can therefore be said that synaptic plasticity is a consequence of the postsynaptic membrane potential at the time of transmission.
	The postsynaptic depolarization often has a correlation with how much time there is left until firing, and might be the reason why this mechanism is called Spike--Time Dependent Plasticity(STDP).
%	The mechanistic model for synaptic plasticity can be considered one for the main reasons behind simulating neurons in the time domain.
%	STDP is intrinsically stable, something that enable a higher [FORSTERKNING] of synaptic plasticity.
	This mechanistic learning rule gives a strong motivation for utilizing spiking neuron simulations in the nodes of an ANN.
	%STDP is a mechanistic learning rule that gives a strong motivation for utilizing spiking neuron simulations in the nodes of an ANN.
	%This mechanistic learning rule creates a strong motivation for utilizing spiking neuron simulations for the nodes in the ANN. 
	Such ANNs are often referred to as Spiking Artificial Neural Networks(SANN).
	%This creates a graded responce of the $NMDA-R$ dependent on the depolarization of the neuron before transmission\cite{CITE}.
	
	Despite its advantages, SANN is seldomly used in technology\cite{CITE}.
	This is partially caused by the computational complexity of SANN simulations\cite{CITE}.
	MASSE SKAL TIL FOR AT SANN ER ERROR-fri.
	The theory for frequency based ANNs is also well established, and can not direcly be used in SANN.
	FREKVENS-simulering kan ikkje overføres til SPIKE-simulering. Float-variabel vs. diskrete overføringer.
	-Error: kan ungåes ved å ha fleire computational time steps. MEN dette gjør simulering tyngre.
	
	Error har i dette arbeidet blitt analysert, og der sees at NIM har akkumulasjon av feil.
	Forsøk på å unngå dette : KANN



%	It is shown that the speed at which biological neural systems compute certain input, 
%		can not be achieved unless the spike timing is a part of the neural computation. %XXX Cite
%	SKRIVE OM InnØret og VentroLateralNuclei?
%%
%	Networks of nodes that simulates signal propagation by spikes have been referred to as Spiking Artificial Neural Networks(SANN). %XXX Cite
%	SANN have been used for simulations of neural systems[CITER] as well as for technology[CITE]. %TODO TODO
%%	It is used for technology both to approach the fast computation of biological neural systems as well as its learning capabilities.

\newpage

	Despite its advantages, SANN have not been used much in technology.  %TODO CITE!
	This is partially because of the computational complexity of SANN simulations[CITE], 
		and because ``frequency based ANNs'' is well established.
	In this work, an attempt to make spiking neuron simulations more effective is conducted.
	SNAKKE om at det beste er om det går an å lage en modell som har muligheten for å benytte 2.gen teorier for simulering av SANN.


Motivasjon! Kvifor simulere neuronet!

Problemet!  - (Såppass kaotisk med ANN at feil kan føre til enorme utslag)
			- feil som gir feil oppførsel [VEID OPPIMOT]  for treig
Skriv kven denne teksten er skrevet for. Skriv at eg ikkje går ut fra noe neuroscience bakgrunn, men en grunnleggende matematisk bakgrunn og en relativt god oversikt over C++ er gått ut fra at leser har..
%Skriv kva gruppe eg skriv til. Kva bakgrunn ser eg for meg at dei har. Eg har tatt med litt bakgrunnsinformasjon om nevro dersom leser ikkje har utpregende kjennskap til dette området. Eg har også forsøkt å ta med litt meir i avsnittene som omhandler C++, ettersom leser også kan være fra neuroscience minjøer uten utpreget kjennskap til programmering. Tilfeller vil difor oppstå der leser har god kunnskap til området, og i dette tilfellet bes leser å skim these sections.


	

%	Nevn LIFE ---  sjå ssayNEVR3002\_proprieception 
%
%	Simulering av SN: feilaktig simuleing: kvifor der det drit å få feil?\\
%	
%	Vanskelig å unngå akumulativ feil!\\



%	Elements like robustness, ability to handle fuzzy information, fault and failure tolerance, and learning makes ANNs the best tool for handling certain input\cite{jainEtAl}.

% Mi avgrensning.  --Kva har eg utelatt. Kvifor. osv..
	Skriv at eg bare har sett på LIF neuronet. Men dette er den mest brukte neuron modellen i SANN simulering.
	Ser bare på enkleste form for numerisk integrasjon. Kvifor? Begrunn. Skriv at forbedringer kan mest sansynlig gjøres med like stor impact for begge modeller..



%Kort motivasjon for å lese vidare. Tenk at dette er dritkjedelig for sensor, og at eg må motivere han til å lese vidare (eit avsnitt)
%
%  Ting som må være med :
% 		A statement med målet til teksten. Kvifor vart prosjektet gjort?
% 		Nok bakgrunnsinfo for å forstå kvifor det er viktig å lese vidare.
% 		Proper accnowledgement of previous work on which I am building. Nok referanser til at leser kan gå til biblioteket og finne støttelitteratur før han leser vidare.
% 		The introduction should be focused on the thesis question(s).  All cited work should be directly relevent to the goals of the thesis.  
% 		Scope of project: Kva er med, og kva er ikkje!
% 		Verbal table of context. Vær sikker på at det er veldig klart kva som er bakgrunnsinfo og kva som er mitt arbeid.


%XXX RAPPORT DISPOSISJON:
% Motivasjon: Kvifor er dette gjort?
	Det er mykje dataen ikkje er så flink på. \\
	For slike oppgaver brukes ofte ANN\\
	
	Nevrovitenskapen's fokus på spike times har flere årsaker. Læring og også signal processing i noen spesielle situasjoner.
	Dette har smittet over på computational neuroscience, både for bruk i simulering av nye teorier og påtenkt: for ANN i teknologi.
	SANN i ANN er veldig nytt felt, og er ikkje skikkelig utprøvd kanskje hovudsaklig på grunn av at dette krever tunge utregninger(lang tid).

% Problemområde: Kva er gjort?
	Det er kanskje spesiellt nyttbart for å utvikle proteser i biomedisinsk bevegelse, etter at LIFE er utviklet.
	LIFE er ...  og muliggjør en direkte kommunikasjon mellom neuron og teknologi. Dette krever avansert signalbehandlign som kan forstå signala.
	Dette tenker eg er veldig gunstig å gjøre med nevro-emulatorer lagt som simuleringer av neuronet i datamaskina.
	Dette skaper en spesiell motivasjon for å utvikle SANN slik at det er mulig å nytte i teknologi(sanntids-utførelse).

% Kva har andre gjort? 		(skive om SANN)
	SANN består i dag av mange noder som simulerer neuronet ved numerisk integrasjon av input.
	Det er en balanse mellom gode simuleringsresultat og lave computational costs(effektivitet); Dersom man vil ha mindre simuleringsfeil kan man øke den temporale oppløyringa, men dette øker computational load.
% Mi avgrensning.  --Kva har eg utelatt. Kvifor. osv..
	I dette arbeidet vil eg sjå på muligheten for å simulere neuronet ved å bruke den algebraiske løysinga til neuron modellen som er brukt.
	Modellen som er sett på er 'the Leaky Integrate-and-Fire(LIF) neuron model', da dette er den modellen som oftest er brukt for SANN.'
	Andre neuron modeller har ikkje blitt undersøkt, selv om metoden i bruk absolutt kan nyttes for desse også.
% Disposisjon: Utvida innholdsfortegnelse.

	%Skriv kva gruppe eg skriv til. Kva bakgrunn ser eg for meg at dei har. Eg har tatt med litt bakgrunnsinformasjon om nevro dersom leser ikkje har utpregende kjennskap til dette området. Eg har også forsøkt å ta med litt meir i avsnittene som omhandler C++, ettersom leser også kan være fra neuroscience minjøer uten utpreget kjennskap til programmering. Tilfeller vil difor oppstå der leser har god kunnskap til området, og i dette tilfellet bes leser å skim these sections.
	% Skriv: Eg har plassert eit kapittel i appendix om synaptisk overføring i appendix. Dette er for å få en meir helhetlig oversikt over det biologiske systemet, men siden det ikkje er direkte knyttet til oppgaven(dårlig formulert) er det ikkje med som en del av hovedteksten. Dette kan brukes for utfyllende informasjon og motivasjon for SANN.









% // vim:fdm=marker:fmr=//{,//}


\chapter{Background Theory}
	Before we looking av Artificial Neural Networks(ANN), different aspects of ANN history and biology have to be introduced.
	This chapter is intended to introduce the reader to the necessary background information to make the theory in this work comprehendable.

	First, theory about the simulated system is ... OSV (sjå FDP)...

	Introduksjon til kapittel: motiver leser til å få bakgrunnsinfo!
	-Neurale systemet er en shitbra signal-prosesseringsenhet som har en del egenskaper som digitale signal-processore ikkje har- List opp..
	
% DISPOSISJON
% 	Intro: Begynn beldig vidt: "neuron" er en samlebetegnelse for en spesiell type celler med signalling properties (kan bli eksitert). [REF].
%  		IDE: -> Skriv litt om at den vide grupperingen gjør at simulering er vanskelig (men ikkje her: MEN i ANN-seciton!)
%  		- Dette ligger i komplekse neurale nettvert med "recurrent connections" og høg grad av 'connectivity'.(I CNS er det estimert til å være 1000+ input per neuron)
% 		- Koblingen mellom 
% 




% "Although the human brain contains an extraordinary number of these cells (in the order 10E11 neurons), which can be classified into at least a thousand different types, all nerve cells share the same architecture." Kandell kap 2.



\section{Biological Neural Systems}
	
	In the late 1800s, Camillo Golgi developed a way of staining nervous tissue so that complex networks became apperent in nervous tissue.
	%In the late 1800s, Camillo Golgi developed a way of staining nervous tissue so that complex networks of nerve cells (neurons) became apperent.
	Santiago Ramon Y Cajal used Golgi's technique in such a way that individual neurons could be separated, and it was observed that nervous tissue was not a continuous web but a network of discrete cells. 
	He proposed what has later been known as the neuron doctrine; That the basis of intelligence is individual ``brain cells'' that can process incoming transmissions and 
		can send output transmissions as a consequence of receiving enough depolarizing input transmissions.
	For their contribution, Ramon Y Cajal and Golgi shared the 1906 Nobel's price in Physiology and Medicine.
	\cite{NeuroscienceExploringTheBrain3edKAP2}
	%TODO Cite Bear kap. 2.

	Modern neuroscience follows the neuron doctrine. 
	Each node in a neural network is called a neuron and the connection between neurons are called synapses.
	When the presynaptic neuron ``fires'' an action potential, the following synaptic transmission cause the postsynaptic neuron to become excited or inhibited.
	%All synapses have direction(propagates information in one direction) and transmits if the presynaptic neuron ``fires'' an action potential.
	%Synapses can be exitatory or inhibitory.
	Transmissions in excitatory synapses increase the postsynaptic membrane potential, causing that neuron to approach firing. %sending an action potential.
	Inhibitory transmissions does the opposite, and inhibits the postsynaptic neuron with respect to firing.
	%When a neuron fires an action potential, a transmission is eventually initialized for all the neuron's output synapses.  % redundant? eller viktig å poengtere?
	When the neuron fires an action potential, a transmission in all the neuron's output synapses is the result.
	\cite{PrinciplesOfNeuralScience4edKAP02}
% TODO TODO TODO CITE!


% TODO TODO FINN BEDRE FIGUR! TODO TODO TODO 
\begin{figure}[hbt!p]
	\centering
	\includegraphics[width=0.65\textwidth]{ModellAvNeuronet}
	\caption{A illustrative model of the neuron. The signal propagation goes from the left to the right in this figure;
			Synaptic integration at the dendrites, action potential through the axon and finally transmission throught the output synapses. 
			The aspects of the cell body is not immediately relevant to signal processing, and is not taken into account in the model used. }
	\label{figFigurAvNeuronet}
\end{figure}
	
	In this section, the most important elements of neural signal processing are presented, enabling the reader to become more familiar with how neural networks process information.
	%It is recommended that the reader utilize this section as a reference work when different methods of neural simulation are presented.
	It is recommended that the reader utilize this section as a reference work when different methods of neural simulation are discussed, later in the text. % la til "later in this text", no.  Er det bra?
	% DERSOM NESTE SETNING skal være med: Skriv om "the neuron", slik at dette er sant!
	%Before discussing how the neuron processes information, the organization of the neuron have to be reviewed. %REVIEWED er dårlig TODO
	%We will start with the organization of the neuron, before moving on to a fundamental signal processing mechanism of the neuron; The electrochemical properties of the neuron membrane.




%	The most important elements of signal processing in neural systems is presented in this section,
%		enabling the reader to become familiar with how biological neural networks function.
%	Fundamental elements for the neural signal processing is presented, and it is recommended to use this section as a work of reference when 	we later discuss simulating the biological neuron.
																																				%going through the remainder of this text.
	



	

%Using Golgi's technique, Ramon Y Cajal, Santiago stained nervous tissue in such a way that individual neurons could be separated and 




		
% 	1) Membran som er stengt for ion
% 	2) Dendrite som som mottar depolarizing input 
% 	3) Axon
% 	4) Synapse
% Avslutt med å skrive litt om retninga til signalet: inn i dendritt (eller soma), integrert i intracellular fluid (ser på det nært axon hillock), overstiger terskel i axon hillock, AP i axon, syn.trans.
	
	\subsection{The Neuron}
		% TODO TODO TODO Dersom det passer: Skriv om lay-out til neuronet. (Dette er lovet fra setninga, over). Evt. fjærn denne setninga!)
		%
		In terms from graph theory, a biological neural network is a directed, cyclic graph.
		The nodes are called neurons and the edges between the nodes are called synapses.
		% Skriv om: neste setning! Vil heller formidle at i dette kapittelet skal eg gå gjennom geografien til neuronet.
		In addition to the synapse, the neuron contain three elements that are fundamental for signal processing.
		% skrive at det er eit PHOSPHOlipid bilayer?
		The most important element is the neuronal membrane. % that is a lipid bilayer with a low permeability to ions, enabling an electrical potential across the cell membrane.
		%The neuron membrane is a lipid bilayer that has a low permeability to ions, enabling an electrical potential across the cell membrane.
% 		%
%		The axon hillock initiates an action potential when the membrane potential becomes more positive than the firing threshold, causing an action potential to propagate through the axon.
%%
% 	Skal eg liste opp alle her, eller skal eg bare nevne det, og gå vidare til membran?		

		



	%xxx 1) Membran som er stengt for ion
		Each neuron is surrounded by a phospholipid bilayer cell membrane with a low permeability to ions, enabling a different concentration of ions over the membrane.
		%5													%%												%enabling the intracellular fluid to have a different consectration of different ions than the fluid outside the membrane.
		%under: ikkje "spesialiserte", men "dedidkerte"
		All neuron membranes has ionic pumps dedicated to create an ionic concentration gradient over the membrane.
		The different ionic pumps push the corresponding ions ``upstream'' in relation to the ionic concentration gradient, resulting in an electrochemical potential over the membrane.
		%The membrane potential at rest generally lies at about $-65mV$.
		The resting membrane potential of a neuron generally lies at about $-65mV$. 
		\cite{NeuroscienceExploringTheBrain3edKAP3} %XXX Dersom dei to avsnitta over blir kobla sammen: flytt denne under heile avsnittet. TODO Do this?
%TODO CITE: Bear kap 3 ?

		If specialized ionic gates permeable to certain ions are opened, these ions can flow freely through the gate.
		Depending on which ions are let thought, the neuron membrane is either hyperpolarized(more negative membrane potential) or depolarized(more positive membrane potential).
		%Depending on which ions are let thought, the neuron membrane to either be hyperpolarized(more negative membrane potential) or depolarized(more positive membrane potential).
		%This cause the neuron membrane to either be hyperpolarized(more negative membrane potential) or depolarized(more positive membrane potential), depending on which ions are let thought.
		If the membrane potential becomes more positive than the firing threshold of the neuron, an action potential is initiated at the axon hillock at the base of the neuron's axon. 
		\cite{PrinciplesOfNeuralScience4edKAP07}
		%\cite{NeuroscienceExploringTheBrain3edKAP3}\cite{PrinciplesOfNeuralScience4edKAP07} %XXX Dersom dei to avsnitta over blir kobla sammen. TODO Do this?
		% TODO Med unntak av siste linje kan alt finnes i Kandel:07. Siterer difor bare K:07. Er dette rett?
%TODO CITE: Bear kap 3 og 4. F.eks.     (Gjelder om det var kobla med det over.. No er det kanskje nok å cite en av kap. ?)
%TODO 		Eller: Kandell kap 7,9 (?)



%TODO TODO TODO Finn figur som handler om depol. og memb. pot. TODO TODO TODO




% Lag ny plan for det som no kommer. Skal jo skrive om denrite og axon i egene avsnitt. No trenger eg bare å skrive om korleis lay-out er på neuronet!
% 	- komme inn på at signalet går i en retning: fra dendrite til axon-terminal.
% 	- nevne såvidt kva som skjer ved axon terminal (synaptic vesicles og NT). NT spres gjennom noko kalla syn.cleft, og exiterer eller inhibits postsyn neuron.


	\subsection{The Axon and the Action Potential}

		Voltage--gated sodium and potassium channels are located along the membrane of the axon.
		If the membrane potential is more positive than the ``firing threshold'' of the neuron, these channels open, causing the membrane to have a transient positive increase in membrane potential.
		%In the membrane of the axon there are voltage--gated sodium and potassium channels that open if the level of depolarization is more than some value, referred to as the ``firing threshold'' of the neuron.
%This value is referred to as the ``firing threshold'' of the neuron.
		%When these channels are activated, the membrane potential will transiently have a large positive increase in membrane potential.
		Through passive transmission of the electrical charge due to diffusion of ions, the membrane potential at the next site of voltage gated channels becomes more positive than the firing threshold. %, and the process is repeated.
		The process is repeated, and the active propagation of the action potential continues until the end of the axon(the axon terminal).
		\cite{PrinciplesOfNeuralScience4edKAP09}
%%%Through passive transmission of the electrical potential, the next voltage gated channels will open as a result of going above the gate threshold. 
		%This establishes the active aspect of action potential propagation, and results in a self carrying propagating through the axon.
		%T0DO CITE: \cite{PrinciplesOfNeuralScience4edKAP09}.


		The two most important voltage gated channels for the active part of action potential propagation are the sodium and the potassium channels.
		%The $Na^{2+}$ channel is more responsive than the $K^+$ channel, and enables the $Na^{2+}$ ion that has the highest concentration outside the cell to flow into the cell.
		%This cause the membrane potential to become more positive.
		The $Na^{2+}$ channel is most responsive and enables the $Na^{2+}$ ion to flow freely. %, that has the highest concentration outside the cell, to flow into the cell.
		%The highest concentration of $Na^{2+}$ ions is on the outside of the neuron, resulting in a flow of positively charged ions into the neuron.
		The highest concentration of $Na^{2+}$ ions is on the outside of the neuron, resulting in an inflow of positively charged ions that depolarize the neuron.
		%%
		The potassium ion has the highest concentration inside the cell, and activation of the somewhat slower $K^+$ channel cause a flow of positively charged ions out of the cell.
		%Because both channels close after a short while, and the $K^+$ channel is slower than the $Na^{2+}$ channel, the action potential cause a membrane potential as shown in fig. \ref{figActionPotential}. 
		%Because both channels close after a short while, and the $K^+$ channel is slower than the $Na^{2+}$ channel, the action potential cause a transient membrane potential as shown in fig. \ref{figActionPotential}. 
		Because the $K^+$ channel is slightly less responsive than the $Na^{2+}$ channel, and that both channels only stay open for a short while, 
			the transient membrane potential of the action potential have a form as shown in fig. \ref{figActionPotential}. 
		\cite{PrinciplesOfNeuralScience4edKAP09}
		%TODO CITE: \cite{PrinciplesOfNeuralScience4edKAP09}.
	

\begin{figure}[hbt!p]
    \centering
    %\includegraphics[width=0.65\textwidth]{AP_IonFlow}
    \includegraphics[width=0.65\textwidth]{AP_IonFlow2}
 	  \caption{The action potential. Activation of the $Na^{2+}$ channel cause positively charged ions to flow into the neuron, depolarizing the neuron. The slower $K^+$ channel has the opposite effect.
				Both channels close after a short while, and the membrane potential returns to the resting value after a small overshoot.
				(Figure from \cite{PrinciplesOfNeuralScience4edKAP09}).
				%Figure taken from  \cite{PrinciplesOfNeuralScience4edKAP09}, page 158.
			%SKRIV MEIR! %TODO TODO TODO
			}
    \label{figActionPotential}
\end{figure}

% TODO TODO TODO TODO Før eller etter neste avsnitt: Skriv om at signalet er konstant gjennom heile aksonet, causing the syn.trans. til å være av samme størrelse. 
% TODO Trur eg skal skrive om dette heilt sist i dette avsnittet, for å lede inn på syn.trans.


		After a successful opening of the voltage--gated ion channels, internal mechanisms of the channels cause the ion channels to close again.
		The channels stay closed for a short while, enabling the active sodium--potassium pump to reestablish the ionic concentration gradient over the membrane.
		During this time, the voltage--gated channels remain closed, and it is impossible to elicit a new action potential.
		This mechanism cause what is referred to as the absolute refraction period for the neuron, and is important both to hinder the action potential to ``travel back'' along the axon and to limit the firing frequency of the neuron.
		\cite{NeuroscienceExploringTheBrain3edKAP4} %TODO TODO TODO Sjekk: les gjennom kapittelet, og se om alt stemmer! TODO TODO TODO
		%TODO CITE f.eks. Bear, kap 4

		%When the action potential reach the axon terminal at the distal end of the axon, 
		The axon is organized as a tree, with a trunk by the soma called the axon hillock.
		The branches of the ``axonic tree'' are called axon collaterals. % and split the axon in two. %%% TA MED SISTE?
		At the far end of each axon collateral, lies the axon terminal where the neuron's output synapses are located.
		\cite{NeuroscienceExploringTheBrain3edKAP2}
		% TODO CITE!

%		%TODO TODO TODO TODO TODO TODO TODO TODO TODO TODO TODO TODO TODO TODO TODO  Skriv ferdig siste avsnittet som skal lede leser over på synapsen! TODO TODO TODO TODO TODO TODO TODO TODO TODO TODO TODO TODO TODO TODO TODO 
%		An important result of the active signal propagation of the action potential is that the membrane potential of the action potential is the same OVER HEILE AKSONET:
% %TODO Heilt på slutten: For å lede leser over på synapsen:
% An important result of this is that the presynaptic membrane at the synapses receives an equal depolarization, independent of its location. 
% The release of neurotransmitters is a result of this depolarization, and the action potential enshures an equal depolarization of the presynaptic membrane.
% This gives that the transmission is dependent only on the synaptic connection, or ``the synaptic weight''.
% \cite{PrinciplesOfNeuralScience4edKAP09}














% //{ GAMMELT:
% ------------------
% GAMMELT:
% ------------------
% 	
% %	The electrochemical charge, represented by charged ions or molecules, spreads passively through the neuron.
% %	When the membrane potential at the axon hillock is more positive than the firing threshold, an action potential is initialized.
% %%%%%%%%
% 	The action potential is an active mechanism in the axon membrane that temporarily opens voltage--gated $K^+$ and $Na^{2+}$ channels as a result to the membrane potential to be more positive than the firing threshold. 
% % TODO Ikkje skriv for mykje om axonet: Spar litt til ssecTheAxon!
% 	This mechanism cause the membrane potential to become more positive, and an unstable mechanism that temporarily depolarize the axon membrane is the result.
% 	The ions diffuse in the intracellular fluid in the axon, causing $K^+$ and $Na^{2+}$ channels are opened further along the axon.
% 	This is repeated until the signal reach the axon terminal at the distal end of the axon.
% 
% 	Because the action potential propagation is active, the intracellular signal does not diminish along the axon.
% 	This cause the membrane potential to be of a relatively constant size, irrespective of distance traveled.
% 	This enables the axon to be LASKDJFLAKSJDFLAKSJDFLKJ, enabling the axon to 
% 
% %	SIDEN AP ER AKTIVT, minker ikkje signalet!
% 
% 
% % skrive meir? :	The $K^+$ channel ER RASKEST.
% 	
% 	The neuron's output synapses lies at different the axon terminals in the axon.
% 
% %-----------
% %	This mechanism cause the membrane potential to become more positive, and we have an unstable mechanism that cause the membrane potential to become temporally depolarized.
% %	%This cause the signal to actively be propagated along the axon, and makes the distance of a synapse along the axon irrelevant to the transmitted signal.
% %	When this signal reach the axon terminal, the synapses at the axon terminal is activated and synaptic transmission is the result.
% %	This signal does not decrease with the distance along the axon, causing distal synapses to receive the same signal as early synapses along the axon.
% 	
% 
% 	
% 	
% 
% 
% \newpage
% 
% 
% GAMMELT Fra into-delen:
% % TODO Drit i dendritt!
% 	%xxx 2) Dendrite som mottar depolarizing input
%  		%%% 						%% 		endre   receives   til eit anna ord!
% 		The dendrite is the part of the neuron that receives most of the excitatory input connections to the neuron, while most inhibitory input synapses arrive close to the cell body(soma).
% 		When the neuron receives an excitatory input transmission, the neuron is said to be ``\emph{excited}''.
% % TODO Skriv om ionekanaler? (Receptorer) ????
% 		This cause the membrane potential to become more positive, leading the neuron towards firing an action potential.
% 		Most inhibitory input synapses arrive close to the soma, and cause the membrane potential to become more negative, or hyperpolarized.
% 		This inhibits neuronal firing. %TODO Fortsett her, go CITE
% 		%Most excitatory input connections arrive at the dendrite. %CITE
% 
% 	%xxx 3) Axon
% 		When the neuron is depolarized sufficiently, an action potential is initiated at the axon hillock.
% 		The action potential is an unstable mechanism, causing a bolean (all--or--nothing) signal to spread through the axon.
% 		When this signal reaches the ``axon terminal'', where the presynaptic part of the synapse lies, different mechanisms cause neurotransmitters to be released into the synaptic cleft
%  			(see appendix \ref{appendixSecPresynapticSynapticPartOfTransmission} for a more complete discussion of the action potential and the presynaptic elements of synaptic transmission).
% 		The neurotransmitters diffuse passively through the synaptic cleft, %TODO Skriv kva syn.cleft er! TODO
% 		The ones that come in contact with postsynaptic ligand--gated receptors, activate these receptors and opens an ionic gate in the postsynaptic neuron.
% 		The resulting ion flow cause the postsynaptic neuron to become depolarized or hyperpolarized, changing the postsynaptic membrane potential.
% 		
% % 	4) Synapse
% 
% 
% 	
% 
% 
% 
% %		The neuron is a cell with a special property; 	The electrochemical properties of the neurons enables advanced signal processing
% 
%  //}



	\subsection{The Synapse}

		The size of the synaptic transmission is propotional to the membrane potential at the axon terminal.
		%TODO TODO TODO No er det for stort hopp! Gjør om, slik at det blir meir flytende overgang fra siste setninga, over, til neste setninga!
		One important result of the active signal propagation of the action potential is that the increase in neuron potential is independent of how long the signal have to travel.
		When the action potential reach the axon terminal, voltage--gated $Ca^{2+}$ channels in the active zone of the terminal opens and $Ca^{2+}$ enters the cytosol of the axon terminal. %\cite{PrinciplesOfNeuralScience4edKAP10}.
		The axon terminal contacts bag--like organelles called synaptic vesicles filled with different neurotransmitters.
		%The axon terminal contains synaptic vesicles, bag--like organelles that contain different neurotransmitters.
		When $Ca^{2+}$ enters the cytosol of the axon terminal, synaptic vesicles are pulled toward the membrane where they fuse and release its content into the synaptic cleft.
		%The neurotransmitters diffuse out in the fluid of the synaptic cleft, and some come in contact with postsynaptic ligand--gated channels that enable different ions to flow freely into/out of the postsynaptic neuron.
		The neurotransmitters diffuse out in the fluid of the synaptic cleft, and some come in contact with postsynaptic receptors. %ligand--gated channels that enable different ions to flow freely into/out of the postsynaptic neuron.
		When the right neurotransmitter bind to a specific group of receptors called ligand--gated channels, an ionic channel is opened in the postsynaptic membrane.
		Depending on the channel(and thus the ions that are let through), this can either depolarize(excite) or hyperpolarize(inhibit) the postsynaptic neuron.
						%% 											%%  , the postsynaptic neuron is either depolarized(excited) or hyperpolarized(inhibited).
		\cite{PrinciplesOfNeuralScience4edKAP10}
		%TODO Sjekk om det er rett å cite KANDELL:10
		
		%The synaptic vesicles are bag--like organelles that contain different neurotransmitters


\begin{figure}[hbt!p]
    \centering
    \includegraphics[width=0.95\textwidth]{synapticTransmissionFraKandell}	%Figure taken from  \cite{PrinciplesOfNeuralScience4edKAP10}, page 183.
 	  \caption{\textbf{Synaptic transmission in an excitatory synapse.} 
			An action potential arriving at the terminal of the presynaptic axon cause $Ca^{2+}$ to enter the presynaptic cytosol, causing synaptic vesicles to fuse with the membrane.
			The containing neurotransmitters are released into the synaptic cleft.
			The released neurotransmitters diffuse passively across the synaptic cleft and bind to specific receptors in the postsynaptic membrane, causing the connected ion channels to open.
			%This example depolarize the postsynaptic neuron. %% Redundant->sjå textbf(fig.-overskrifta)
			(Figure from \cite{PrinciplesOfNeuralScience4edKAP10}).
			}
    \label{figActionPotential}
\end{figure}

		%The size of one transmission is often referred to as the postsynaptic potential(PSP) following the transmission.
		In neural simulation, the size of a synaptic transmission at a particular synapse is often referred to as the synaptic weight of this synapse.
		The synaptic weight is plastic, and different mechanisms like long--term potentiation(LTP) and long--term depression(LTD) increase and decrease the synaptic weight.
		In neuroscience, synaptic plasticity is what is seen as the basis of learning and memory\cite{NeuroscienceExploringTheBrain3edKAP25}. %(DENNE CITE gjelder bare denne linja..) (Ta vekk?)
		%This is what is seen as the basis of learning\cite{NeuroscienceExploringTheBrain3edKAP25}. %(DENNE CITE gjelder bare denne linja..) (Ta vekk?)
		%The exact firing time have recently been found to have a large impact on synaptic plasticity.
		The subject of synaptic transmission and plasticity in biological systems have been covered extensively in appendix \ref{appendixSynPlast}.
	%TODO TODO TODO Skriv ferdig. Skal eg ha med STDP, her?
		%\cite{} MULIG: Neural Networks and Brain Function: KAP 1 (Introduction) s.7

	\subsection{Signal Propagation in the Biological Neuron}
		- oppsymmer AP, spatiotemporal effekt av axon, synaptic transmission, EPSP/IPSP, ...

% // vim:fdm=marker:fmr=//{,//}

	
%\section{Artificial Neural Networks}

% SKRIV OM! FOkuser på at det er noken ting som ikkje er så bra å gjøre i PC.
% 	Så skriv litt om pragmatic ANN (som løysing på dette)

%	\subsection{A Review of ANN History}



\section{Artificial Neural Systems : A Review of ANN History}
	\label{ssecHistoryOfANN}
	The pragmatic use neural network simulations started with the ``McCulloch--Pitts neuron'' in 1943.
	%Warren McCulloch, an early neuroscientist and the young mathematician Walter Pitts formalized the models of the neuron and proposed the first artificial neuron simulator. %artificial neural network.
	Warren McCulloch, an early neuroscientist and the young mathematician Walter Pitts initiated a formalized discussion about the mechanics of the neuron and the use of neuron simulations in technology. %artificial neural network.
	This resulted in the first neuron emulator(artificial neuron). %, later referred to as the McCulloch--Pitts neuron. 
	%When a network of nodes consisting of the artificial neural was set up, McCulloch and Pitts created the very first artificial neural network(ANN).\cite{MccullochPittsHistorie} %TODO Sjekk referansen! TODO
%
	%TODO Glatt ut: Gjør slik at det eg god flyt i teksten i det som står under her! TODO
	Artificial Neural Networks based on the McCulloch--Pitts neuron model has later been referred to as the first generation ANN\cite{Maass97networksof}.
	%What has later been referred to as the first generation ANN is based on the McCulloch--Pitts neuron\cite{Maass97networksof}.
	%One example of a first generation ANN is the Rosenblatt's Preceptron\cite{HaykinANNbok}.
	Each node is modelled as a boolean device(with an on--off response), where the node sends output if the immediate input level is large enough.
	The first generation ANN can therefore be said to be a network of simple threshold gates, %filters called threshold gates.
		and does not take into consideration the depolarization state of each node. %, and is a tremendous simplification of the biological neuron.
	%This does not take into consideration the depolarization (state) of each node, and is a tremendous simplification of the biological neuron.
	%The first generation ANN thus does not take into consideration the depolarization of the neuron, and is a great simplification of the biological neuron.
	One famous example of an ANN classified as a first generation ANN is Rosenblatt's Perceptron\cite{HaykinANNbok}.

\begin{figure}[hbt!p]
	\centering
	\includegraphics[width=0.75\textwidth]{sigmoidCurve}
	\caption[The sigmoid curve that is often used to compute the output of a node in a second generation ANN]{Sigmoid curve $\frac{1}{1+e^{-x}}$, often used as the activation funtion in a second generation ANN.}
	\label{figSigmoidCurve}
\end{figure}

	The second generation ANN gives a better simulation of the neuron. % in the frequency domain.
	%A better simulation of the neuron considers the neuron in the frequency domain.
	Each node computes the output level as a floating point number based on the immediate input to the node.
	From sec. \ref{secBiologicalNeuralSystems}, the biological neuron is introduced as a node that sends output when the depolarization goes to suprathreshold levels.
	A continuous propagation of a floating point number can therefore be said to represent the frequency of such transmissions as a function of present input.
%%%
	The function used for computing the output is referred to as the \emph{activation function} of the node, and is found to give the best results if it is given by the continuously differentiable sigmoid function\cite{HaykinANNbok}.
	%The function used for computing the output is referred to as the \emph{activation function} of the node.
	%The activation function is found to give the best results if the function is a continuously differentiable sigmoid function\cite{HaykinANNbok}.
\begin{equation}
	\sigma(x)=\frac{1}{1+e^{-x}}   %TODO TODO TODO TODO TODO TODO Lag heller en figur for å vise sigmoid function! (Bytt ut ligning med fig!) TODO TODO TODO TODO TODO TODO TODO
\end{equation}

	As the biological neuron has a state that gives the depolarization value of the neuron, the stateless computation in the McCulloch--Pitts neuron is gross simplification of the original system.
	It could be more correct to consider the neuron as being stateless in the frequency domain, and the stateless computation of a second generation ANN can be said to be more correct than in a first generation ANN.
	%Because it is more right to consider the neuron as stateless in the frequency domain, the stateless computation in a second generation ANN is more correct than the stateless computation in the McCulloch--Pitts neuron model.
	As the concept of frequency only makes sense for time intervals of a certain size, precise simulations with small computational time steps does not necessarily give accurate simulation results for an ANN of the second generation. 
	%As the concept of frequency only makes sense for time intervals of a certain size, a second generation ANN can not be used for accurate simulations with small computational time steps.
	%%%
		%%the second generation ANN only give accurate simulations for computational time intervals where is makes sense to talk about mean frequency.
%%%%%%%%%%
	%It is intuitive that the frequency representation only give good simulation results for time intervals where it makes sense to talk about mean frequency.
	%This model thus only gives accurate simulations for a very coarse temporal resolution(large computational time steps), and does not take into account temporal effects caused by the time of firing. %%
	%%This simulation can therefore only be said to only give an accurate simulation for a very coarse temporal resolution(large computational time steps).
	For more precise simulations of the neuron or simulation where temporal mechanisms in the neuron are important, the frequency representation in second generation ANNs can therefore not be used.  %TODO Finn noe å CITE!
	Such simulations can only be executed with a neuron simulation model that considers the depolarization of each neuron.
% TODO TODO TODO TODO TODO TODO TODO TODO TODO TODO TODO TODO TODO TODO TODO TODO TODO TODO TODO TODO TODO TODO TODO TODO TODO TODO TODO TODO TODO TODO TODO TODO TODO TODO TODO TODO 
% TODO Skriv: Da må depolarisasjonen explisitt simuleres. Dette gjøres i Spiking ANN TODO Bra overgang til neste subsection! TODO
% TODO TODO TODO TODO TODO TODO TODO TODO TODO TODO TODO TODO TODO TODO TODO TODO TODO TODO TODO TODO TODO TODO TODO TODO TODO TODO TODO TODO TODO TODO TODO TODO TODO TODO TODO TODO 
	%For more precise simulations of the neuron or of temporal mechanisms, the frequency representation in second generation ANNs therefore can not give good simulation results. %TODO Finn noe å CITE!
	%%This representation therefore is unsuitable for more precise simulations of neural networks and of mechanisms that depend on temporal elements(like STDP learning rules). 
%TODO TODO TODO TODO Cite :  Finn steder dette står/ting å cite! TODO TODO TODO TODO


	
% TODO Kommenterte ut følgende uten å sjekke resultatet. TODO Sjekk om alt er med når dette er kommentert ut, og i såfall: slett det.. TODO
% 	A direct simulation of the signal propagation mechanisms of the neuron is often referred to as a ``spiking'' artificial neuron model. %\cite{Maass97networksof}.
% 	The depolarization of the neuron is simulated by numerical integration of all events that change the neuron's depolarization.
% 	The most commonly used model for spiking neuron simulations is the Leaky Integrate--and--Fire(LIF) neuron, where the depolarization of a neuron is simulated as a leaky integration of depolarizing input\cite{florian03}.
% 	Because all aspects considered important in signal processing are simulated, this model can be used for precise simulation of the neuron and can therefore be used to test neuron models and theories about neural signal propagation.
% 	%TODO LINJA over: Endre litt, og cite. F.eks. Maass97networksof?? (XXX Last leddsetning does not tell what I intended..)
% 
% 	The LIF model describes the depolarization of a neuron as a leaky integration of the neuron's excitatory and inhibitory input, where the depolarization value diminish(towards the resting membrane potential) over time.
% 	%The leaky aspect of the neuron can be implemented by subtracting a certain ration of the last computed depolarization, every time iteration.
% %%	Artificial Neural Networks that utilize this simulation model for its nodes is sometimes referred to as Spiking ANN(SANN) and belong to the \emph{third generation ANN}. %TODO Cite en art. av Wulfram Gerstner
% 	%%Artificial Neural Networks that utilize this simulation model for its nodes is sometimes referred to as Spiking ANN(SANN) and belong to what is referred to as the \emph{third generation ANN}. %TODO Cite en art. av Wulfram Gerstner
% %%	
% 	When the simulated depolarization of a node is excited above the firing threshold, a spike is initiates, causing transmission through all the node's output edges. %synapses.
% 	The signal is propagated as discrete spikes, very similar to the signal processing of a biological neuron\cite{Kunkle02pulsedneural}.
% 	%TODO Setninga under: Virker  som om eg snakker om LIF neuron. Dette er feil, i såfall. Ikkje LIF som er SANN. TODO FIKS!
% 	Artificial Neural Networks with this simulation model for its nodes are sometimes referred to as Spiking ANN(SANN) and belong to the \emph{third generation ANN}\cite{Maass97networksof}. %TODO Cite en art. av Wulfram Gerstner
% 	%%Artificial Neural Networks that utilize this simulation model for its nodes is sometimes referred to as Spiking ANN(SANN) and belong to what is referred to as the \emph{third generation ANN}. %TODO Cite en art. av Wulfram Gerstner
% 	
% 
% 	To summarize this section about ANN history, there are three generations of artificial neural networks, each getting closer to the biological neuron in behaviour.
% 	%What propagates thought the network, how this is computed and what it represents differs  
% 	The first generation of artificial neurons where so--called threshold gates, with a boolean output that was [true] if the summed input were above some threshold. %TODO CITE!
% 	Nodes of the second generation gave, in some respects, a better simulation of the biological neuron.
% 	The output is not given as discrete states given by the input but as a continuous function that can be interpreted as the firing frequency of the node. % of the level of input to the node.
% 	With this interpretation it can be said that ANNs of this generation gives a simulation that is closer to the biological neuron in behaviour.
% 	%If the transmission through the output synapse of a neuron is seen as the firing frequency of that neuron, it can be said that this generation of ANN gives a simulation that is closer to the biological neuron.
% 	%The third generation ANN is supposed to be an accurate simulation of the neuron, and is as close to the biological neuron as possible.
% 	The third generation ANN is supposed to give an accurate simulation of the neuron, and achieves this by simulation the neuron's depolarization directly. 
% 	The neuron has an internal state representing the depolarization and fires if this value goes to supra--threshold levels.
% 	The signal is propagated in the same manner as in the biological neuron, where excitatory synapses increase the postsynaptic depolarization and inhibitory synapses decrease the postsynaptic depolarization.
% 	%Errors in the simulation comes as a consequence of numerical errors or from the neuron model used.
% 	Because the nodes of a SANN is a direct simulation of the mechanisms involved in the neuron's depolarization, only numerical errors and errors in the neuron model used separates the simulation results from the behaviour of a real neuron.
% 	%Only numerical errors in the digital simulation and errors in the neuron model used separates the simulated result from the behaviour of a real neuron.
% 
% %%%	XXX TA MED?    Den observante leser vil dermed se at med kvar ny generasjon ANN, så kommer vi nermere bio-neuronet. 




% 2.gen er bedre enn første, fordi begge er "state less". For tidsdomenet blir dette heilt feil. For frekvensdomenet blir det mindre feil. (Heilt rett dersom du ser bort fra syn.p. og modulatory neurotransmitters.

%It is actually so close that the word ``simulation'' will occasionally be used in this report.  	% ".. actually so closa that .." DÅRLIG. Fiks?

%In the third generation ANN it is the action potentials or the "spikes", that is responsible for information processing.  %TODO Skriv om slutten / Feil ord.. 		".. or the "spikes" that is responsible for the information flow.
%This ANN model is therefore often referred to as ``Spiking Artificial Neural Network''(SANN).

%If the transmissions between nodes is viewed as the firing frequency of the neuron, we can say that the continuous output value represents the output frequency as a function of the input frequency over the time step in the simulation.

%The nodes of the third generation ANN became even more similar to the biological neuron, as the output of a node depend solely on the state of the node.






%	\subsection{Synaptic Plasticity and motivation for SANN}
%		- skiv om Hebbian learning: ustabilt. \\
%		- skiv om STDP og at dette er en viktig grunn til å bruke SANN. \\
%		Det er truleg at begge 'learning rules' har sannhet. Det ville difor vært bra å kunne benytte begge, ivertfall i forskningssamanheng.

%TODO TODO TODO TODO TODO TODO TODO TODO TODO TODO TODO TODO TODO TODO TODO TODO TODO TODO TODO TODO TODO TODO TODO TODO TODO TODO TODO TODO TODO TODO TODO TODO TODO 
%TODO TODO TODO TODO TODO TODO TODO TODO TODO TODO TODO TODO TODO TODO TODO TODO TODO TODO TODO TODO TODO TODO TODO TODO TODO TODO TODO TODO TODO TODO TODO TODO TODO Har tenkt å endre litt:
% XXX Har gjort dette fra å være subsection til å være section(sjå under=.
% 		Dette innebærer at 	1) eg må endre label fra ssecDepolarizationSimulationByNIM til å være secSpikingANNsimulation, eller noke.
% 							2) Eg kan endre det slik at depol-error section for NIM kan ligge her? Dette må eg tenke meir på.
\section{Spiking Artificial Neural Networks}
%\subsection{Depolarization Simulation by Numerical Integration}
	\label{ssecDepolarizationSimulationByNIM}
	A node in a SANN is a simulation of the neuron's depolarization.
	When the depolarization level crosses the firing threshold from below, a spike is initiated and the signal is propagated.
	Many formal spiking neuron models exist, where the most common is the $LIF$ neuron model\cite{florian03}. %TODO Sjekk om dette står der!
%	Many formal spiking neuron models exist, but we will in this text focus on the Leaky Integrate--and--Fire(LIF) neuron.
%	The depolarization level of a neuron can be simulated utilizing different formal neuron models, but what is common for all neural simulation models is that a spike is initiated when the depolarization crosses the firing threshold.
	
	
\begin{figure}[tb!hp]
	\centering
	\includegraphics[width=0.85\textwidth]{figRCcircuit}
	\caption[A schematic diagram of the $LIF$ neuron model]
			{A schematic diagram of the $LIF$ neuron model. 
			Each node can be modelled as the circuit inside the dashed circle on the right side of the model.
			Depolarizing input is represented by the input current $I(t)$, and when the potential over the capacitor is larger than the firing threshold at time $t_i^{(f)}$, a spike $\delta(t-t_i^{(f)})$ is generated. 
			The left--hand side of the figure shows a model of synaptic transmissions as a low--pass filtering of the presynaptic action potentials $\delta(t_j^{(t)})$, generating a input current $\alpha(t-t_j^{(f)})$ to neuron $i$.
			(Figure from \cite{gerstnerKistler2002KAP04}).
% TODO TODO Amund seier at denne referansen oppfattes som om det er figur pluss teks som kommer fra artikkelen. Gjøre om på dette? Hvordan?
% TODO TODO TODO TODO TODO TODO TODO TODO TODO TODO TODO TODO TODO TODO TODO TODO TODO TODO TODO TODO TODO TODO TODO TODO TODO TODO TODO TODO TODO TODO TODO TODO TODO 
			}
	\label{figRCcircuitAvNeuronet}
\end{figure}
	The $LIF$ model is a simple phenomenological model of the biological neuron, and is highly popular due to its simplicity.
	%The neuron's depolarization is simulated by numerical integration of all membrane--potential changing input, including leakage.
	%A basic electrical $RC$ circuit, consisting of a resistance in parallel with a capacitor can be used as an analogy to explain the $LIF$ model.
	%The driving current $I_{tot}(t)$ can be split into two components, $I_R(t)$ that represents the current that ``leaks'' through the resistance and the input current $I_{input}(t)$\cite{gerstnerKistler2002KAP04}. 
	The leaky integration of the $LIF$ neuron can be modelled by the electrical circuit shown in fig. \ref{figRCcircuitAvNeuronet}.
	When the membrane potential $v(t)$ crosses the neuron's firing threshold $\tau$, a spike is initiated causing transmissions through all the neuron's output synapses.
	The neuron's membrane potential is then reset to the resting membrane potential $v_r<\tau$ \cite{gerstnerKistler2002KAP04}.
	%A spike cause transmissions through all the neuron's output synapses and the neuron's membrane potential to be reset to the resting membrane potential $v_r<\tau$.

	%If the  TODO Innfør ligning 12 fra \ref{florian03}.
	% TODO Gjør det her!
\begin{equation}
	\lim\limits_{t\to t^{(f)}; t>t^{(f)}} v(t) = v_r
\end{equation}

	To the author's knowledge, the $LIF$ neuron have previously only been simulated by numerical integration.
	%To the author's knowledge, the only way of simulating the $LIF$ neuron in the digital computer is by numerical integration.
	All depolarizing current and the leakage is integrated numerically by summing all synaptic input as discrete transmissions and subtracting the leakage after each computational time step.
	If the efficiency of the synapse from neuron $j$ to neuron $i$ is modelled by $\omega_{ij}$, the total input current can be found by 
\begin{equation}
	I_i(t_n) = \sum_j \omega_{ij} \sum_f a(t_n - t_j^{(f)}) + \xi_i(t_n)
\end{equation}
	%TODO TODO Slutten av neste setning er nok ikkje fra {florian03}. Sjekk dette, og isåfall endre neste setning.. TODO TODO Fiks det slit ag neste setning stemmer med florian03-referansen. Sjekk ligningene hans..
	where $a(t- t_j^{(f)})$ is given by the neuron's synaptic input currents and $\xi_i(t_n)$ comes from external sources of depolarizing current, e.g. current induced through a probe or the activation of a sensory neuron\cite{florian03}.

%TO O TD XXX XXX XXX HER HER HER OTOD TODO TODO TDODO TOO TODO TODO
	Leakage can be simulated by subtracting a fraction of the present membrane potential every time step.
	%%The leakage is simulated by subtracting a fraction of the present depolarization every time step.
	%The fraction is defined by the ``leakage constant'' $\alpha=\tau^{-1}$, where $\tau$ is the system's time constant.
	Because computational resources are limited, a finite temporal resolution(discrete time) is utilized.
	This involves discrete steps in simulated time, and the previously computed depolarization level have to be used to find the leakage.
	This introduces a delay of the size of the computational step:
	%The previously computed depolarization level is often used to find the leakage, delayed by the size of the computational step:
\begin{equation}
	%I_R(t_n) = \alpha v(t_{n-1})
	l(t_n) = \alpha v(t_{n-1})
\end{equation}
	The discretization of time thus introduces an error defined by the size of the computational time step;
	%This introduces an error defined by the size of the computational time step;
		If the computational time steps are increased, the size of the simulation error is also increased.
	%Accurate simulations can be designed by decreasing the size of the computational time steps, but this also increases the computational load of the simulation.
	More accurate simulations can be designed by making the computational time step smaller, but this simultaneously increase the computational load of the simulation. %Xxx , but this makes the computational load larger. ELLER NOKE. XXX
	%More accurate simulations can be designed by decreasing the computational time step, but this also increases the computational load.
	
	% 	The simplest form of the $LIF$ neuron utilize sample--and--hold numerical integration.

%	This can be simulated by implementing a direct simulation of the $LIF$ neuron's depolarization, as modelled by the electrical $RC$ circuit in fig. \ref{figRCcircuitAvNeuronet}.

% TODO ?: Write eq. 4.4, page 95 in gerstnerKistler2002KAP04 (threshold criterion).

	%The depolarization of a node is a time integral of all depolarizing input and the total ``leakage'' of depolarization value,  and is implemented by numerical integration. %and can be implemented by numerical integration.
	%The size of the leakage is computed by the last computed value for the neuron's depolarization, delayed by the size of the computational time step.
	%XXX The error from each iteration, referred to as the local truncation error, thus increases with the size of the computational time step.
	%Accurate simulations can therefore be designed by decreasing the size of the computational time step.

% TODO TODO TODO TODO TODO TODO TODO TODO TODO TODO TODO TODO TODO TODO TODO TODO TODO TODO TODO TODO TODO TODO TODO TODO TODO TODO TODO TODO TODO TODO TODO TODO TODO TODO TODO 
% TODO TODO TODO TODO TODO TODO TODO TODO TODO TODO TODO TODO TODO TODO TODO TODO TODO TODO TODO TODO TODO TODO TODO TODO TODO TODO TODO TODO TODO TODO TODO TODO TODO TODO TODO  Lag figuren som viser ANN på nytt! Litt slurvete, nå.
% TODO TODO TODO TODO TODO TODO TODO TODO TODO TODO TODO TODO TODO TODO TODO TODO TODO TODO TODO TODO TODO TODO TODO TODO TODO TODO TODO TODO TODO TODO TODO TODO TODO TODO TODO 
\begin{figure} %[hbt!p]
    \centering
    \subfloat[Model of Spiking ANN Connections]{
        \label{figAuronE:subfigModel}
        \includegraphics[width=0.45\textwidth]{auronE_fig}
    }
    %%
    %%
    \subfloat[Depolarisation Time Course]{
        \includegraphics[width=0.55\textwidth]{auronE_depol}
        \label{figAuronE:subfigDepolarization}
    }
    \caption[An artificial neural circuit to illustrate numerical integration of the $LIF$ neuron. Schematic model and simulation results.]
			{ % TODO TODO TODO Litt rart å bruke auron, her? Vil gjerne bruke auron, men dette må innføres først. Do It! TODO
			(\ref{figAuronE:subfigModel}) A schematic model of the synaptic connections in a neural circuit intended to illustrate neural integration.
			The synaptic connections in fig. \ref{figAuronE:subfigModel} are represented as a factor of the firing threshold, meaning that a transmission thought a synapse with $\omega_{ij}=1$ will cause the postsynaptic auron to fire.
			The neural circuit $[A1, A9]$ is thus self sustaining, and cause synaptic transmissions through the synapse from auron $[A*]$ to auron $E$.
			\mbox{(\ref{figAuronE:subfigDepolarization}) The} resulting depolarization curve for auron $E$.
			Every auron but $A7$ is connected to auron $E$, making the effect of leakage prominent.
			Auron $A7$ is disconnected, causing a small decrease in auron $E$'s depolarization every ninth iteration.
			%%Auron $A7$ is disconnected to show the effect of the leakage, that can be seen by the decrease in auron $E$'s depolarization every ninth iteration.
			%(The figure is generated by \emph{$_s{AuronSim}$}, a simulator utilizing numerical integration implemented in the preliminary project to this work) 
			(The figure is from the preliminary project and is generated by $auroSim_s$, the part of $auroSim$ that ulilize numeical integration)
			%(The figure is generated by \emph{${AuronSim}_s$}, the part of a simulator implemented in the preliminary project to this work that utilize numerical integration) 
			%(The figure is generated by \emph{AuronSim}, a simulator utilizing numerical integration implemented in the preliminary project to this MS thesis) %this work)
                % TODO TODO Make auron A7->A9. This is better for the text.. TODO TODO Also make figure figAuronE:subfigModel more pretty: redraw! TODO TODO TODO
				}
    \label{figExperiment2}
\end{figure}


%TODO TODO TODO TODO TODO TODO TODO TODO TODO TODO TODO TODO TODO TODO TODO TODO TODO TODO TODO TODO TODO TODO TODO TODO TODO TODO TODO TODO TODO TODO TODO TODO TODO 
%TODO Flytt neste 5 linjene til en anna plass. Funker ikkje så bra her?  Kanskje heller i experiment: Efficiency Mesurements (?)..
%TODO TODO TODO TODO TODO TODO TODO TODO TODO TODO TODO TODO TODO TODO TODO TODO TODO TODO TODO TODO TODO TODO TODO TODO TODO TODO TODO TODO TODO TODO TODO TODO TODO 
% 	If all nodes are updated each time step, the computational load scale linearly with the number of nodes and the inverse of the size of the computational time step.
% 	By halving the size of the computational time step, the computational load increase as if the number of nodes are doubled.
% 	By having precise simulation algorithms, fewer time iterations can be utilized to accomplish the same accuracy for the simulation.
% 	This explains that the accuracy of simulation algorithms can be used as a good measure of efficiency, and establish the motivation for having precise simulation algorithms.
% 	More sophisticated numerical integration techniques are therefore often used to accomplish a high accuracy in numerical simulations\cite{PlesserStraubeMorrisonPlesser2007}.


% TODO TODO TODO TODO TODO TODO TODO TODO TODO TODO TODO TODO TODO TODO TODO TODO 
% TODO TODO TODO TODO TODO TODO TODO TODO TODO TODO TODO TODO TODO TODO TODO TODO  Flytt det over, over til comparison of the two models!
% TODO TODO TODO TODO TODO TODO TODO TODO TODO TODO TODO TODO TODO TODO TODO TODO  		(Bra opplegg, som eg bør skrive om!) Fokuser dermed på at spatiotemporal delay simuleres og bruker resurser.
% TODO TODO TODO TODO TODO TODO TODO TODO TODO TODO TODO TODO TODO TODO TODO TODO 


		
% // vim:fdm=marker:fmr=//{,//}

% 	\input{NIM-simulations}




%\chapter{Articles}
%	

\section{Introduction}


%	TODO Skriv om årsaken til like sitering: at all uncited material har eg funnet fram til sjølv. Viktig å poengtere dette i intro!\\


	The digital computer can compute tasks that can be stated as simple algorithms.
	It can compute algebra much faster than any human being, and can be an efficient tool. %is an efficient tool for ???.
	Digital systems are not directly capable of performing associative tasks or tasks involving pattern recognition or learning\cite{CITE}.
	An example of this is the anti--bot service ``citeUlike'' that makes it hard for so--called bots to e.g. register account on web sites%%
	\cite{CITE?_artikkel_som_snakker_om_citeUlike, Heimesida_til_citeUlike}.
%	To execute such tasks, the digital computer needs to run algorithms that emulates such skills.
%	%To be able to execute such tasks, the digital computer needs algorithms that enable this.
	Pattern recognition is useful in tasks that involve recognition of objects\cite{RobotisertIndustri, Bildegjenkjenning} and patterns \cite{artikkelSomViserMoenstergj.kj., enTilOmDet, endaEn, endaEn}
% TODO   Lån bok "mønstegjenkjenning" på bib, og skriv om dette! TODO 


	Bionics is a common description for technology inspired by nature\cite{CITE}. %XXX
	Artificial Neural Networks(ANNs) is an example of such technology;
		ANNs emulates the signal processing mechanism of a network biological neurons by letting each node be a simulation of the neuron\cite{CITE}. %XXX
	%Such networks of biological neurons is what will be referred to as Biological Neural Networks(BNNs) in this text.
	%Such networks of neural simulators emulates the signalling processing capabilities of networks of biological neurons, referred to as Biological Neural Networks(BNN) in this text.
	While the digital computer process information by algorithms, networks of neurons can be said to process information by 
		pattern recognition on its input\cite{CITE}. % the network's input\cite{CITE}.
	The two presented computational systems thus utilize different computational schemes, 
		but both can emulate the computational scheme of the other to accomplish certain tasks. % to be able to perform certain tasks.
	%The two presented computational systems thus utilize different computational schemes, but both can emulate the computational scheme utilized by the other. % to be able to perform certain tasks.
	By classification of input and producing output based on previously learned patterns and memory, biological neural systems are capable of performing algorithmic tasks. % with serial execution.
	Digital systems are likewise able to emulate neural abilities by simulating networks of artificial neurons\cite{CITE}. %TODO finn CITE
	%Digital systems are likewise able to emulate neural networks by utilizing ANN algorithms\cite{CITE}. %TO DO finn noke som passer!
	%Digital systems can likewise emulate neural networks by the use of ANNs\cite{CITE?}.
	%Like a biological agent is able of performing algorithms by emulating a serial computation/execution, the digital computer is able of emulating the biological computation by emulating an ANN by algorithms.
%%%%%

	%TODO Nevne litt om kor mykje det er brukt. Gjennomgang: ANN i teknologi! TODO (og da kommer det fram at det er 2.gen ANN TODO

	Neurons propagate information by discrete output transmissions\cite{CITE}.
	%As presented in sec. \ref{secBiologicalNeuralSystems}, neurons propagate information by discrete output transmissions.
	Transmissions are initialized when the depolarization of the neuron, given as a leaky integral of input, goes beyond some threshold\cite{CITE}.
%	When the integral of input goes beyond some threshold, referred to as the firing threshold, the output synapses have a transmission and the node's state is reset\cite{CITE}.
	%A transmission happens when the integral of input goes above some threshold\cite{CITE}.
	This cause an action potential in the neuron that gives transmission through all the neuron's output synapses.
	%Synaptic transmissions occur after what is referred to as action  potentials of the presynaptic neuron.
	The size of these transmissions does not vary with the magnitude of the neuron's input, 
		but is thought to be defined by the strength of the synaptic connection alone\cite{CITE}. %TODO Sitere? Eller bare hene på forrige?
%		but is thought to be a function of the strength of the synaptic connection only\cite{CITE}. %TOD O Sitere? Eller bare hene på forrige?
	It is possible to propagate information as e.g. the frequency or the exact timing of these action potentials(``spikes'')\cite{ frekvensCITE, ExactTimingCITE}.
%	It is possible to propagate information as e.g. the frequency or the exact timing of synaptic transmissions\cite{ frekvensCITE, ExactTimingCITE}.
	%It is possible to propagate information as e.g. the frequency of synaptic transmissions or the exact timing of synaptic transmissions\cite{ frekvensCITE, ExactTimingCITE}.
	%It is possible to transmit graded information in this system, e.g. as the frequency of such transmissions or the exact timing of action\cite{ frekvensCITE, ExactTimingCITE}.
	%Information can, however, be said to propagate as a consequence of the frequency of such transmissions\cite{CITE}. %TODO
%	The main branch of ANN used for technology models information propagation by a floating point number defined by the immediate input, and can thus be said to represent the neuron in the frequency domain\cite{FDP_report}. %emulated neurons. %TODO
	The main branch of ANN technology models information propagation by a floating point number, given as a function of the neuron's immediate input\cite{CITE}. %TODO CITE
	These ANNs can thus only be said to represent the neuron in the frequency domain\cite{FDP_report}.
		%and can thus be said to simulate the neuron in the frequency domain\cite{FDP_report}. %emulated neurons. %TODO
%	The main branch of ANN technology models information propagation by a floating point number. This variable is given as a funtion of the neuron's immediate input, and can thus only be said to represent the neuron in the frequency domain\cite{NEVR3004OmModellane, FDP_report}

	%Simulating the neuron in the frequency domain is a great simplication of the system.
%\begin{quote}
%	When modelling neurons in the frequency domain, information of spike timing is lost\cite{FDP_report}.
%\end{quote}

	Simulating the neuron in the frequency domain is a major simplification of the system, 
		as all information about timing is lost. %the spike timing is lost.
		%and all information of spike timing is lost in an ANN that only propagates the activation level of neurons.
	Such models can therefore not be used for exact simulations of the neuron or where the relative spike time of neurons is important\cite{NEVR3004OmModellane}. %TODO TODO Finn anna referanse. Dette er lett å finne CITE på! TODO
% TODO Skriv om Hebb, gå over mot STDP etterkvart.. 
%	Simulating the neuron in the frequency domain is a great simplification of the system, and all information of spike timing is lost.
%	An ANN that propagates a floating point number between its nodes is therefore a great simplification of the system, 
%		and can not be used for simulations to be used with scientific intent of where the relative spike time of neurons is important.
% TODO TODO TODO Skriv neste setn. på nytt. Få heller fokus på at Hebb er ustabilt! TODO TODO TODO TODO
	%As one of the main reasons behind utilizing ANNs is learning and adaptation, mechanisms for synaptic plasticity is important for ANNs.
	An element of particular importance is synaptic plasticity, what is seen as learning in neuroscience\cite{CITE, NEVR3001synPlast, FDP_report}. %KANDEL? Originalartikkel?   og kanskje FDP
%	An element of particular importance is synaptic plasticity, what is seen as the background of learning in neuroscience\cite{CITE, CITE2}.
	%XXX Local learning rule? Var ikkje det noke?
	In frequency based ANNs, local learning rules can e.g. be a funtion of the presynaptic and postsynaptic spike frequency, $r_j'$ and $r_i$.
	%\cite{CITE}.
	%In frequency based ANNs, a local learning rule has to be a funtion of the presynaptic and postsynaptic firing frequency.
	%In ANNs, this is often modelled as a function of the presynaptic and postsynaptic firing frequency\cite{CITE}.
\begin{equation}
	\Delta \omega_{ij} = \sum_j C r_i r_j' \quad,\qquad
	\begin{tabular}{l}
 		$r_j'$ 			\tiny{ is the presynaptic neuron' s firing frequency} \\
		$r_i$  			\tiny{ is the receiving neuron's firing frequency}  \\
		$\omega_{ij}$ 	\tiny{ it the synaptic weight between neuron $j$ and $i$} \\ %\tiny{ is the synaptic weight form neuron $j$ to neuron $i$} \\
		$C > 0$ \\
	\end{tabular}
	\label{eqHebbsPostulate}
\end{equation}
	where $\omega_{i,j}$ represents the magnitude of the synaptic connection between neuron $j$ and neuron $i$.
	This is a mathematical interpretation of what is referred to as ``Hebbian learning'' after Donald A. Hebb who first proposed this mechanism.
\begin{quote}
	When an axon of cell A is near enough to excite a cell B and repeatedly or persistently takes part in firing it, some growth process or metabolic change takes place in one or both cells such that A’s efficiency, as one of the cells firing B, is increased. \cite{Hebb1949Kap4}
\end{quote}

	Hebb's postulate or the mathematical interpretation presented in equation \eqref{eqHebbsPostulate} only
		describes positive change in synaptic weight and is obviously unstable;
	Any correlation between the two neurons' firing frequency would cause a stronger connection between them and thus increase the correlation.
%	Any correlation between the two neurons would cause a stronger connection between them and thus increase the correlation in firing frequency.
	Numerous attempts have been made create stable learning rules in frequency based ANNs, with still increasing complexity\cite{CITE, MASSE, MEIR, EndaFleir - EnTil}. %TODO Finn ref. FINN UT korleis eg lager cite: [2, 4-6] (streken).
%%
	%One mechanism that requires special attention is Spike Timing Dependent Plasticity(STDP)\cite{NEVR3004OmModellane}.

%TODO Finn figur. Legg inn figur om STDP, her! (eller skal det være fig. i intro?		TODO

%TODO TODO TODO Sjekk alle referanser i neste avsnitt! Har ikkje sjekka. Bare stjåle fra NEVR3003. TODO TODO TODO
	In 1987, Gustafsson et al. proposed that the synaptic weight gain from a transmission varies with the postsynaptic neuron's depolarization at the time of transmission\cite{Gustafsson03011987}. %TODO Sjekk artikkelen igjen. Er veldig usikker på om dette stemmer!
%	This opened for a graded increase in synaptic weight.
	%TODO Spesielt viktig å sjekke neste! TODO
	At about the same time, Levy and Steward found that synaptic transmission could cause Long--Term Depression, a decrease in synaptic weight\cite{Levy1983791LTDetterSTDP}. % as a consequence of synaptic transmission\cite{Levy1983791LTDetterSTDP}.
	%At about the same time, Levy and Steward found that synaptic connections could undergo Long--Term Depression, a decrease in synaptic weight as a result of synaptic transmissions\cite{Levy1983791LTDetterSTDP}.
%TODO Neste setning er litt dårlig TODO skriv om!
	These two findings makes stable local learning rules possible, as a single transmission can give graded synaptic plasticity from negative to positive weight change\cite{CITE}.
	%These two findings makes it possible with stable local learning rules, as the synapse can undergo a graded synaptic plasticity from negative to positive weight change after a single synaptic transmission\cite{CITE}.
%	These two findings give that a synapse can undergo a graded synaptic plasticity ranging from positive to negative weight change, and could be the basis of a stable local learning rule\cite{CITE}.
	%These two findings explains that a synapse can undergo a graded synaptic plasticity ranging from positive to negative weight change after transmission, and could be the basis of a stable local learning rule\cite{CITE}.
	%These two findings explains how it is possible with a graded synaptic plasticity ranging from positive to negative change in synaptic weight. %CITE? XXX
	The N-methyl-D-aspartic acid($NMDA$) receptor is found to be of a particular importance in this context.
	%The $NMDA$ receptor is found to be important in this context.
%	 %\cite{RossumStableHebbVedSTDP}, %TODO Sjekk teksten til artikkl, ikkje bare tittel..
%	\cite{NEVR3003STDP, RossumStableHebbVedSTDP},

	$NMDA$ receptors consists of ionic gates that open when the receptor is exposed to the right neurotransmitters\cite{CITE}.
	%In addition to other ions involved in neural depolarizing, this channel lets $Ca^{2+}$ ions flow into the neuron. Calcium is an ion that takes part in regulating synthesis of new receptors\cite{CITE,C}. 
	As opposed to the $AMPA$ receptor, this channel enable $Ca^{2+}$ ions to flow into the neuron.
	This ion is thought to take part in regulating the synthesis of new $AMPA$ receptors and is considered important for synaptic plasticity as well as transmission\cite{CITE}.
	%The $Ca^{2+}$ ion is thought to take part in regulating the synthesis of new $AMPA$ receptors, and is considered important for synaptic transmission and plasticity\cite{CITE}.
%%
	%The negative resting membrane potential of about  $-65mV$ pulls positively charged ions to it from outside\cite{CITE}. 
	The $NMDA$ channel is blocked by a $Mg^{2+}$ ion that covers the opening, stopping all ion flow through the channel\cite{CITE}.
	%The $NMDA$ channel is blocked by a $Mg^{2+}$ ion covering the opening, stopping all ion flow through the channel\cite{CITE}.
%%%%%%%%%%%%%
	%Outside of the $NMDA$ channel lies a $Mg^{2+}$ ion that covers the opening and blocks any flow of ions thought the channel\cite{CITE}.
	%Outside the opening, there is a $Mg^{2+}$ ion that inhibits a flow of ions thought the channel\cite{CITE}.
	%There is a $Mg^{2+}$ ion that lies in the opening and inhibits the flow of ions thought the opening\cite{CITE}.
	When the (negative) membrane potential is sufficiently depolarized, the block is no longer pulled towards the gate and ions is able to pass through the channel. %TODO TODO TODO VÆR HEILT SIKKER på at det er neg. potensial inni i forhold til utafor! TODO
	%When the (negative) membrane potential is sufficiently depolarized, the block is pushed away from the gate and ions can flow through the channel\cite{CITE}.
	%The probability of pushing away the block can be seen as a stochastic function of the membrane potential. %FINN CITE? EVT BEGRUNN MEIR!
	Due to the number of $NMDA$ receptors and variations in $Mg^{2+}$ blocks, this creates a graded magnitude of the $Ca^{2+}$ inflow and thus variations in synaptic plasticity\cite{CITE}.
	%Due to the number of $NMDA$ receptors and variations in the block in $NMDA$ receptors, this creates a graded response for the magnitude of the $Ca^{2+}$ inflow as a function of the membrane potential at the time of transmission\cite{CITE}. %TODO
	Synaptic plasticity can thus be modelled as a function of the postsynaptic membrane potential at the time of transmission\cite{CITE}.
%	Synaptic plasticity thus have a graded response of the postsynaptic membrane potential at the time of transmission\cite{CITE}.
%	It can therefore be said that synaptic plasticity is a consequence of the postsynaptic membrane potential at the time of transmission.
	The postsynaptic depolarization often has a correlation with how much time there is left until firing, and might be the reason why this mechanism is called Spike--Time Dependent Plasticity(STDP).
%	The mechanistic model for synaptic plasticity can be considered one for the main reasons behind simulating neurons in the time domain.
%	STDP is intrinsically stable, something that enable a higher [FORSTERKNING] of synaptic plasticity.
	This mechanistic learning rule gives a strong motivation for utilizing spiking neuron simulations in the nodes of an ANN.
	%STDP is a mechanistic learning rule that gives a strong motivation for utilizing spiking neuron simulations in the nodes of an ANN.
	%This mechanistic learning rule creates a strong motivation for utilizing spiking neuron simulations for the nodes in the ANN. 
	Such ANNs are often referred to as Spiking Artificial Neural Networks(SANN).
	%This creates a graded responce of the $NMDA-R$ dependent on the depolarization of the neuron before transmission\cite{CITE}.
	
	Despite its advantages, SANN is seldomly used in technology\cite{CITE}.
	This is partially caused by the computational complexity of SANN simulations\cite{CITE}.
	MASSE SKAL TIL FOR AT SANN ER ERROR-fri.
	The theory for frequency based ANNs is also well established, and can not direcly be used in SANN.
	FREKVENS-simulering kan ikkje overføres til SPIKE-simulering. Float-variabel vs. diskrete overføringer.
	-Error: kan ungåes ved å ha fleire computational time steps. MEN dette gjør simulering tyngre.
	
	Error har i dette arbeidet blitt analysert, og der sees at NIM har akkumulasjon av feil.
	Forsøk på å unngå dette : KANN



%	It is shown that the speed at which biological neural systems compute certain input, 
%		can not be achieved unless the spike timing is a part of the neural computation. %XXX Cite
%	SKRIVE OM InnØret og VentroLateralNuclei?
%%
%	Networks of nodes that simulates signal propagation by spikes have been referred to as Spiking Artificial Neural Networks(SANN). %XXX Cite
%	SANN have been used for simulations of neural systems[CITER] as well as for technology[CITE]. %TODO TODO
%%	It is used for technology both to approach the fast computation of biological neural systems as well as its learning capabilities.

\newpage

	Despite its advantages, SANN have not been used much in technology.  %TODO CITE!
	This is partially because of the computational complexity of SANN simulations[CITE], 
		and because ``frequency based ANNs'' is well established.
	In this work, an attempt to make spiking neuron simulations more effective is conducted.
	SNAKKE om at det beste er om det går an å lage en modell som har muligheten for å benytte 2.gen teorier for simulering av SANN.


Motivasjon! Kvifor simulere neuronet!

Problemet!  - (Såppass kaotisk med ANN at feil kan føre til enorme utslag)
			- feil som gir feil oppførsel [VEID OPPIMOT]  for treig
Skriv kven denne teksten er skrevet for. Skriv at eg ikkje går ut fra noe neuroscience bakgrunn, men en grunnleggende matematisk bakgrunn og en relativt god oversikt over C++ er gått ut fra at leser har..
%Skriv kva gruppe eg skriv til. Kva bakgrunn ser eg for meg at dei har. Eg har tatt med litt bakgrunnsinformasjon om nevro dersom leser ikkje har utpregende kjennskap til dette området. Eg har også forsøkt å ta med litt meir i avsnittene som omhandler C++, ettersom leser også kan være fra neuroscience minjøer uten utpreget kjennskap til programmering. Tilfeller vil difor oppstå der leser har god kunnskap til området, og i dette tilfellet bes leser å skim these sections.


	

%	Nevn LIFE ---  sjå ssayNEVR3002\_proprieception 
%
%	Simulering av SN: feilaktig simuleing: kvifor der det drit å få feil?\\
%	
%	Vanskelig å unngå akumulativ feil!\\



%	Elements like robustness, ability to handle fuzzy information, fault and failure tolerance, and learning makes ANNs the best tool for handling certain input\cite{jainEtAl}.

% Mi avgrensning.  --Kva har eg utelatt. Kvifor. osv..
	Skriv at eg bare har sett på LIF neuronet. Men dette er den mest brukte neuron modellen i SANN simulering.
	Ser bare på enkleste form for numerisk integrasjon. Kvifor? Begrunn. Skriv at forbedringer kan mest sansynlig gjøres med like stor impact for begge modeller..



%Kort motivasjon for å lese vidare. Tenk at dette er dritkjedelig for sensor, og at eg må motivere han til å lese vidare (eit avsnitt)
%
%  Ting som må være med :
% 		A statement med målet til teksten. Kvifor vart prosjektet gjort?
% 		Nok bakgrunnsinfo for å forstå kvifor det er viktig å lese vidare.
% 		Proper accnowledgement of previous work on which I am building. Nok referanser til at leser kan gå til biblioteket og finne støttelitteratur før han leser vidare.
% 		The introduction should be focused on the thesis question(s).  All cited work should be directly relevent to the goals of the thesis.  
% 		Scope of project: Kva er med, og kva er ikkje!
% 		Verbal table of context. Vær sikker på at det er veldig klart kva som er bakgrunnsinfo og kva som er mitt arbeid.


%XXX RAPPORT DISPOSISJON:
% Motivasjon: Kvifor er dette gjort?
	Det er mykje dataen ikkje er så flink på. \\
	For slike oppgaver brukes ofte ANN\\
	
	Nevrovitenskapen's fokus på spike times har flere årsaker. Læring og også signal processing i noen spesielle situasjoner.
	Dette har smittet over på computational neuroscience, både for bruk i simulering av nye teorier og påtenkt: for ANN i teknologi.
	SANN i ANN er veldig nytt felt, og er ikkje skikkelig utprøvd kanskje hovudsaklig på grunn av at dette krever tunge utregninger(lang tid).

% Problemområde: Kva er gjort?
	Det er kanskje spesiellt nyttbart for å utvikle proteser i biomedisinsk bevegelse, etter at LIFE er utviklet.
	LIFE er ...  og muliggjør en direkte kommunikasjon mellom neuron og teknologi. Dette krever avansert signalbehandlign som kan forstå signala.
	Dette tenker eg er veldig gunstig å gjøre med nevro-emulatorer lagt som simuleringer av neuronet i datamaskina.
	Dette skaper en spesiell motivasjon for å utvikle SANN slik at det er mulig å nytte i teknologi(sanntids-utførelse).

% Kva har andre gjort? 		(skive om SANN)
	SANN består i dag av mange noder som simulerer neuronet ved numerisk integrasjon av input.
	Det er en balanse mellom gode simuleringsresultat og lave computational costs(effektivitet); Dersom man vil ha mindre simuleringsfeil kan man øke den temporale oppløyringa, men dette øker computational load.
% Mi avgrensning.  --Kva har eg utelatt. Kvifor. osv..
	I dette arbeidet vil eg sjå på muligheten for å simulere neuronet ved å bruke den algebraiske løysinga til neuron modellen som er brukt.
	Modellen som er sett på er 'the Leaky Integrate-and-Fire(LIF) neuron model', da dette er den modellen som oftest er brukt for SANN.'
	Andre neuron modeller har ikkje blitt undersøkt, selv om metoden i bruk absolutt kan nyttes for desse også.
% Disposisjon: Utvida innholdsfortegnelse.

	%Skriv kva gruppe eg skriv til. Kva bakgrunn ser eg for meg at dei har. Eg har tatt med litt bakgrunnsinformasjon om nevro dersom leser ikkje har utpregende kjennskap til dette området. Eg har også forsøkt å ta med litt meir i avsnittene som omhandler C++, ettersom leser også kan være fra neuroscience minjøer uten utpreget kjennskap til programmering. Tilfeller vil difor oppstå der leser har god kunnskap til området, og i dette tilfellet bes leser å skim these sections.
	% Skriv: Eg har plassert eit kapittel i appendix om synaptisk overføring i appendix. Dette er for å få en meir helhetlig oversikt over det biologiske systemet, men siden det ikkje er direkte knyttet til oppgaven(dårlig formulert) er det ikkje med som en del av hovedteksten. Dette kan brukes for utfyllende informasjon og motivasjon for SANN.









% // vim:fdm=marker:fmr=//{,//}

%	\input{artikkel_06.02.2012/theory}
%	\input{artikkel_06.02.2012/metode}
%	\input{artikkel_06.02.2012/results}
%	
%  		- Skriv om at med dEstimatedTaskTime-opplegget er det ikkje nødvendig å simulere spatiotemporal delay. I aksonet for eksempel. Dette er kjempebra for neuronsimulering.
% 			- Mulighet for "multi-compartment model with single compartment implementation".
% 		
% 		- Skriv om at det var teit å bruke deriverte ved overføringer. Bedre ville det vore å bare definert det slik at kvart neuron oppdaterte(rekalkulerte) kvar iterasjon. 
%
% 		- Directions for further work


%TODO Bruk: 	"This study shows ..." 	Veldig rett formulering! XXX


%XXX skrive om Lovelace and Cios(2008) som "proposed a very simple spiking neuron(VSSN) model, og "Simplicity an Efficiency of Integrate--and-fire neuron models" mener er drit. Denne bruker en forenkling av SANN, men diesmann og Plesser mener den er dårlig. Skriv om at eg går motsatt retning, og bruker meir avanserte metoder for å finne meir correct resultat(som betyr meir effektiv simulering).



% TODO Synaptisk plasticity! Kan bruke både aktivitetsbasert og spikebasert!   TODO 



\section{Summary}

The mechanisms of biological neuron networks, the computational system of biological beings, is not fully understood.
% How biological neural networks, the computational system of biological beings, function is not fully understood.
On a low level, neuroscientists have found that networks of neurons propagate information by discrete action potentials.
An action potential causes a transmission through all the neuron's output synapses, leading to the increase or decrease in the postsynaptic neuron's value.
This value, referred  to as \emph{the depolarization} of the neuron, is the result of a leaky integration of synaptic input transmissions.
% This value, referred to as \emph{depolarization}, can be considered to be the result of a leaky integration of synaptic input transmissions.
% This value, referred to as \emph{depolarization}, can be seen as a leaky integration of synaptic input transmissions.

Digital simulations have discrete time, and a neuron's depolarization is often simulated by numerical integration.
%Digital simulations have discrete time, and continuous mechanisms like leakage can be simulated by numerical integration.
This is done by adding synaptic input and subtracting an estimate of the neuron's leakage.
In this work, the previous time step's value is utilized when computing leakage for the $NIM$ model (\emph{sample--and--hold integration}). % in the implementation that utilizes numerical integration

This study shows that the error from each computational time step varies like a stochastic variable, and that the total error is defined as the integral of all local errors. 
This results in a diverging simulation error, unless the local truncation error has an expectancy value of zero.
In an attempt to avoid this, a novel simulation scheme has been developed that does not involve numerical integration.
% In order to avoid this, ideas from systems theory have been utilized to develop a new simulation scheme that does not depend on numerical integration.
Using the concept of \emph{time windows}, time intervals where the neuron's depolarizing inflow is held constant, a neural simulator was developed that utilize the algebraic value equation in these intervals.
% By the concept of \emph{time windows}, time intervals where the neuron's depolarizing inflow is held constant, a neural simulator was developed that utilize the algebraic value equation in these intervals.
% %%%%
Software intended to make differences in design of the two simulation schemes have been designed and implemented, $auroSim$.
The artificial neuron has the functional lay--out of the biological neuron, with four distinct subelement types, [$i\_dendrite$, $i\_auron$, $i\_axon$, $i\_synapse$].
The abstract \emph{i\_\{element\}} types are inherited to \emph{s\_\{element\}} and \emph{K\_\{element\}}, model specific classes.
All common aspects between the two simulation models can thus be placed in the ancestor \emph{i\_\{element\}} class, making principal differences in design of the two simulation schemes prominent.

It is shown experimentally that although the $\kappa M$ simulation scheme is computationally more complex, the simulation is more effective. %it gives a more effective simulation.
Because the $\kappa M$ simulation scheme produces less errors, longer computational time steps can be used to achieve the same accuracy.
This makes it possible to utilize fewer computational time steps to achieve the same degree of simulator accuracy, enabling a more effective simulation.
It is also shown that the absolute error of the algebraic simulation scheme is bounded, something that could be of importance in complex ANN simulations.
% It is also shown that the algebraic simulation scheme has a bounded error, something that might be important in complex ANN simulations.


%XXX XXX XXX XXX XXX XXX XXX XXX XXX XXX XXX 
% \section{Contribution of Thesis}
% 	-modell
% 	-simulation scheme
% 	-implications of results
% 	-future directions


\section{Discussion}  
	One question that presents itself is the importance of a gradually increasing cumulative error.
	The most immediate errors are the ones that alter the length of an inter--spike interval.
	These are represented as the derivative of the spike--time error curves in fig. \ref{figSpikeTimeErrorExperiment2}; % or \ref{figExperiment2ErrorInTenSineOscillations};
		when an inter--spike interval has an erroneous length, the spike--time error is changed by this amount.
	Fig. \ref{figSpikeTimeErrorExperiment2} shows that in the first period of the forcing function, the $\kappa M_{100}$ spike--time error change with about the same rate as the $NIM_{1.000}$ error.
	After spike nr. $20$, the derivative of the spike--time error is larger in the $NIM_{1.000}$ simulation than in the $\kappa M_{100}$ simulation.
	This illustrates a significant efficiency improvement, as the $NIM$ simulation has a temporal resolution that involves ten times as many time steps as the $\kappa M$ simulation.
% 	The $NIM$ simulation has a temporal resolution that involves ten times as many time steps, and illustrates a significant efficiency improvement by utilizing the $\kappa M$ simulation model.
% 	Since the $NIM$ simulation have a temporal resolution that involves ten times as many time steps as the $\kappa M$ simulation, this illustrates a significant efficiency improvement for the novel model.
% 	Since the $NIM$ simulation has ten times as many time steps, this shows a significant efficiency improvement when utilizing the $\kappa M$ simulation model.

	Fig. \ref{figExperiment2ErrorInTenSineOscillations} shows the spike time errors for the same experiment, simulated over a longer time interval. 
%	Fig. \ref{figExperiment2ErrorInTenSineOscillations} shows the spike time error in an artificial neuron simulated over a longer time interval.
	One can observe the cumulative property of the $NIM$ error as a gradual increase in the absolute spike--time error. %, over the duration of the experiment. 
%	One can observe the cumulative property of the $NIM$ error, since the absolute spike--time error increases over the duration of the experiment. 
% XXX Remove the next two sentences? XXX
% 	Also note that the rate of change, represented by the magnitude of the oscillations in the spike--time error, increases in the course of the experiment.
% 	The reason for this is unknown. %xxx Ta med? This is important for discussion! TODO
%%%
	To compare the $\kappa M_{100}$ simulation's spike--time error with the $NIM_{10.000}$ simulation's error, the difference in absolute error is presented in fig. \ref{appendixDifferenceInErrorFig}.
	This figure shows that in the second half of the experiment, the $\kappa M_{100}$ error is generally less than the $NIM_{10.000}$ simulation's error. %for the $NIM_{10.000}$ simulation.
	This implies an even greater efficiency improvement, as the $NIM$ simulation has a number of time steps that is two orders of magnitude larger than the $\kappa M$ simulation's.
	In all conducted experiments, this effect becomes larger for longer simulations.
% 	It appears that this effect becomes larger for longer simulations.

	Reproducibility has been an important element in the conducted experiments in this work.
% 	In the experiments conducted in this work, reproducibility of results have been in focus.
%%%	The design of the comparison software, and the implementation of fundamental simulation elements have therefore been documented in this text.
	The most important elements of the simulation software are well documented, and the forcing functions in the experiments are represented by algebraic functions.
% 	The forcing functions in the conducted experiments are represented as algebraic functions.
% 	The forcing functions are therefore represented as algebraic functions.
	It is possible that the use of algebraic forcing functions limits the validity of the results, since the input to a node in a neural network is far from being a smooth algebraic function.
	Experiment 2 considers a sinusoidal forcing function, where neither the value nor the derivative is constant for any time interval. % at any time.
	This can be used as a basis in a Fourier series to produce any periodic signal.
	This forcing function can therefore be seen as a component in any signal, and is considered to be an appropriate algebraic function for efficiency measurements. %be a good signal for efficiency mesurements.
	A stochastic Wiener process could also be used, but this would make the experiments harder to validate for others.
% 	
% 	The use of algebraic input functions makes reproduction of the results simpler, but might be a limitation for the validity of the results.
%%
	To simplify further analysis and for a thorough study of the implementation, $auroSim$ has been published under \emph{GPL}.
%	To simplify further analysis and for a thorough study of the conducted experiments, $auroSim$ has been published under \emph{GPL}.
	The source code can be found under branch \emph{master} in the git repository located at \emph{https://github.com/leikanger/masterProject} \cite{gitRepoCommit}.


	One element that could be worth examining, is the ability of the $\kappa M$ simulation model to simulate the neuron by other formal neuron models.
% 	Something that could be worth examining is the ability of $\kappa M$ to simulate the neuron by other formal neuron models.
	The $LIF$ neuron model is often used because it is simple, and does not involve complex operations.
	Other neuron models are reported to produce more accurate simulation results \cite{gerstnerKistler2002}. %\cite{gerstnerKistler2002KAP04}.
% 	Some other neuron models are reported to give more accurate simulation results\cite{gerstnerKistler2002}.
% 	Other formal neuron models is reported to give more accurate simulation results\cite{gerstnerKistler2002}.
	The $\kappa M$ simulation scheme is thought to be applicable for any neuron model where the depolarization is described by an ordinary differential equation.
	As long as the value equation is defined as a function of a single variable, \emph{time windows} can be defined, and the $\kappa M$ simulation model can be utilized.
	The use of $\kappa M$ for systems defined by partial differential equations or sets of ordinary differential equations, is also an area that could be worth examining.
% 	These models can probably be simulated with $\kappa M$ by substituting the $LIF$ neuron's equation with the alternative model's value equation. %, in the implementation.
	%These models can be simulated by substituting the $LIF$ neuron's value equation with the alternative model's equation. %, in the implementation.
% 	It is probable that a $\kappa M$ simulation of these models is as precise as for the $LIF$ model, enabling equally effective simulations of these neuron models.
% 	This can be done by substituting the $LIF$ neuron's depolarization equation with an other neuron model's in the implementation.

	When edge transmission is implemented as the derivative of synaptic flow, transmissions are only needed when there is an altered activation level for the presynaptic neuron.
% 	When edge transmissions as the derivative is implemented, transmissions are only needed when there is an altered activation level for the presynaptic neuron.
% 	Edge transmissions as the derivative enables the synapse to transmit, only when there is an altered activation level for the presynaptic neuron.
%	Edge transmissions as the derivative enables the synapse to transmit, only when the presynaptic neuron has an altered activation level.
% 	Edge transmissions as the derivative enables the synapse to conduct transmissions only when the presynaptic node has an altered activation level.
% 	The use of edge transmissions as the derivative, causes transmissions to be needed, only when the presynaptic node has an altered action level.
% 	The use of edge transmissions as the derivative seemed like a good idea, since transmissions are only needed when the presynaptic node has an altered activation level.
	When a double precision floating point data type is used, with the smallest increase defined to be $10^{-308}$, it is highly unlikely that the activation level of a node remains constant over any time interval.
% 	Because a double precision floating point data type is used, with a precision down to $10^{-308}$, it is highly unlikely that the activation level of a node is constant over a time step.
	The concept of edge transmission as the derivative does not decrease the efficiency of a simulation, but it does not improve it either.
% 	The concept of edge transmission as the derivative therefore does not improve efficiency noticeably.
	It does increase the complexity of the design/implementation, and is recommended to be removed for further uses of $auroSim$.
%%%% Ta med? Ta vekk? ???
% 	Because it does not affect the simulation results, or other comparisons in this report, edge transmission for the nodes has not been altered
%	Because it does not affect the simulation results or other comparisons in this report, edge transmission as the derivative has not been altered in this work.
	
	%todo todo todo todo todo todo todo todo todo todo Skriv om bruk av andre integrasjonsmodeller: Burde sjekka f.eks. runkegutta' todo todo todo todo todo todo todo todo todo todo 


% 	Why the oscillations in spike--time error grows in longer $NIM$ simulations is unknown.
% 	- Muligheten for å bruke KM for andre neuron-models. Skriv at så lenge man har en algebraisk funksjon kan KM brukes. Får en 'bounded error'.
% 	- Mulighet for å interface'e med 2.gen. ANN og andre filter. (skrevet dette før?) Sjekk, og evt. skriv her..
	


\section{Conclusion}
	%TODO Konsklusjon: NIM error auker mens KM error er stabil. Som forutsett..

	This work introduces an entirely new way of considering a neuron's activation level.
	The novel formalism considers what the neuron's depolarization would approach, $\kappa$, if no firing interrupts it.
	The $\kappa$--formalism enables the use of algebra to find the neuron's depolarization, as well as the immediate firing frequency of the neuron.
	Combined with the concept of \emph{time windows}, time intervals where the depolarizing inflow is held constant, spiking neuron simulations can be conducted without the use of numerical integration.
% 	Combined with the concept of \emph{time windows}, it is possible to conduct a spiking neuron simulation that does not utilize numerical integration.



	The traditional Numerical Integration Model($NIM$) and $\kappa M$ is compared theoretically and experimentally in this report. % analitically in this report.
%	Section \ref{ssecAnalysisOfErrorsForTheTwoModels} shows that because the neuron has a positive 
	The analysis of the $NIM$ model shows that the local truncation error has stochastic elements, and that the global truncation error diverges unless the local errors have a expectancy value $\hat{e} = 0$. %that is zero.
	The $\kappa M$ error is a result of a delayed update from a variable that varies within a bounded domain, producing a bounded error. % causing the error to be limited.

% 	The neuron has a potential unlimited increase in depolarization value, since this value has a positive increase till the firing threshold before being reset, every inter--spike interval.
% 	This happens every inter--spike interval, causing a potentially unlimited increase in value. %(positive derivative).
% 	Any systematic local truncation error based on this value therefore produce a diverging error for the simulated depolarization. % neuron's depolarization.

	The two simulation models were implemented in a common framework, and accuracy comparisons was conducted. % on the resulting software.
	These comparisons are relevant since the differences between the models were isolated and potential faults in the common framework affects both models equally.
%FORRIGE(amund endra foreslår å endre. Dette prøver eg å gjør, over) GAMML:	Accuracy comparisons thus give more valid results, as the differences between the models were isolated and potential faults in the implementation of a simulator affects both simulations.
%	Accuracy comparisons thus give more valid results, as the differences between the models were isolated and other implementation details are shared.
%% Ta med? Ta vekk?
% 	The time course of the simulated neuron's depolarization has been compared to a high--resolution simulation to find the two models' error.
	It is shown that in the course of $15$ periods  of a sinusoidal forcing function, the $\kappa M_{100}$ simulation, a $\kappa M$ simulation with $100$ time steps per forcing function period,
		generally produces more accurate results than a $NIM_{10.000}$ simulation.
% 		produces more accurate results than a $NIM_{10.000}$ simulation.
	This is a significant efficiency improvement, as the $NIM_{10.000}$ simulation has a number of time steps that is two orders of magnitude larger than for the $\kappa M_{100}$ simulation.
	All results imply that this effect becomes larger for longer simulations, making the $\kappa M$ simulation model a significant improvement of today's spiking neuron simulation model.

%	
%1	Experiments conducted on the resulting software, $auroSim$, implies that the conducted error analysis is accurate.
%1	The $NIM$ simulations have an accumulation of error, while the $\kappa M$ simulations have a bounded error.
%1	In only $15$ periods of the sinusoidal forcing function, it is shown that the $\kappa M_{100}$, a $\kappa M$ simulation with $100$ time steps per sinus period, 
%1		produce a more accurate result than a $NIM_{10.000}$ simulation.
%1	This implies a significant efficiency improvement, as the $NIM_{10.000}$ simulation has a number of time steps that is two orders of magnitude larger than the $\kappa M_{100}$ simulation.
%1	All results implies that this effect grow larger for longer simulations.


% //{ KOMMENTERT UT!
% \chapter{UFERDIG.. kladd:}
% 
% 	If for example all inter--spike intervals are increased by the same factor, one can say that this is a similarity transform of 
% 		the real inter--spike intervals. % and the results can be transformed back.
% 	For ``offline simulations'' with e.g. scientific intent, the resulting depolarization and spike times can thus be transformed back.
% 	If the emulator is to be used for real--time applications, it is hard to avoid that the transformed results are utilized instead of the correct values.
% 	The gradually increasing spike time error affects the activity level of the neuron, altering the mean firing frequency of the neuron;
% 		When the firing time error increase by some factor $C_{e}$ in the course of the interval $\Delta t$, 
% 		the resulting firing frequency error for that interval have a magnitude $e_f = \frac{C_e}{\Delta t}$.
% 
% 	\begin{itemize}
% 		\item[-] Årsak til at feilen er mindre for KANN: intra--iteration time accuracy, use of algebraic function, 
% 		\item[-] Bruk av edge transmission: Kvifor er dette bortkasta arbeid? Skriv også at det BARE auker kompleksiteten på implementasjonen.
% 		\item[-] Limitations: bruk av algebraisk test-funk. Kanskje dette gjør eksperimentet ubra? Snakk om Fourier-series som lineærkombinasjon av sinusoidal functions.
% 		\item[-] Sammenligning: burde kanskje også sammenligna med en etablert software, som NEURON eller [det fra ås]. Dette ville gjort resultata av effektivitetsanalysen meir overbevisandes.
% 				\emph{vart ikkje gjort fordi dette ikkje var hovudelemented i prosjektet}.
% 		\item[-] Burde sjekka meir avanserte integrasjonsmetoder for NIM.
% 		\item[-] Task-scheduler: muliggjør meir effektiv simulering av spatio-temporal effects.
% 		\item[-] Diskuter kvifor oscillations in the error increase for longer simulations.
% %Diskuter litt om kvifor oscillations i feilen auker, for SANN. Kaffaen er dette? Det ser ut som om feilen svinger meir(deriverte er større) etterkvart, for SANN. Dermed er det ikkje bare den absolutte feilen som auker, men også svingningene av feilen..
% 		\item[-] 
% 	\end{itemize}
% %By considering the neuron's activation level by the value the depolarization would approach in the absence of a firing threshold, 
% 	
% 	\
% 	\subsection{Limitations}
% 	\subsection{Implications of Results}
% 
% 	
% \section{gammel discussion}
% 
% As the neuron fires when the value crosses the firing threshold and an algebraic equation is utilized to find the depolarization value,
% 	the exact firing time $t^{(f)}$ can be estimated by the equation $v(t^{(f)}) = \tau$. 
% 	%the exact firing time can be estimated by an equation that equals the algebraic formula to the firing threshold. %TODO Skriv litt om på slutten.
% The $\kappa$ formalism can be used to simulate the neuron by utilizing the concept of 'time windows', intervals where $\kappa$ is constant.
% A changed $\kappa$ initializes a new time window, and the initial value for the value equation is found as the last value in the previous time window.
% %A changed $\kappa$ initializes a new time window, and the firing time estimate needs to be updated.
% Time windows are thus fundamental for the use of the $\kappa$ formalism in a neural simulator.
% %When a new time window is initialized, the depolarization value and firing time estimate is updated for the neuron.
% 
% % TODO TA vekk: Dette er ikkje en diskurs!
% % When a new time window is initialized, the depolarization value and a new firing time estimate is computed.
% % %When $\kappa$ is updated, the depolarization value of this time is computed and saved along the time of initiation of the new time window.
% % The computation of a new firing time estimate and all other aspects involved in initiating a new time window are only computed once, 
% % 	at the end of a computational time step.
% % This saves much computational resources, and cause these computations to be executed as often as the computations of leakage in $NIM$.
% % The computational complexity still makes the $\kappa M$ simulation more demanding on the system, and the per--iteration efficiency is lower than for the $NIM$ model.
% % %It also generates a smaller error OR SOMETHING..
%  
% 
% 
% Two experiment have been set up to assess the discussed theory.
% The first experiment considers a constant depolarizing inflow, and illustrates that the $\kappa$ formalism can be used to simulate the depolarization of the neuron. 
% The $\kappa M_{100}$ simulation produce the algebraically correct spike times for all spikes, 
% 	indicating that the $\kappa M$ error analysis varies with the change in activation level.
% As the $NIM_{100}$ simulation produce a cumulative error of notable size, the first experiment justifies the conclutions from the theoretical error analysis in sec. \ref{ssecAnalysisOfErrorsForTheTwoModels}.
% 
% 
% The second experiment considers a single sensory neuron with a forcing function given by one and a half period of a sine function. %TODO sine?
% This study shows that errors of approximately the same magnitude is achieved for a $\kappa M_{100}$ simulation with a temporal resolution 
% 	of $100$ time steps per sensory function period as a $NIM_{1.000}$ simulation with ten times the number of time steps.
% The results also indicate that the $NIM$ simulation produce a cumulative error, while the $\kappa M$ simulation have 
% 	an error that only varies with the phase of the forcing function.
% To test the extend of the hypothesized cumulation of error for $NIM$, % and the stability property of the $\kappa M$ error,
% 	the experiment was repeated with a simulation time interval that is ten times as long.
% The cumulative property of $NIM$ is prominent, and the results also verifies the hypothesized stability property of the $\kappa M$ error;
% 	Instead of having an accumulation of truncation errors from numerical integration, $\kappa M$ error varies within constant bounds. %a constant interval. 
% An error from a delayed update can also possibly be dampened by methods from systems theory or numerical estimation.
%  	
% 
% %%% XXX HAR FLYTTET OPP...
% 	One question that presents itself is the importance of a gradually increasing cumulative error.
% 	%One question that presents itself is the importance of a gradually increasing accumulation of error.
% 	If for example all inter--spike intervals are increased by the same factor, one can say that this is a similarity transform of 
% 		the real inter--spike intervals. % and the results can be transformed back.
% 	For ``offline simulations'' with e.g. scientific intent, the resulting depolarization and spike times can thus be transformed back.
% 	%For simulations with a scientific intent, the resulting depolarization and spike times can thus possibly be transformed back.
% 	If the emulator is to be used for real--time applications, it is hard to avoid that the transformed results are utilized instead of the correct values.
% 	%This is hard if the emulator is to be used for real--time applications.
% %
% %	Problems arise when this is to be used for real time applications;
% %		Unless the simulation results are transformed back, the transformed results are utilized instead of the correct values.
% %	Transforming the results back demands algebraic equations for the error, and is hard to achieve for $NIM$ simulations.
% %%%%%%%%
% %	Another problem with having a gradually increasing error is that the mean firing frequency is affected by this gradual increase in absolute error.
% %	If the firing time error increase by some factor $C$ in the course of the interval $\Delta t$, 
% %		the resulting firing frequency error for that interval have a size $e_f = \frac{C}{\Delta t}$.
% %	%The implications of errors are therefore hard to predict in a complex neural network.
% %	%If there is a motivation for doing accurate simulations with a maximal error, the $\kappa M$ simulation scheme is a contribution to the field of computational neuroscience	and/or neural--inspired  cybernetics.
% %%%%%
% %%  %%
% 	%If there is a gradually increasing error, the mean firing frequency of the neuron is affected.
% 	The gradually increasing spike time error affects the activity level of the neuron, altering the mean firing frequency of the neuron;
% 		When the firing time error increase by some factor $C_{e}$ in the course of the interval $\Delta t$, 
% 		the resulting firing frequency error for that interval have a magnitude $e_f = \frac{C_e}{\Delta t}$.
% 	%If it is important that the neuron simulations are accurate without a gradual increase in error,
% 	%	the $\kappa M$ simulation scheme accomplish a small and limited absolute error at a relatively low computational cost.
% %%%
% 	
% 
% The concept of edge transmission, where the signal is propagated as the derivative of synaptic transmission seems like a good idea.
% Only the subset of input synapses with a changed transmission level is considered by the postsynaptic node.
% % Because floating point precision is utilized for representing $\kappa$, 
% % 	it is highly unlikely that the activation level remains constant to this resolution.
% % The concept of edge transmission increase the complexity of the design, and it is recommended that information is propagated as 
% % 	synaptic flow instead of its derivative.
% When $\kappa$ is represented with a floating point precision, it is unlikely that the activation level remains constant 
% 	to this accuracy from one computational time step to the next.
% The concept of edge transmission as the derivative thus only increase the complexity of the design, without impoving simulator efficiency.
% %The concept of edge transmission as the derivative therefore does not involve any efficiency improvement, and only increases the complexity of the design.
% It is therefore recommended that information propagation is implemented as synaptic flow instead of its derivative in future work.
% % It is highly unlikely that the activation level remains constant to the resolution of the floating point data format,
% % 	the consept of edge transmission as the derivative therefore does not involve any efficiency improvement.
% % Transmission as the derivative therefore only increase the complexity of the design, and it is recommended that information is propagated
% % 	as synaptic flow instead of its derivative.
% 
% 
% 	%TODO TODO Diskuter det med umiddelbar vs. transient-kurve for overføring! Kanskje KM er bedre for dette? (bruker ikkje dirac-delta)
% 
% \section{Limitations}
% 	%Her kan eg DISKUTERE begrensninger med dette arbeidet(diskutere frem og tilbake. Kanskje ende opp positivt?
% 	%VELDIG bra for å få full score på kor reflektert arbeidet er!
% 
% 	%\subsection{The Model} xxx
% 		%The $\kappa M$ simulation scheme is possible after a couple of simple concepts woven together to make is possible to utilize the algebraic equation for the neuron's depolarization.
% 		%Because the simulation model utilize the algebraic solution to the $LIF$ neuron's differential equations, the author can only think of two aspects that can be limiting for the $\kappa M$ simulation model.
% 		The concept of 'time windows' makes it possible to utilized algebraic equations to simulate the depolarization of the neuron.
% 		The model used in this work is the $LIF$ neuron model.
% 		This is a simple neuron model with many abstractions, and gives less correct simulation results than more advanced models\cite{CITE}.
% 		It is still used in this work, as this model is the most commonly used neuron model in computational neuroscience\cite{CITE}.
% 
% %%TODO Skriv 
% %%TODO [leikanger] hevder at sånnOgSånn, difor har eg brukts Sånn.. 		Hevder at, raporterer at, sier at, har gjort, ...
% 
% 		[CITEAUTHOR] reports that e.g. the nonlinear IF model is a more precise and balanced neuron model\cite{CITE}.
% 		It is likely that all models that can be represented by a funtion of a single variable can be simulated by the novel simulation scheme. %$\kappa M$.
% %		In this case, the alternating model's equation for depolarization is used instead of that of the $LIF$ model.
% 		%The alternating model's equation for depolarization have to be used instead of that of the $LIF$ model.
% 		%The algebraic solution to the $LIF$ model only has to be substituted with the equation for the alternative model.
% %		As these models are harder to simulate numerically, it is believed that the efficiency improvement could be even larger for these models. 	%%%
% 		This is something that aught to be tested.		
% 
% %		%The nonlinear IF model have not been examined, but could be especially [EGNET]suited for simulation by the $\kappa M$.
% %		%[SKRIVE MEIR OM DETTE]
% %		%The nonlinear integrate--and--fire model have not been examined, but the native use of the exponential function in $\kappa M$ makes
% %		%	it probable that $\kappa M$ can be relatively computationally more effective than a $NIM$ design.
% %		This might be the case for other neuron models as well, and is worth further examination.
% 
% 		An other aspect that might be a limitation for the $\kappa M$ implementation, is how synaptic transmission is modelled.
% 		Because $\kappa M$ propagates information as a second generation ANN, with transmission of the activation level, 
% 			it is hard to imagine whether all aspects of the propagation of spikes are preserved.
% 		With the $NIM$ design, synaptic transmission can e.g. be implemented as transient transmission curves instead of instantaneous transmissions.
% 		It is possible that this is possible for $\kappa M$ as well, by letting $\kappa_{ij}$ have a transient curve.
% 		%It is possible that $\kappa M$ also makes this possible, by letting $\kappa_{ij}$ have a transient curve.
% 		It is therefore recommended that this aspect is examined further.
% 
% %		The third aspect comes from the implementation and computational complexity of the $\kappa M$ simulation scheme.
% 
% %	\begin{itemize}
% %		\item Bruk av $LIF$ modellen:
% %			\begin{itemize}
% %				\item [-] ikkje så bra modell
% %				\item [+] mest brukte modellen
% %			\end{itemize}
% %		\item Diskuter korleis KM kan utvides for f.eks. nonlinear model.
% %			\small{KM kan gjøre det lettare å bruke transiente overføringskurver(enn dirac delta)}
% %		\item Synaptisk plasticity
% %	\end{itemize}
% %		\subsubsection{Further Work (-ikkje med, men viktig å hugse-)}
% 
% 	%\subsection{The Implementation} xxx
% 		An important focus for the experiments conducted in this work have been to make it simpler to verify the results.
% 		%An important focus in the experiments conducted in this work have been on reproducibility.
% 		The source code is therefore freely available\cite{gitRepoCommit} and all experiments have been conducted with an algebraic forcing function.
% 		It is thus simple to examine the source code and verify the results, as well as examine other elements/situations.
% 		%It is thus possible to examine the source code and reproduce the results as well as further examine other elements.
% 		Other sensory functions can be examined by declaring them in the file \emph{sensoryFuntions.h} 
% 			and constructing sensory aurons with the pointer to that function as argument.
% 		% All experiments have been conducted with an algebraic forcing function and with a focus on making it possible for others to reproduce the results. 
% 
% % //{ KOMMENTERT UT
% %	This work [OMFATTER] two different projects.
% %	The first, and in the beginning the most important project, have been to compare the two $LIF$ neuron simulation schemes.
% %	Both simulators have been implemented from scratch, in order to emphasize the differences between the two models.
% %	%Both simulators was implemented from scratch, to be able to see important differences between the two models.
% %	
% %	The second part of the report that considers the efficiency comparison between $\kappa M$ and $NIM$, % the two simulation models, 
% %		and utilize the resulting simulator software from the first part.
% %	%The second part of the project, involving efficiency comparison between the two models used this implementation when comparing $\kappa M$ and $NIM$.
% %	Both implementations have nodes that are designed as the biological neuron, with four subelements representing the functionality
% %		of [dendrite, soma, axon, synapse].
% %	%Both implementations are designed as a biological neural system, with four sub--compartment in each node.
% %	It can thus be said that both models are designed so that spatiotemporal delay is simulated directly in the artificial neuron.
% %			% as a direct simulation of the biological neuron.
% %%% 	%%	  %%%
% %	It is found that this is not necessary, and only introduce more computational load for the simulation.
% %	Because the task scheduler can be utilized for both neural simulation models, this has not been persuaded any further in the efficiency comparison. %XXX persuaded? Sjekk om dette er feil skrivemåte.. (forfulgt..)
% %	It is possible that implementing the task scheduler for the $NIM$ simulation model introduce more computational load, but it is unlikely that this is more demanding than simulating intracellular signal propagation.
% %	%Delay can instead be implemented by scheduling tasks to happen after the defined delay, and no more computational resources have to be used.
% %	As the efficiency improvement from utilizing the task scheduler only affects the run time and not the results,
% %		utilizing the task scheduler would not affect the simulation results.
% %	The efficiency comparison done in this work would therefore be unaffected by doing this.
% %%the result of a comparison that only considers the accuracy of the simulation results.
% %	It is still recommended that further development of $auroSim$ utilize the task scheduler developed in this work%
% %	(the source code of $auroSim$ can be found at \cite{gitRepoCommit}).
% %	%(for the source code of $auroSim$, if is referred to \cite{gitRepoCommit}).
% %	%It is still recommended that future uses of $auroSim$ utilize the task scheduler developed in this work.
% %	%This would only affect the total run time of the simulation, not the accuracy of the simulation results.
% %
% %	%It is also believed that a multiple--compartment model can be simulated in a single--compartment implementation.
% %	%This is an important direction for further work, as it can greatly improve the efficiency of scientific simulations utilizing multiple--compartment models of the neuron.
% %	
% %	%An other aspect that makes the $\kappa M$ implementation more complex is the use of edge transmission as the derivative.
% %	%TODO Skrive om dette?
% %
% %%TODO Ha med, eller ikkje? XXX:
% %%	The results from the efficiency comparison might have been more credible if an established simulation software was used for the $NIM$ simulation.
% %%	%The best approach, when comparing the efficiency of the two models could be to use established software for the $NIM$ simulation.
% %%	As both implementations are designed and written by the author, the $NIM$ implementation can be criticized for being 
% %%		less accurate than other $NIM$ implementations.
% %%	The approach where both simulation models are implemented from a common framework have still been utilized, 
% %%		to minimize the effect of errors originating from the common framework on the comparison. %on the results.
% %%		%as errors originating from the common framework thus affects both simulation models equally.
% %%	%As both simulation models are implemented in a common framework, and it has been important that all common aspects between the
% %%	%	two models are similar, this approach have still been utilized.
% %%	This causes suboptimal elements from the framework to affect both models, and the two models have a more equal [UTG.PKT].
% %%	%It is believed that all elements that are sub--optimal affects both models equally.
% %
% %
% %%	The efficiency comparison done in this work considers the error from $\kappa M$ and $NIM$ simulations.
% %%	The results from a $NIM$ simulation with a temporal resolution that is two orders of magnitude larger than for the compared simulations is used as the correct time course for the neuron's depolarization.
% %%	%The results that are used as the correct time course for the neuron's depolarization is found by a $NIM$ simulation with a number of time steps that is two orders of magnitude larger than the compared simulations.
% %%	%This approach to finding the correct value is validated, as it is shown that the truncation error is proportional to the size of the computational time step.
% %%	The node is designed to emulate a sensory neuron, where the change in depolarization varies with the sensed signal, enabling algebraic 
% %%		functions to define each experiment.
% %	//}
% 
% %	Algebraic input functions are chosen to make the experiments replicable.
% 	By an infinite sum of different sine functions with different phase and frequency, any signal can be modelled exactly\cite{CITE}.
% 	The forcing function utilized when assessing efficiency is therefore chosen to be a sine function.
% 	%The forcing function utilized when assessing efficiency have been used as this can be considered a single [LEDD] in a Fourier series.
% 	To assess all possible signals, however, an infinite amount of these experiments have to be considered.
% %%%%%%
% 	Because the experiment does not define the time scale, the aspect of different frequencies is irrelevant in this context.
% 	The size of $\kappa$ in relation to $\tau$ is defined and gives the inter--spike interval, but the amplitude of the sine function is not linked with the time scale of the simulation.
% 	%The amplitude of the forcing function oscillations is not directly linked to the time scale, and implies that the experiment done in this work is but an example of simulator efficiency.
% 	This implies that the efficiency experiment is but an example of simulator efficiency.
% 	%The domain of the forcing function is chosen so that errors are prominent while still being plausible.
% 	The forcing function is chosen so that it is a relevant example that is plausible for neural networks.
% 	%The forcing function is chosen so that the example is relevant and plausible for neural network, so that it is a relevant example.
% 	Because the forcing function of the efficiency comparison in chapter \ref{chExperimentalEfficiencyMeasurement} has the constraint that $\kappa$ is above threshold for the whole simulation, the experiment has been repeated with a forcing function without this constraint.
% 	%Because the forcing function of the efficiency comparison in chapter \ref{chExperimentalEfficiencyMeasurement} is above threshold for the whole simulation to make errors prominent, the experiment has been repeated with a forcing function without this constraint.
% 	%Because it is above the firing threshold for the whole simulation to make differences prominent, a simulation has also been conducted with a forcing funtion that goes below this value.
% 	For the sake of completeness, method and results of this experiment are presented in appendix \ref{appendixExperiment3}.
% % 		, and shows a similar effect.
% %	The results are presented in appendix \ref{appendixExperiment3} for the sake of completeness, and shows a similar effect.
% 	%The results from this experiment have not been included in the main text, but a plot of the results is presented in appendix \ref{appendixExperiment3}.
% 
% 	%Algebraic input functions are chosen to facilitate
% 	%Algebraic input functions are chosen for the sake of replication
% 	%Algebraic input functions enable a easier repetition of experiments, and also makes analysis of e.g. the simulated solution possible.
% 	A neuron's spiking input from other nodes in a neural network have a complex character, and resembles white noise in the time domain.
% 	%The chaotic input from a network of neurons can almost be seen as white noise in the time domain.
% 	%The chaotic input from a network of neurons can almost be seen as white noise.
% 	The resulting depolarization of the receiving $LIF$ neuron is the integral of this input, and can thus resemble a Wiener process.
% 	To test more realistic neural input flows, one should focus less on replication, and utilize a stochastic process to define neural input.
% 	%To test a more realistic neural input flow, one could step away from optimizing experiments for replication, and e.g. utilize a Wiener process to define neural input.
% %%%%%	
% %	An algebraic forcing function can be seen as only an example of all possible input functions found in white noise.
% %	All functions can be represented as a Fourier series of trigonometric functions[CITE].
% %	The author therefore thinks that the use of a sine forcing function makes the experiment more general, 
% %		and that this particular input can give general results. %answers.
% %	%As all functions can be represented as a Fourier series of trigonometric functions[CITE], 
% %	%	the author believes that the use of the sine function makes the forcing function more general.
% %	It is left for further work to assess forcing functions that gives a sub--threshold $\kappa$. %TODO Eller ta det inn i denne rapporten?
% %	%The use of an algebraic function defined as a sine function can thus be seen as only one 
% 	It is left for further research to assess forcing functions that are defined by e.g. a Wiener process or by synaptic input from an ANN.
% %	It is left for further work to assess forcing functions that also gives sub--threshold activation levels, 
% %		and to assess the two models when a node receives input from an ANN.
% 	To simplify further analysis, the source code of $auroSim$ have been published under GPL and the version presented in this report can be found under branch \emph{master} in the git repository found in \emph{https://github.com/leikanger/masterProject}\cite{gitRepoCommit}.
% 	%To simplify further analysis, the software developed in this work have been published under GPL,
% 	%	and the version presented in this report can be found with the commit id $5e1e609ef_{\ldots}$ at 
% 		%\cite{gitRepoCommit}.
% 	%	\emph{github.com/leikanger/masterProject}\cite{gitRepositorySida}\cite{gitRepoCommit}.
% 	%	\emph{github.com/leikanger/masterProject}\cite{gitRepositorySida}. %TODO TODO TODO Cite viser ikkje vev-sida. Fiks webcitation!
% 	
% 	
% 
% 
% 
% 
% %	\begin{itemize}
% %		\item Har ikkje implementert synaptic plasticity
% %		\item Kanskje skrive om effektivitetsanalysen: at KM er implementert for likt NIM, og vil difor 'suffer' av dette.
% %			KM vil difor bruke meir comp. resources enn nødvendig, noke som kan ha sett KM i dåligare lys.\\
% %			MEN det er alikevel valget å gjøre det slik, siden hovedfokuset i denne teksten er en teoretisk sammenligning(og utvikling av KM)
% %		\item Eg burde brukt meir bibliotek, istadenfor å implementere alt selv (i C++)
% %		\item Eg kunne sammenligna KM med etablert software for effektivitetssammenligning.
% %			\begin{itemize}
% %				\item[+] Dette ville gjort resultata mindre avhengig av min implementasjon av $NIM$.
% %				\item[-] Implementasjonene ville da blitt meir ulike med tanke på f.eks. effektiviteten av tids-simuleringa. Dette ville difor kunne bli feilaktig.
% %				\item[-] Den teoretiske sammenligningen ville ikkje blitt like insiktsfull: mange små detaljer ville forsvunnet.
% %			\end{itemize}
% 
% %		\item bare sett på enkelt-neuron(ikkje ANN)
% %			\begin{itemize}
% %				\item[+] Bedre for å reprodusere resultat!
% %				\item[-] Mulig det ikkje gir eit velbalansert svar. I kvit støy finner man alle mulige situasjoner mens
% %					algebraisk funk. vil være rimelig smalt i fohold.
% %				\item[-/+] Diskutere korleis en synaptic flow vil opptrå. Skrive at denne vil ha en kontinuerlig funk. Så kvit støy er umulig.
% %					Heller den integrerte av kvit støy -- wiener process.
% %%				\item[+] Mulig å finne algebraisk løysing
% %			\end{itemize}
% % TODO TODO TODO Skal eg også ta med neste? (har eg ikkje allerede skrevet om dette?) TODO TODO TODO
% %		\item Bruk av edge transmission as the derivative. 
% %			\begin{itemize}
% %				\item[-] Øker kompleksiteten til design uten å forbedre effektivitet
% %				\item[-] Fører til små feil som må handteres for Kappa,  uten å forbedre effektivitet
% %				\item[+] Tvang meg til å tenke gjennom desse aspektene, og lære at dette er bortkastet.
% %			\end{itemize}
% %	\end{itemize}
% 
% 
% \section{Concluding Remarks}
% 	%(Skriv into om K-formalismen først) TODO
% 
% 	In this work, a new formalism is developed to denote the activation level of a neuron.
% % NESTE SETN ER DÅRLIG: litt feilaktig. TODO TODO Skriv meir presist/ bedre/ rettere. (Det er ikkje at man benytter K-formalismen, men time windows.. TODO
% 	By utilizing the concept of time windows and synaptic flow, it is shown that algebraic equations can be used directly in spiking neuron simulators.
% 	%By utilizing the concept of time windows and synaptic flow, it is shown that algebraic equations can be used to make activation based spiking neuron simulators.
% 	%By utilizing the concept of time windows and synaptic flow, it is shown that a mechanistic model can be used for activation based spiking neuron simulators.
% %	By utilizing this formalism, it is shown that a mechanistic model can be used to design activation based neural simulations.
% %	%By utilizing the new $\kappa$ formalism to denote the activation level of the neuron, 
% %TODO Skriv neste stetninga slika at alle aspekt ved den er interesant og har innhold! TODO
% 	By computing the neuron's depolarization every time the activation level is altered, the neuron's spike time estimate can also be updated.
% 	%As the neuron's depolarization is updated every time the activation level is altered, this can be used as the initial value in the equations to estimate the neuron's spike time.
% 	When the neuron is estimated to fire in the course of the present time step, this estimate can be used as the simulated firing time as $\kappa$ is defined to be constant during a computational time step.
% %	When the the neuron is estimated to fire in the present time step, this estimate will not be altered, and the node can fire.
% 	%By updating the neuron's depolarization every time the activation level is altered, the neuron's spike time can be can not only be found on a reactive basis(firing after threshold crossing), but the spike can be scheduled at the estimated spike time.
% %XXX No gir den heller ikkje heilt meining.. "Ka vil han med dette?"
% 	%By updating the neuron's depolarization every time the activation level is altered, the neuron's spike time can be fond on a proactive basis, not only by firing after the depolarization variable's threshold crossing(reactive basis).
% 	This makes it possible to have spike times with an arbitrary precision, and what is called intra--iteration time accuracy is the result.
% %	This makes it possible for the neuron to spike at a time instance with an arbitrary resolution. %a floating point variable's resolution.
% 	Theoretical analysis conclude that this gives more accurate simulation results.
% 	%It is theoretzised that this gives more accurate simulation results.
% %	For all conducted experiments, it is shown that this gives more accurate simulation results.
% 
% 	Theoretical and experimental analysis indicate that the error varies with the derivative of the neuron's activation level in $\kappa M$. %,
% 	%	and the absolute simulation error is bounded since synaptic inflow is bounded.
% 	This results in a bounded error for $\kappa M$ simulations, as opposed to the error produced when simulating the neuron by numerical integration.
% 	%TODO Fortesett på dette: Veldig bra intro til å skrive at man trenger 10 ganger så mange 
% %%%%%%%%%%%%%%%%%%%%%%%%%%%
% 	If efficiency can be measured by the temporal resolution needed to accomplish some simulation accuracy requirement, this study shows that utilizing the $\kappa M$ results in a large increase in efficiency.
% %	If the size of the computational time step is used to measure efficiency, this study shows that a large increase in efficiency can be the result of utilizing the $\kappa M$ simulation scheme.
% 	In the course of one and a half period of a sinusoidal input, the $\kappa M$ simulation gives more accurate results than a $NIM$ 
% 		simulation with ten times the number of computational time steps.
% % 	In the course of one and a half period of a sinusoidal input, a $\kappa M$ simulation gives more accurate results than a simulation
% % 		that utilize numerical integration of input, with ten times the number of time steps.
% % 		%the currently used simulation model based on numerical integration of input.
% 	The comparative improvement is further increased if the experiment is simulated over $15$ periods of this input;
% 		The absolute error is generally less in the $\kappa M_{100}$ simulation than in the $NIM_{10.000}$ simulation. %, with a computational time step that is $1\%$ of $\kappa M_{100}$'s time steps.
% %		In the second half of this experiment, the absolute error is less in the $\kappa M_{100}$ simulation than in a $NIM_{10.000}$ simulation with a computational time step that is $1\%$ of $\kappa M_{100}$'s time steps.
% 	The $NIM_{10.000}$ simulation has a computational time step that is $1\%$ of the $\kappa M_{100}$'s, and the $\kappa M$ can be said to be ``as efficient as'' a $NIM$ simulation that has a number of time steps that is two orders of magnitude larger.
% 	This involves a simulator efficiency improvement of an increasing magnitude for longer simulations.
% 	%The $\kappa M$ simulation thus generates a smaller error in this experiment than a $NIM$ simulation that have a number of time steps that is two orders of magnitude larger.
% %%%%%%% TODO TODO DODO TODOD TODO TODO SKRIV SISTE TO SETN OM! TODO TODO TDOO TODO
% %	It is concluded that the $\kappa M$ error varies within a bounded domain, while the $NIM$ error is cumulative without an upper limit.
% %	%%					%%							%%									%and does not have an upper limit.
% %	%It is concluded that the comparative efficiency improvement of the $\kappa M$ is partially a result of the stability property of the $\kappa M$ error and the cumulative property of the $NIM$ error.
% %	% It is concluded that the comparative efficiency of the two models is further improved for longer simulations.
% %	The effect of the error properties of the two models are prominent already after 15 periods of a sinusoidal input,
% %		and all results implicate that these difference will diverge further for longer simulations.
% %	%The effect of the error properties of the two models are prominent even thought 15 periods of sinusoidal input can be considered a short simulation.
% %	%Fifteen periods of a sinusoidal input can be considered a short simulation, but the effect of the error properties of the two models are prominent. 
% %	%The effect will be larger for longer simulations, implying that the $\kappa M$ could be a break through when it comes to neural simulation.
% %	%The observed effect will be larger for longer simulations.
% %	%The comparative efficiency of the two models will thus be further improved in longer simulations.
% %	
% %
% %	%The $\kappa$ formalism ... %skrive om at det er mulig å bruke til anna også(?)
% %	%A new formalism for modelling a neuron's activation level is the result.
% %	%The $\kappa$ mathematics enables a more precise discussion about a neuron's activation level.
% %	%skriv f.eks. om kor forvirra dette er no, og kor teit det er å bruke gj.sn. fyringsfreq. som umiddelbart aktivitetsmål!
% %
% 		
% 	
% 
% 	The novel spiking ANN model propagate information like a second generation ANN, while still being capable of computing spikes.
% 	This is something that makes theory from frequency based ANN relevant.
% 	Second generation ANNs or other digital filters, can therefore be interfaced directly by letting its floating point output give $\kappa$ for the interface node.
% 	%This can also be done by a ``sensory node'', as in the experiments conducted in this work, and is thus valid for $NIM$ as well.
% 	By utilizing a ``sensory node'', similar effect can be achieved for input to a $NIM$ node, but only $\kappa M$ is able to produce a signal that can be utilized by a second generation ANN without signal processing.
% %	
% %	This is valid for both $NIM$ and $\kappa M$, but only $\kappa M$ is able of producing a floating point output signal without signal processing and/or estimation.
% 	Elements from second generation ANNs can also be utilized direcly in a $\kappa M$ spiking ANN;
% 		$\kappa$ can be used to compute the neuron's ``immediate frequency'' as well as the present and future depolarization values. %, and $\kappa M$ can thus be used for simulating second generation ANNs as well as spiking ANNs.
% 	%E.g. learning rules can be applied directly, as $\kappa$ can be used to compute the neuron's ``immediate frequency'' as well as the present and future depolarization values.
% 	With a few constraints, $\kappa M$ can therefore be used for either a second generation ANN or a spiking ANN.
% 	This makes $\kappa ANN$ very useful for examining theoretical neuroscience, as a full $\kappa M$ simulation can find info om begge..
% 
% %	%Begrunn meir: skriv at 2.gen er godt utvikler, mens SANN er heilt nytt.
% 
% %TODO TODO TODO TODO TODO TODO TODO TODO TODO TODO TODO TODO TODO TODO TODO  MEIR FUTT I SLUTT! TODO TODO TODO TODO TODO TODO TODO TODO TODO TODO TODO TODO 
% 
% 
% 
% 
% 
% 
% 
% 
% % Negative ting med arbeidet mitt:
% % 	- Begge simulatorene er designet som NIM-simulatoren, der axonet simulerer spatiotemporal effekter. Dette er dårlig, og eg gjekk vekk fra dette designet. Det henger likevel igjen fra tidlig i prosjektet, 
% % 		og bør designes på nytt i videre arbeid.
% % 	- Synaptisk transmission i KM er implementert som den deriverte. Dette var for å effektivisere oppdateringen av postsynaptisk neuron, da bare de som var oppdatert trengte å resummeres. 
% % 		(Selv om summering er en lett operasjon kan neuronet ha veldig mange input-synapser).
% % 		Mengden med inputsynapser fører selfølgelig til at de aller fleste neuron (så mange at man kan kalle det "alle") får endret K kvar iter.
% % 		Det ville difor vært bedre å bruke en enklere direkte implementasjon, da dette er mindre "error prone" og er lettere å vedlikeholde. Dette er eit viktig aspekt i videreføringen av dette arbeidet!
% % 	- KM har meir konstant workload, men er dette egentlig bra? Diskuter om resultatet alltid er maks(meir enn det burde være).
% % 		Bra diskurs-materiale! (fordi eg trur at det ender med en følelse om at det er bra, spess for real-time applications)
% % 	- Skriv at når man 'for the case of reproducibility' går vekk fra å bruke eit kaotisk ANN, over til å bruke en enkelt node kan dette tviste effektivitetsanalysen. Dette er lite truleg, pga formen på signalet.
% % 		Hovedfokus i dette prosjektet var å sammenligne accuracy. Dette blir ikkje endra om det kommer fra $\xi(t)$ eller fra synaptisk input. Den analytiske input funksjonen er også designa til å være slik at verken verdien eller den deriverte av en vilkårlig grad er konstant. Bra.
% % 		Diskuter frem og tilbake.. slutt med å diskuter for denne approach.
% 
% % Det meste som var felles (i i\_auron, i\_...) var skriving til logfil osv. (STEMMER DETTE?) Kan isåfall diskutere at det aller meste er heilt ulikt for KM!
% 
% % Skrive om at eg ser i etterkant at eg ikkje burde implementert KM noden som NIM noden. KM noden trenger ikkje simulere intracellular delay, men kan bruke scheduler. (Dette kan for så vidt NIM også)
% 
% % Siden dette også var en sammenligning av design av de to SANN simulator modellene, har to implementasjoner blitt skrevet fra grunnen og sammenlignet. Det hadde kanskje vore bedre å sammenlignet KM med SANN implementert av andre når effektivitetssammenligne, men dette ville tatt bort en del fra første aspektet ved denne oppgaven(teoretisk sammenligning av modellene). BLA BLA. Men kunne være interresant. Men sammenligningen ville vært mellom KM-sample--and--hold og NIM med meir avanserte integrasjonsmetoder.
% 
% 
% % VIDARE ARBEID:
% % 	- Undersøke om det er mulig å gjøre det samme for andre neuron modeller. Kanskje KM er meir egnet til f.eks. exponential neuron model?
% % 	- Transduction mellom generasjoner av ANN(og også andre filter)
% % 	- Har bare sett på det enkle eksempelet med sensor-neuron. Dette er bra for reproducibility, men kan kanskje gjøre noko anna i dette forferdelig komplekse systemet. VIDERE FORSKNING kan difor være å undersøke om dette også stemmer for synaptisk input(ANN).
% % 	- Eg går bare ut ifra at begge modellene har nytte av å bruke meir avanserte numeriske metoder. Kor stor denne nytten er bør undersøkes. Dette har eg satt som utenfor the scope of this project. Dette er viktig element i for further recearch!
% 
%//} 
 
 
% // vim:fdm=marker:fmr=//{,//}

%	\input{artikkel_06.02.2012/concludingRemarks}
%
% 	\appendix
% 	
% TODO TODO TODO TODO TODO TODO TODO TODO TODO TODO TODO TODO TODO TODO TODO  Skriv om denne fila: Skriv tekst og greier. No er det replikat av artikkel-appendix
\chapter{Mathematical Derivations}

\section{Algebraic Solution to the LIF Neuron's Depolarisation}
\label{appendixAlgebraicSolution}
	The subthreshold behaviour of the LIF neuron model can be modelled as a general leaky integrator.
	\begin{equation}
		\begin{split}
			\dot{v}(t)&= \dot{v}_{in}(t) - \dot{v}_{out}(t) \\
				&= I - \alpha v(t)
		\end{split}
		%\nonumber
		\label{appendix:eqDifferentialEquation}
	\end{equation}
		where I represents the input flow and $\alpha$ represents the leakage constant of the value.

		Laplace transform gives
	% TODO TODO TODDO TODO TODO Legg utledning av uttrykk i appendix?
	\begin{equation}
		\begin{split}
			sV(s)-v_0 		&= \frac{I}{s} - \alpha V(s) 			\qquad, \; \qquad v_0 = v(t_0) 				\\
			(s+\alpha)V(s) 	&= \frac{I}{s} + v_0 														\\
			V(s) 			&= \frac{1}{s+\alpha}\left( \frac{I}{s} + v_0 \right)
		\end{split}
		\nonumber
	\end{equation}

	And 
	\begin{equation}
		\begin{split}
			v(t)  	&= 		\mathscr{L}^{-1}\bigg\{ V(s) \bigg\}  									\\
			 		&=		\frac{I}{\alpha} - \frac{I}{\alpha} e^{-\alpha t_w} + v_0 e^{-\alpha t_w} \qquad, \; t_w = t - t_0
% TODO TODO TODO TODO Sjekk om det stemmer at init-value blir trukket fra, som over (t_w = t - t_0)
		\end{split}
		\label{appendix:eqValueEquationUTLEDING}
 	\end{equation}

	The value equation for the leaky integrator with initial value $v_0$ is only valid for time intervals where $I$ and $\alpha$ remain constant.
	This includes any time window, as defined in sec. \ref{ssecTheAlgebraicSolution}.
	%The value equation for the leaky integrator with initial value $v_0$ is only valid for time intervals where $\kappa$ and $\alpha$ remain constant.
	%Such an interval is referred to as a time window, as defined in sec. \ref{ssecTheAlgebraicSolution}.
%	%%We arrive at the value equation for the leaky integrator with initial value $v_0$.
%	%%It is important to emphasize that the value equation \eqref{appendix:eqValueEquationUTLEDING} is only valid for time intervals where $\alpha$ and $I$ remain constant.
%%
	The variable that represents time is measured from the start of the current time window, $t_w = t - t_0$.

\section{Firing Time for the LIF Neuron}
The remainder of the interspike period can be found by 
\label{appendixFiringTime}
		\begin{equation}
			\begin{split}
					v\left(t_w^{(f)}\right)			 							&= \tau \qquad 										\\	%,\qquad\qquad\tau = \text{firing threshold} 	\\
					\kappa - \left( \kappa - v_0 \right) e^{-at_w^{(f)}}  		&= \tau 											\\
			%		(v_0-\kappa)e^{-\alpha t^^{(f)}}							&= \tau-\kappa 										\\
					e^{-\alpha t_w^{(f)}} 			 						&= \frac{\kappa - \tau}{\kappa - v_0} 					\\
					t_w^{(f)}													&= -\alpha^{-1} \, \ln \left( \frac{\kappa - \tau}{\kappa - v_0} \right) 					
			\end{split}
			\label{appendix:eqFiringTimeEstimate_withoutRefractionPeriod}
		\end{equation}
		
		If we define an absolute refraction period $t_r$ for the neuron, where the depolarization will remain constant after firing, this can be added to \eqref{appendix:eqFiringTimeEstimate_withoutRefractionPeriod}.
		%We will call $t^{(f)}$ for the remainder of the depolarizing phase of the inter--spike period, $p_r(\kappa)$.
		We can view $t_w^{(f)}$ as the remainder of the depolarizing phase of the inter--spike period, and denote it $p_r(\kappa, v_0)$.
		\begin{equation}
			p_r(\kappa, v_0)	 = -\alpha^{-1} \, \ln \left( \frac{\kappa - \tau}{\kappa - v_0} \right) + t_r
		\end{equation}



% // vim:fdm=marker:fmr=//{,//}



\chapter{Development of A Novel Neuron Simulation Scheme}
	% TODO Write an intro to this chapter
	The subthreshold build--up of membrane potential of a LIF node have the form of a general leaky integrator.
	In the preliminary work to the final thesis, the author proposed a novel simulation scheme based directly on the solution to the LIF neuron's differential equations.
	In this work, the simulation model has been further developed and tested experimentally.
	This chapter will focus on the mathematical modelling and different necessary concepts for the resulting equations to be utilized in a simulator. %, before an experiment is put up in chapter ??
	%	In this section, the mathematical modelling of the LIF neuron and different necessary concepts are introduced.
%%	In the preliminary work to the final thesis, the author proposed a novel simulation technique that utilize the algebraic solution to the LIF neuron's differential equations.

%	Because the system can be modelled as a capacitor circuit, the author have experimented on utilizing the leaky integrator of the familiar capacitor equation to simulate the neuron's depolarization instead of numerical integration.
	

	
\section{Spiking Neuron Simulation Based on Synaptic Flow}
	\label{secDevelopmentOfTheNovelANNmodel}
	A system that behaves like a leaky itegrator is a bucket with a set of small holes at the bottom.
	%An intuitive leaky integrator is a bucket with a set of small holes at the bottom.
	If the LIF neuron is visualized as a leaky bucket with input from a gutter, excitatory synaptic input can be represented by an agent pouring cups of water into that gutter.
	%If the LIF neuron is visualized as a leaky bucket with input from a gutter, synaptic transmissions is represented by pouring cups of water into this gutter.
	When the number of agents pouring water into the gutter becomes very large and the size of each transmission is small, this can again be visualized as rain.
	%The resulting water level in the leaky bucket can be simulated by either counting the number of raindrops or by estimating the corresponding flow in the input gutter and utilizing the algebraic solution to find the water level.

	The resulting water level in the leaky bucket can either be simulated by counting the number of raindrops(and computing the size of the leakage in every computional time step)
	%The resulting water level in the leaky bucket can be simulated by either counting the number of raindrops(and computing the leakage after each computational time step)
		or by estimating the corresponding flow through the gutter and utilizing the algebraic solution to find the water level.
%%%%%
	%If the simulation has a bounded temporal accuracy(discrete time), the author believes that a more accurate simulation result can be achieved when the algebraic solution is utilized to simulate the systems value. 
	For simulations with a bounded temporal accuracy(discrete time), more accurate simulations may be achieved by utilizing the algebraic solution to simulate the systems value.
	This is tested in chapter \ref{chExperimentalEfficiencyMeasurement}.
	%Experiments test this is set up in chapter \ref{chExperimentalEfficiencyMeasurement}.
	%For simulations with a bounded temporal accuracy(discrete time), it is found that a more accurate simulation result can be achieved when the algebraic solution is utilized to simulate the systems value.
	%TODO Dropp setninga over? NEI: første biten er jævla bra! 			Men det etter her er litt dårlig: avslører for mykje om kva eg finner ut?
%%                 %%                                                          %%                                                  %%                                              % input is represented as a flow.
	%This implies that fewer iterations are needed to accomplish some accuracy goal, and a more efficient simulator model is the result.
	In this section, the mathematics and necessary concepts for a flow simulation is developed and presented.




% 	The subthreshold integration of a LIF neuron can be visualized as a leaky bucket with input from a gutter.
% 	Excitatory synaptic input can further be represented by an agent pouring cups of water into that gutter.
% 	%The subthreshold integration of a LIF neuron can be thought of as a leaky bucket with small holes at the bottom.
% 	%If the LIF neuron is modelled as a leaky bucket with input from a gutter, excitatory synaptic input can be represented by pouring cups of water into that gutter. %this gutter.
% 	When the number of incoming synaptic connections are very large and the size of each transmission is small, this can again be visualized as rain.
% 	The resulting water level in the bucket can either be simulated by counting the number of rain drops and estimating the size of each, or by estimating the corresponding flow out of the gutter and utilizing the 
% 		algebraic solution to find the water level. %algebraic solution to the differential equations to find the water level.
% 	If the simulation has a bounded temporal resolution(discrete time), it is found that a more accurate simulation can be achieved by considering depolarizing flow instead of discrete synaptic transmissions.
% 	In this section, the mathematics and necessary concepts for flow simulation are developed and presented.

	\subsection{The Algebraic Solution to the LIF Neuron's Value}
	\label{ssecTheAlgebraicSolution}
		Subthreshold integration in the LIF neuron is defined by general leaky integrator's differential equations\cite{gerstnerKistler2002KAP04}.
		\begin{equation}
			\begin{split}
				\dot{v}(t)&= \dot{v}_{in}(t) - \dot{v}_{out}(t) \\
					&= I(t) - \alpha v(t)
			\end{split}
			%\nonumber
			\label{eqDifferentialEquation}
		\end{equation}
		The inflow is represented by $\dot{v}_{in}(t) = I(t)$, and $\dot{v}_{out}(t)$ represents the ``leakage'' of the neuron's depolarization value.
		The leakage is thus given as the neuron's present depolarization level scaled by the system's leakage constant $\alpha$.
		The algebraic solution to \ref{eqDifferentialEquation} is derived in appendix \ref{appendixAlgebraicSolution}.
		For time intervals where $\kappa$ and $\alpha$ are constant, it is found that the system's subthreshold depolarization is given by %can be found by TODO Skriv om! "is given by" er dårlig!
		\begin{equation}
			v(t_v) = \kappa - \left( \kappa - v_0 \right) e^{-at_v} 	\quad,\; \kappa = \frac{I}{\alpha} % \quad,\;t_v = t-t_0
			\label{eqValueEquation}
		\end{equation}

		The variable $v_0$ represents the initial value for the neuron's depolarization and $t_v$ is the time from the start of the considered time interval\mbox{($t_v = t - t_0$)}.
% Var sammenkobla med Recall that equation \ref{eqValueEquation} ...
%TODO Lag figur på nytt! Endre litt på teksten som står (t_p -- time from start of period    er dårlig. Bl.a.)
\begin{figure}[htb!p]
    \centering
    \includegraphics[width=0.65\textwidth]{demonstrasjonAvUlikeKappaforVerdifunksjonen}
 	  \caption[Illustration of how time windows can be utilized to simulated the neuron by the algebraic equation]{
	%		A leaky integrator can be simulated by utilizing the concept of time windows.
			The figure shows how the concept of time windows enables the use of \eqref{eqValueEquation} for simulating the neuron's depolarization.
			In the time interval $t_p = [0, 100]$, $\kappa_0 = 0.7$ is valid.
			At time $t_p = 100$, $\kappa$ is changed to $\kappa_1 = 0.5$, before it finally is set to $\kappa_2 = 1$ at time $t_p = 150$.
			}
\end{figure}
		Recall that equation \ref{eqValueEquation} only is valid for time intervals where $\kappa$ and $\alpha$ remain constant.
		To formalize such an interval for later discussions, the concept of time windows is introduced. % defined.
		\begin{mydef}
			A time window is a time interval where $\kappa$ and $\alpha$ are constants, within one inter--spike period.
			\label{defTimeWindow}
		\end{mydef}

		When the neuron's input flow is changed or the neuron fires an action potential, a new time window is initialized.
		The initial value $v_0$ can be found by computing the last value of the previous time window, and $t_0$ is acquired by saving the time of initiation for the new time window.



	\subsection{The Action Potential}
	\label{ssecTheActionPotential}
	As introduced in sec. \ref{secBiologicalNeuralSystems}, the neuron fires an action potential when the depolarization value crosses the firing threshold.
	%In continuous time, 
	The firing time for a neuron in continuous time can therefore be found by the equation $v(t_f) = \tau$, where $\tau$ is the firing threshold for the neuron.

	\begin{equation}
		\begin{split}
				v\left(t_w^{(f)}\right)			 							&= \tau \qquad 										\\	%,\qquad\qquad\tau = \text{firing threshold}
				\kappa - \left( \kappa - v_0 \right) e^{-at_w^{(f)}}  		&= \tau 											\\
		%		(v_0-\kappa)e^{-\alpha t^^{(f)}}							&= \tau-\kappa 										\\
				e^{-\alpha t_w^{(f)}} 			 						&= \frac{\kappa - \tau}{\kappa - v_0} 					\\
				t_w^{(f)}													&= -\alpha^{-1} \, \ln \left( \frac{\kappa - \tau}{\kappa - v_0} \right) 					
		\end{split}
		\label{eqDevelopmentOfFiringTimeEstimateEq}
	\end{equation}

	If an absolute refraction time $t_r$ is defined for the neuron where the depolarization remain constant after firing, this value is added to eq. \eqref{eqDevelopmentOfFiringTimeEstimateEq}.
	%If an absolute refraction time $t_r$ is defined for the neuron where the depolarization remain constant after firing, $t_r$ has to be added to eq. \eqref{eqDevelopmentOfFiringTimeEstimateEq}.
	An other way of viewing the resulting equation is as the remainder of current inter--spike interval, $p_r(\kappa, v_0)$.

	%It is shown in appendix \ref{appendixFiringTime} that the firing time, represented as the remainder of the current inter--spike period can be estimated by % is given by
\begin{equation}
	p_r(\kappa, v_0)  	= -\alpha^{-1} \, \ln \left( \frac{\kappa - \tau}{\kappa - v_0} \right) + t_r
	\label{eqEstimatedTimeToFiring}
\end{equation}

	As eq. \eqref{eqEstimatedTimeToFiring} is derived from \eqref{eqValueEquation}, the same constraints are valid;
		The estimate for the remainder of the current inter--spike interval is only valid until a new time window is initialized.
	%This means that when a new time window is initiated, the old firing time estimate becomes invalid.
%%
	If depolarizing inflow is defined to be constant during a computational time step, a firing time estimate in the current iteration can not change before the estimated time. % the neuron fires.
	%If depolarizing inflow is defined to be constant during a computational time step, a firing time estimate in the present time step can not change before that time. %the neuron fires.
	The estimated firing time can therefore be utilized as the simulation's firing time, and an action potential can be initiated with an intra--iteration time accuracy defined by the data format used in the computations. %, e.g. the \emph{double} data formate. %given by e.g. the \emph{float} data format.
	%The estimated firing time can therefore be utilized as the actual firing time, and an action potential can be initiated at that precise moment.
%%
	The set of possible spike times thus have a near--continuous temporal resolution, only limited by the accuracy of the format used. %e.g. the double precision floating point format.

% asdf@jeje12

% XXX Er det for langt hopp? Vil gjerne gå over til neste section: synaptic flow of activation level.
	An inter--spike interval is finalized by the neuron firing an action potential, after which the neuron's depolarization is reset to the membrane resting potential before the process starts anew.
	%After an action potential, the neuron's depolarization is reset to the membrane resting potential and the process starts anew.
	The current estimate of the total inter--spike interval can be computed by eq. \eqref{eqEstimatedTimeToFiring} from the neuron's reset potential $v_r$.
	%An immediate estimate of the total inter--spike interval can be found by computing eq. \eqref{eqEstimatedTimeToFiring} from the neuron's reset potential $v_r$.
	%The total inter--spike interval can therefore be estimated as the remainder of the inter--spike period from the neuron's reset potential $v_r$.
\begin{equation}
	p_{isi}(\kappa) = p_r(\kappa, v_r)% IKKJE: + t_r
	\label{eqEstimateOfInterSpikePeriod}
\end{equation}
	This equation will show important when we next consider synaptic flow of activation level.
	
	%This process can be modelled by 


    \subsection{Synaptic Flow}
	\label{ssecSynapticFlow}
%	Neural input that changes the neuron's depolarization can be separated into two sets, a subclass of synaptic input that changes the postsynaptic neuron's depolarization and other depolarizing input.
%	Synaptic depolarizing input can be mediated through ligand--gated channels, as introduced in section \ref{ssecTheBiologicalSynapse}.
%	%The synaptic part of depolarizing input can be mediated through ligand--gated channels, as introduced in section \ref{ssecTheBiologicalSynapse}.
%	This is what will be referred to as synaptic input in the remainder of this text.

%todo todo todo todo todo todo todo todo todo todo todo todo todo todo todo todo todo todo todo todo todo todo todo todo todo todo todo todo todo todo todo todo todo 
%todo todo todo todo           Lag en figur som viser skematisk kva input eit neuron har(K_ij og xi_i)                   todo todo todo todo todo todo todo todo todo 
%todo todo todo todo todo todo todo todo todo todo todo todo todo todo todo todo todo todo todo todo todo todo todo todo todo todo todo todo todo todo todo todo todo 

	Let all synaptic input be modelled as the flow $\kappa_{ij}$, where $j$ represents the presynaptic neuron and $i$ the receiving neuron.
	Other input that changes neuron $i$'s depolarization is represented by $\xi_i(t)$.
	The final value for the neuron's depolarization, $\kappa_i = \frac{I_i}{\alpha}$, is defined as the sum of all the neuron's input flows.
	%The final value for the neuron's depolarization $\kappa_i$ is defined by the sum of all input flows for neuron $i$.
	If $\mathscr{D}$ is the set of integers representing neuron $i$'s presynaptic neurons, the total inflow during the $n$'th iteration can be written as
	%The total inflow in the $n$'th iteration can therefore be written as

		\begin{equation}
% TODO HUGS: K = I/a : dermed må I være sum(k_ij + xi)*alpha
			% I_{i, t_n} = \sum_{j} \kappa_{ij, t_n} + \xi_{i, t_n}
			\begin{split}
			I_{i, t_n} 	&= \kappa_{i,t_n} \cdot \alpha \\
						&= \left( \sum_{j} \kappa_{ij, t_n} + \xi_i(t_n) \right) \alpha \quad,\; j\in\mathscr{D}
			\end{split}
			\label{eqSynapticIntegrationForKANN}
		\end{equation}

	The most important depolarizing input for neural signal processing is synaptic input\cite{PrinciplesOfNeuralScience4edKAP10}.
	%The most important depolarizing input when it comes to neural signal processing is synaptic input\cite{PrinciplesOfNeuralScience4edKAP10}.
	Synaptic input will therefore be the main focus in this section.
	%The main focus of this section is therefore synaptic transmissions.
	%The main focus of this section will therefore be synaptic transmissions.
	The funtion $\xi_i(t)$, representing other input, have different forms for different depolarizing sources and have to be modelled separately for each such source.
	%Other input $\xi_i(t)$ have different forms for different sources and have to be modelled separately for different such mechanisms.
	%The form of other input $\xi_i(t)$ varies for different sources of the signal and have to be modelled separately for each such mechanism. 
	%XXX BRA XXX: One example of another source for changing a neuron's depolarization is the instrumentation done by sensory neurons. %TODO TODO TODO Skriv om dette en plass, og referer dit!  

\begin{figure}[hbt!p]
	\centering
	\includegraphics[width=0.70\textwidth]{epsp_ipsp}
	\caption[Illustration of neural integration of synaptic input]{
			A simulation of neural integration of synaptic input. 
			Excitatory Postsynaptic Potentials(EPSP) increase the membrane potential of the postsynaptic neuron and thus excite the neuron toward firing.
			Inhibitory Postsynaptic Potentials(IPSP) hyperpolarizes the postsynaptic neuron, and inhibits the postsynaptic neuron with respect to firing.
			When the membrane potential at the axon hillock crosses the firing threshold, set to $-10mV$, an action potential is fired.
			%Figuren kommer fra http://techlab.bu.edu/resources/software_view/epsp_ipsp/
			%The simulation result presented in the figure is produced with the educational ``\emph{EPSP IPSP}'' software intended to illustrate EPSP and IPSP after synaptic transmissions.
			(The figure is found on the website of the educational ``\emph{EPSP IPSP}'' software intended for illustration of EPSP and IPSP after synaptic transmissions).
			% TODO Gjør forrige setninga mindre, og FÅ MED AT DET IKKJE ER EG SOM HAR LAGA DEN!
				}
	\label{figIllustrationOfEPSPandIPSP}
\end{figure}


	Let the synaptic weight $\omega_{ij}$ be defined as the postsynaptic change in depolarization after one synaptic transmission. 	
	%Let the synaptic weight $\omega_{ij}$ be defined as the postsynaptic change in depolarization after one transmission in the synapse.
	Synapse $j$'s contribution to the total change in depolarization after a time interval $\Delta t$ can therefore be written as the number of transmissions in that interval scaled by the synaptic weight $\omega_{ij}$.
	%In discrete time simulations, this can be written as
	\begin{equation}
% TODO Skriv det som N
%		\Delta v_i(\Delta t) = f_j(t_{n-1})\Delta t \cdot\omega_{ij} = \frac{\omega_{ij}}{p_{isi}(t_{n-1}}
		%\Delta v_{i, t_n}(\Delta t) = N_{j,t_n}\cdot\omega_{ij, t_n} %								%= f_j(t_{n-1})\Delta t \cdot\omega_{ij} % = \frac{\omega_{ij}}{p_{isi}(t_{n-1}}
		\Delta v_{i}(\Delta t_n) = N_{j,\Delta t}\cdot\omega_{ij, t_{n-1}} \qquad,\; j\in\mathscr{D}%								%= f_j(t_{n-1})\Delta t \cdot\omega_{ij} % = \frac{\omega_{ij}}{p_{isi}(t_{n-1}}
	\end{equation}
	where $N_{j,t_n}$ represents the number of transmissions in the synapse from neuron $j$ to neuron $i$ in the time interval $\Delta t_n$, 
	and $\omega_{ij,t_{n-1}}$ the synaptic weight updated at time $t_{n-1}$.
	%where the variable $N_{j,t_n}$ represents the number of transmissions from neuron $j$ in time interval $\Delta t_n$, and $\omega_{ij, t_{n-1}}$ represents the synaptic weight updated at time $t_{n-1}$.
	%where the number of transmissions is found by the last computed firing frequency of the presynaptic neuron $f_j(t_{n-1})$ multiplied by the length of the time interval $\Delta t$.

	%In the flow simulation model($\kappa M$), a continuous variable representing the present estimate of the inter--spike interval can be sent instead of the integer number of transmissions, enabling a higher resolution for the propagated signal.
	In the flow simulation model($\kappa M$), a continuous variable representing the present estimate of the inter--spike interval can be sent instead of the integer number of transmissions. 
	This enables a higher resolution for the propagated signal and thus a more accurate simulation.
	%For a time interval where the presynaptic activation level $\kappa_j$ is constant(a time window for the presynaptic neuron), synaptic flow of activation level can be written as
	For a time interval where the presynaptic activation level $\kappa_j$ is constant, synaptic flow of activation level can be written as
	\begin{equation}
	%	\kappa_{ij} = \frac{ \omega_{ij} }{ p_{isi}(\kappa_{j})}\Delta t
		\kappa_{ij, t_n} = \frac{ \omega_{ij, t_n} }{ p_{isi}(\kappa_{j, t_n}) } \Delta t \qquad,\; j\in\mathscr{D}% TODO SKRIV kva \Delta t   er for noke! TODO TODO SKVIVE DET SOM FREKVENS, først? = f(t) \omega \cdot \Delta t
	\end{equation}

	For a simulation with constant computational time steps $\Delta t = C_t$, this constant can be incorporated into the equation for synaptic flow $\kappa_{ij}$.
	%If a simulation with constant computational time steps $\Delta t = C_t$ is considered, this constant can be incorporated into the equation for synaptic flow $\kappa_{ij}$.
	%
	%If we let the simulation be carried out with constant time steps $\Delta t = C_t$, this constant can be incorporated into the equation for synaptic flow $\kappa_{ij}$.
	% ELLER:
	%Let the simulation be carried out with constant time steps $\Delta t = C_t$.
	%This constant can then be incorporated into the equation for synaptic flow $\kappa_{ij}$.
	We arrive at the equation for synaptic flow of activation level for constant time steps:
	\begin{equation}
		\kappa_{ij} = \frac{ \omega_{ij} }{ p_{isi}(\kappa_{j})} \qquad,\;j\in\mathscr{D}
		\label{eqSynapticTransmissionForKANN}
	\end{equation}
	
	When synaptic plasticity is introduced, it is important to remember that synaptic weight is scaled by the constant $C_t$.
	For consistency, it is important to scale synaptic plasticity by the same factor.

%[Her stod tidligare en analyse av feilen for de to ANN modellene]. Dette er flyttet inn i analysisOfTheTwoModels.tex



	\section{New Aspects to be Considered for the Novel Model}
		The use of the theory presented in this chapter introduce new aspects that have to be considered as well as opportunities for the simulator. % implementation.
		%The use of the theory presented in this chapter introduce a some new considerations and opportunities for the simulator. % implementation.
		%The use of the theory presented in this chapter introduce a some new considerations and opportunities for the implementation.
		Because the activation level $\kappa$ is updated many times before the neuron fires, time windows have to be utilized to be able to simulate utilizing the $\kappa M$. %a spiking neuron by $\kappa M$.
		From equation \ref{eqEstimatedTimeToFiring} and \ref{eqEstimateOfInterSpikePeriod}, the next firing and the inter--spike period can be estimated with a floating point accuracy.
		This enables a synaptic signal propagation of a number with a higher resolution, and the execution of an action potential at the computed firing time instant.
		
		When equation \ref{eqEstimatedTimeToFiring} have given an estimate that is in the present computational time step, an action potential is simulated.
		%When the estimated firing time is in the present time iteration, an action potential is simulated.
		The simulated firing is not involved in signal propagation as in the $NIM$ model, but is an additional capability for the $\kappa M$ simulation model.
		%The main reason for simulating the action potential in the $\kappa M$ is to compute mechanisms like STDP, as presented in appendix \ref{appendixSynapticPlasticity}.
		The neuron fires when the estimated task time is in the present computational time step.
		To be able to efficiently make use of this proactive firing time simulation scheme, a task scheduler have to be devised specifically for this purpose.
		%This proactive firing scheme in $\kappa M$ requires a task scheduler to be able to efficiently simulate the neuron.

		%TODO Kanskje ha "synaptic transmission as the derivative", her? Da kan eg også ha "recalculation of kappa", her..


%TODO TODO TODO TODO TODO TODO TODO TODO TODO TODO TODO TODO TODO TODO TODO TODO TODO TODO TODO TODO TODO TODO TODO TODO TODO TODO TODO TODO TODO TODO TODO TODO TODO 
%TODO TODO TODO TODO TODO Skriv noke nytt, her. Skal flytte "Task Scheduling" til "General Design of Simulator"::"Time" Det er fortidlig å ha det her.
%TODO TODO TODO TODO TODO TODO TODO TODO TODO TODO TODO TODO TODO TODO TODO TODO TODO TODO TODO TODO TODO TODO TODO TODO TODO TODO TODO TODO TODO TODO TODO TODO TODO 
		
% 		\subsection{Task Scheduling}
% 			
% 			Two alternatives for scheduling tasks have been tested for the simulator.
% 			The first is based on a continuously updated linked list of linked lists with tasks. %that can be considered a variable array.
% 			When a task is scheduled for execution e.g. in the iteration after the next, the object's pointer is inserted into the second inner list of the outer linked list.
% 			Before every time step, the first element of the outer list is popped and all the tasks of the inner list is inserted into \emph{pWorkTaskQueue}.
% 			This gives a list of lists that gives the relative time of scheduled tasks, where each list contains jobs scheduled for future time iterations.
% 			
% 			An alternative approach is to implement time scheduling by letting the \emph{time\_interface} abstract class have a variable \emph{double dEstimatedTaskTime}.
% 			This element is updated every time the neuron's firing time estimate is updated and checked by \emph{time\_class::doTask()} when time is iterated:
% 				If an element is scheduled for execution during the next computational time step, the pointer to that element is inserted into \emph{pWorkTaskQueue}.
% 			As introduced in section \ref{ssecTime}, this causes the task to be executed during the correct computational time step, 
%  			%This causes the task to be executed during the correct computational time step, 
% 				and the double precision floating point variable \emph{dEstimatedTaskTime} enables an intra--iteration time accuracy for tasks if \emph{pWorkTaskQueue} is
% 				 sorted by this variable.
% 
% 			The two methods was tested by comparing the total run time for a similar experiment set up.
% 			Because the second alternative is simpler to implement and thus simpler to maintain,
% 				and it was found to have about the same grade of efficiency(almost $5\%$ faster for the conducted experiment),
% 				%and have about the same grade of efficiency(about $5\%$ faster for the conducted experiment), 
% 				this approach is used for time scheduling in the implementation.
% 				%the alternative with the \emph{time\_interface::dEstimatedTaskTime} is used for time scheduling in this implementation.
% 			%The second alternative was somewhat more efficient($<5\%$ faster run time) in addition to being simpler to implement and maintain.
% 			%This alternative was therefore chosen.
% 			
% 			\subsubsection{Task Scheduling for Other Tasks}
% 				%The task scheduler utilize a variable from \emph{
% 				As the task scheduler use a member variable from \emph{class time\_interface}, task scheduling can be used for all classes that is part of the simulation.
% 				An important example of this is the \emph{synapse}: % The synaptic transmission for all output synapses of a node can therefore 
% 					When the neuron fires, the auron object of the node can write to all the node's output synapses' \emph{dEstimatedTaskTime} variable.
% 				The time can be written to the present time plus the predefined axonic delay before that synapse's transmission.
% 				In this way, a more efficient axon delay can be simulated with floating point accuracy.
%TODO TODO TODO TODO TODO TODO TODO TODO TODO TODO TODO TODO TODO TODO TODO TODO TODO TODO TODO TODO TODO TODO TODO TODO TODO TODO TODO TODO TODO TODO TODO TODO TODO TODO TODO TODO TODO TODO TODO TODO TODO TODO TODO TODO TODO TODO TODO TODO TODO TODO TODO TODO TODO TODO TODO TODO TODO TODO TODO TODO TODO TODO TODO TODO TODO TODO TODO TODO TODO TODO TODO TODO TODO TODO TODO TODO TODO TODO TODO TODO TODO TODO TODO TODO TODO TODO TODO TODO TODO TODO TODO TODO TODO TODO TODO TODO TODO TODO TODO TODO TODO TODO TODO TODO TODO TODO TODO TODO TODO TODO TODO TODO TODO TODO TODO TODO TODO TODO TODO TODO TODO TODO TODO TODO TODO TODO TODO TODO TODO TODO TODO TODO TODO TODO TODO TODO TODO TODO TODO TODO TODO TODO TODO TODO TODO TODO TODO TODO TODO TODO TODO TODO TODO TODO TODO TODO TODO TODO TODO TODO TODO TODO TODO TODO TODO TODO TODO TODO TODO TODO TODO TODO TODO TODO TODO TODO TODO TODO TODO TODO TODO TODO TODO TODO TODO TODO TODO TODO TODO TODO TODO TODO TODO TODO TODO TODO TODO TODO TODO TODO TODO TODO TODO TODO TODO TODO TODO TODO TODO TODO TODO TODO TODO TODO TODO TODO TODO TODO TODO TODO TODO TODO TODO TODO TODO TODO TODO TODO TODO TODO TODO TODO TODO TODO TODO TODO TODO TODO TODO TODO TODO TODO TODO TODO TODO TODO TODO TODO TODO TODO TODO TODO TODO TODO TODO TODO TODO TODO TODO TODO TODO TODO TODO TODO TODO TODO TODO TODO TODO TODO TODO TODO TODO TODO TODO TODO TODO TODO TODO TODO TODO TODO TODO TODO TODO TODO TODO TODO TODO TODO TODO TODO TODO TODO TODO TODO TODO TODO TODO TODO TODO TODO TODO TODO TODO TODO TODO TODO TODO TODO TODO TODO TODO TODO TODO TODO TODO TODO TODO TODO TODO TODO TODO TODO TODO TODO TODO TODO TODO TODO TODO TODO TODO TODO TODO TODO TODO TODO TODO TODO TODO TODO TODO TODO TODO TODO TODO TODO TODO TODO TODO TODO TODO TODO TODO TODO TODO TODO TODO TODO TODO TODO TODO TODO TODO TODO TODO TODO TODO TODO TODO TODO TODO TODO TODO TODO TODO TODO TODO TODO TODO TODO TODO TODO TODO TODO TODO TODO TODO TODO TODO TODO TODO TODO TODO TODO TODO TODO TODO TODO 





	
% // vim:fdm=marker:fmr=//{,//}

		% Skal inneholde:  		\section{The novel model: Modelling}  : 	\label{ssecModellingRefractionTimeToLimitFiringFrequency}
	%\include{timeAndError} ... eller noke



\appendix
	
\chapter{Synaptic Transmission and Plasticity in the Biological Neuron}
\label{appendixSynPlast}



\bibliography{bibliografi}
%\bibliographystyle{abbrvnat}
\bibliographystyle{plain}
\end{document}

% // vim:fdm=marker:fmr=//{,//}
