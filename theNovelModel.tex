
\section{Spiking Neuron Simulation Based on Synaptic Flow}
	\label{secDevelopmentOfTheNovelANNmodel}
	An intuitive leaky integrator is a bucket with a set of small holes at the bottom.
	%Because this system is simple to visualize, it is 
	If the LIF neuron is visualized as a leaky bucket with input from a gutter, excitatory synaptic input can be represented by an agent pouring cups of water into that gutter.
	%If the LIF neuron is visualized as a leaky bucket with input from a gutter, synaptic transmissions is represented by pouring cups of water into this gutter.
	When the number of agents pouring water into the gutter becomes very large and the size of each transmission is small, this can again be visualized as rain.
	%The resulting water level in the leaky bucket can be simulated by either counting the number of raindrops or by estimating the corresponding flow in the input gutter and utilizing the algebraic solution to find the water level.
	The resulting water level in the leaky bucket can either be simulated by counting the number of raindrops(and computing the size of the leakage in every computional time step)
	%The resulting water level in the leaky bucket can be simulated by either counting the number of raindrops(and computing the leakage after each computational time step)
		or by estimating the corresponding flow through the gutter and utilizing the algebraic solution to find the water level.
%%%%%
	%If the simulation has a bounded temporal accuracy(discrete time), the author believes that a more accurate simulation result can be achieved when the algebraic solution is utilized to simulate the systems value. 
	When the simulation has a bounded temporal accuracy(discrete time), it is found that a more accurate simulation result can be achieved when the algebraic solution is utilized to simulate the systems value.
	%TODO Dropp setninga over? NEI: første biten er jævla bra! 			Men det etter her er litt dårlig: avslører for mykje om kva eg finner ut?
%%                 %%                                                          %%                                                  %%                                              % input is represented as a flow.
	%This implies that fewer iterations are needed to accomplish some accuracy goal, and a more efficient simulator model is the result.
	In this section, the mathematics and necessary concepts for a flow simulation is developed and presented.




% 	The subthreshold integration of a LIF neuron can be visualized as a leaky bucket with input from a gutter.
% 	Excitatory synaptic input can further be represented by an agent pouring cups of water into that gutter.
% 	%The subthreshold integration of a LIF neuron can be thought of as a leaky bucket with small holes at the bottom.
% 	%If the LIF neuron is modelled as a leaky bucket with input from a gutter, excitatory synaptic input can be represented by pouring cups of water into that gutter. %this gutter.
% 	When the number of incoming synaptic connections are very large and the size of each transmission is small, this can again be visualized as rain.
% 	The resulting water level in the bucket can either be simulated by counting the number of rain drops and estimating the size of each, or by estimating the corresponding flow out of the gutter and utilizing the 
% 		algebraic solution to find the water level. %algebraic solution to the differential equations to find the water level.
% 	If the simulation has a bounded temporal resolution(discrete time), it is found that a more accurate simulation can be achieved by considering depolarizing flow instead of discrete synaptic transmissions.
% 	In this section, the mathematics and necessary concepts for flow simulation are developed and presented.

	\subsection{The Algebraic Solution to the LIF Neuron's Value}
	\label{ssecTheAlgebraicSolution}
		Subthreshold integration in the LIF neuron is defined by general leaky integrator's differential equations\cite{gerstnerKistler2002KAP04}.
		\begin{equation}
			\begin{split}
				\dot{v}(t)&= \dot{v}_{in}(t) - \dot{v}_{out}(t) \\
					&= I(t) - \alpha v(t)
			\end{split}
			%\nonumber
			\label{eqDifferentialEquation}
		\end{equation}
		The inflow is represented by $\dot{v}_{in}(t) = I(t)$, and $\dot{v}_{out}(t)$ represents the ``leakage'' of the neuron's depolarization value.
		The leakage is thus given as the neuron's present depolarization level scaled by the system's leakage constant $\alpha$.
		The algebraic solution to \ref{eqDifferentialEquation} is derived in appendix \ref{appendixAlgebraicSolution}.
		For time intervals where $\kappa$ and $\alpha$ are constant, it is found that the system's subthreshold depolarization is given by %can be found by TODO Skriv om! "is given by" er dårlig!
		\begin{equation}
			v(t_v) = \kappa - \left( \kappa - v_0 \right) e^{-at_v} 	\quad,\; \kappa = \frac{I}{\alpha} % \quad,\;t_v = t-t_0
			\label{eqValueEquation}
		\end{equation}

	
		The variable $v_0$ represents the initial value for the neuron's depolarization and $t_v$ is the time from the start of the considered time interval\mbox{($t_v = t - t_0$)}.
		Recall that equation \ref{eqValueEquation} only is valid for time intervals where $\kappa$ and $\alpha$ remain constant.
		To formalize such an interval for later discussions, the concept of time windows is introduced. % defined.
		\begin{mydef}
			A time window is a time interval where $\kappa$ and $\alpha$ are constants, within one inter--spike period.
			\label{defTimeWindow}
		\end{mydef}
		When the neuron's input flow is changed or the neuron fires an action potential, a new time window is initialized.
		The initial value $v_0$ can be found by computing the last value of the previous time window, and $t_0$ is acquired by saving the time of initiation for the new time window.


%TODO Lag figur på nytt! Endre litt på teksten som står (t_p -- time from start of period    er dårlig. Bl.a.)
%TODO TODO TODO TODO TODO TODO TODO TODO TODO TODO TODO TODO 
\begin{figure}[htb!p]
    \centering
    \includegraphics[width=0.65\textwidth]{demonstrasjonAvUlikeKappaforVerdifunksjonen}
 	  \caption{
	%		A leaky integrator can be simulated by utilizing the concept of time windows.
			The figure shows how the concept of time windows enables the use of \eqref{eqValueEquation} for simulating the neuron's depolarization.
			In the time interval $t_p = [0, 100]$, $\kappa_0 = 0.7$ is valid.
			At time $t_p = 100$, $\kappa$ is changed to $\kappa_1 = 0.5$, before it finally is set to $\kappa_2 = 1$ at time $t_p = 150$.
			}
\end{figure}

	\subsection{The Action Potential}
	\label{ssecTheActionPotential}
	As introduced in sec. \ref{secBiologicalNeuralSystems}, the neuron fires an action potential when the depolarization value crosses the firing threshold.
	%In continuous time, 
	The firing time for a neuron in continuous time can be found by the equation $v(t_f) = \tau$, where $\tau$ is the firing threshold for the neuron.
	It is shown in appendix \ref{appendixFiringTime} that the firing time, represented as the remainder of the current inter--spike period can be estimated by % is given by
\begin{equation}
	p_r(\kappa, v_0)  	= -\alpha^{-1} \, \ln \left( \frac{\kappa - \tau}{\kappa - v_0} \right) + t_r
	\label{eqEstimatedTimeToFiring}
\end{equation}

	As the equation for the remainder of the inter--spike period is derived from \eqref{eqValueEquation}, the estimate is only valid for as long as $\kappa$ and $\alpha$ remains constant.
	This means that when a new time window is initiated, the old firing time estimate becomes invalid.
%%
	%If the depolarizing inflow is defined to be constant during a computational time step, a firing that is estimated to happen during the present time iteration can not change prior to this time.
	%If the depolarizing inflow is defined to be constant during a computational time step, a firing time estimate during the present time step can not change before that time. %the neuron fires.
	If the depolarizing inflow is defined to be constant during a computational time step, a firing time estimate in the present computational time step can not change before that time. %the neuron fires.
	The estimated firing time can therefore be utilized as the actual firing time, and an action potential can be initiated at that precise moment.
%%
	The set of possible spike times thus have a near--continuous temporal resolution, only limited by the accuracy of the format used. %e.g. the double precision floating point format.
	If e.g. the double precision floating point format is utilized, the IEEE standard defines the smallest number to be given by an exponent of $-308$\cite{kreyszig8edKAP17}. 
	%For e.g. the double precision floating point format, the IEEE standard defines the smallest number to be given by an exponent of $-308$\cite{kreyszig8edKAP17}. 
	This implies an accuracy where the numbers are separated by a step down to $10^{-308}$ time units.
	%For a discussion about what this results in for the simulation error, it is referred to sec. \ref{ssecAnalysisOfErrorsForTheTwoModels}.
	If all tasks are executed according to estimated spike times, a task planned slightly before another will therefore be initiated before that task despite being scheduled in the same computational time step.
	% TODO Skriv om neste setning, slik at det bare er ei leddsetning!
	%In some situations, this will have a large effect on mechanisms defined by the relative spike times of two neurons, e.g. Spike--Time Dependent Plasticity as mentioned in appendix \ref{appendixSynapticPlasticity}.
	This will have a large effect on mechanisms defined by the relative spike times of two neurons, e.g. Spike--Time Dependent Plasticity as mentioned in appendix \ref{appendixSynapticPlasticity}.
	%This could have a large effect on mechanisms defined by the relative spike times of two neurons, e.g. Spike--Time Dependent Plasticity as mentioned in appendix \ref{appendixSynapticPlasticity}.

%TODO TODO TODO TODO TODO TODO TODO TODO TODO TODO TODO TODO TODO TODO TODO 
%TODO TODO TODO TODO TODO TODO TODO TODO TODO TODO TODO TODO TODO TODO TODO  Finn figur som viser AP. Skriv at dette er formen på det biologiske AP.
%TODO TODO TODO TODO TODO TODO TODO TODO TODO TODO TODO TODO TODO TODO TODO 


%TODO TODO TODO TODO TODO TODO TODO TODO TODO TODO TODO TODO TODO TODO TODO HER ER EG!

	The perhaps most important effect of having a near--continuous temporal resolution for the simulation error is that the next inter--spike interval is initiated at the correct time. %computed time instant.
	After an action potential(and the predefined absolute refraction period), the neuron can start charging the membrane potential at the right time. %computed time instant. 
	%%
	% TODO Har eg definert "reactive firing scheme" tidligere?
	%With the reactive firing scheme in simulations utilizing numerical integration, the firing have to be delayed to the next iteration to preserve causality in a neural network. %TODO Finn referanse! TODO
	With the reactive firing scheme in simulations utilizing numerical integration, where the neuron fires as a reaction to the depolarizing going to suprathreshold levels, 
		the firing have to be delayed to the next iteration to preserve causality in a neural network. %TODO Finn referanse! TODO
	%With a discretization of possible spike times, the neuron have to wait for the iteration after the threshold crossing to preserve causality in the neural network.
	This  introduces a small delay of up to one computational time step before the neuron can start depolarizing again.
	%This gives a small delay of up to one computational time step before the neuron can start depolarizing again.
%% 										%% 																	%% 														%% 			har ikkje innført 	 efficiency som funksjon av accuracy, enda.
	%A precise initiation of the next inter--spike interval caused by the near--continuous temporal resolution for possible spike times removes this error mechanism, and might be important for the efficiency of the neural simulator. %xxx
	A precise initiation of the next inter--spike interval caused by the near--continuous temporal resolution for possible spike times removes this error mechanism, and might be important for the accuracy of the simulated depolarization.
	%A precise initiation of the next inter--spike interval will therefore remove an important error mechanisms in spiking neuron simulations.
	%%A precise initiation of the next inter--spike interval will remove an error mechanism that might give an important part of the total error in spiking neuron simulations that utilize numerical integration. %when simulating depolarization by numerical integration.
	%For a more elaborate discussion of the error induced by discrete possible spike times, it is referred to sec. \ref{ssecAnalysisOfErrorsForTheTwoModels}.
	For a more elaborate discussion of the error induced by the discretization of time and discrete possible spike times, it is referred to sec. \ref{ssecAnalysisOfErrorsForTheTwoModels}.

% XXX Er det for langt hopp? Vil gjerne gå over til neste section: synaptic flow of activation level.
	An inter--spike interval is finalized by the neuron firing an action potential, after which the neuron's depolarization is reset to the membrane resting potential before the process starts anew.
	%After an action potential, the neuron's depolarization is reset to the membrane resting potential and the next inter--spike interval starts.
	%After an action potential, the neuron's depolarization is reset to the membrane resting potential and the process starts anew.
	The total inter--spike interval can therefore be estimated as the remainder of the inter--spike period from the neuron's reset potential $v_r$.
\begin{equation}
	p_{isi}(\kappa) = p_r(\kappa, v_r)% IKKJE: + t_r
	\label{eqEstimateOfInterSpikePeriod}
\end{equation}
	This equation will show important when we next consider synaptic flow of activation level.
	
	%This process can be modelled by 


    \subsection{Synaptic Flow}
	\label{ssecSynapticFlow}
	Neural input that changes the neuron's depolarization can be divided into two groups, a subclass of synaptic input that changes the postsynaptic neuron's depolarization and other depolarizing input.
	%Other input include the activation of a sensory neuron as a consequence of the sensed signal, and is represented by the 
	The synaptic part of depolarizing input can be mediated through ligand--gated channels, as introduced in section \ref{ssecTheBiologicalSynapse}.
	%Synaptic input that alters the postsynaptic membrane potential can be mediated through ligand--gated channels, as introduced in section \ref{ssecTheBiologicalSynapse}.
	This class of neural input is what will be referred to as synaptic input in the remainder of this text.
	%This class of neural input will be referred to as synaptic input in the remainder of this text.
%%
%%%	Input that change the depolarization of the neuron comes in many forms. 
%	A class of these is a subset of synaptic input that alters the postsynaptic neuron's depolarization.
%	This class of input will be referred to as synaptic input in the remainder of this text.

	Let all synaptic input be modelled as the flow $\kappa_{ij}$, where $j$ represents the presynaptic neuron and $i$ the receiving neuron.
	Other input that changes neuron $i$'s depolarization is represented by $\xi_i(t)$.
	The final value for the neuron's depolarization, $\kappa_i$, is defined by the sum of all the neuron's input flows.
	%The final value for the neuron's depolarization $\kappa_i$ is defined by the sum of all input flows for neuron $i$.
	The total inflow in the $n$'th iteration can therefore be written as

		\begin{equation}
% TODO HUGS: K = I/a : dermed må I være sum(k_ij + xi)*alpha
			% I_{i, t_n} = \sum_{j} \kappa_{ij, t_n} + \xi_{i, t_n}
			I_{i, t_n} = \kappa_{i,t_n} \cdot \alpha = \left( \sum_{j} \kappa_{ij, t_n} + \xi_i(t_n) \right) \cdot \alpha
			\label{eqSynapticIntegrationForKANN}
		\end{equation}

	The most important depolarizing input, when it comes to neural signal processing, is synaptic input\cite{PrinciplesOfNeuralScience4edKAP10}.
	Synaptic transmissions will therefore be the main focus of this section.
	%The most important aspect for the neuron's signal processing capabilities comes as a consequence of synaptic transmission in networks of neurons, and will be the main focus in this section.
	The form of other input $\xi_i(t)$ varies for different sources of the signal and have to be modelled separately for each such mechanism. 
	%XXX BRA XXX: One example of another source for changing a neuron's depolarization is the instrumentation done by sensory neurons. %TODO TODO TODO Skriv om dette en plass, og referer dit!  

\begin{figure}[hbt!p]
	\centering
	\includegraphics[width=0.70\textwidth]{epsp_ipsp}
	\caption{A simulation of neural integration of synaptic input. 
			Excitatory Postsynaptic Potentials(EPSP) increase the membrane potential of the postsynaptic neuron and thus excite the neuron toward firing.
			Inhibitory Postsynaptic Potentials(IPSP) hyperpolarizes the postsynaptic neuron, and inhibits the postsynaptic neuron with respect to firing.
			When the membrane potential at the axon hillock crosses the firing threshold, set to $-10mV$, an action potential is fired.
			%Figuren kommer fra http://techlab.bu.edu/resources/software_view/epsp_ipsp/
			%The simulation result presented in the figure is produced with the educational ``\emph{EPSP IPSP}'' software intended to illustrate EPSP and IPSP after synaptic transmissions.
			(The figure is found on the website of the educational ``\emph{EPSP IPSP}'' software intended for illustration of EPSP and IPSP after synaptic transmissions).
			% TODO Gjør forrige setninga mindre, og FÅ MED AT DET IKKJE ER EG SOM HAR LAGA DEN!
				}
	\label{figFigurAvNeuronet}
\end{figure}


	Let the synaptic weight $\omega_{ij}$ be defined as the postsynaptic change in depolarization after one transmission in the synapse.
	Synapse $j$'s contribution to the total change in depolarization after a time interval $\Delta t$ can therefore be defined as the number of transmissions in that interval, scaled by the synaptic weight $\omega_{ij}$.
	In discrete time simulations, this can be written as
	\begin{equation}
% TODO Skriv det som N
%		\Delta v_i(\Delta t) = f_j(t_{n-1})\Delta t \cdot\omega_{ij} = \frac{\omega_{ij}}{p_{isi}(t_{n-1}}
		%\Delta v_{i, t_n}(\Delta t) = N_{j,t_n}\cdot\omega_{ij, t_n} %								%= f_j(t_{n-1})\Delta t \cdot\omega_{ij} % = \frac{\omega_{ij}}{p_{isi}(t_{n-1}}
		\Delta v_{i}(\Delta t_n) = N_{j,\Delta t_n}\cdot\omega_{ij, t_{n-1}} %								%= f_j(t_{n-1})\Delta t \cdot\omega_{ij} % = \frac{\omega_{ij}}{p_{isi}(t_{n-1}}
	\end{equation}
	where the variable $N_{j,t_n}$ represents the number of transmissions from neuron $j$ in time interval $\Delta t_n$, and $\omega_{ij, t_{n-1}}$ represents the synaptic weight updated at time $t_{n-1}$.
	%where the number of transmissions is found by the last computed firing frequency of the presynaptic neuron $f_j(t_{n-1})$ multiplied by the length of the time interval $\Delta t$.

	In the flow simulation model($\kappa M$) presented in this text, a continuous variable representing the present estimate of the inter--spike interval can be propagated instead of the integer number of transmissions,
		enabling a higher resolution for the propagated signal.
	For a time interval where the presynaptic activation level $\kappa_j$ is constant(a time window for the presynaptic neuron), synaptic flow of activation level can be written as
	\begin{equation}
	%	\kappa_{ij} = \frac{ \omega_{ij} }{ p_{isi}(\kappa_{j})}\Delta t
		\kappa_{ij, t_n} = \frac{ \omega_{ij, t_n} }{ p_{isi}(\kappa_{j, t_n}) } \Delta t % TODO SKRIV kva \Delta t   er for noke! TODO TODO SKVIVE DET SOM FREKVENS, først? = f(t) \omega \cdot \Delta t
	\end{equation}

	If a simulation with constant computational time steps $\Delta t = C_t$ is considered, this constant can be incorporated into the equation for synaptic flow $\kappa_{ij}$.
	%If we let the simulation be carried out with constant time steps $\Delta t = C_t$, this constant can be incorporated into the equation for synaptic flow $\kappa_{ij}$.
	% ELLER:
	%Let the simulation be carried out with constant time steps $\Delta t = C_t$.
	%This constant can then be incorporated into the equation for synaptic flow $\kappa_{ij}$.
	We arrive at the equation for synaptic flow of activation level for constant time steps:
	\begin{equation}
		\kappa_{ij} = \frac{ \omega_{ij} }{ p_{isi}(\kappa_{j})}
		\label{eqSynapticTransmissionForKANN}
	\end{equation}
	
	When synaptic plasticity is introduced, it is important to remember that synaptic weight is scaled by the constant $C_t$.
	%If synaptic plasticity is introduced, it is important to remember that synaptic weight is scaled by the constant $C_t$.
	For consistency, it is important to scale synaptic plasticity by the same factor.


\section{Time and Error for the Two Models}
 	\label{ssecAnalysisOfErrorsForTheTwoModels}
% TODO TODO TODO Skriv på nytt! TODO TODO TODO Introen er litt dårlig.
% TODO TODO TODO TODO TODO TODO TODO TODO TODO TODO TODO TODO TODO TODO TODO TODO TODO TODO TODO TODO TODO TODO TODO TODO TODO TODO TODO TODO TODO TODO TODO TODO TODO 
% TODO TODO TODO TODO TODO TODO TODO TODO TODO TODO TODO TODO TODO TODO TODO TODO TODO TODO TODO TODO TODO TODO TODO TODO TODO TODO TODO TODO TODO TODO TODO TODO TODO 
% TODO TODO TODO TODO TODO TODO TODO TODO TODO TODO TODO TODO TODO TODO TODO TODO TODO TODO TODO TODO TODO TODO TODO TODO TODO TODO TODO TODO TODO TODO TODO TODO TODO 
% Repetisjon fra introen om ANN.tex
% F.eks. ta vekk første 3 linjene, under. Skriv om det under det igjen!
% 	Digital simulations are conducted with discrete time, that is with a finite temporal resolution.
% 	The computational load of a simulator increase with the temporal resolution, and a finite temporal resolution enables a simulation to have finite computational load.
% 	This simplification also introduce the truncation error to the simulation.
% %	This simplification enables the simulation with finite computational resources, but also limits the accuracy of the simulation.
% 
% 	%From the discretization of time, truncation errors arise
% 	%From the discretization of time comes the truncation error.
% 	As the variable in question is updated explicitly every iteration, the value utilized for computing effects like leakage is the previously computed value.
% 	This cause a delay of up to one time iteration, and is the background of the local truncation error(the truncation error from each time step) for the simulation.
% 	Because the Numerical Integration Method($NIM$) is fundamentally different from the simulation model based on synaptic flow($\kappa M$), as presented in sec. \ref{secDevelopmentOfTheNovelANNmodel},
% 		 the two models are analyzed separately.
% 	%As the Numerical Integration Method($NIM$) is fundamentally different from the simulation model based on synaptic flow($\kappa M$), introduced in this text, the two models will be analyzed separately.
% 	%Because the two simulation models considered in this text is fundamentally different, the error mechanisms of the two models will be analyzed separately.
% 	All analysis done in this text are of the unimproved model, where numerical computations are executed by a simple sample--and--hold technique.
% 	Optimization by e.g. estimating intermediate values can be utilized for both models, but will not be considered in this text.
	
%not from the article, any more.. FRA ARTIKKELEN: (KOPIER TILBAKE TIL ARTIKKELEN!)
        When simulating time variant variables in discrete--time environments such as the digital computer, truncation errors arise from the discretization of time.
		Mechanisms than make the variable time variant are computed based on the previously updated value instead of continuously updating the  value.
		%Mechanisms that cause the variable to change as a consequence of time alone are computed based on the previously updated value instead of continuously updating the value. %a continuously updated value.
		%%This implies that all depolarizing input and the leakage during the time step does not influence that time step's simulated leakage.
		For a $NIM$ simulation, this implies that all depolarizing input and the effect of leakage during the time step does not influence the total size of that time step's leakage.
%%		The leakage is computed based on the depolarization value at the initiation of the computational time step.
		The size of this discretization error increase for larger computational time steps.
		%As this effect becomes larger for larger computational time steps, it can be stated that the size of this error vary with the size of the computational time step.
		As mentioned in section \ref{ssecDepolarizationSimulationByNIM}, a simulation with a smaller error can therefore easily be designed by increasing the temporal resolution of the simulation.
			%but this is not a preferable solution as it also increases the computational load. 
		This is not a preferable solution as it also increases the computational load of the simulation. 


        %When simulating time variant variables in discrete--time  environments such as the digital computer, truncation errors arise from the discretization of time.
        %The variable is updated based on the previous time step's value, delayed up to $\Delta t$ time units.
        %This introduces an error that vary with the size of the computational time step.
%%%
        %Because the Numerical Integration Model($NIM$) is fundamentally different from a simulation model that considers depolarizing flow($\kappa M$) as the one presented in sec. \ref{secDevelopmentOfTheNovelANNmodel}, 
        Because the Numerical Integration Model($NIM$) is fundamentally different from a simulation model that considers depolarizing flow($\kappa M$),
			the two models' error mechanisms are analyzed separately.
%
        All analysis done in this text are of the unimproved models, implemented with a simple sample--and--hold numerical integration technique.
        Optimization by e.g. estimating the intermediate values in each time iteration can be utilized for both models, but will not be covered in this text.
% TODO 										 (KOPIER TILBAKE TIL ARTIKKELEN!)



%TODO TODO TODO TODO TODO TODO TODO TODO TODO TODO TODO TODO TODO TODO TODO 
%TODO TODO TODO TODO TODO TODO TODO TODO TODO TODO TODO TODO TODO TODO TODO  Find citations(references) for the above section.
%TODO TODO TODO TODO TODO TODO TODO TODO TODO TODO TODO TODO TODO TODO TODO 
	\subsection{Numerical Integration Method($NIM$)}
	\label{ssecErrorForNIM}
% TODO Må ha definert uttr. tidligere:xxx Local truncation error,xxx Inter--spike truncation error,xxx Global truncation error, 
	%Every inter--spike interval is completed when the depolarization crosses the firing threshold, causing the neuron to fire the next spike. %action potential.
	%Every inter--spike interval is completed when the depolarization goes to suprathreshold levels, causing the neuron to fire the next spike. %action potential.
	Every inter--spike interval is completed when the depolarization goes to suprathreshold levels, causing the initiation of the next spike. % to fire the next spike. %action potential.
	The neuron's depolarization therefore goes through a net rising phase(from the resting membrane potential $v_r$ to the firing threshold $\tau$) in the course of an inter--spike interval. 
%%
	%A rising phase implies that a delayed value is less than the correct value. 
	%A rising phase defines that a delayed value is less than the correct value. 
	A rising phase means that a delayed earlier value is smaller than the correct value. 
	%As the leakage is proportional to the depolarization, utilizing an earlier value thus gives less leakage for the discrete--time $NIM$ simulation.
	%As the leakage is proportional to the depolarization, the computed leakage therefore is less in the discrete--time $NIM$ simulation due to the discretization of time.
	As the leakage is proportional to the depolarization, the computed leakage therefore is less in the discrete--time $NIM$ simulation,
		and the size of the error is defined by the size of the computational time step.
	%As the leakage is proportional to the depolarization, utilizing a delayed value therefore makes the leakage smaller for the discrete--time $NIM$ simulation.
	%As the computed leakage is proportional to this value, the discrete--time $NIM$ simulation produce a smaller leakage. % is the result in a $NIM$ simulation.
%%
% 	%A rising phase implies that a delayed value generally is less than the correct value, something that gives a smaller leakage in a $NIM$ simulation.
% 	A rising phase implies that a delayed value is less than the correct value, something that gives a smaller leakage in a $NIM$ simulation.
%%%
	%As the delay is the direct result of the computational time step, the size of the computational time step defines the size of the local truncation error.
	%When simulating the neuron in discrete time, the delay is the direct result of the size of the computational time step and defines the connection between the size of the computational time step and the error.
%%%
	%In the course of an inter--spike interval, all local truncation errors caused by this effect are integrated up to what will be referred to as the inter--spike truncation error, and causes the threshold crossing to happen too early.
	In the course of an inter--spike interval, all local truncation errors caused by this effect are integrated up to what will be referred to as the inter--spike truncation error,
		and causes the threshold crossing to happen before too early.
%%%
% %% ELLER
% 	A rising phase implies that a delayed value generally is less than the correct value.
% 	During the whole inter--spike interval, all local truncation errors caused by this effect are integrated, and the total leakage that period is less for the discrete simulation than for the original system(in continuous time).
% %% end: ELLER
% 	This cause a positive error for the simulated depolarization, and the next spike is initiated to early.
% 	%This cause the depolarization to be to large for the discrete--time simulation, and the next spike is initiated to early.
	An early firing gives an earlier start for the next inter--spike interval, causing the depolarization of that inter--spike interval to have been integrated for too long. %be too large for any given time instant.


	
% 	An error with opposite sign ...
	An opposite error comes as a consequence of having discrete possible firing times.
	The action potential is defined to happen as a consequence of the depolarization crossing the firing threshold from below.
	To preserve causality in a network of artificial neurons, the action potential therefore have to be delayed to the time step after the threshold crossing.
	This introduces a small delay before firing, causing a delayed transmission and a delayed initiation of the subsequent inter--spike interval.
	%UNDER: Heller skrive at dette lager en motsatt feil (av den over)?
	As previously described, an erroneous time of initiation of an inter--spike interval will cause an error for the simulated depolarization for all time after that instant. 
	%As previously mentioned, an erroneous time of initiation of an inter--spike interval will cause an error for the simulated depolarization, that inter--spike interval.
	%As previously mentioned, an erroneous time of initiation of an inter--spike interval will cause an error for the simulated depolarization caused by the late initiation of that inter--spike interval.



%	An opposite error comes from having discrete possible firing times for the neuron.
%	To implement causality in a neural network and assure that synaptic transmissions comes after firing, the firing of an action potential can be delayed to the subsequent time iteration after the threshold crossing.
%	This cause a small delay before the initiation of the next inter--spike interval and therefore a negative error for the depolarization in that inter--spike interval. 
%	%This cause a small delay before the initiation of the next inter--spike interval, and cause an opposite effect for the neuron's depolarization than the previously mentioned error. %erroneous leakage.

%TODO TODO  The cumulative property of the $NIM$ error is the result. TODO TODO Skriv om akkumulering av feil! Positiv feil + stokastisk negativ feil.
 	The net inter--spike truncation error is given by the relative size of the two mechanisms.
	%The size of the two error mechanisms varies, and the net inter--spike truncation error is given by the relative size of the two mechanisms.
	The error from an erroneous leakage caused by the discretization of time vary from having a size of $e_l=0$ if the neuron uses an eternity to reach the firing threshold,
	%The error from the first error mechanisms vary from having a size of $e_l=0$ if the neuron uses an eternity to reach the firing threshold,
		 to the size of the correct leakage if the depolarization goes all the way from $v_r$ to $\tau$ in one iteration.
	The error caused by having discrete possible firing times varies from $e_d=0$ if the threshold crossing happens at the very end of the iteration to 
		having a size given by the size of a full computational time step if the threshold crossing happens    at the very beginning of the time step. %immediately after the initiation of that time step.
	%The global truncation error is thus very hard to predict, and its differential have an appearance of being stochastic.
	%The differential of the global truncation error is therefore very hard to predict, especially as the second error mechanism have an appearance of being stochastic.
	The differential of the accumulated truncation error, referred to as the global truncation error, is therefore very hard to predict. %, especially as the second error mechanism have an appearance of being stochastic.
	%If the inter--spike truncation error is systematic in any way, the global truncation error will diverge for $t\to\infty$ as the error after one inter--spike period can be seen as the differential of the global truncation error.
	As the global truncation error can be defined as the sum of all previous inter--spike truncation errors, 
		any systematic local truncation error(with an expectancy value different from zero) therefore cause the global truncation error to diverge in a $NIM$ simulation as $t\to\infty$.
	%If the inter--spike truncation error is systematic in any way, the global truncation error will diverge for $t\to\infty$. % as the error after one inter--spike period can be seen as the differential of the global truncation error.
	%If the increase of the global truncation error after an inter--spike period is systematic in any way, 
%%%
	%This is probably not preferable, and should be avoided.
	%This is probably not preferable.
	%This is not preferable.
	



















	
% // vim:fdm=marker:fmr=//{,//}
